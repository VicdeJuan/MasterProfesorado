\documentclass[palatino]{apuntesURJC}

\title{Educación, Sociedad y Familia}
\author{Víctor de Juan Sanz}
\date{16/17 C2}

% Paquetes adicionales

\begin{abstract}
Apuntes tomados en clase de la asignatura \textit{Educación, Sociedad y Familia}, impartida por Ana Romero, doctora en Filosofía y profesora en la URJC.
\end{abstract}
% --------------------

\begin{document}
\pagestyle{plain}
\maketitle

\tableofcontents
\newpage
% Contenido.


\chapter{Educación}

\section{Ideas previas}

El \concept[Saber\IS práctico]{saber práctico} es el que se aprende en el hacer.
%
Que se aprenda haciendo no significa que no haya teoría.
%
Por ejemplo, la educación es un saber práctico.


El \concept[Saber\IS pragmático]{saber pragmático}, sólo si sirven para algo.
%
Reducir el saber práctico al saber pragmático, nos cargamos la esencia.
%
Con las personas por ejemplo, se ve claro.
%
Una persona se valora por lo que es, no para lo que sirve.


La educación se puede considerar un arte.
%
No es una mera aplicación técnica ni de recetas.

La reflexión no es un lujo sino un elemento indispensable.
%
Para guiar esta reflexión, hay 4 preguntas fundamentales, que hacerse en orden:
%
\begin{itemize}
	\item ¿Qué? ¿Qué estoy haciendo? ¿Qué están aprendiendo?
	\item ¿Porqué? ¿Porqué quiero hacer esto? ¿Porqué un alumno no se comporta bien?
	\item ¿Para qué? ¿Para qué quiero hacer esto? ¿Para qué
	\item ¿Cómo? \subitem Solamente cuando las 3 anteriores están contestadas, tiene sentido preguntarnos por el cómo.
\end{itemize}



\textbf{Conclusión}

Hay que hacer para aprender, pensando sobre el sentido de la acción educativa (la teoría: el porqué y el para qué de lo que hago) y pensando también sobre la propia acción para mejorarla.

\subsection{La Educación como fenómeno humano}

El hombre es el único animal que necesita aprender a comportarse como lo que es, como ser humano.
%
Además, puede comportarse como lo que no es, puede comportarse inhumanamente.

El ser humano necesita de otros seres humanos para aprender.
%
Los casos de seres humanos salvajes (criados en la naturaleza), con inmensas dificultades para socializar, avalan este argumento.
%
El resto de animales tienen una \textbf{dotación instintiva} mucho mayor.
%
Si a un animal lo abandonas con animales de otra especie, desarrolla comportamientos de la especie, debido a los instintos.
%
En el ser humano, no ocurre tal.

El ser humano tiene 3 aspectos que hacen que el ser humano pueda ser educado.
\begin{itemize}
	\item \textbf{\concept[Indeterminación\IS Biológica]{Indeterminación biológica:}} el ser humano nace inacabado, tanto a nivel cultural como a nivel fisiológico.
	%
	Esto permite (y a la vez exige) el aprendizaje.
	\subitem Es clave en este aspecto la racionalidad del hombre.
	\item \textbf{Libertad:} tomar las propias decisiones y de elegir los propios fines a los que uno se dirige y a los que dirige su vida.
	\item \textbf{Trascendencia (capacidad) y necesidad del otro:} El ser humano es un animal social.
\end{itemize}

Vamos a seguir profundizando.

Delval, un psicólogo español de gran renombre, acuñó el término \concept[Plasticidad]{plasticidad}.
%
La capacidad de seguir aprendiendo. Además, esta capacidad es muy grande.
%
No hay fin para el aprendizaje del hombre, tampoco a nivel neurobiológico, como constatan los últimos estudios.

La plasticidad humana se muestra también en la mano.
%
La mano no tiene una funcionalidad concreta, como son las membranas de un palmípedo \footnote{rana}.
%
Al no tener una funcionalidad concreta, decimos que es abierta, que tiene una \concept[Indeterminación\IS funcional]{indeterminación funcional}.


Ortega dice:

La educación es una necesidad natural y es de carácter cultural.
%
Es natural por lo constatado anteriormente.
%
Y decimos que su carácter es cultural, porque la manera de educar depende de la cultura.
%
Se educa distinto en las distintas culturas y en las distintas épocas.


Existen 4 aptitudes que sitúan al hombre en diferente plano al resto de vivientes.
\begin{itemize}
	\item \textbf{Utilización de un \concept[Lenguaje\IS simbólico]{lenguaje simbólico}} El ser humano utiliza un lenguaje de signos para transmitir el pensamiento y para comunicarse. El lenguajes simbólico es exclusivo del ser humano.

	Otros lenguajes que utilizan ciertas especies animales (las más desarrolladas) es el \concept[Lenguaje\IS Dígito]{lenguaje dígito}.
	%
	Este lenguaje está siempre asociado a expresar sus necesidades o las necesidades de su especie.
	%
	El lenguaje simbólico va más allá de simplemente expresar necesidades. Por ejemplo, la poesía.

	\item \textbf{El uso de la técnica} La utilización de la naturaleza para crear herramientas.
	\item \textbf{La ética} la distinción entre el \textit{ser} y el \textit{deber ser}.
	\item \textbf{El arte} La producción de objetos artísticos solamente es propia del ser humano.
\end{itemize}

Como el ser humano es educable, es también manipulable. Precisamente porque tiene tanto por aprender, precisamente por la plasticidad, puede ser manipulado.
%
Cuando hablamos de \concept[Manipulación]{manipulación} hablamos de un uso abusivo del carácter conductor del proceso educativo que no deja lugar o que ahoga la libertad personal y que niega o ahoga el enfrentamiento del educando con la realidad que vive.
%
¿Qué ocurre en las actitudes manipulativas? Muestran que se trata a la persona como un medio al servicio de los propios fines.
%
Además, se silencian algunas de las dimensiones de la libertad.

¿Cómo evitar la manipulación?
%
Para evitar la manipulación, hay que buscar el bien del educando.
%
Lo que diferencia la acción manipulativa de la acción educativa es el fin.
%
La manipulativa es según los  fines del manipulador, la educativa es según los fines particulares del educando.

Lo más esencial para evitar la manipulación es hacerles capaces de pensar por sí mismos.
%
Y no sólo hacerles capaces de pensar por sí mismos, sino de que tengan ese \textbf{hábito}.

Otra posibilidad es mostrar el fin que se busca con la acción educativa.


\textit{En la educación formal:}
Para evitar la manipulación en la actitud educativa es bueno incluir y hacer partícipes a los padres.
%
Esta acción evita la manipulación por parte del profesor. ¿Por qué?
%
Lo habitual es que los padres busquen el bien de sus hijos. Así, el contar con las padres en la acción educativa, ejerce como pared de contención.


\section{La persona y la educación}

Vamos a ver 5 notas que describen a la persona. Estas 5 notas se llaman \concept[Aspecto Fenomenológico]{aspecto fenomenológicos}.

\begin{itemize}
	\item \textbf{Intimidad:} Todos los seres vivos tienen como característica \textit{inmanecia}.
	%
	Lo que se guarda y lo que queda dentro.
	%
	Todo ser vivo lleva a cabo acciones inmanentes, todo ser vivo tiene un mundo interior.
	%
	El ser humano en concreto, tiene un mundo interior que solamente cada ser conoce (salvo que decida darlo a conocer).
	%
	Ese mundo interior que existe en las personas, es un mundo en el que cada uno se puede introducir; cada uno puede adentrarse, sumergirse en su propia intimidad.
	%
	Además, ese adentrarse/sumergirse puede hacerse sin que nadie se de cuenta.
	%
	Por otro lado, es un mundo al que puedo abrirme: ¿Cuáles son mis deseos? ¿Cuáles son mis criterios?
	%
	En lo que llamamos adolescencia del ser humano, se produce el descubrimiento del mundo interior. \footnote{Este es uno de los grandes descubrimientos de la adolescencia.}

	\subitem Esta intimidad puede crecer y hacerse más profunda.

	\subitem Lo íntimo es tan esencial que tiene un sentimiento propio que la protege: la vergüenza.
	%
	Cuando nuestra intimidad aparece al descubierto, aparece \textit{naturalmente} el sentimiento de vergüenza o pudor.

	\item \textbf{Manifestación:} La capacidad para sacar de sí mismo la intimidad.
	%
	La manifestación de la intimidad se hace a través de 3 vías:
	\subitem \textbf{Cuerpo: } Es un mediador entre la interioridad y la exterioridad. La persona manifiesta su intimidad a través del cuerpo.
	%
	Lo que llamamos \concept[Lenguaje\IS no verbal]{Lenguaje no verbal}
	%
	Esto se ve especialmente en una parte del cuerpo: el rostro.
	%
	Incluso, dentro del rostro, la mirada es especialmente expresiva.\footnote{Por ejemplo, cuando quieres ignorar a alguien, dejas de mirarle.}


	\subitem \textbf{Lenguaje:} \footnote{entendiendo lenguaje como lenguaje verbal.}
	%
	A través del lenguaje hago público, comparto con otros aquello que antes estaba dentro de mi.

	El lenguaje tiene mayor capacidad expresiva que el cuerpo y puede permanecer (a través de la escritura sobretodo).

	\subitem \textbf{Acción:}
	%
	Las personas nos damos a conocer con nuestros actos.
	%
	La persona no es sólo sus acciones, pero es aquello que hace. \footnote{No sé si esta afirmación es lógicamente consistente en sí misma.}

	\item \textbf{Libertad:} el ser humano es dueño de sí mismo.
	%
	Es principio y origen de sus actos. Ahora bien, al ser dueña de sus actos, es también dueña de su vida y de su destino.

	\subitem El hombre es esencialmente libre. Aunque tenga mermadas sus libertades (como podría ser una persona en un campo de concentración), la libertad sigue siendo posible.
	%
	Siguen existiendo decisiones que tomar, entre ellas:

	\item \textbf{Capacidad de dar-darse-autoposeerse-libertad: autotrascendencia}
	Llamamos \textit{darse} cuando una persona extrae algo de su intimidad y se lo da a otra persona que lo acepta.

	El ser humano puede dar:

	\subitem \textbf{De lo que hace}

	\subitem \textbf{De lo que tiene:} antropológicamente más complejo. Hay 2 niveles del tener: el nivel material y el nivel inmaterial.
	%
	El hombre puede prestar sus posesiones (un coche) y puede prestar consejo, su experiencia...

	\subitem \textbf{De lo que es:} Este es el nivel más profundo de donación.
	%
	Esa capacidad de dar y de recibir lo dado, se llama \concept[Amor]{amor}.
	%
	Y la intimidad personal se nutre, entre otras cosas, de lo que otros dan.

	\item \textbf{Carácter Dialógico:} El ser humano es el ser que habla.
	%
	El hombre es constitutivamente dialogante, ya que el hombre necesita hablar (hablar, no sólo comunicarse
	%
	\footnote{Porque la verdadera comunicación humana, en todo su espectro, no puede darse de manera completa sin el lenguaje, y además el lenguaje oral, que es la manera más rica de comunicarse.
	%
	Las personas que están privadas del lenguaje oral, ven mermada su capacidad de abstracción})
	%
	para desarrollarse.

\end{itemize}


%% Apendices (ejercicios, examenes)
\appendix

\chapter{---}
% -*- root: ../SociedadyFamilia.tex -*-


\chapter{Comentario informe Delors}

\newcommand{\cita}[1]{"\textit{#1}"}
\newcommand{\clase}[1]{\textit{#1}}

\section*{Capítulo 4} 

\paragraph{Los cuatro pilares de la educación:\\}


El planteamiento educativo del siglo XXI no debe tener como finalidad la acumulación de una cantidad ingente de conocimientos para recurrir a ellos a lo largo de la vida. Dado que no hay fin para el aprendizaje del hombre, este debe ser capaz de llevar a cabo un proceso continuo de actualización y enriquecimiento de ese saber que le permita adaptarse a la sociedad cambiante. 

Como hemos visto en clase, cuando abordamos la tarea educativa hay que pensar en la persona en toda su complejidad (todas sus dimensiones). Para ello, el texto nos define cuatro aprendizajes fundamentales sobre los cuales se asentará el conocimiento de la persona: \cita{aprender a conocer”, “aprender a hacer”, “aprender a vivir juntos” y “aprender a ser}. Cuando desde el punto de vista educativo nos basamos en ellos, estamos educando a la persona desde su totalidad sin obviar ningún aspecto. 

El aprendizaje de \cita{aprender a conocer}, pretende el desarrollo de la dimensión psíquica de la persona. Según el texto, \cita{se debe despertar del alumno la curiosidad intelectual, estimular el sentido crítico y permitir descifrar la realidad, adquiriendo al mismo tiempo una autonomía de juicio}. Por lo tanto, la tarea del profesor no se basa sólo en transmitir habilidades técnicas y productivas, sino que también debe suscitar que crezcan hábitos, orientados al desarrollo de la voluntad y de la propia inteligencia.
%
Cada vez con mayor frecuencia, los progresos del conocimiento se producen en los puntos donde convergen varias disciplinas. Por ello, la especialización educativa debe sustentarse en una amplia cultura general y tener la posibilidad de estudiar a fondo un pequeño número de materias. 

La educación, como todo saber de carácter práctico, es un saber que se aprende en el hacer. En la actualidad, la sustitución del trabajo humano por máquinas nos lleva a la necesidad de desarrollar una educación en la cual \cita{aprender a hacer} y \cita{aprender a conocer} se encuentres fuertemente imbricadas. Para la adaptación del alumnado al futuro mercado laboral será necesario potenciar sus capacidades cognitivas en decremento de su preparación para desempeñar una tarea bien definida. 

Mediante el aprendizaje de \cita{aprender a vivir juntos} se pretende una educación que permita evitar o solucionar los conflictos de manera pacífica a la vez que se mejora las relaciones sociales. La Comisión plantea a la comunidad educativa dos orientaciones para lograr este objetivo:

\begin{itemize}
\item Logar que la persona sea capaz de ponerse en el lugar de los demás y comprender sus reacciones. Para alcanzar esta actitud de empatía, primero deberá conocerse así mismo. 

\subitem El educador del siglo XXI, debe buscar métodos de solución de conflictos que permitan el diálogo e intercambio de argumentos entre los alumnos.

\item Permitir la participación de los jóvenes en proyectos cooperativos, que logren disminuir los conflictos entre individuos.
\end{itemize}

Por último, la enseñanza de \cita{aprender a ser} aparece como un compendió de las tres anteriores, al permitir todas ellas el desarrollo global de la persona. Se busca dotar al individuo de las herramientas que le permitan comprender el mundo que le rodea y comportarse de forma responsable y justa.


\section*{Capítulo 5}

Uno de los conceptos que se definen en este capítulo es la educación permanente que se tiene que adaptar a las sociedades modernas de forma que abarque toda la existencia y se ajuste a todas las dimensiones de la sociedad, adquiriendo el concepto de “educación a lo largo de la vida”, título del presente capítulo aunque no es la primera vez que se menciona el tema en el informe (Capítulo 4).
%
Así se busca brindar a las personas los medios necesarios para alcanzar un equilibrio entre el trabajo y el aprendizaje.
%
Se educa para saber vivir.
%
No basta con instruir a las personas o capacitarlas de inteligencia, sino que es necesario que las personas sepan manejarse en su propia vida respondiendo a las preguntas que les plantee la vida.
%
Por tanto, el educador tiene que equipar a los alumnos para que por si solos respondan dichas preguntas.
%

La educación a lo largo de la vida debe permitirle al ser humano tomar conciencia de sí mismo y de su entorno y desempeñar su función social en el mundo del trabajo y la vida pública.
%
Esto es, el educando es el agente principal en el proceso educativo, pero hay que tener en cuenta también los agentes externos como la familia, las instituciones educativas y la sociedad.
%
Como dijo Kant \cita{El hombre necesita del hombre para ser hombre} por lo que la sociedad juega un papel importante en su proceso educativo.
%
Como se indica en el presente capítulo del Informe Delors, la responsabilidad de la sociedad en el ámbito de la educación se amplía porque se considera la educación como un proceso pluridimensional que no se limita a la adquisición de conocimientos ni depende únicamente de las instituciones educativas.
%
Por tanto, es importante tener en cuenta todos esos factores externos que influyen en el aprendizaje del educando.

Uno de estos factores, tratado en este informe, es la familia indicando que establece enlace entre los aspectos afectivos y cognoscitivos asegurando así mismo la transmisión de valores y normas.
%
Sin embargo, a veces esos valores familiares chocan con los de las instituciones educativas por lo que es muy importante la comunicación entre familia y centros educativos para conseguir un aprendizaje armonioso de forma que dichos factores externos se complementen.
%
En definitiva, el objetivo de la educación es moldear al alumno y no manipularle por lo que tienen que buscar el crecimiento del educando y no el propio provecho para lo que familia e instituciones educativas se tienen que complementar adecuadamente.


Otra forma de inserción social del alumnado es mediante la incorporación al mundo laboral por lo que es clave una conexión entre el sistema educativo y las empresas.
%
Se busca así que cobren consciencia de las limitaciones y oportunidades que ofrece la vida profesional y que adquieran madurez y habilidades sociales.
%
De esta forma pueden conciliar el saber teórico con el saber práctico.
%
La educación es un \clase{saber práctico}, un \clase{saber hacer}, pero \clase{todo saber práctico necesita una buena teoría.}


En este capítulo se trata también el tiempo libre que existe y la necesidad de aprovechar dicho tiempo para que el ser humano se siga enriqueciendo.
%
Entre otras cosas están las instituciones culturales como bibliotecas o museos, pero en un mundo tecnológico hay que prestar especial interés a los medios de comunicación.
%
Por ello se destaca la importancia de la figura del profesor a la hora de inculcar valores en el educando de forma que éste sea capaz de formular juicios y a partir de éstos adquirir una conducta.
%
En definitiva, se busca hacer razonar y reflexionar con rigor al educando (hábitos de la inteligencia) y el hacer que obren de forma justa (hábitos morales).
%
Estos hábitos son los que le van a permitir ser dueño de su propia vida y ser responsable de sus actos permitiendo que las potencias racionales crezcan.


El presente capítulo de este informe termina indicando cómo la educación permanente se ha ampliado en la actualidad antes los nuevos requerimientos de una sociedad en continua transformación.
%
El ser humano necesita poner en práctica de forma continua sus conocimientos y raciocinio para pensar y actuar y hacerlo en sociedad.
%
Es decir, la educación juega un papel fundamental a lo largo de toda la vida ya que permite el perfeccionamiento humano permitiendo alcanzar cada vez metas más complejas, y, puesto que las personas viven en sociedad, es muy importante aprovechar las posibilidades que ésta ofrece en la educación.
%

En definitiva, el fin de la educación es conseguir el crecimiento de las personas en todas sus dimensiones prestando especial atención a las potencias racionales para lo que los agentes externos como familia, instituciones educativas y sociedad juegan un papel importante en la educación del agente principal, el educando.



\section*{Capítulo 7}

El capítulo 7 del informe Delors se titula \cita{El personal docente en busca de nuevas perspectivas}.
%
En él se hace un análisis de la situación actual de la educación en el mundo y a partir de este análisis, el autor establece lo que en opinión deben ser las pautas a seguir para mejorar la calidad de la educación de manera global.


Sabemos que \clase{educar  es más difícil que enseñar porque para enseñar se precisa saber, pero para educar se precisar ser}.
%
En este capítulo se hace especial mención a que los profesores deben ser algo más que "enseñadores".
%
El profesor debe ser educador y para ello se recomienda que pueda tener una formación continua, que pueda compaginar su labor de docente con un trabajo a tiempo parcial en el mundo empresarial, que pueda tener un año sabático con el que poder abrir su mente viajando y conociendo otras culturas, etc.
%
En resumidas cuentas, los educadores además de saber y ser competentes en las disciplinas que ellos imparten, tienen que ser, y para ser hay que conocer mundo, abrir la mente y relacionarse con los demás.
%
Nadie es si se encierra en su cuarto de estudio sin relacionarse con el mundo exterior, ya que, como decíamos en la sección anterior citando a Kant, \cita{el hombre necesita del hombre para ser hombre}. 


Uno de los criterios que salvaguardan a los hombres de la manipulación es el hecho de hacerles capaces de pensar por sí mismos.
%
En este capítulo se hace referencia de una manera explícita en varias partes del informe:

\begin{itemize}
\item \cita{para que puedan adquirir la autonomía, la creatividad y la curiosidad intelectual que son los complementos necesarios de la adquisición de saber, el maestro debe mantener forzosamente una cierta distancia entre la escuela y el entorno, para que los niños y adolescentes tengan ocasión de ejercer su sentido crítico}
\item \cita{Son el trabajo y el diálogo con el docente lo que contribuye a desarrollar el sentido crítico del alumno}.
\end{itemize}



El informe trata sobre la educación de manera global, centrándose principalmente en la educación primaria y secundaria y en los agentes principales y externos de la misma.
%
Coincide con lo que vimos en clase acerca de que para que haya educación se debe tratar de potenciar todas las dimensiones de la persona (salvo el desarrollo físico que se desarrolla de manera natural) y tiene que tratar de hacer crecer y potenciar esas dimensiones de una manera armónica.
%
Entre las nuevas perspectivas que se presentan a los futuros docentes está la de contar con todos los agentes externos involucrados en la educación que son evidentemente los docentes y las instituciones educativas, pero también a la familia y a la sociedad en su conjunto.
%

En el informe, por un lado considera esencial que la familia forme parte del proceso de educación y colabore de una manera activa en dicho proceso, tema desarrollado más ampliamente en el capítulo 5.
%
Además, cita un estudio realizado Filipinas en la que un grupo de padres colaboró activamente en el proceso de educación de sus hijos con resultados muy positivos.

Por otro lado, da un toque de atención a la sociedad y sobre todo a aquellas personas que piensan que no forman parte de la educación de los menores y juzgan que todos los males de nuestra sociedad actual se deben a políticas educativas que ellos consideran malas.
%
Cuando en realidad no es así.
%
\cita{Es la sociedad misma, con todos sus elementos, a la que corresponde subsanar las graves deficiencias que afectan su funcionamiento y reconstituir los elementos indispensables para la vida social y las relaciones interpersonales}.




\chapter{Circo de las mariposas}

\paragraph{Explicar las condiciones de cómo se da la educación en el corto "El circo de las Mariposas"}.

\newcommand{\nick}{Will\xspace}
\newcommand{\mendez}{Méndez\xspace}


El corto trata sobre el crecimiento como persona del protagonista. 
%
En el primer circo en el que trabaja \nick sólo se da una única condición: \textbf{la contribución}. 
%
\nick tiene su papel en el circo, pero no es un papel suficientemente importante para que se sienta realizado. 
%
Tal vez no sienta que lo que hace aporta o que su contribución es valiosa... Por otro lado, tampoco parece muy desinteresada su colaboración sino coartada. 
%
El trato que recibe del primer director no es favorecedor de ninguna de las demás condiciones.

\nick no es aceptado por ser. No se siente valorado por nadie. 
%
Difícilmente \nick puede alcanzar la plenitud como hombre si no es aceptado, ni por los demás ni por él mismo y menos si cree que nadie le valora, si no tiene el aprecio de nadie...
%
Esa falta de aceptación y aprecio resulta en él en agresividad y rechazo como posibles mecanismos de defensa para reducir el sufrimiento que causa el rechazo continuo recibido.
%
Pero también faltan otras condiciones. 
%
Difícilmente puede sentirse parte del circo y de la sociedad. De hecho, podríamos decir que no se considera ni siquiera hombre:
"Un hombre, si es que puede llamarse así" es la frase con la que \nick es introducido en el circo y que, según \mendez, el propio \nick se cree.
%
Este trato que recibe pone el énfasis en sus imperfecciones y no en sus capacidades.
%
Hasta él mismo se ha creído que su única \textbf{competencia} es la de mostrar sus imperfecciones, como se aprecia en el diálogo con el contorsionista:

- Aquí no tenemos un show de rarezas.
- ¿Qué quieres decir? Todos los circos tienen uno. 

Pero para que \nick llegue a sentirse competente y que es capaz de hacer algo por sí mismo necesita que se den previamente otras condiciones.
%
Las casas no se construyen por el tejado.



Podemos tomar a \mendez como educador ejemplar, ya que consigue establecer las condiciones necesarias para que \nick pueda crecer y perfeccionarse como hombre, en todas sus dimensiones.
%
Lo primero de todo es el sentido de la \textbf{aceptación}.
%
\nick necesita sentirse aceptado y \mendez es lo que le ofrece. 
%
Es aceptado y tratado con normalidad por los miembros del circo.
%
Además, no sólo es aceptado tal y como es, sino \textbf{apreciado} desde el primer momento: "Eres magnífico" le dice \mendez al conocerle.

Una vez ya están más o menos establecidas esas condiciones se pueden ir construyendo otras. 
%
La siguiente, que también contribuye positivamente al sentimiento de aprecio y aceptación es la \textbf{contribución}.
%
Llega un momento, bastante al principio en el que \nick puede \textbf{contribuir}. 
%
Mientras el fortachón está pegando carteles, él es quien hace la pubilicidad, quien da la información a voz en grito. 
%
Aunque no tenga extremidades, ha encontrado su manera de contribuir al circo y eso le hace tener más autoestima, persumiblemente gracias al aumento de aceptación y aprecio recibido.
%
Sin embargo, esa contribución no es suficiente para sentirse plenamente parte del circo. 
%
"¿También formas parte del show?" le pregunta un niño. "No, no exactamente". 
%
Forma parte del grupo, pero no del show que es lo que da identidad al grupo. 
%
No ha encontrado (todavía) su propia identidad en el grupo.
%
Siente que \textbf{pertenece} al grupo, que pertenece a la caravana pero no al show.
%
Cuando llegue el momento de pertenecer por completo al grupo, de pertenecer también al show, \nick podrá realizarse como persona.
%
Solamente es necesario que encuentren cómo puede \nick contribuir al show, de encontrar en lo que \nick es competente y que pueda hacer con confianza.

\mendez muestra a \nick que los demás han tenido unas condiciones previas difíciles, pero que han sabido \textit{resurgir de las cenizas}.
%
\mendez creyó en ellos, y ahora ellos son competentes y tienen confianza.
%
\nick se siente diferente y se lo dice:

- "Ellos son diferentes."

- "Sí, tu tienes una ventaja.
%
Cuanto mayor es la lucha, mayor es el la gloria del triunfo."

\mendez le plantea retos para que vea su \textbf{competencia} y pueda aumentar su \textbf{autoconfianza}.
%
Poco a poco, y cada vez retos mayores.
%
El primero, aceptar que los demás le valoren y acepten. 
%
Después, aceptarse a sí mismo y dejar de creer que está maldito de nacimiento.
%
Estos retos ya los hemos comentado, pero el siguiente es el más interesante. Lentantarse. \mendez cree en él y cree que es capaz de conseguirlo (además, tal vez, de no querer sobreproteger y apartar las dificultades).
%
Sin embargo, \nick no tiene suficiente confianza, no se cree competente. 
%
Aunque sea capaz y sea realmente competente (porque, como vemos después, es capaz de levantarse) , al no tener confianza, no se siente competente.
%
Cuando lo consigue, cuando se consigue levantar, 
%
descubre su \textbf{competencia}, y al ya \textit{sentirse apreciado externamente}, \textbf{aumenta su confianza} en sí mismo.


El último reto que se encuentra no se lo plantea nadie. 
%
Es un reto que le pone la propia vida ante él y del que nadie le puede sacar.
%
Los anteriores eran oportunidades para superarse, sin apartar las dificultades. 
%
Caerse al agua requiere una respuesta y al tener confianza en sí mismo puede afrontar el reto.

Cuando descubren que puede nadar, todos están sorprendidos, incluido el propio \nick.
%
Ahora que ya se siente aceptado y valorado, ahora que ya es competente y tiene confianza han encontrado una manera de \textbf{contribuir} al show y de que \nick \textbf{pertenezca} plenamente al grupo.
%
\nick ha vivido un bonito proceso de crecimiento y ahora está en unas condiciones perfectas para seguir creciendo.




\printindex
\end{document}
