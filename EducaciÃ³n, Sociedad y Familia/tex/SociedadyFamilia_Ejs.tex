% -*- root: ../SociedadyFamilia.tex -*-


\newcommand{\cita}[1]{"\textit{#1}"}
\newcommand{\clase}[1]{\textit{#1}}


\section*{Capítulo 4} 

\paragraph{Los cuatro pilares de la educación:\\}


El planteamiento educativo del siglo XXI no debe tener como finalidad la acumulación de una cantidad ingente de conocimientos para recurrir a ellos a lo largo de la vida. Dado que no hay fin para el aprendizaje del hombre, este debe ser capaz de llevar a cabo un proceso continuo de actualización y enriquecimiento de ese saber que le permita adaptarse a la sociedad cambiante. 

Como hemos visto en clase, cuando abordamos la tarea educativa hay que pensar en la persona en toda su complejidad (todas sus dimensiones). Para ello, el texto nos define cuatro aprendizajes fundamentales sobre los cuales se asentará el conocimiento de la persona: \cita{aprender a conocer”, “aprender a hacer”, “aprender a vivir juntos” y “aprender a ser}. Cuando desde el punto de vista educativo nos basamos en ellos, estamos educando a la persona desde su totalidad sin obviar ningún aspecto. 

El aprendizaje de \cita{aprender a conocer}, pretende el desarrollo de la dimensión psíquica de la persona. Según el texto, \cita{se debe despertar del alumno la curiosidad intelectual, estimular el sentido crítico y permitir descifrar la realidad, adquiriendo al mismo tiempo una autonomía de juicio}. Por lo tanto, la tarea del profesor no se basa sólo en transmitir habilidades técnicas y productivas, sino que también debe suscitar que crezcan hábitos, orientados al desarrollo de la voluntad y de la propia inteligencia.
%
Cada vez con mayor frecuencia, los progresos del conocimiento se producen en los puntos donde convergen varias disciplinas. Por ello, la especialización educativa debe sustentarse en una amplia cultura general y tener la posibilidad de estudiar a fondo un pequeño número de materias. 

La educación, como todo saber de carácter práctico, es un saber que se aprende en el hacer. En la actualidad, la sustitución del trabajo humano por máquinas nos lleva a la necesidad de desarrollar una educación en la cual \cita{aprender a hacer} y \cita{aprender a conocer} se encuentres fuertemente imbricadas. Para la adaptación del alumnado al futuro mercado laboral será necesario potenciar sus capacidades cognitivas en decremento de su preparación para desempeñar una tarea bien definida. 

Mediante el aprendizaje de \cita{aprender a vivir juntos} se pretende una educación que permita evitar o solucionar los conflictos de manera pacífica a la vez que se mejora las relaciones sociales. La Comisión plantea a la comunidad educativa dos orientaciones para lograr este objetivo:

\begin{itemize}
\item Logar que la persona sea capaz de ponerse en el lugar de los demás y comprender sus reacciones. Para alcanzar esta actitud de empatía, primero deberá conocerse así mismo. 

\subitem El educador del siglo XXI, debe buscar métodos de solución de conflictos que permitan el diálogo e intercambio de argumentos entre los alumnos.

\item Permitir la participación de los jóvenes en proyectos cooperativos, que logren disminuir los conflictos entre individuos.
\end{itemize}

Por último, la enseñanza de \cita{aprender a ser} aparece como un compendió de las tres anteriores, al permitir todas ellas el desarrollo global de la persona. Se busca dotar al individuo de las herramientas que le permitan comprender el mundo que le rodea y comportarse de forma responsable y justa.


\section*{Capítulo 5}

Uno de los conceptos que se definen en este capítulo es la educación permanente que se tiene que adaptar a las sociedades modernas de forma que abarque toda la existencia y se ajuste a todas las dimensiones de la sociedad, adquiriendo el concepto de “educación a lo largo de la vida”, título del presente capítulo aunque no es la primera vez que se menciona el tema en el informe (Capítulo 4).
%
Así se busca brindar a las personas los medios necesarios para alcanzar un equilibrio entre el trabajo y el aprendizaje.
%
Se educa para saber vivir.
%
No basta con instruir a las personas o capacitarlas de inteligencia, sino que es necesario que las personas sepan manejarse en su propia vida respondiendo a las preguntas que les plantee la vida.
%
Por tanto, el educador tiene que equipar a los alumnos para que por si solos respondan dichas preguntas.
%

La educación a lo largo de la vida debe permitirle al ser humano tomar conciencia de sí mismo y de su entorno y desempeñar su función social en el mundo del trabajo y la vida pública.
%
Esto es, el educando es el agente principal en el proceso educativo, pero hay que tener en cuenta también los agentes externos como la familia, las instituciones educativas y la sociedad.
%
Como dijo Kant \cita{El hombre necesita del hombre para ser hombre} por lo que la sociedad juega un papel importante en su proceso educativo.
%
Como se indica en el presente capítulo del Informe Delors, la responsabilidad de la sociedad en el ámbito de la educación se amplía porque se considera la educación como un proceso pluridimensional que no se limita a la adquisición de conocimientos ni depende únicamente de las instituciones educativas.
%
Por tanto, es importante tener en cuenta todos esos factores externos que influyen en el aprendizaje del educando.

Uno de estos factores, tratado en este informe, es la familia indicando que establece enlace entre los aspectos afectivos y cognoscitivos asegurando así mismo la transmisión de valores y normas.
%
Sin embargo, a veces esos valores familiares chocan con los de las instituciones educativas por lo que es muy importante la comunicación entre familia y centros educativos para conseguir un aprendizaje armonioso de forma que dichos factores externos se complementen.
%
En definitiva, el objetivo de la educación es moldear al alumno y no manipularle por lo que tienen que buscar el crecimiento del educando y no el propio provecho para lo que familia e instituciones educativas se tienen que complementar adecuadamente.


Otra forma de inserción social del alumnado es mediante la incorporación al mundo laboral por lo que es clave una conexión entre el sistema educativo y las empresas.
%
Se busca así que cobren consciencia de las limitaciones y oportunidades que ofrece la vida profesional y que adquieran madurez y habilidades sociales.
%
De esta forma pueden conciliar el saber teórico con el saber práctico.
%
La educación es un \clase{saber práctico}, un \clase{saber hacer}, pero \clase{todo saber práctico necesita una buena teoría.}


En este capítulo se trata también el tiempo libre que existe y la necesidad de aprovechar dicho tiempo para que el ser humano se siga enriqueciendo.
%
Entre otras cosas están las instituciones culturales como bibliotecas o museos, pero en un mundo tecnológico hay que prestar especial interés a los medios de comunicación.
%
Por ello se destaca la importancia de la figura del profesor a la hora de inculcar valores en el educando de forma que éste sea capaz de formular juicios y a partir de éstos adquirir una conducta.
%
En definitiva, se busca hacer razonar y reflexionar con rigor al educando (hábitos de la inteligencia) y el hacer que obren de forma justa (hábitos morales).
%
Estos hábitos son los que le van a permitir ser dueño de su propia vida y ser responsable de sus actos permitiendo que las potencias racionales crezcan.


El presente capítulo de este informe termina indicando cómo la educación permanente se ha ampliado en la actualidad antes los nuevos requerimientos de una sociedad en continua transformación.
%
El ser humano necesita poner en práctica de forma continua sus conocimientos y raciocinio para pensar y actuar y hacerlo en sociedad.
%
Es decir, la educación juega un papel fundamental a lo largo de toda la vida ya que permite el perfeccionamiento humano permitiendo alcanzar cada vez metas más complejas, y, puesto que las personas viven en sociedad, es muy importante aprovechar las posibilidades que ésta ofrece en la educación.
%

En definitiva, el fin de la educación es conseguir el crecimiento de las personas en todas sus dimensiones prestando especial atención a las potencias racionales para lo que los agentes externos como familia, instituciones educativas y sociedad juegan un papel importante en la educación del agente principal, el educando.



\section*{Capítulo 7}

El capítulo 7 del informe Delors se titula \cita{El personal docente en busca de nuevas perspectivas}.
%
En él se hace un análisis de la situación actual de la educación en el mundo y a partir de este análisis, el autor establece lo que en opinión deben ser las pautas a seguir para mejorar la calidad de la educación de manera global.


Sabemos que \clase{educar  es más difícil que enseñar porque para enseñar se precisa saber, pero para educar se precisar ser}.
%
En este capítulo se hace especial mención a que los profesores deben ser algo más que "enseñadores".
%
El profesor debe ser educador y para ello se recomienda que pueda tener una formación continua, que pueda compaginar su labor de docente con un trabajo a tiempo parcial en el mundo empresarial, que pueda tener un año sabático con el que poder abrir su mente viajando y conociendo otras culturas, etc.
%
En resumidas cuentas, los educadores además de saber y ser competentes en las disciplinas que ellos imparten, tienen que ser, y para ser hay que conocer mundo, abrir la mente y relacionarse con los demás.
%
Nadie es si se encierra en su cuarto de estudio sin relacionarse con el mundo exterior, ya que, como decíamos en la sección anterior citando a Kant, \cita{el hombre necesita del hombre para ser hombre}. 


Uno de los criterios que salvaguardan a los hombres de la manipulación es el hecho de hacerles capaces de pensar por sí mismos.
%
En este capítulo se hace referencia de una manera explícita en varias partes del informe:

\begin{itemize}
\item \cita{para que puedan adquirir la autonomía, la creatividad y la curiosidad intelectual que son los complementos necesarios de la adquisición de saber, el maestro debe mantener forzosamente una cierta distancia entre la escuela y el entorno, para que los niños y adolescentes tengan ocasión de ejercer su sentido crítico}
\item \cita{Son el trabajo y el diálogo con el docente lo que contribuye a desarrollar el sentido crítico del alumno}.
\end{itemize}



El informe trata sobre la educación de manera global, centrándose principalmente en la educación primaria y secundaria y en los agentes principales y externos de la misma.
%
Coincide con lo que vimos en clase acerca de que para que haya educación se debe tratar de potenciar todas las dimensiones de la persona (salvo el desarrollo físico que se desarrolla de manera natural) y tiene que tratar de hacer crecer y potenciar esas dimensiones de una manera armónica.
%
Entre las nuevas perspectivas que se presentan a los futuros docentes está la de contar con todos los agentes externos involucrados en la educación que son evidentemente los docentes y las instituciones educativas, pero también a la familia y a la sociedad en su conjunto.
%

En el informe, por un lado considera esencial que la familia forme parte del proceso de educación y colabore de una manera activa en dicho proceso, tema desarrollado más ampliamente en el capítulo 5.
%
Además, cita un estudio realizado Filipinas en la que un grupo de padres colaboró activamente en el proceso de educación de sus hijos con resultados muy positivos.

Por otro lado, da un toque de atención a la sociedad y sobre todo a aquellas personas que piensan que no forman parte de la educación de los menores y juzgan que todos los males de nuestra sociedad actual se deben a políticas educativas que ellos consideran malas.
%
Cuando en realidad no es así.
%
\cita{Es la sociedad misma, con todos sus elementos, a la que corresponde subsanar las graves deficiencias que afectan su funcionamiento y reconstituir los elementos indispensables para la vida social y las relaciones interpersonales}.



\section{Circo de las mariposas}

Explicar las condiciones de cómo se da la educación en el corto "El circo de las Mariposas".
