

\newcommand{\nick}{Will\xspace}
\newcommand{\mendez}{Méndez\xspace}


El corto trata sobre el crecimiento como persona del protagonista. 
%
En el primer circo en el que trabaja \nick sólo se da una única condición: \textbf{la contribución}. 
%
\nick tiene su papel en el circo, pero no es un papel suficientemente importante para que se sienta realizado. 
%
Tal vez no sienta que lo que hace aporta o que su contribución es valiosa... Por otro lado, tampoco parece muy desinteresada su colaboración sino coartada. 
%
El trato que recibe del primer director no es favorecedor de ninguna de las demás condiciones.

\nick no es aceptado por ser. No se siente valorado por nadie. 
%
Difícilmente \nick puede alcanzar la plenitud como hombre si no es aceptado, ni por los demás ni por él mismo y menos si cree que nadie le valora, si no tiene el aprecio de nadie...
%
Esa falta de aceptación y aprecio resulta en él en agresividad y rechazo como posibles mecanismos de defensa para reducir el sufrimiento que causa el rechazo continuo recibido.
%
Pero también faltan otras condiciones. 
%
Difícilmente puede sentirse parte del circo y de la sociedad. De hecho, podríamos decir que no se considera ni siquiera hombre:
"Un hombre, si es que puede llamarse así" es la frase con la que \nick es introducido en el circo y que, según \mendez, el propio \nick se cree.
%
Este trato que recibe pone el énfasis en sus imperfecciones y no en sus capacidades.
%
Hasta él mismo se ha creído que su única \textbf{competencia} es la de mostrar sus imperfecciones, como se aprecia en el diálogo con el contorsionista:

- Aquí no tenemos un show de rarezas.
- ¿Qué quieres decir? Todos los circos tienen uno. 

Pero para que \nick llegue a sentirse competente y que es capaz de hacer algo por sí mismo necesita que se den previamente otras condiciones.
%
Las casas no se construyen por el tejado.



Podemos tomar a \mendez como educador ejemplar, ya que consigue establecer las condiciones necesarias para que \nick pueda crecer y perfeccionarse como hombre, en todas sus dimensiones.
%
Lo primero de todo es el sentido de la \textbf{aceptación}.
%
\nick necesita sentirse aceptado y \mendez es lo que le ofrece. 
%
Es aceptado y tratado con normalidad por los miembros del circo.
%
Además, no sólo es aceptado tal y como es, sino \textbf{apreciado} desde el primer momento: "Eres magnífico" le dice \mendez al conocerle.

Una vez ya están más o menos establecidas esas condiciones se pueden ir construyendo otras. 
%
La siguiente, que también contribuye positivamente al sentimiento de aprecio y aceptación es la \textbf{contribución}.
%
Llega un momento, bastante al principio en el que \nick puede \textbf{contribuir}. 
%
Mientras el fortachón está pegando carteles, él es quien hace la pubilicidad, quien da la información a voz en grito. 
%
Aunque no tenga extremidades, ha encontrado su manera de contribuir al circo y eso le hace tener más autoestima, persumiblemente gracias al aumento de aceptación y aprecio recibido.
%
Sin embargo, esa contribución no es suficiente para sentirse plenamente parte del circo. 
%
"¿También formas parte del show?" le pregunta un niño. "No, no exactamente". 
%
Forma parte del grupo, pero no del show que es lo que da identidad al grupo. 
%
No ha encontrado (todavía) su propia identidad en el grupo.
%
Siente que \textbf{pertenece} al grupo, que pertenece a la caravana pero no al show.
%
Cuando llegue el momento de pertenecer por completo al grupo, de pertenecer también al show, \nick podrá realizarse como persona.
%
Solamente es necesario que encuentren cómo puede \nick contribuir al show, de encontrar en lo que \nick es competente y que pueda hacer con confianza.

\mendez muestra a \nick que los demás han tenido unas condiciones previas difíciles, pero que han sabido \textit{resurgir de las cenizas}.
%
\mendez creyó en ellos, y ahora ellos son competentes y tienen confianza.
%
\nick se siente diferente y se lo dice:

- "Ellos son diferentes."

- "Sí, tu tienes una ventaja.
%
Cuanto mayor es la lucha, mayor es el la gloria del triunfo."

\mendez le plantea retos para que vea su \textbf{competencia} y pueda aumentar su \textbf{autoconfianza}.
%
Poco a poco, y cada vez retos mayores.
%
El primero, aceptar que los demás le valoren y acepten. 
%
Después, aceptarse a sí mismo y dejar de creer que está maldito de nacimiento.
%
Estos retos ya los hemos comentado, pero el siguiente es el más interesante. Lentantarse. \mendez cree en él y cree que es capaz de conseguirlo (además, tal vez, de no querer sobreproteger y apartar las dificultades).
%
Sin embargo, \nick no tiene suficiente confianza, no se cree competente. 
%
Aunque sea capaz y sea realmente competente (porque, como vemos después, es capaz de levantarse) , al no tener confianza, no se siente competente.
%
Cuando lo consigue, cuando se consigue levantar, 
%
descubre su \textbf{competencia}, y al ya \textit{sentirse apreciado externamente}, \textbf{aumenta su confianza} en sí mismo.


El último reto que se encuentra no se lo plantea nadie. 
%
Es un reto que le pone la propia vida ante él y del que nadie le puede sacar.
%
Los anteriores eran oportunidades para superarse, sin apartar las dificultades. 
%
Caerse al agua requiere una respuesta y al tener confianza en sí mismo puede afrontar el reto.

Cuando descubren que puede nadar, todos están sorprendidos, incluido el propio \nick.
%
Ahora que ya se siente aceptado y valorado, ahora que ya es competente y tiene confianza han encontrado una manera de \textbf{contribuir} al show y de que \nick \textbf{pertenezca} plenamente al grupo.
%
\nick ha vivido un bonito proceso de crecimiento y ahora está en unas condiciones perfectas para seguir creciendo.
