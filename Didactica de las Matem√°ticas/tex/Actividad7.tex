\section{Análisis de la Unidad Didáctica (UD) de SM\\\quad\quad\small{y comparativa con la UD de Editex}:}


\paragraph{Puntos a mejorar:}


\begin{itemize}
\item \textbf{Estructura:} no encontramos clara la estructura de la UD.
	\subitem No hay una descripción de los objetivos, aunque podemos encontrarlos dispersos en la ruta de aprendizaje.
	\subitem La planificación no está detallada. Sólo hay una sugerencia de temporalización, pero sin concretar qué parte del temario se podría tratar en cada sesión. \\
En este sentido, la UD que encontramos (de Editex) tenía una exposición clara de los objetivos, aunque la planificación es inexistente en ambas.

\item \textbf{Evaluación:} Los criterios de evaluación son muy escasos. En la UD de Editex este aspecto estaba mejor tratado porque estaban más especificados los criterios y tenían relación con los estándares de Aprendizaje definidos por el BOE.

\item \textbf{Contenido transversal:} Un aspecto muy útil que SM no tiene es incluir apartado de contenidos de otras asignaturas para los que la UD de Matemáticas es necesaria y útil, y viceversa.
\footnote{Este apartado ha sido incluido tras la revisión en clase.}

\end{itemize}


\paragraph{Puntos positivos:}

\begin{itemize}
\item La inclusión del apartado de \textbf{conocimientos previos}. Ahorra el tiempo al docente de buscar los conocimientos previos que deberían haber adquirido años anteriores.
Además, se sugiere en cada epígrafe algunos conceptos para repasar.
Este aspecto nos parece muy positivo y en la UD de Editex no aparecía. 

\item \textbf{Recursos:} Hay una muchas sugerencias de recursos (varias referencias a Geogebra, sugerencia de concursos para aplicar \textit{gamificación}) para utilizar durante las sesiones.
La UD de Editex no tenía un apartado de recursos, aunque sí sugiere escasas pautas metodológicas (\textit{Cómo trabajar la unidad}).
\end{itemize}
