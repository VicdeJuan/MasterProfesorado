\section{Análisis de la Unidad Didáctica (UD) de SM\\\quad\quad\small{y comparativa con la UD de Editex}:}

\label{sec1}

\paragraph{Puntos a mejorar:}


\begin{itemize}
\item \textbf{Estructura:} no encontramos clara la estructura de la UD.
	\subitem No hay una descripción de los objetivos, aunque podemos encontrarlos dispersos en la ruta de aprendizaje.
	\subitem La planificación no está detallada. Sólo hay una sugerencia de temporalización, pero sin concretar qué parte del temario se podría tratar en cada sesión. \\
En este sentido, la UD que encontramos (de Editex) tenía una exposición clara de los objetivos, aunque la planificación es inexistente en ambas.

\item \textbf{Evaluación:} Los criterios de evaluación son muy escasos. En la UD de Editex este aspecto estaba mejor tratado porque estaban más especificados los criterios y tenían relación con los estándares de Aprendizaje definidos por el BOE.

\item \textbf{Contenido transversal:} Un aspecto muy útil que SM no tiene es incluir apartado de contenidos de otras asignaturas para los que la UD de Matemáticas es necesaria y útil, y viceversa.
\footnote{Este apartado ha sido incluido tras la revisión en clase.}

\end{itemize}


\paragraph{Puntos positivos:}

\begin{itemize}
\item La inclusión del apartado de \textbf{conocimientos previos}. Ahorra el tiempo al docente de buscar los conocimientos previos que deberían haber adquirido años anteriores.
Además, se sugiere en cada epígrafe algunos conceptos para repasar.
Este aspecto nos parece muy positivo y en la UD de Editex no aparecía. 

\item \textbf{Recursos:} Hay una muchas sugerencias de recursos (varias referencias a Geogebra, sugerencia de concursos para aplicar \textit{gamificación}) para utilizar durante las sesiones.
La UD de Editex no tenía un apartado de recursos, aunque sí sugiere escasas pautas metodológicas (\textit{Cómo trabajar la unidad}).
\end{itemize}


\newpage\section{¿Qué preguntas harías a la hora de analizar si las actividades y las metodologías de la ud son las adecuadas?}

\begin{itemize}
\item Cómo son las actividades? ¿Mecánicas y rutinarias? ¿Retadoras? ¿Trabajan conceptos o sólo procedimientos? ¿Son todas iguales cambiando los números, o cambian también el tipo y el tema?

\item ¿Qué objetivo tiene la unidad? ¿Se corresponde con el propuesto en el currículo oficial?

\item ¿Cuáles son los conocimientos previos necesarios antes de llegar a este tema? ¿Se tratan o se dan por supuestos?

\item ¿Qué errores espero? ¿Qué puedo hacer para solucionarlos?

\item ¿Se pueden plantear metodologías activas o cooperativas para trabajar la unidad? ¿Y recursos digitales o materiales manipulativos?

\item ¿Se trabajan las competencias transversales (comunicación, digital...)? ¿Sirven para trabajar todos los objetivos?

\item ¿Cómo es la evaluación a partir de las actividades? ¿Se puede evaluar la comprensión matemática adquirida con las actividades propuestas?

\item ¿Cuál es tu opinión personal? ¿te parecen adecuadas? ¿Qué cambiarías?
\end{itemize}





\newpage\section{Realizar un análisis de las actividades y metodologías propuestas en la ud de SM utilizando como guía el documento "Análisis de las actividades de la ud.docx"}

Basándonos en la U.D. de SM:

\begin{enumerate}
\item \textbf{¿Qué tipo de actividades son (demandan muchos conocimientos del alumno? ¿son rutinarias? ¿memorísticas? ¿trabajan conceptos o solo procedimientos? }

Son un cúmulo de tareas que sirven para afianzar (rutinarias) e interiorizar (memorísticas) los conocimientos del alumno. Algunas de ellas son más exigentes que demandan mayor conocimiento.

\item \textbf{¿Qué variables didácticas movilizan? ¿sólo cambian los números o cambia sustancialmente la actividad de manera que los alumnos necesitan otras estrategias distintas a la anterior? }

Las actividades propuestas se van alternando dependiendo del contenido impartido, por ejemplo la act.101 trata con la música y la act.97 con la economía para los procesos de modelización. La act.67 y otras hacen referencia a cálculo con calculadora y hay otra para resolver gráficamente con Geogebra.

\item \textbf{¿Qué errores o dificultades esperas que tengan los alumnos al hacer esas actividades? ¿qué harías para resolverlos? }

En una primera visión de la misma, esperamos errores de redondeo, los errores más comunes cometidos en Álgebra y cálculo del valor absoluto.
Para resolverlos proponemos PROHIBIR el lápiz y que los errores aparezcan en rojo para evitar que repitan el fallo.

\item \textbf{¿Qué objetivos tienen? }

Entender los números reales y aplicarlos en las diferentes situaciones e la vida cotidiana.

\item \textbf{¿Trabajan los alumnos en grupo para resolverlas? }

Esa parte queda a libre elección del profesor ya que no se menciona explícitamente.

\item \textbf{¿Planifican usar materiales manipulativos o tecnologías digitales? ¿Calculadora? }

Se utilizan bastante según viene especificado en criterio de evaluación B.1.12

\item \textbf{¿Se fomenta la comunicación con esas actividades? }


Los objetivos 01, 02, 05 06 hacen énfasis sobre la comunicación lingüística.

\item \textbf{¿Cómo están secuenciadas? ¿Cubren todos los objetivos?}

Las actividades están secuenciadas de una manera difícil de entender.\\
Si, y además hay una trazabilidad entre las actividades y las competencias y objetivos de la U.D.

\item \textbf{¿Cómo es la evaluación de la comprensión matemática a partir de las actividades?}


Nos resulta acorde a los objetivos y las competencias de la U.D.

\item \textbf{¿Cuál es tu opinión personal? ¿te parecen adecuadas? ¿Qué cambiarías?}

Nuestra opinión personal respecto a esta unidad didáctica está reflejada en la sección \ref{sec1}

\end{enumerate}


