\section{Actividad 1}


\paragraph{“Eres profesor de matemáticas 1º de ESO y es tu primer día de clase”
¿Qué les dirías a tus alumnos? ¿Qué harás durante esta primera clase?}

\textit{Nota:} Ya han tenido la presentación con su tutor y la tuya es la primera clase del día.


\begin{enumerate}
\item 	\textbf{Presentación del profesor}, incluyendo algunos aspectos como porqué me gustan las matemáticas o porqué soy profesor. 
\subitem Motivación y recordatorio de la diferencia del paso de primaria a secundaria.
\item \textbf{Presentación del alumnado:} cada estudiante se presenta diciendo su nombre y su asignatura favorita o sus hobbies.
\item \textbf{Presentación de la asignatura:} El temario, el horario, la evaluación y la metodología (incluyendo los materiales, tipo de cuaderno, boli y no típex, para que quede constancia de los errores).
\item \textbf{Actividad en grupo} Como un juego de lógica. Por grupos de 3 mejor que por parejas, ya que es más fácil que salga adelante si algún estudiante reacciona con timidez.
\subitem También puede ser buena idea hacer una competición, ya que ésta nos permitiría empezar a ver la personalidad de los chavales: quién se lo toma en serio, quién es más pasota, quién lidera y quién no, etc.
\end{enumerate}

Otros aspectos que podrían pensarse como la prueba de nivel, es mejor dejarlos para otro momento más avanzado, en el que ya se hayan desoxidado después de las vacaciones.

