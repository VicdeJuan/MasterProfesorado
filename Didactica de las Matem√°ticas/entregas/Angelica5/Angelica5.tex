\documentclass[palatino,nochap,nobuilddate]{apuntesURJC}

\usepackage{soul}
\title{Registros de representación}

\author{Víctor de Juan,
Pedro de la Mata Gómez,
Virginia Vadillo Lacasa,
Gustavo Adolfo Martínez Risque}

\date{16/17 C1}

\newcommand{\specialcell}[2][c]{%
  \begin{tabular}[#1]{@{}c@{}}#2\end{tabular}}

% Paquetes adicionales

% --------------------



\begin{document}

\maketitle

\pagestyle{plain}

\begin{table}[hbtp]
\centering
\begin{tabular}{|c|c|}
\hline
\textbf{N grupo} & \textbf{Apellidos, Nombre}
\\\hline
Grupo 3 & \begin{tabular}{c}
de Juan Sanz, Víctor \\\hline
de la Mata Gómez, Pedro \\\hline
Vadillo Lacasa, Virginia \\\hline
Adolfo Martínez Risque, Gustavo
\end{tabular}
\\\hline
\end{tabular}
\caption{Descripción de los miembros del grupo}
\end{table}

\section{Introducción}

Utilizando \cite{Bruno},\cite{Duval},\cite{Ismenia},\cite{Oviedo}.

\section{Contextualización}

\section{Unidad didáctica}

En esta unidad didáctica sólo hemos encontrado conversiones entre registros, pero ningún tratamiento. 
%
Creemos que es algo natural, ya que las diversas representaciones de una función son únicas, salvo cambios de escala en la representación gráfica (que no aportan información nueva), y salvo que empleemos distintos valores en la tabla en la representación tabular (lo que tampoco aportaría ninguna información relevante). 

A continuación, pasamos a describir detalladamente, concepto por concepto de los tratados en la unidad didáctica, los registros de representación identificados.
%
Al final de la descripción detallada, incluimos una tabla resumen.

\subsection{Concepto de función}

En la definición de una función encontramos 2 registros distintos: 
%
el \ul{registro natural} al explicar en lenguaje natural el concepto de función y el \ul{registro gr\'afico} para dar una representación gráfica del concepto de función. 
%
Esto sería una \textbf{conversión} entre \textit{registros semióticos} de un mismo objeto matemático.

\begin{figure}[hbtp]
\centering
\includegraphics[scale=0.4]{img/img1.jpg}
\caption{\textcolor{red}{Tema 1, arriba a la izquierda} Conversión del registro natural al registro gráfico.}
\label{img1}
\end{figure}

\subsection{Funciones dadas por tablas de valores}


\begin{figure}[hbtp]
\centering
\includegraphics[scale=0.4]{img/img2.jpg}
\caption{Conversión del registro natural al registro gráfico.}
\label{img2}
\end{figure}

\subsection{Proporcionalidad}

\subsection{Pendiente}

\subection{Funciones lineales}

\subsection{Constantes}

}
\begin{table}[hbtp]
\begin{tabular}{|c||c|c|c|c|c|c|c|}
\hline 
			&\textsc{Icónico} &\textsc{Tabular} &\textsc{Gráfico} &\textsc{Aritmético} &\textsc{Natural}  &\textsc{Imagen}\\\hline
Concepto de función &no   & no & si & no & si  & \ref{img1}\\\hline
\specialcell[c]{Funciones dadas \\por tablas de valores} & no & no & si & si & si & \ref{img2}\\\hline
Proporcionalidad & si & si & si & si & si  & \ref{img3}\\\hline
Pendiente & no & no & si & si & si  & \ref{img4}\\\hline
Funciones lineales & si & si & si & si & si  & \ref{img5} \\\hline
Constantes & si & si & si & si & si  & \ref{img6}\\\hline
\end{tabular}
\caption{Cuadro resumen de los registros encontrados en la unidad didáctica.}
\label{Tablaresumen}
\end{table}


\section{Conclusiones}

\printindex

\appendix

\bibliographystyle{abbrv}
\bibliography{angelica5}  % angelica5.bib es el nombre del fichero que contiene


\end{document}