\documentclass[a4paper]{article}

% Las líneas que empiezan por % son comentarios

\usepackage[utf8x]{inputenc}  % Esto es para que LaTeX reconozca tildes
\usepackage[spanish]{babel} % Esto para que las cadenas estén en español (por ejemplo, la fecha)
\usepackage{amsmath} % Esto nos permite poner fórmulas matemáticas

\title{Hola mundo}
\author{Yo}
\date{\today}

\begin{document}
\maketitle

\section{Mi primera sección}

\subsection{Mi primera subsección}

\subsubsection{Mi primera subsubsección}

\paragraph{Mi primer párrafo} O, de cómo me di cuenta de que no hay más niveles de sección.

Este es mi segundo párrafo, así que voy a aprovechar para poner \textbf{una negrita}, una \textit{cursiva} y hasta un poco de texto en \texttt{monoespacio}\footnote{O en modo typewriter, y así aprovecho para poner una nota al pie automáticamente}.

\begin{itemize}
	\item También puedo poner listas
	\item Como esta
\end{itemize}

Y además, puedo escribir matemáticas con $1 + 1 = 2$ o, para que vayan en su propia línea, con \[ \int f(x) = 32 + \frac{3}{4} \cdot 1 + \dotsb + 3^2 + a_3 + b_{ij} \]

A modo de curiosidad, podemos ver que para \LaTeX\            muchos espacios            en blanco son lo mismo que uno sólo.
De hecho,
un salto de línea
sin una línea blanca en medio no genera un nuevo párrafo en el documento.

Ahora sí que tenemos un nuevo párrafo.
\end{document}
