\documentclass[palatino,nochap]{apuntesURJC}

\title{Fundamentos de la Investigación Educativa}
\author{}
\date{16/17 C1}


\def\citeapos#1{\citetitle{#1} (\citeauthor{#1}, \citeyear{#1}) \cite{#1}}

% Paquetes adicionales

% --------------------

\begin{document}
\pagestyle{plain}
\maketitle

\tableofcontents
\newpage
% Contenido.

\section{Entrega 1}

\paragraph{Tema de investigación. 
%
Pregunta de investigación que nos gustaría responder dentro de ese tema (mayor nivel de concreción).
%
Valoración de los mismos.}

\textbf{Tema} La influencia de los videojuegos en el desempeño académico.

\textbf{Preguntas de investigación: }

\begin{itemize}
\item ¿Existen videojuegos entre los más jugados entre los adolescentes que redunden en un mayor desempeño académico?
\item Atendiendo a alguna clasificación de videojuegos (cooperativos/individualistas, bélicos/deportes/plataformas/estrategia, móviles/consolas …) ¿Cuáles ayudan y cuáles no? ¿A qué ayuda cada uno? 
\item ¿Cuál es la curva de influencia de cada videojuego en el desempeño académico general o en alguna asignatura concreta respecto del tiempo jugado al día? ¿Es siempre decreciente (con lo que no ayuda), siempre creciente (cuanto más juego, mayor desempeño) o tiene un máximo? 
\end{itemize}



\section{Entrega 2}

\paragraph{Justificación de la investigación (la valoración de la Tarea 1 puede servir de base)\\}
\textbf{Documentación bibliográfica sobre el tema.
%
Tres artículos relacionados. 
%
Sólo los títulos de los textos, referenciados según normativa APA o ISO.}


Los videojuegos son una manera de empleo del ocio y tiempo libre por parte de muchos y cada vez más adolescentes y jóvenes. 
%
La industria del videojuego es cada vez más grande e incluso algunos videojuegos se están empezando a considerar deportes.
%
Es de vital importancia conocer en profundidad los efectos de emplear el tiempo libre con videojuegos supone en otros aspectos de nuestra vida, y en el caso de adolescentes y jóvenes en el desempeño académico. 
%
¿Fomentan los videojuegos el aprendizaje o lo dificultan?


Sobre este tema se ha trabajado bastante. 
%
Sobretodo en alumnos de primaria (y el impacto en las inteligencias múltiples \cite{del2015videogames} ) aunque también hay algunas investigaciones referentes a adolescentes y jóvenes.
%
En Brasil se ha llevado a cabo una investigación \cite{PimenPimen2014xw} en unos términos similares a los nuestros. 
%
¿Cuales son las relaciones entre el tiempo libre con juegos digitales y el aprendizaje? 
%
Además, proporcionan un buen estudio del arte sobre el que trabajar.

Dado que este es un tema algo controvertido, con sus férreos defensores y detractores, es interesante estudiar la perspectiva negativa que los videojuegos \cite{OnlineGamesYoung} (aunque trate un subconjunto de los videojuegos como son los juegos online.)





\section{Entrega 3}

\paragraph{Enunciar los objetivos del trabajo (aquellos que me llevan a poder responder a la pregunta de investigación).\\}
\textbf{ Determinar (justificadamente) si la metodología que emplearíamos sería cuantitativa, cualitativa, o mixta.}


\section{Entrega 4}

Definiciones de los términos y conceptos utilizados.
%
Cómo recoger datos.


%% Apendices (ejercicios, examenes)

\bibliographystyle{abbrv}
\bibliography{tex/Bibliografia}  % memoria.bib es el nombre del fichero que contiene

\printindex
\end{document}
