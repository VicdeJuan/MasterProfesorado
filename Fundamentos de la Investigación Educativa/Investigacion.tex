\documentclass[palatino,nochap]{apuntes}

\title{Fundamentos de la Investigación Educativa}
\author{}
\date{16/17 C1}

% Paquetes adicionales

% --------------------

\begin{document}
\pagestyle{plain}
\maketitle

\tableofcontents
\newpage
% Contenido.

Tema investigación: "El trato de las creencias personales de los estudiantes".

Plantear preguntas iniciales: La relación de la aceptación de las manifestaciones religiosas con la integración social de la minoría/integración.

No vale tener sesgo. Si yo pienso que $A$, tengo que evitar condicionar mi mirada a encontrar que $A$ es cierto. De hecho, puede ser buena idea buscar que $B$ es cierto y $A$ no.

\textit{La pregunta que me gustaría sería: ¿es mejor para el país un modelo laico (Francia) o aconfesional? Pero claro, mejor en qué sentido... }


\textit{Otra posibilidad: comparativa de rendimiento de colegios en base a la diversidad religiosa profesada. Rendimiento en cuanto a no violencia, conflictos, también puede ser interesante, no solo académicos.}


\textbf{Plan b:} Relación de tener un profesor particular de matemáticas con el desempeño académico (independientemente del profesor primeramente).




%% Apendices (ejercicios, examenes)
\appendix

\chapter{---}
\input{tex/Investigacion_Ejs.tex}

\printindex
\end{document}
