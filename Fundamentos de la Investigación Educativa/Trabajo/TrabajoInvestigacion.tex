\documentclass[palatino,miniheader]{apuntesURJC}

\title{Un proyecto para beneficiar, desde la Informática, al resto de disciplinas escolares y viceversa.}
\author{Víctor de Juan Sanz}
\date{16/17 C1}

\usepackage[acronym]{glossaries} 		% Glosario/Acrónimos

\makeglossaries
\newacronym{CPSES}{CPSES}{Computer Programming Self-Efficacy Scale}
\newcommand{\CPSES}{\textit{\gls{CPSES}} \cite{CPSES}}


\def\citeapos#1{\citetitle{#1} (\citeauthor{#1}, \citeyear{#1}) \cite{#1}}
%\newcommand{\citeapos}[1]{\citetitle{#1} (\citeauthor{#1}, \citeyear{#1}) \cite{#1}}

% Paquetes adicionales

% --------------------

\begin{document}
\pagestyle{plain}
\maketitle

\tableofcontents
\newpage
% Contenido.


%\chapter*{Abstract}

%%%%%%%%%%%%%%%%%%%%%%%%%%%%%%%%%%%%%%%%%%%%%%%%%%%%%%%%%%%%%%%%%%%%%%%%%%%%%%%%%%%%%%%%%%%%%%%%%%%%%%%%%%%%%%%%%%%%%%%%%%%%%%%%%%%%%%%%%%%%%%%%%%%%%%%%%%%%%%%%%%%%%%%%%%%%%%%%%%%%%%%%%%%%%%%%%%%%%%%%%%%%%%%%%%%%%%%%%%%%%%%%%%%%%%%%%%%%%%%%%%%%%%%%%%%%%%%%%%%%%%%%%%%%%%%%%%%%%%%%%%%%%%%%%%%%%%%%%%%%%%%%%%%%%%%%%%%%%%%%%%%%%%%%%%%%%%%%%%%%%%%%%%%%%%%%%%%%%%%%%%%%%%%%%%%%%%%%%%%%%%%%%%%%%%%%%%%%%%%%%%%%%%%%%%%%%%%%%%%%%%%%%%%%%%%%%%%%%%%%%%%%%%%%%%%%%%%%%%%%%%%%%%%%%%%%%%%%%%%%%%%%%%%%%%%%%%%%%%%%%%%%%%%%%%%%%%%%%%%%%%%%%%%%%%%%%%%%%%%%%%%%%%%%%%%%%%%%%%%%%%%%%%%%%%%%%%%%%%%%%%%%%%%%%%%%%%%%%%%%%%%%%%%%%%%%%%%%%%%%%%%%%%%%%%%%%%%%%%%%%%%%%%%%%%%%%%%%%%%%%%%%%%%%%%%%%%%%%%%%%%%%%%%%%%%%%%%%%%%%%%%%%%%%%%%%%%%%%%%%%%%%%%%%%%%%%%%%%%%%%%%%%%%%%%%%%%%%%%%%%%%%%%%%%%%%%%%%%%%%%%%%%%%%%%%%%%%%%%%%%%%%%%%%%%%%%%%%%%%%%%%%%%%%%%%%%%%%%%%%%%%%%%%%%%%%%%%%%%%%%%%%%%%%%%%%%%%%%%%%%%%%%%%%%%%%%%%%%%%%%%%%%%%%%%%%%%%%%%%%%%%%%%%%%%%%%%%%%%%%%%%%%%%%%%%%%%%%%%%%%%%%%%%%%%%%%%%%%%%%%%%%%%%%%%%%%%%%%%%%%%%%%%%%%%%%%%%%%%%%%%%%%%%%%%%%%%%%%%%%%%%%%%%%%%%%%%%%%%%%%%%%%%%%%%%%%%%%%%%%%%%%%%%%%%%%%%%%%%%%%%%%%%%%%%%%%%%%%%%%%%%%%%%%%%%%%%%%%%%%%%%%%%%%%%%%%%%%%%%%%%%%%%%%%%%%%%%%%%%%%%%%%%%%%%%%%%%%%%%%


\chapter{Introducción y Justificación}
\section{Selección del tema}

Con el fin de conseguir una mayor interacción entre las asignaturas de la educación secundaria y que esa interacción sea fructífera, es necesario innovar mediante la experimentación de diversas metodologías y el estudio de su efecto.
%
Intuitivamente uno podría pensar que, cuanta más interacción haya entre las asignaturas, cuanta más utilidad tengan unas para otras, mayor será el grado de adquisición de competencias por el estudiante.
%
Dado que la intuición no es un método fiable para la toma de decisiones, proponemos un acercamiento científico al tema. 
%
Técnicamente, un acercamiento científico a un subtema: el beneficio que ofrece incluir contenido y enfoques interdisciplinarios en el estudio de la Informática (más concretamente desde la programación).
%
Suponemos también que esta interdisciplinaridad tendrá un efecto \textit{doble} provocando un aumento de las destrezas adquiridas por los estudiantes en la programación y un aumento en los conocimientos relativos a la otra disciplina.

\section{Identificación, delimitación, valoración y formulación del problema. Justificación}

La elección de fomentar la interdisciplina desde la Informática se debe a que estos últimos años se caracterizan por la creciente demanda de puestos de trabajo relacionados con la Ingeniería Informática.
%
En un periodo de crisis, la Ingeniería Informática se ha convertido en una de las profesiones con menor tasa de paro. 
%
Incluso, muchas personas que no han estudiado Ingeniería Informática acaban desarrollándose laboralmente en el ámbito informático, como puede ser el caso de  matemáticos, físicos e ingenieros de otras ramas (como Industriales y de Telecomunicaciones - aunque estos últimos están algo más relacionados con el tema).

Estas afirmaciones son compartidas por investigaciones científicas (\cite{CSIsImportant},\cite{CSArguing}). 
%
De hecho, \cite{CSArguing}, además de constatar la importancia de la Ingeniería Informática, argumenta que debería incrementarse la importancia de la Informática en la educación básica y secundaria.

Por otro lado, no sólo parece positivo incluir en la educación competencias relativas a la Ingeniería Informática (más allá de la Ofimática), sino que también es importante hacerlo de la mejor manera posible.
%
Algunos sistemas educativos, debido a las metodologías aplicadas, no consiguen una satisfactoria adquisición de destrezas y competencias por parte de los estudiantes, provocando un rechazo hacia la programación porque no han llegado a superar las primeras dificultades \cite{CSArguing}. 
%
Es por ello que consideramos necesario innovar en la docencia informática, entendiendo innovar como "cualquier cambio sustancial novedoso que da lugar a mejores resultados" (Raquel Garrido).

En esta investigación tratamos de aportar una metodología (y la evidencia sobre su funcionamiento) con la que los alumnos puedan desarrollar mejores competencias Informáticas. 
%
Debido al enfoque multidisciplinar de la metodología, estudiaremos también los beneficios producidos en las otras disciplinas.
%
Desarrollaremos una propuesta concreta de cómo ejercer la docencia de Informática en los últimos cursos de la educación secundaria (en lo relativo a la programación) que, previsiblemente, tendrá como resultado un mejor rendimiento en otras asignaturas y una mayor adquisición de la destreza necesaria para programar.
%
Ambas consecuencias podrían ser debidas al enfoque interdisciplinar.

El estudio \textit{The Impact of an Interdisciplinary Space Program on Computer Science Student Learning} \cite{Interdiscipline} realiza una experiencia parecida en E.E.U.U. Sin embargo, pretendemos completar esa investigación en dos sentidos: 
\begin{itemize}
	\item Realizar un estudio similar en un sistema educativo muy diferente, como es el Sistema Educativo Español. 
	%
	Estas diferencias del sistema tienen 2 vertientes, tanto las relativas al propio sistema (en España los estudiantes no pueden elegir los \textit{major} que tienen en E.E.U.U.) y las relativas al currículo de las asignaturas referidas a la Ingeniería Informática.
	\item Estudiar el doble efecto de la interdisciplinaridad: el efecto desde la Informática sobre las otras disciplinas y viceversa.
\end{itemize}


%%%%%%%%%%%%%%%%%%%%%%%%%%%%%%%%%%%%%%%%%%%%%%%%%%%%%%%%%%%%%%%%%%%%%%%%%%%%%%%%%%%%%%%%%%%%%%%%%%%%%%%%%%%%%%%%%%%%%%%%%%%%%%%%%%%%%%%%%%%%%%%%%%%%%%%%%%%%%%%%%%%%%%%%%%%%%%%%%%%%%%%%%%%%%%%%%%%%%%%%%%%%%%%%%%%%%%%%%%%%%%%%%%%%%%%%%%%%%%%%%%%%%%%%%%%%%%%%%%%%%%%%%%%%%%%%%%%%%%%%%%%%%%%%%%%%%%%%%%%%%%%%%%%%%%%%%%%%%%%%%%%%%%%%%%%%%%%%%%%%%%%%%%%%%%%%%%%%%%%%%%%%%%%%%%%%%%%%%%%%%%%%%%%%%%%%%%%%%%%%%%%%%%%%%%%%%%%%%%%%%%%%%%%%%%%%%%%%%%%%%%%%%%%%%%%%%%%%%%%%%%%%%%%%%%%%%%%%%%%%%%%%%%%%%%%%%%%%%%%%%%%%%%%%%%%%%%%%%%%%%%%%%%%%%%%%%%%%%%%%%%%%%%%%%%%%%%%%%%%%%%%%%%%%%%%%%%%%%%%%%%%%%%%%%%%%%%%%%%%%%%%%%%%%%%%%%%%%%%%%%%%%%%%%%%%%%%%%%%%%%%%%%%%%%%%%%%%%%%%%%%%%%%%%%%%%%%%%%%%%%%%%%%%%%%%%%%%%%%%%%%%%%%%%%%%%%%%%%%%%%%%%%%%%%%%%%%%%%%%%%%%%%%%%%%%%%%%%%%%%%%%%%%%%%%%%%%%%%%%%%%%%%%%%%%%%%%%%%%%%%%%%%%%%%%%%%%%%%%%%%%%%%%%%%%%%%%%%%%%%%%%%%%%%%%%%%%%%%%%%%%%%%%%%%%%%%%%%%%%%%%%%%%%%%%%%%%%%%%%%%%%%%%%%%%%%%%%%%%%%%%%%%%%%%%%%%%%%%%%%%%%%%%%%%%%%%%%%%%%%%%%%%%%%%%%%%%%%%%%%%%%%%%%%%%%%%%%%%%%%%%%%%%%%%%%%%%%%%%%%%%%%%%%%%%%%%%%%%%%%%%%%%%%%%%%%%%%%%%%%%%%%%%%%%%%%%%%%%%%%%%%%%%%%%%%%%%%%%%%%%%%%%%%%%%%%%%%%%%%%%%%%%%%%%%%%%%%%%%%%%%%%%%%%%%%%%%%%%%%%%%%%%%%%%%%%%%%%%%%%%%%%%%%%%%%%%%%%%%%%


\chapter{Objetivos}

Los objetivos de esta investigación son:
\begin{itemize}
	\item Cómo mejora la autoeficacia\footnote{la definición se encuentra en \ref{defn::autoeficacia}} en la programación (utilizando \CPSES) gracias a haber estado desarrollando un programa con una utilidad directa.
	
	\begin{defn}[Autoeficacia]
	\label{defn::autoeficacia}

	Utilizamos la definición dada por (J.E. Ormrod, 2008) \cite{autoef}

	La autoeficacia, también llamada eficacia personal, es el grado o fuerza de la creencia de un individuo en su propia habilidad para completar tareas y alcanzar metas.

	\end{defn}

	%%%
	\item Seleccionar un grupo de estudiantes de 1 de Bachillerato en un instituto donde se imparta Informática.  
	%
	El grupo será del bachillerato de ciencias, dado que el currículo de Informática de 1 de Bachillerato Tecnológico incluye la programación.
	%%%
	\item Dilucidar si el desarrollo por parte de los alumnos de un programa de preguntas y respuestas sobre el temario de otras asignaturas (como fechas de Historia, vocabulario de Inglés, formulación Química [tanto Orgánica como Inorgánica]) supone una mejora en el rendimiento en la asignatura sobre la que verse el programa desarrollado.
\end{itemize}

%%%%%%%%%%%%%%%%%%%%%%%%%%%%%%%%%%%%%%%%%%%%%%%%%%%%%%%%%%%%%%%%%%%%%%%%%%%%%%%%%%%%%%%%%%%%%%%%%%%%%%%%%%%%%%%%%%%%%%%%%%%%%%%%%%%%%%%%%%%%%%%%%%%%%%%%%%%%%%%%%%%%%%%%%%%%%%%%%%%%%%%%%%%%%%%%%%%%%%%%%%%%%%%%%%%%%%%%%%%%%%%%%%%%%%%%%%%%%%%%%%%%%%%%%%%%%%%%%%%%%%%%%%%%%%%%%%%%%%%%%%%%%%%%%%%%%%%%%%%%%%%%%%%%%%%%%%%%%%%%%%%%%%%%%%%%%%%%%%%%%%%%%%%%%%%%%%%%%%%%%%%%%%%%%%%%%%%%%%%%%%%%%%%%%%%%%%%%%%%%%%%%%%%%%%%%%%%%%%%%%%%%%%%%%%%%%%%%%%%%%%%%%%%%%%%%%%%%%%%%%%%%%%%%%%%%%%%%%%%%%%%%%%%%%%%%%%%%%%%%%%%%%%%%%%%%%%%%%%%%%%%%%%%%%%%%%%%%%%%%%%%%%%%%%%%%%%%%%%%%%%%%%%%%%%%%%%%%%%%%%%%%%%%%%%%%%%%%%%%%%%%%%%%%%%%%%%%%%%%%%%%%%%%%%%%%%%%%%%%%%%%%%%%%%%%%%%%%%%%%%%%%%%%%%%%%%%%%%%%%%%%%%%%%%%%%%%%%%%%%%%%%%%%%%%%%%%%%%%%%%%%%%%%%%%%%%%%%%%%%%%%%%%%%%%%%%%%%%%%%%%%%%%%%%%%%%%%%%%%%%%%%%%%%%%%%%%%%%%%%%%%%%%%%%%%%%%%%%%%%%%%%%%%%%%%%%%%%%%%%%%%%%%%%%%%%%%%%%%%%%%%%%%%%%%%%%%%%%%%%%%%%%%%%%%%%%%%%%%%%%%%%%%%%%%%%%%%%%%%%%%%%%%%%%%%%%%%%%%%%%%%%%%%%%%%%%%%%%%%%%%%%%%%%%%%%%%%%%%%%%%%%%%%%%%%%%%%%%%%%%%%%%%%%%%%%%%%%%%%%%%%%%%%%%%%%%%%%%%%%%%%%%%%%%%%%%%%%%%%%%%%%%%%%%%%%%%%%%%%%%%%%%%%%%%%%%%%%%%%%%%%%%%%%%%%%%%%%%%%%%%%%%%%%%%%%%%%%%%%%%%%%%%%%%%%%%%%%%%%%%%%%%%%%%%%%%%%%%%%%%%%%%%%%%%%%%%%%%%%%


\chapter{Marco teórico}

Numerosas investigaciones (\cite{CSIsImportant},\cite{CSArguing} entre otras) evidencian la necesidad de la adquisición de competencias informáticas en la educación secundaria para un satisfactorio desempeño en la educación posterior y, a la larga, en la vida profesional.
%
A la hora de que los alumnos adquieran esas destrezas es imprescindible analizar cuál es la mejor metodología para que los estudiantes las adquieran.
%
En este sentido, \cite{StudentCenter} demuestra que la autoeficacia\footnote{Definida en \ref{defn::autoeficacia}} (utilizando la escala \CPSES) en \textit{Computer Science} aumenta significativamente cuando la metodología empleada es a base de problemas y tareas, siendo el docente una mera referencia a la que consultar las dudas que puedan surgir.
%
\label{studentbased}
%
Otro estudio llega una conclusión parecida \cite{StudentCenterVSLectures}:
%
la docencia basada en explicaciones del docente (\textit{Teacher based}) da lugar a una mejora en la autoeficacia peor que la mejora obtenida con una metodología centrada en el estudiante.
%
\label{groupsbased}
%
Otra investigación \cite{ABPCS} constata que en el ámbito de la programación también se aumentan las destrezas adquiridas trabajando en grupo sobre problemas y proyectos que los estudiantes encuentren interesantes y entretenidos. 


Por otro lado, Mitch Resnick, el investigador que desarrolló \concept{Scratch} \cite{scratch}, describe la programación como un medio por el que aprender y comprender.
%
Scratch se desarrolló para que los alumnos de primeros cursos de primaria pudieran acercarse a la programación y esto les ayudara con el \textit{enfoque procedimental} necesario en la programación.
%
Podemos definir el \concept{pensamiento procedimental} como la consecución del modelo de solución de problemas de Polya \cite{Polya} basado en 4 pasos:
1) Comprender el problema. 2) Idear un plan (formular una estrategia general). 3) Ejecutar ese plan (formular una prueba detallada). 4) Verificar los resultados. 
%
Estos son los 4 pasos que se aplican continuamente en la resolución de problemas de la Ingeniería Informática y que, según Resnick, pueden resultar de ayuda para el resto del conocimiento académico.

El objeto de esta investigación es llevar una idea basada en la de Resnick a la educación secundaria, con adolescentes que ya son capaces de un pensamiento más abstracto.
%
¿De qué manera la resolución de problemas informáticos (de programación) puede ayudar a mejorar el rendimiento en otras asignaturas que, a priori, no tienen relación como pueda ser la Historia o el Inglés?
%
Aprovechando las evidencias científicas propondremos ejercer la docencia \textit{centrada en el estudiante}.
%
Nos preguntamos también:
%
¿de qué manera, la modificación de la metodología de enseñanza de la programación en secundaria para enfocarla a mejorar el rendimiento en otras asignaturas redunda en una mejora significativa del aprendizaje de la programación?


%%%%%%%%%%%%%%%%%%%%%%%%%%%%%%%%%%%%%%%%%%%%%%%%%%%%%%%%%%%%%%%%%%%%%%%%%%%%%%%%%%%%%%%%%%%%%%%%%%%%%%%%%%%%%%%%%%%%%%%%%%%%%%%%%%%%%%%%%%%%%%%%%%%%%%%%%%%%%%%%%%%%%%%%%%%%%%%%%%%%%%%%%%%%%%%%%%%%%%%%%%%%%%%%%%%%%%%%%%%%%%%%%%%%%%%%%%%%%%%%%%%%%%%%%%%%%%%%%%%%%%%%%%%%%%%%%%%%%%%%%%%%%%%%%%%%%%%%%%%%%%%%%%%%%%%%%%%%%%%%%%%%%%%%%%%%%%%%%%%%%%%%%%%%%%%%%%%%%%%%%%%%%%%%%%%%%%%%%%%%%%%%%%%%%%%%%%%%%%%%%%%%%%%%%%%%%%%%%%%%%%%%%%%%%%%%%%%%%%%%%%%%%%%%%%%%%%%%%%%%%%%%%%%%%%%%%%%%%%%%%%%%%%%%%%%%%%%%%%%%%%%%%%%%%%%%%%%%%%%%%%%%%%%%%%%%%%%%%%%%%%%%%%%%%%%%%%%%%%%%%%%%%%%%%%%%%%%%%%%%%%%%%%%%%%%%%%%%%%%%%%%%%%%%%%%%%%%%%%%%%%%%%%%%%%%%%%%%%%%%%%%%%%%%%%%%%%%%%%%%%%%%%%%%%%%%%%%%%%%%%%%%%%%%%%%%%%%%%%%%%%%%%%%%%%%%%%%%%%%%%%%%%%%%%%%%%%%%%%%%%%%%%%%%%%%%%%%%%%%%%%%%%%%%%%%%%%%%%%%%%%%%%%%%%%%%%%%%%%%%%%%%%%%%%%%%%%%%%%%%%%%%%%%%%%%%%%%%%%%%%%%%%%%%%%%%%%%%%%%%%%%%%%%%%%%%%%%%%%%%%%%%%%%%%%%%%%%%%%%%%%%%%%%%%%%%%%%%%%%%%%%%%%%%%%%%%%%%%%%%%%%%%%%%%%%%%%%%%%%%%%%%%%%%%%%%%%%%%%%%%%%%%%%%%%%%%%%%%%%%%%%%%%%%%%%%%%%%%%%%%%%%%%%%%%%%%%%%%%%%%%%%%%%%%%%%%%%%%%%%%%%%%%%%%%%%%%%%%%%%%%%%%%%%%%%%%%%%%%%%%%%%%%%%%%%%%%%%%%%%%%%%%%%%%%%%%%%%%%%%%%%%%%%%%%%%%%%%%%%%%%%%%%%%%%%%%%%%%%%%%%%%%%%%%%%%%%%%%%%%


\chapter{Metodología de la Investigación}

\section{Descripción}
Definiremos 
\begin{itemize}
	\item \textbf{Variable independiente:} Metodología impartida en la clase de Informática.
	\item \textbf{Variables dependientes:} 
	\subitem \textbf{1.- Rendimiento en otra disciplina} que llamaremos \concept{asignatura complementaria}, como la Química Orgánica, Química Inorgánica, Historia o Inglés.
	\subitem Técnicamente se podrían realizar \textit{n} experimentos con las \textit{n} asignaturas definidas. 
	%
	Para esta investigación sólo vamos a tomar 1 asignatura, a elegir por los sujetos del experimento.
	\subitem \textbf{2.- Autoeficacia en Programación.}
	\item \textbf{Grupos:}
	\subitem Dividiremos el grupo de Informática en 2.

	A los que aplicaremos la  metodología propuesta (grupo experimental) y a los que no (grupo control).
	%
	Los alumnos del grupo experimental escogerán como asignatura complementaria una de las siguientes materias: Química Orgánica, Química Inorgánica, Historia o Inglés. 
	


	\subitem La división de los alumnos de la asignatura de Informática en los grupos \textit{control} y \textit{experimental} se hará equilibradamente (utilizando la media académica como heurístico del rendimiento escolar).
	\subitem Además, para poder realizar un análisis estadístico más profundo, tomaremos como referencia los alumnos del Bachillerato Tecnológico que no cursen Informática.
	%
	Así podremos contrarrestar el efecto de variables extrañas que afecten a todos por igual, como puede ser un cambio de profesor, avanzar el curso y en el temario de la asignatura complementaria, por lo que los alumnos cada vez habrán adquirido más destrezas...
\end{itemize}


La primera recogida de datos serán 2 cuestionarios a los 3 grupos definidos.
%
\label{descDatos}
%
El primer cuestionario será uno sobre conocimientos relativos la materia complementaria.
%
El segundo cuestionario será para evaluar la autoeficacia en Informática (según \CPSES).

Después, a los alumnos del grupo experimental se les propondrá que hagan, por grupos (como ya hemos sugerido anteriormente \ref{groupsbased}), un programa para evaluar sus conocimientos sobre la materia complementaria.
%
El funcionamiento de la clase será \textit{basado en el estudiante} ya que, como hemos mencionado anteriormente (\ref{studentbased}), es una metodología más efectiva que la \textit{basada en el docente}.
%
Mientras el grupo experimental ha recibido esta propuesta, el grupo control recibirá las instrucciones que el docente considere oportunas para desarrollar un software similar (como podría ser el de un cajero automático).

Una vez finalizado el tiempo (especificado en el siguiente punto \ref{tiempos}), se volverán a pasar los 2 cuestionarios a los 3 grupos.

\section{Muestra escogida}

La muestra escogida es un único grupo de 1 de Bachillerato Tecnológico de un centro concreto, debido a que es la única muestra a la que podemos acceder.

La composición de los alumnos podría ser la siguiente:

\begin{table}[hbtp]
\centering
\begin{tabular}{|l|c|}
\hline
Alumnos \textbf{totales} & 48\\\hline
Cursando Informática & 30 \\\hline\hline
Grupo \textbf{control} & 15 \\
Grupo \textbf{experimental} & 15 \\
Grupo \textbf{contrarresto} V.E. & 18 \\\hline
\end{tabular}
\caption{Distribución del alumnado de la muestra}
\end{table}

El experimento podría repetirse con otras muestras procedentes de otros centros educativos con las características definidas.


\section{Planificación}
\label{tiempos}
Un programa muy sencillo de preguntas y respuestas ha sido desarrollado por un programador amateur para esta investigación a modo de ejemplo en, aproximadamente, treinta minutos.
%
Esperamos que los alumnos sean capaces de aprender los rudimentos de la programación como para hacer un software con las características deseadas en tres semanas.
%%%%%%%%%%%%%%%%%%%%%%%%%%%%%%%%%%%%%%%%%%%%%%%%%%%%%%%%%%%%%%%%%%%%%%%%%%%%%%%%%%%%%%%%%%%%%%%%%%%%%%%%%%%%%%%%%%%%%%%%%%%%%%%%%%%%%%%%%%%%%%%%%%%%%%%%%%%%%%%%%%%%%%%%%%%%%%%%%%%%%%%%%%%%%%%%%%%%%%%%%%%%%%%%%%%%%%%%%%%%%%%%%%%%%%%%%%%%%%%%%%%%%%%%%%%%%%%%%%%%%%%%%%%%%%%%%%%%%%%%%%%%%%%%%%%%%%%%%%%%%%%%%%%%%%%%%%%%%%%%%%%%%%%%%%%%%%%%%%%%%%%%%%%%%%%%%%%%%%%%%%%%%%%%%%%%%%%%%%%%%%%%%%%%%%%%%%%%%%%%%%%%%%%%%%%%%%%%%%%%%%%%%%%%%%%%%%%%%%%%%%%%%%%%%%%%%%%%%%%%%%%%%%%%%%%%%%%%%%%%%%%%%%%%%%%%%%%%%%%%%%%%%%%%%%%%%%%%%%%%%%%%%%%%%%%%%%%%%%%%%%%%%%%%%%%%%%%%%%%%%%%%%%%%%%%%%%%%%%%%%%%%%%%%%%%%%%%%%%%%%%%%%%%%%%%%%%%%%%%%%%%%%%%%%%%%%%%%%%%%%%%%%%%%%%%%%%%%%%%%%%%%%%%%%%%%%%%%%%%%%%%%%%%%%%%%%%%%%%%%%%%%%%%%%%%%%%%%%%%%%%%%%%%%%%%%%%%%%%%%%%%%%%%%%%%%%%%%%%%%%%%%%%%%%%%%%%%%%%%%%%%%%%%%%%%%%%%%%%%%%%%%%%%%%%%%%%%%%%%%%%%%%%%%%%%%%%%%%%%%%%%%%%%%%%%%%%%%%%%%%%%%%%%%%%%%%%%%%%%%%%%%%%%%%%%%%%%%%%%%%%%%%%%%%%%%%%%%%%%%%%%%%%%%%%%%%%%%%%%%%%%%%%%%%%%%%%%%%%%%%%%%%%%%%%%%%%%%%%%%%%%%%%%%%%%%%%%%%%%%%%%%%%%%%%%%%%%%%%%%%%%%%%%%%%%%%%%%%%%%%%%%%%%%%%%%%%%%%%%%%%%%%%%%%%%%%%%%%%%%%%%%%%%%%%%%%%%%%%%%%%%%%%%%%%%%%%%%%%%%%%%%%%%%%%%%%%%%%%%%%%%%%%%%%%%%%%%%%%%%%%%%%%%%%%%%%%%%%%%%%%%%%%%%%%%%%%%%%%%%


\chapter{Para completar}
\section{Recolección de datos}

Procederíamos a la recolección de datos como hemos descrito en \ref{descDatos}.

\section{Análisis de los datos}

\subsection{Descripción de los datos}

Una vez obtenidos los datos, presentaríamos en forma de tablas y gráficas los datos obtenidos.

\subsection{Análisis estadístico}

Utilizando el grupo de \textit{contrarresto}, procederíamos a realizar diversos contrastes estadísticos.


\subsubsection{Mejora en Informática}

Sea $µ_B$ la media del grupo \textit{contrarresto} (base), $µ_C$ la media del grupo control y $µ_E$ la media del grupo experimental.
%
Consideramos las 3 medias en el cuestionario \CPSES.

Ahora, procederíamos a contrastar la hipótesis nula $H_0: (µ_C - µ_B) < (µ_E - µ_B)$.

Este mismo contraste podría realizarse con los cuartiles, para dilucidar si la mejora es mayor o menor en alumnos de mejor rendimiento, de peor, o en los intermedios.

\subsubsection{Mejora en la disciplina complementaria}

Sea $µ_B$ la media del grupo \textit{contrarresto} (base), $µ_C$ la media del grupo control y $µ_E$ la media del grupo experimental.
%
Consideramos las 3 medias en el cuestionario relativo al conocimiento sobre la otra asignatura.

Ahora, procederíamos a contrastar la hipótesis nula $H_1: (µ_C - µ_B) < (µ_E - µ_B)$.

Este mismo contraste podría realizarse con los cuartiles, para dilucidar si la mejora es mayor o menor en alumnos de mejor rendimiento, de peor, o en los intermedios.



%%%%%%%%%%%%%%%%%%%%%%%%%%%%%%%%%%%%%%%%%%%%%%%%%%%%%%%%%%%%%%%%%%%%%%%%%%%%%%%%%%%%%%%%%%%%%%%%%%%%%%%%%%%%%%%%%%%%%%%%%%%%%%%%%%%%%%%%%%%%%%%%%%%%%%%%%%%%%%%%%%%%%%%%%%%%%%%%%%%%%%%%%%%%%%%%%%%%%%%%%%%%%%%%%%%%%%%%%%%%%%%%%%%%%%%%%%%%%%%%%%%%%%%%%%%%%%%%%%%%%%%%%%%%%%%%%%%%%%%%%%%%%%%%%%%%%%%%%%%%%%%%%%%%%%%%%%%%%%%%%%%%%%%%%%%%%%%%%%%%%%%%%%%%%%%%%%%%%%%%%%%%%%%%%%%%%%%%%%%%%%%%%%%%%%%%%%%%%%%%%%%%%%%%%%%%%%%%%%%%%%%%%%%%%%%%%%%%%%%%%%%%%%%%%%%%%%%%%%%%%%%%%%%%%%%%%%%%%%%%%%%%%%%%%%%%%%%%%%%%%%%%%%%%%%%%%%%%%%%%%%%%%%%%%%%%%%%%%%%%%%%%%%%%%%%%%%%%%%%%%%%%%%%%%%%%%%%%%%%%%%%%%%%%%%%%%%%%%%%%%%%%%%%%%%%%%%%%%%%%%%%%%%%%%%%%%%%%%%%%%%%%%%%%%%%%%%%%%%%%%%%%%%%%%%%%%%%%%%%%%%%%%%%%%%%%%%%%%%%%%%%%%%%%%%%%%%%%%%%%%%%%%%%%%%%%%%%%%%%%%%%%%%%%%%%%%%%%%%%%%%%%%%%%%%%%%%%%%%%%%%%%%%%%%%%%%%%%%%%%%%%%%%%%%%%%%%%%%%%%%%%%%%%%%%%%%%%%%%%%%%%%%%%%%%%%%%%%%%%%%%%%%%%%%%%%%%%%%%%%%%%%%%%%%%%%%%%%%%%%%%%%%%%%%%%%%%%%%%%%%%%%%%%%%%%%%%%%%%%%%%%%%%%%%%%%%%%%%%%%%%%%%%%%%%%%%%%%%%%%%%%%%%%%%%%%%%%%%%%%%%%%%%%%%%%%%%%%%%%%%%%%%%%%%%%%%%%%%%%%%%%%%%%%%%%%%%%%%%%%%%%%%%%%%%%%%%%%%%%%%%%%%%%%%%%%%%%%%%%%%%%%%%%%%%%%%%%%%%%%%%%%%%%%%%%%%%%%%%%%%%%%%%%%%%%%%%%%%%%%%%%%%%%%%%%%%%%%%%%%%%%%%%%%%%%%%%%%%%%%


\chapter{Resultados esperados}

Es de esperar que ambas hipótesis estadísticas sean aceptables con niveles de confianza del 95\% o mayores.
%

\section{Líneas de futuro abiertas}

En este estudio se ha utilizado sólo una de las otras disciplinas sugeridas (Historia, Inglés, Química Orgánica, Química Inorgánica). 
%
Pueden plantearse futuras investigaciones sobre cuál es el efecto en las otras disciplinas, incluso plantear nuevas disciplinas con las que rehacer el estudio, como las Matemáticas o la Geografía.

Otra característica del estudio es que los alumnos no saben programar al llegar a este curso y se les enseña, desde cero, en 1 de Bachillerato.
%
Esta premisa puede ser falsa en algunos centros. 
%
Al menos en la Comunidad de Madrid, en 2 y 3 de ESO existe una optativa que el centro puede elegir dar que se llama \textit{Tecnología, Programación y Robótica}. 
%
Esta optativa abre dos posibilidades nuevas para esta investigación:
\begin{itemize}
 	\item Realizar la investigación en los cursos de 2 y 3 de ESO, en esa asignatura en la que aprenden a programar.
 	\item Realizar la investigación en el mismo curso de Bachillerato, pero enfocando la docencia de la Informática al ejercicio y mejora de las destrezas de programación que los alumnos hayan adquirido en los cursos anteriores, en vez de enfocarla al aprendizaje.
 	%
 	De esta manera, se podría proponer una metodología con la que mejorar más eficazmente las habilidades de programación de los alumnos.
 \end{itemize} 

\section{Conclusiones}

Los alumnos que hayan aprendido a programar desarrollando un software que ellos mismos pueden utilizar para estudiar otra asignatura habrán adquirido una mayor destreza en la programación que los alumnos que hayan aprendido a programar siguiendo la metodología habitual.

Además, los alumnos que han programado su propio software para evaluar sus conocimientos sobre otra asignatura habrán desempeñado mejor en esa asignatura que los demás alumnos.

De estos 2 "hechos" concluiríamos que parece mejor enseñar a programar planteando el proyecto de desarrollar un software de preguntas y respuestas para evaluar los propios conocimientos sobre otra asignatura.


%% Apendices (ejercicios, examenes)

\bibliographystyle{abbrv}
\label{bibliografia}
\bibliography{tex/Bibliografia}  % memoria.bib es el nombre del fichero que contiene

%\printindex
\end{document}
