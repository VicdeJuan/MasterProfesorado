\section{Tema 1}

\paragraph{Círculo virtuoso de reflexión en la acción}
La educación es un conocimiento práctico; es un \textit{saber} que consiste en \textit{saber hacer} y que se aprende en el propio \textit{hacer}.
%
Esto suscita una interesante reflexión.
%
Si se aprende haciendo, tras año de experiencia en la profesión, se debería haber acumulado un gran saber.
%
Es bien sabido que esto no ocurre así.
%
Los profesores más veteranos tienen más experiencia pero no necesariamente tienen más dominio del arte de enseñar que los profesores jóvenes. 

Podría darse el caso de que haya profesores que con los años de experiencia hayan ido perdiendo dominio del arte. 
%
Es verdad que hay otras variables tremendamente influyentes en el proceso de mejora y aprendizaje de los profesores, más allá de los años de experiencia.
%
Chojin, en su canción "\textit{Soy y no soy}" hace una afirmación muy acertada: \textit{Cuando aprendo no es por la experiencia en sí sino por el momento que me pille dispuesto y abierto}.
%
Se puede enseñar, pero depende del educando aprender.
%
El aprendizaje y la mejora del educador como docente no depende de sus años de práctica, sino de la actitud dispuesta y abierta a aprender y agrandar el saber hacer.
%
Si el educador no realiza un trabajo de reflexión y evaluación sobre su propia acción educativa, difícilmente va a poder seguir aprendiendo la labor docente.

\begin{figure}[hbt]
\centering
\includegraphics[scale=0.5]{img/Circulo.png}
\caption{Círculo de mejora de la acción educativa.}
\small{Fuente: Material para las clases de \profe.}
\label{Circulo}
\end{figure}

\paragraph{Educar para ayudar a crecer} A veces confundimos educación con formación. 
%
Es habitual pensar en la educación como el medio para introducirse en un mercado laboral y para ello es necesaria una formación en conocimientos bastante amplia.
%
Poco a poco se va transformando la sociedad en su conjunto hacia un concepto de educación más amplio, más integrador de las diferentes facetas humanas.

"Education is not the learning of facts, but the training of the mind to think" \footnote{En español: La educación no es el aprendizaje de hechos, sino el entrenamiento de la mente para pensar} es una frase de Einstein.
%
Einstein no hablaba de crecer sino de aprender a pensar. 
%
Una interpretación de esta frase se basa en entender que \textit{pensar} es una acción muy amplia. 
%
De hecho, el crecimiento que se busca en la educación es un crecimiento de las potencias propias (Inteligencia y Voluntad), por lo que, tal vez, Einstein no andaba desencaminado.

Es reconfortante y alentador descubrir que la propuesta educativa dada a los futuros educadores es la que uno traía ya pensada de casa.
%
Es confirmador de la vocación ver que la idea y el sueño de uno para su acción profesional coincida con la de los expertos en la materia.


\subparagraph{Armonía de todas las dimensiones} 
%
Es tremendamente aclaradora la metáfora de la poda.
%
Para forjar el carácter de los educandos es necesario cultivar y dejar crecer, pero puede haber momentos de poda.
%
La poda de una viña, por ejemplo, no quita solamente los sarmientos muertos o los que no dan fruto, sino que también puede interesar cortar sarmientos que dan poco fruto para que vuelvan a crecer más fuertes y fructíferos.

Esta metáfora de la poda resulta interesante. 
%
¿Merecerá la pena un castigo\footnote{Entendiendo castigo desde la definición Psicológica asociada a los refuerzos} en una situación ambigüa o mediocre para que el educando pueda crecer?



\paragraph{Manipulación} 
%
Este es un tema que se discute poco desde un punto de vista ético y moral.
%
Todas las personas buscan influir en los demás, desde decisiones poco importantes hasta decisiones vitales.
%
¿Quién no ha vivido una situación en la que alguien se estaba equivocando\footnote{o eso se pensaba} y se le ha intentado conducir y corregir? 
%
Eso podría estar claro que no es una manipulación ya que se trata por el bien del otro.
%
¿Y si, en esa corrección, el corrector sale beneficiado y busca el cambio de actitud del corregido porque también sale beneficiado?
%
Los límites de la manipulación se difuminan porque es difícil ponderar la importancia que se le está dando al bien de la otra persona y al bien propio.
%
¿Y si busco el bien de un tercero?
%
Como docente podría interesar influir en un determinado estudiante para que modificara su actitud, no tanto por su propio bien personal, sino por el bien de los demás alumnos.
%
¿Quién no ha escuchado "Tu siéntate ahí, calladito y no molestes."?
%
Se están silenciando dimensiones propias de la libertad y no se está buscando el bien para esa persona.
%
¿Es esto una manipulación? ¿Se considera simplemente influencia? ¿Es éticamente correcta?

Como conclusión es importante destacar que, cuando mostramos el fin de una determinada influencia o persuasión, estamos salvaguardándonos de caer en una manipulación.
%
Además, es importante hacer capaz a la otra persona de pensar por sí misma. 



\paragraph{Cuando no educar: Nunca}
Una tentación de la profesión docente es a reducir al ámbito escolar la labor educativa.
%
Si un profesor se encuentra con alumnos por la calle que están realizando un acto de vandalismo, la opción fácil es no educar pasando desapercibido y no interviniendo.
%
Tal vez en un acto vandálico es más fácil intervenir.
%
¿Y si están de botellón? ¿Sería necesario intervenir? 
%
Si hacemos caso a que no se debe no educar nunca, lo ideal sería intervenir.

Un ejemplo menos drástico y más fácil sería a la hora de cruzar la acera. 
%
El docente es educador también fuera del centro escolar y su ejemplo es importante.
%
¿Qué educación ofrece a los alumnos que los profesores crucen mal? 
%
¿O que fumen?
%
Es importante ser ejemplares en todo momento, no sólo en el centro o sus proximidades.


