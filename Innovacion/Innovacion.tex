\documentclass[palatino,nochap]{apuntesURJC}

\title{Innovacion}
\author{Víctor de Juan}
\date{16/17 C1}

% Paquetes adicionales

% --------------------

\begin{document}
\pagestyle{plain}
\maketitle

\tableofcontents
\newpage
% Contenido.

\section{Portafolios}

\paragraph{Primera entrada del portafolio: ¿Qué es innovar en educación?}

Hacer las cosas de otra manera. "Si siempre haces lo mismo, no esperes resultados distintos." decía Einstein y para mi innovar es hacer de manera distinta teniendo como objetivo resultados distintos.

Mis expectativas para la asignatura son fundamentalmente ejercitar mi capacidad de innovar de cara a dar las clases. 

Tras comentar en clase la idea de innovación, he descubierto que hay otro aspecto muy importante: que sea un cambio duradero con mucha adopción.
%
Si es un cambio que no mejora las cosas, no es una innovación. 
%
Si nadie está dispuesto a adoptar la nueva forma, tampoco es una innovación.

En esta clase escuché por primera vez el término \concept{Educación bulímica}, que significa \textit{Estudio para vomitar en el examen}.



Estudiando un informe sobre la visión que se tiene de las matemáticas veo resultados que me resultan sorprendentes. 
%
Quien rechaza las mates se hace, no nace. En primaria nadie rechaza las mates, en Secundaria sí.
%
Además, todas las personas reconocen que las matemáticas son realmente útiles, quienes se les dan bien cómo quienes no se les dan bien.

A quien le gustan las mates, las consideran fáciles. A quién no le gustan, las consideran difíciles. 50\% de alumnos que rechazan las mates es por culpa de los malos profesores. 30\% de alumnos que no rechazan las mates es gracias a sus profesores. 


\paragraph{Segunda entrada del portafolio}


En verano estuve de voluntariado en Perú y una semana fue dar clase de mates a chavales de secundaria de allí. 
%
Les ocurría que sabían despejar las ecuaciones perfectamente. Hacían todos los pasos bien, hasta que al llegar al final, $3x = 24 \to x=\frac{24}{3} = ?$. 
%
El último paso no eran capaces de hacerlo. ¿Cómo es posible que lleguen al curso en el que están y sepan despejar sin saberse las tablas de multiplicar? 
%
¡Qué sistema educativo tan ineficiente! 
%
Pasan de curso sin los conocimientos necesarios... construyen el conocimiento con unas lagunas bestiales.

Viendo los errores de primero de grado propuestos por Raquel me doy cuenta que no hay tanta diferencia entre el modelo de allí, con el modelo de aquí; 
%
en cuanto a construir un conocimiento sólido sin lagunas.

\paragraph{Entrada 3: TIC}

Hoy hemos visto utilizar una \concept{PDi(P)}\textbf{ - Pizarra Digital interactiva (Portátil)} y unas cuantas aplicaciones muy útiles para trabajar con tics en el aula.

Entre ellas: Graphing calculator 3D (para dibujar superficies e intersecciones de superficies).
%
Esta herramienta complementa muy bien a geogebra, puesto que los problemas de geometría de segundo de bachillerato son en tres dimensiones y geogebra no tiene opción a dibujar en 3 dimensiones.

Otra herramienta simulaba el comportamiento de objetos atraídos por la fuerza de la gravedad. 
%
Lo primero es dibujar los objetos sin gravedad. Después, al activar la gravedad los objetos se mueven y se puede ver el comportamiento de un péndulo, de un lanzamiento vertical y estudiar a qué altura llega... ¡Los enunciados de los problemas pueden ser vídeos!

A la hora de dibujar en matemáticas, estas herramientas son geniales. Ayudan a visualizar mucho mejor y sobretodo, ahorran mucho tiempo de hacer los dibujos.

La explicación sobre los colores primarios de 15 segundos es la mejor explicación que he escuchado en la vida. 
%
Tener el \textit{flash} sobre los colores primarios de la luz permite mezclaros en el momento y moverlos sin apenas esfuerzo. 
%
Al mezclarlos, se comprueba claramente que son los colores complementarios (los colores que utilizan las impresoras para conseguir todos los demás).

Ha sido muy constructiva esta sesión, aunque hubiera estado mejor poder probarlas, ya que escribir sin mirar a dónde estás escribiendo no me parece algo muy intuitivo. 
%
¿Cuánto entrenamiento hace falta para acostumbrarse?

\paragraph{Entrada 4: Más recursos}

Hoy ha sido un bombardeo de recursos para utilizar en clase. Paginas web, blogs, educalab... 
%
A día de hoy se me queda lejano porque no he profundizado sobre los recursos. 
%
Parece que voy a tener que estudiarme y bucear por todos estos recursos para descubrir cuáles me gustan más, cuáles me parece que pueden ser más útiles, etc.
%
La cantidad de tiempo invertido en filtrar 



\section{Ideas interesantes}

\begin{itemize}
	\item ¿Porqué no, hablar todos los años de quién ha ganado la medalla Fields?
\end{itemize}







%% Apendices (ejercicios, examenes)
\appendix

\chapter{---}
% -*- root: ../Innovacion.tex -*-

\section{Conglomerado de recursos}

\begin{itemize}
	\item
	Estudios internacionales de Evaluación
	http://www.mecd.gob.es/inee/publicaciones/estudios-internacionales.html 
	\item 
	Instituto NAcional de Tecnologias Eduativas y Formación del profesorado 
	http://educalab.es/intef
	\item 
	Seamos gansos
	https://www.youtube.com/watch?v=K5G8gRvx7nQ
	\item 
	9 gestos cotidianos que los matemáticos hacemos de otra manera:
	http://verne.elpais.com/verne/2015/11/13/articulo/1447413460_147289.html?id_externo_rsoc=FB_CM
	\item 
	La escuela en 2030
	http://www.elmundo.es/espana/2014/10/21/54455b9f22601d22738b458e.html
	\item 
	El cerebro necesita emocionase para aprender
	http://economia.elpais.com/economia/2016/07/17/actualidad/1468776267_359871.html
	\item 
	La neuroeducación demuestra que emocion y conocimiento van juntos
	http://blogs.elpais.com/ayuda-al-estudiante/2013/12/la-neuroeducacion-demuestra-que-emocion-y-conocimiento-van-juntos.html
	\item 
	mates y neurociencia
	https://escuelaconcerebro.wordpress.com/2012/03/20/matematicas-y-neurociencia/
	\item 
	actividad cerebral del alumno durante la clase magistral
	http://ined21.com/actividad-cerebral-del-alumno-durante-la-tradicional-clase-magistral/
	\item 
	Aprendizaje cooperativo y neuroeducación: guiando la poda sináptica
	https://escuelaconcerebro.wordpress.com/2016/08/18/aprendizaje-cooperativo-y-neuroeducacion-guiando-la-poda-sinaptica/

\end{itemize}


\printindex
\end{document}
