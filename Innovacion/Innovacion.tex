\documentclass[palatino,nochap,miniheader]{apuntesURJC}

\usepackage{soul}
\usepackage{exmath}


\title{\textsc{Innovación educativa y TICS aplicadas a la enseñanza de las Matemáticas}}
\author{Víctor de Juan,
Pedro de la Mata Gómez,
Virginia Vadillo Lacasa,
Gustavo Adolfo Martínez Risque}
\date{16/17 C1}


\renewenvironment{leftbar}[2][\hsize]
{
    \def\FrameCommand
    {
        {\color{#2}\vrule width 3pt}
        \hspace{0pt}
    }
    \MakeFramed{\hsize#1\advance\hsize-\width\FrameRestore}
}
{\endMakeFramed}


%%% A COLOR:
\newcommand{\guscolor}{OliveGreen}
\newcommand{\virgicolor}{Goldenrod}
\newcommand{\pedrocolor}{NavyBlue}
\newcommand{\victorcolor}{Bittersweet}


% EN blanco y negro

%\newcommand{\guscolor}{black!90!white}
%\newcommand{\virgicolor}{black!70!white}
%\newcommand{\pedrocolor}{black!40!white}
%\newcommand{\victorcolor}{black!15!white}



%\newenvironment{opin}[2]{
%   \textcolor{#1}{\textbf{Aportación de #2:}}
%   \begin{leftbar}{#1}
%}{\end{leftbar}}
% Paquetes adicionales


\newenvironment{opin}[2]{
\setcounter{subsubsection}{0}
\setulcolor{#1}
   \textbf{\textcolor{#1}{\underline{\textcolor{black}{Aportación de #2:}}}}
   \begin{leftbar}{#1}
}{\end{leftbar}}



% --------------------

\begin{document}
\pagestyle{plain}
\maketitle

\tableofcontents
\newpage
% Contenido.

\chapter{Portafolios}

Para distinguir correctamente durante el portafolios a quién pertenece la aportación, hay una barra de color a la izquierda del texto, siguiendo la siguiente leyenda:

\begin{table}[h!]
\centering
\begin{tabular}{|c|c|}
\hline
Gustavo & \colorbox{\guscolor}{--------------------}\\
Virginia & \colorbox{\virgicolor}{--------------------}\\
Pedro & \colorbox{\pedrocolor}{--------------------}\\
Víctor & \colorbox{\victorcolor}{--------------------}\\\hline
\end{tabular}
\end{table}

\section{Estructura básica}

\subsection{Innovación en Educación (Tema 1 - 10/10/2016)}

\begin{opin}{\guscolor}{Gustavo}


\subsubsection{¿Qué es innovar en educación para ti?}
Innovar en educación para mí sería buscar cómo cambiar la forma tradicional de impartir las asignaturas para intentar que los alumnos se sientan más atraídos y puedan retener lo aprendido si no es para el resto de su vidas, al menos durante el mayor tiempo posible y así evitar lo que ocurre en muchos casos de que los alumnos aprueban los exámenes y se olvidan.


\subsubsection{Expectativas iniciales de la asignatura}
Personalmente considero que me encuentro entre ese perfil de alumno que aprueba un examen y se olvida casi por completo de  lo estudiado. Es cierto que esta situación se acentuaba cuando las asignaturas de las que me examinaban eran de memorizar. Con las matemáticas era algo diferente porque era necesario conocer la base para seguir avanzando en los cursos posteriores, los cuales además servían de recordatorio de lo estudiado.


Dicho esto, cuando leí el título de esta asignatura (\textit{Innovación Educativa y TICs aplicadas a la enseñanza de las Matemáticas}) me dio la sensación de que iba a conocer cuáles son las nuevas metodologías de trabajo innovadoras que se están poniendo de moda y que son tan eficaces para el aprendizaje como las metodologías tradicionales. Estas nuevas metodologías docentes incluirían cambios drásticos con respecto a la metodología tradicional. También me imaginaba que se haría mucho hincapié en el uso de las nuevas tecnologías dado que facilita la labor para el docente y es una herramienta muy práctica en el aprendizaje.


Sobre el uso de las TICs en general me gustaría añadir una opinión personal. Tengo la sensación de que hay una tendencia generalizada de pensamiento que opina que con las TICs se va a poder hacer todo. Considero que hay que tener mucho cuidado con el uso de las tecnologías para todo. Las TICs sólo sirven cuando ayudan a realizar una determinada labor. Por poner un ejemplo simple, hay ocasiones en las que las TICs son tan complicadas de usar que no solo no ayudan sino que perjudican.

\end{opin}

\begin{opin}{\victorcolor}{Víctor}

\subsubsection{¿Qué es innovar en educación?}

Hacer las cosas de otra manera. "Si siempre haces lo mismo, no esperes resultados distintos." decía Einstein y para mi innovar es hacer de manera distinta teniendo como objetivo resultados distintos.


Tras comentar en clase la idea de innovación, he descubierto que hay otro aspecto muy importante: que sea un cambio duradero con mucha adopción.
%
Si es un cambio que no mejora las cosas, no es una innovación. 
%
Si nadie está dispuesto a adoptar la nueva forma, tampoco es una innovación.

En esta clase escuché por primera vez el término \index{Educación bulímica}\textbf{educación bulímica}, que significa \textit{Estudio para vomitar en el examen}.


\subsubsection{Expectativas de la asignatura}


Mis expectativas para la asignatura son muy altas, ya que la innovación es fundamental de cara a la profesión docente.
%
Para desarrollarme como buen docente necesito principalmente 2 cosas:
\begin{itemize}
	\item Ser capaz de innovar.
 	\item Estar atento de las innovaciones de otras personas.
 \end{itemize} 

Espero que esta asignatura me ayude a pontenciar mi capacidad innovadora y a conocer recursos con los que innovar en clase, aprovechando las buenas ideas de otras personas.


Estudiando un informe sobre la visión que se tiene de las matemáticas veo resultados que me resultan sorprendentes. 
%
Quien rechaza las mates se hace, no nace. En primaria nadie rechaza las mates, en Secundaria sí.
%
Además, todas las personas reconocen que las matemáticas son realmente útiles, quienes se les dan bien cómo quienes no se les dan bien.
%
A quien le gustan las mates, las consideran fáciles. A quién no le gustan, las consideran difíciles. 50\% de alumnos que rechazan las mates es por culpa de los malos profesores. 30\% de alumnos que no rechazan las mates es gracias a sus profesores. 

Estos datos (y los demás que estuvimos comentando en clase) me hacen darme cuenta de lo fundamental de mi labor y me han hecho tomar conciencia de la responsabilidad que lleva esta profesión. Al fin y al cabo, como dice Ben Parker \textit{un gran poder conlleva una gran responsabilidad}.


\end{opin}

\begin{opin}{\pedrocolor}{Pedro}

\subsubsection{Expectativas iniciales de la asignatura}

En la primera semana del Máster ya fui consciente de mis carencias oratorias, dado mi perfil eminentemente técnico. Por lo tanto, esta asignatura se presenta como un “Recurso” que puede potenciar mis habilidades como educador, por su proximidad al campo científico del que procedo.

Dado que actualmente nos movemos en “la era 3.0”, veo necesario el empleo de nuevas técnicas para llegar mejor al alumnado, permitiendo un mejor desarrollo cognitivo que les permita razonar de forma lógica ante su entorno. 

Mi principal objetivo, al finalizar la asignatura, es conseguir adquirir las nociones necesarias para lograr el empleo de nuevas técnicas educativas y aplicarlas en un futuro a mis alumnos, intentando conseguir con ello una mejor compresión de los conceptos matemáticos.

\subsubsection{¿Qué es innovar en educación?}

Cuando hablamos de “Innovación educativa”, estamos introduciendo cambios en los patrones tradicionales de la educación, que conlleva una mejora del mismo. Para llevarlo a cabo, debemos pasar por un proceso de selección, organización y utilización creativa de los recursos humanos y materiales, que den como resultado los objetivos marcados. (Richland, citado por Moreno, 1995).

Tras la lectura de todas las definiciones de Innovación educativa que han dado alumnos de otros años, me gustaría aportar mi propia definición.

“La innovación educativa nace como resultado de un proceso que consiste en la observación y estudio de manera objetiva de las situaciones que se dan en el aula. Teniendo como resultado final la búsqueda de la mejora del proceso de aprendizaje del alumnado, mediante materiales y técnicas nuevas.

Considero que la mejor manera para demostrar que nuestro intento de Innovación Educativa en el aula tiene efectos positivos en el aprendizaje, es lograr captar la atención del alumno.

Con la llegada de la LOGSE, se comienza a pedir a los profesores de secundaria un conocimiento  didáctico sobre la materia que van a impartir. Sus críticas principales radican en el corto período de formación y la consideración de dicho proceso como un mero trámite.

La LOE habla por primera vez sobre la importancia de enseñar competencias y no contenidos, entendiendo por competencias básicas aquellas habilidades que todos los alumnos deben haber alcanzado al finalizar la Educación Secundaria Obligatoria. Como aspecto negativo, esta ley contempla la posibilidad de pasar de curso con materias suspensas, lo que supone potenciar el error de construir conocimientos justo encima de lagunas arrastradas de cursos anteriores.

Con la implantación del Plan Bolonia, se logra dar al Profesorado de Secundaria la necesaria formación teórica (académica y pedagógica) y práctica.  No obstante, es fundamental que el profesor de matemáticas:

\begin{itemize}
%\vspace{-0.2cm}
\item Se encuentre inmerso en un proceso continuo de actualización (científica y didáctica), lográndose mediante la investigación educativa y acercamiento a nuevas TICs. 
%\vspace{-0.2cm}
\item Se encuentre motivado por la asignatura. 
%\vspace{-0.2cm}
\item Conozca los diferentes enfoques que se puede dar a la asignatura, sin limitarse a formalismos. 
%\vspace{-0.2cm}
\item Elija el metodo adecuado de enseñanza para cada etapa o situación de aprendizaje. 

\end{itemize}

Me llama especialmente la atención los resultados arrojados por los estudios de evaluación educativa (PISA o TIMSS), mostrando un deterioro significativo de la enseñanza matemática pese a las reformas educativas que se han llevado a cabo. Y me pregunto: ¿Qué podemos hacer el profesorado para cambiar esta situación? ¿La utilización de recursos educativos en el aula mejoraría el proceso de enseñanza-aprendizaje? ¿Hace falta ser un experto para el uso básico de TICs?


\begin{minipage}[hbtp]{1.0\linewidth}
\centering
\includegraphics[scale=0.1]{img/pedro1.jpg}
\captionof{figure}{Actividad Jornada Metodologías Activas de educación en el aula con Irene Ros \\ ¿Qué haríais para mejorar la educación?}
\end{minipage}

\end{opin}

\begin{opin}{\virgicolor}{Virginia}

\subsubsection{Expectativas de la asignatura}
Desde el primer momento el nombre de esta asignatura me llamó la atención ya que contiene dos puntos muy importantes en el mundo actual, la innovación y las tecnologías de la información y la comunicación. 

En un mundo en el que la tecnología avanza tan sumamente rápido que compañías tecnológicas importantes están en continuo progreso de tal forma que por ejemplo sacan nuevos dispositivos móviles cada año, no me extraña que esas ganas de innovación y uso de las nuevas tecnologías se aplique a diferentes campos. 

Si bien es cierto es que hasta que leí el nombre de esta asignatura no se me ocurrió pensar en su aplicación a la enseñanza. Desde luego me lleve una grata sorpresa porque creo que puede ser algo muy interesante el transmitir los conocimientos de una forma nueva y diferente usando las tecnologías actuales ya que, en mi época, en los 90, carecíamos de la mayoría de los dispositivos y medios que hay actualmente. 

En definitiva, tengo bastante curiosidad en saber que nos deparará esta asignatura, creo que tiene que ser muy interesante y muy positivo cambiar de alguna forma la enseñanza que hasta ahora conozco con una nueva y mejorada que facilite el aprendizaje. También pienso que debe ser una tarea compleja y que hay mucho por aprender y espero que al finalizar esta asignatura comprenda y sea capaz de aplicar nuevos y diferentes métodos de enseñanza que me permitan mejorar la transmisión de mis conocimientos haciendo que el aprendizaje sea más ameno y divertido y los alumnos estén siempre motivados con una asignatura no muy querida como son las matemáticas.


\subsubsection{Innovar en educación}

En el primer día de clase se entró en contacto por un lado con los conceptos que dan nombre a la asignatura: innovación y tecnologías de la información y la comunicación y por otro lado con el estado actual de la enseñanza tras la evolución de las diferentes leyes de educación.

Antes de escuchar y comprender el significado de la palabra innovación en educación que nos contó Raquel, cada uno definimos cómo entendíamos ese concepto. Mi definición fue la siguiente: innovar en educación es buscar nuevas técnicas o métodos de enseñanza utilizando nuevos recursos como por ejemplo las nuevas tecnologías para conseguir transmitir los conocimientos de una forma nueva y diferente que resulte más atractiva y más fácil de comprender.

Estoy de acuerdo que la innovación tiene que ser objeto de algo planificado, para innovar uno tiene que ser consciente de lo que quiere hacer (introducción de algo nuevo: conceptos, métodos, materiales…) y para qué lo hace (mejorar la enseñanza y el aprendizaje).

Siempre me ha llamado la atención y no positivamente que los gobiernos cambien tan a menudo la ley de educación con lo que eso implica para los estudiantes. A mí me pillo la LOGSE lo que supuso dos cambios de centro en dos años, ya que los institutos tenían que abordar más alumnos, y no estaban preparados cuando me tocó cambiarme, por lo que pase por un centro de “paso”. En mi opinión, cambios de legislación de forma tan frecuente lo único que hacen es despistar más al alumnado y al profesorado, lo que genera huelgas y desmotivación.

Evidentemente creo que España necesita un cambio en el sistema educativo entre otras cosas porque al igual que las personas, la tecnología y la vida evolucionan, la educación también lo tiene que hacer adaptándose a las necesidades actuales.

También me hizo mucho que pensar cuando se trató si era adecuada o no la formación del profesorado. Como es lógico para que los alumnos aprendan de forma adecuada el profesor tiene que tener habilidades psicopedagógicas, pero evidentemente saber de lo que está hablando, es decir, conocer en profundidad la asignatura que imparte. Me resultó impactante que los futuros alumnos de magisterio pudieran suspender matemáticas en selectividad, lo que me hizo reflexionar mucho sobre la formación de los futuros maestros de primaria. En mi opinión, me da la sensación que muchos de los que se apuntan a magisterio se piensan que es una carrera sencilla y de letras en la que no existen las matemáticas, ya que la mayoría de los que hacen magisterio realizan el bachillerato de ciencias sociales o humanidades. Teniendo en cuenta que los primeros seis años de educación escolar en matemáticas están a cargo de los maestros, ¿cómo es posible que no se exijan más conocimientos a los futuros maestros en matemáticas? Conozco maestros de primaria que no saben aplicar un porcentaje en su vida real ni cuando van a las rebajas. Es difícil, que los alumnos entiendan la utilidad de las matemáticas si mucho de los maestros no saben aplicarlas en su vida real. Creo que la labor de un maestro es muy importante y como tal habría que ser más exigentes en su formación.

En vista al problema de matemáticas en primaria, los alumnos llegan a secundaria con un rechazo considerable a la asignatura. Me resultó muy representativa la imagen del círculo vicioso, como dicen en su artículo Hidalgo Alonso, Maroto Sáez y Palacios Pico, de las aptitudes de los estudiantes hacia las matemáticas: dificultad – aburrimiento – suspenso –fatalismo - bajo autoconcepto – desmotivación – rechazo - dificultad. Está claro en matemáticas o en cualquier ámbito de la vida cotidiana que algo que te parece difícil o que te crees que no vales para ello provoca un fuerte rechazo. Así, me parece muy importante en este caso la labor del profesor para conseguir que sus alumnos tengan una actitud positiva a las matemáticas haciéndoles ver que es útil en su vida y que todos ellos tienen capacidad para comprenderla, aptitudes que se tienen que implantar desde primaria y reforzar en secundaria. Por ello, en la primera parte de nuestro trabajo en grupo se quiso hacer hincapié en esto y quisimos tratar el tema de la indefensión aprendida y la motivación para demostrar a los alumnos que si quieren, pueden.

En cuanto a información añadida que Raquel nos incluye en el campus por temas, en este quiero destacar el vídeo de los gansos y el artículo “9 gestos cotidianos que los matemáticos hacemos de otra manera”. Con el vídeo de los gansos lo que se muestra es la importancia de trabajar en grupo, ayudándose los unos a los otros y aprendiendo juntos para lograr un objetivo común. El artículo me resultó muy curioso, destacar lo de no echar la lotería, que me he dado cuenta que es lo mismo que hago yo, que sólo echo la de Navidad por no quedarme con cara de boba si le toca a los demás compañeros del trabajo y a mí no por no comprar el décimo. Me hizo gracia lo de jugar con los números de las matrículas porque me recordó a otra cosa que veo yo en las matrículas, pero es que soy química y no matemática pura por ello lo que yo veo son abreviaciones de productos químicos, por ejemplo cuando veo DNT enseguida me acuerdo del explosivo dinitrotolueno, un símil a lo que hacen los matemáticos con los números.

Otro tema que me hizo reflexionar bastante fue la opción de aplicar nuevos métodos de aprendizaje en la clase que implican la participación activa y colectiva de los alumnos mediante el trabajo por proyectos, sustituyendo la típica clase magistral en la que el profesor suelta todo el temario y luego los alumnos tienen que hacer deberes sin parar y sin apenas entender el qué están haciendo y para qué finalidad. Este nuevo método me abrió la mente a nuevas posibilidades en la educación que yo no imaginaba y que me parecen muy interesantes a la hora de motivar y estimular a los alumnos para que quieran aprender más. Me quedo como conclusión con una frase del educador, psicólogo e investigador Alfredo Hernando autor del blog escuela21: \textit{“La clave, por encima de todo lo demás, es enseñar a los alumnos a pensar por sí mismos. Se trata, según resume Hernando, de plantearles preguntas cuyas respuestas no se puedan googlear"}.

\end{opin}


\subsection{Introducción a la Neurodidáctica (Tema 2.1 - 17/10/2016)}
\begin{opin}{\guscolor}{Gustavo}

\subsubsection{Introducción a la neurodidáctica}

Parece una evidencia científica que sin emoción no hay aprendizaje. Es necesario despertar en los adolescentes el interés por las matemáticas en particular y por el conocimiento en general. Y para ello podemos hacer uso de técnicas que permitan emocionar a los alumnos sobre lo que están haciendo y de esta manera conseguir que, seguramente de manera inconsciente para ellos, aprendan los contenidos básicos necesarios para su desarrollo personal.


Hay gente que equipara los términos neurodidáctica y neuroeducación, ambos asociados a la neurociencia. En nuestra clase lo vamos a diferenciar en función de:

\begin{itemize}
\item Neuroeducación: Manera en la que el cerebro aprende
\item Neurodidáctica: Manara en la que el docente lleva a la práctica la neuroeducación
\end{itemize}

\subsubsection{Neuroeducación por otra escuela}

Se puede acceder al video en el siguiente enlace:

\href{https://www.youtube.com/watch?v=QiRqCKUiRDc\&feature=youtu.be}{https://www.youtube.com/watch?v=QiRqCKUiRDc\&feature=youtu.be}
\paragraph{Partes del cerebro. Aprendizaje =  DIVERSION * K}
Las cosas que yo pienso \textmd{se ejecutan} mediante la parte del cerebro prefrontal, que sería la parte de la cabeza que usan los futbolistas para rematar un balón. Se denomina función ejecutiva y se encarga de:
\begin{itemize}
\item La concentración
\item El control de impulsos
\item La memoria a corto plazo
\end{itemize}
La amígdala es la parte del cerebro que se encarga de la emoción y es “la gasolina” de la función ejecutiva.


\begin{minipage}[hbtp]{1.0\linewidth}
\centering
\includegraphics[scale=0.2]{img/cerebrogus.png}
\captionof{figure}{Esquema de la posición de la amígdala en el cerebro humano.}
\end{minipage}

Por simplificar mucho hasta el momento, la parte racional está en el prefrontal y la parte emocional está en la amígdala.


Con esto se explicaría que si estás emocionado con algo, tengas una mejor concentración y por tanto un mejor aprendizaje. En realidad, el cerebro tiene muchas partes y para que el proceso de aprendizaje sea total se deberían emplear todas las regiones y funciones del cerebro.


Como dijo \textbf{Helena Lopez Casares en la conferencia que dio el 20 de septiembre en el Campus Vicálvaro} sobre inteligencias múltiples, hay estudios que demuestran que personas con problemas en la parte del cerebro que controla las emociones son incapaces de tomar una decisión, con lo cual se demuestra la importancia de todas las partes del cerebro deben estar activas y conectadas.

\paragraph{Educación bulímica. Matemáticas, Historia y Filosofía}
Me ha gustado mucho la afirmación que hace Javier Blumenfeld: “Tenemos una especie de educación bulímica: yo te meto contenidos y tú me los vomitas en el examen”. Esto es lo que yo llamo metodología tradicional de la enseñanza.


En este caso sí haría diferenciación por asignaturas y lo aplicaría a mi experiencia personal. En el caso de las matemáticas, aparte de que siempre me han gustado más, he sido capaz de aprenderlas y es muy difícil que se me olviden. O si hay algo que no tengo claro, con un simple repaso sería capaz de recordarlo. Estas matemáticas las he aprendido bajo un contexto de educación bulímica y aun así he sido capaz de aprenderlas. Sin embargo, en asignaturas como historia, filosofía, u otras asignaturas que la manera de aprobar consistía en la memorizar y repetir lo memorizado en el examen, no he recordado nunca nada de adulto.


Sin embargo, es curioso como a día de hoy sí me siento atraído por la historia y por los pensamientos filosóficos. Bajo mi punto de vista es que este cambio se debe a que he alcanzado unos conocimientos y una madurez que me permiten disfrutar de ello. No tenía ningún sentido estudiarlo con la edad que tenía. No disfrutaba de ello. Hace poco hice un tour que te lleva por el Madrid de los Austrias con guía y lo disfruté muchísimo, cosa que era impensable cuando era un niño. Este sería un claro ejemplo de que cuando algo te gusta lo aprendes mejor. Aunque en este caso concreto, tengo la opinión de que es muy difícil motivar a unos alumnos que no han vivido los suficientes años como para entender y disfrutar de la historia y de los filósofos.


\paragraph{Buscando culpables}
En el punto anterior he hecho una exposición desde mi propia experiencia personal. Sin embargo, tengo compañeros y amigos que sí que han sido capaces de aprender más cosas de historia que yo en etapas tempranas de la vida.


¿Cuál será el motivo por el que yo NO he aprendido historia y otras personas que conozco sí? 
¿Será que el colegio en el que ellos estudiaban, fueron capaces de emocionarles mejor en Historia? 
¿O será que yo nunca he puesto interés? 
Tenemos que tener cuidado también con aquellos alumnos que se aprovechen de esto para decir que no han aprendido lo suficiente alegando que el sistema educativo no era el adecuado o que las metodologías de aprendizaje no eran las adecuadas. Hay que hacer analizar todas las situaciones de manera individual para poder discernir aquellos alumnos que realmente han sido sometidos a altos niveles de estrés que han limitado su capacidad de aprendizaje de aquellos alumnos que simplemente no ponían interés.


\paragraph{Entendido el problema. Y ahora …?}
Una vez hecho el diagnóstico de la situación y entendido que como docentes tenemos que motivar para conseguir emocionar al alumnado y de esta manera conseguir que el aprendizaje sea más efectivo, lo que hay que ver ahora es cómo llevar a cabo ese cambio.


Pero antes de empezar a analizar las diferentes alternativas que se plantean como los trabajos por proyectos, aprendizaje basado en problemas, etc me gustaría lanzar una pregunta abierta sin ánimo de criticar estas iniciativas de cambio y es la siguiente, ¿Qué pasaría si con el cambio ocurre que hay alumnos que no se sienten motivados por estas nuevas metodologías? Podríamos volver al problema del cual partimos y volveríamos a tener alumnos desmotivados. Según esto, lo ideal sería una enseñanza personalizada en el individuo y no en el grupo. Pero por otro lado, vivimos en sociedad, somos seres sociales y una enseñanza individual podría dar lugar a perder otro tipo de conocimientos sociales muy importantes. La conclusión es que podamos combinar una educación en la que cada uno aprenda a su velocidad pero en sociedad.

\paragraph{Mens sana in corpore sano}
Me ha interesado mucho también el estudio que asocia deporte con aprendizaje.


Tener alumnos sentados durante tanto tiempo en el aula es antinatural. Después de hacer un ejercicio, sobre todo aeróbica el cerebro funciona mejor.


La irisina se genera al hacer deporte y baja de los músculos al cerebro y favorece la plasticidad neural, que es la base del aprendizaje.


En este caso, las imágenes que aparecen en las transparencias de clase son muy significativas. Se ha evolucionado en muchos aspectos de la vida y sin embargo parece que las aulas permanecen “incambiadas” desde hace muchos años.
\end{opin}


\begin{opin}{\victorcolor}{Víctor}
.


\end{opin}

\begin{opin}{\pedrocolor}{Pedro}

.


\end{opin}

\begin{opin}{\virgicolor}{Virginia}
.


\end{opin}


\subsection{Introducción a la Neurodidáctica (Tema 2.2 - 24/10/2016)}
\begin{opin}{\guscolor}{Gustavo}

\subsubsection{Emoción y Aprendizaje}

En el día de hoy hemos terminado el tema 02 relacionado con la neurodidáctica y me ha impactado la pregunta que nos ha hecho Raquel en clase porque resume muy bien la relación existente entre emoción y aprendizaje. La pregunta es la siguiente:

\textbf{¿Recuerdas lo que estabas haciendo el 11 de Septiembre de 2001 cuando los aviones se estrellaron contra las torres gemelas de Nueva York?}

Es alucinante como un alto porcentaje de la gente a la que le preguntas recuerda exactamente donde estaba y lo qué estaba haciendo.

Se me ocurre otra pregunta similar: \textbf{¿Recuerdas donde viste a España ganar la Copa del Mundo de Futbol de Sudáfrica 2010?}

Como veis estas emociones pueden ser positivas o negativas. Pero lo importante es que haya emoción.

Estos son claros ejemplos de que si hay emociones, hay aprendizaje. Me atrevería a definirlo como un aprendizaje inconsciente y que permanece en la memoria a largo plazo.

Toda esta relación entre emociones y aprendizaje es aún más importante en la adolescencia porque en esta etapa de la vida el cerebro demanda dopamina. Para liberar la dopamina, se necesitan cosas emocionantes. Por ejemplo, cuando nos divertimos segregamos dopamina. El aprendizaje debe ser divertido. 

\subsubsection{La motivación escolar: siete etapas clave}

Volviendo a lo que no terminamos de determinar la semana pasada. El diagnóstico de la situación está claro. ¿Qué podemos hacer para cambiarlo?

Realizar cambios implicaría cambiar con lo establecido y hay muchas personas que tienen miedo a lo desconocido y por eso deciden no cambiar. Solo hay que ver las fotografías de las transparencias de clase para darse cuenta de que al igual que en otros aspectos la sociedad ha evolucionado, en la educación nos hemos quedado estancados.

\label{noevolucionEdu}

\vspace{3cm}
\end{leftbar}
\vspace{-2cm}
\begin{table}[hbt]
\begin{leftbar}{\guscolor}
	\begin{tabular}{cc}
		\begin{minipage}[hbtp]{0.5\linewidth}
			\centering
			\includegraphics[width=0.8\linewidth]{img/coche1.jpg}
			\captionof{figure}{Coche de mediadios del siglo pasado.}
		\end{minipage}
		&
		\begin{minipage}[hbtp]{0.5\linewidth}
			\centering
			\includegraphics[width=0.8\linewidth]{img/coche2.jpg}
			\captionof{figure}{Coche actual.}
		\end{minipage}\\
		\begin{minipage}[hbtp]{0.5\linewidth}
			\centering
			\includegraphics[width=0.8\linewidth]{img/coche3.jpg}
			\captionof{figure}{Educación de mediados del siglo pasado.}
		\end{minipage}
		&
		\begin{minipage}[hbtp]{0.5\linewidth}
			\centering
			\includegraphics[width=0.8\linewidth]{img/coche4.jpg}
			\captionof{figure}{Educación actual.}
		\end{minipage}
	\end{tabular}
	\caption{Resumen de la necesidad de innovación en la educación.}
	%\captionof{tabular}{Resumen de la necesidad de innovación en la educación.}
\vspace{1.5cm}
\end{leftbar}
\vspace{-1.5cm}
\end{table}

\begin{leftbar}{\guscolor}
\vspace{-1.5cm}

Para describir cómo podemos cambiar la situación, me voy a basar en el análisis del artículo del blog “Escuela con cerebro” (\url{https://escuelaconcerebro.wordpress.com/2014/09/18/la-motivacion-escolar-siete-etapas-clave/}) en el que nos hace pensar acerca de lo que podemos hacer en la práctica los profesores para motivar al alumno, cómo conseguir despertar su interés por el aprendizaje (motivación inicial), mantener una implicación regular (motivación de logro) o hacer que el proceso de evaluación sea útil, entre otras cosas.

Analizando las etapas clave

\paragraph{1. Qué curioso}

En los inicios de clase o de las unidades didácticas correspondientes es imprescindible hacer presentaciones activas y variadas que pueden alternar visualizaciones de videos, planteamientos de preguntas al modo socrático clásico, utilización de anécdotas o ejemplos adecuados, etc.

Considero muy importante despertar el interés del alumno a la hora de afrontar un nuevo reto.

\paragraph{2. ¡Esto me interesa!}

Cuando los contenidos que se van a trabajar son contenidos reales cercanos a la vida del alumno y con un enfoque interdisciplinar es más fácil que se motive.

Hay poco que explicar acerca de esta etapa. Puede ser complicado buscar motivaciones individuales para cada alumno, aunque este sería el caso óptimo.

\paragraph{3. ¡Acepto el reto!}

Para no desmotivar al alumnado, los retos han de ser adecuados para que no se sientan cómodos, pero que tampoco se depriman porque les parezcan muy difíciles de abordar.

\paragraph{4. ¡Soy el protagonista!}

Hemos de respetar las preguntas, intervenciones, debates suscitados o análisis entre alumnos sin prisas (no hay excusas con lo de acabar el temario; lo importante no es lo que enseñamos sino lo que aprenden) y permitirles que intervengan en la creación de normas, elección de problemas o estrategias de trabajo.

Me encanta esta afirmación dado que muchas veces, el miedo o vergüenza a ser rechazado por preguntar cosas que deberían saber hace que muchos alumnos se queden estancados y pierdan el interés por la materia.

Esto pasa también en los entornos laborales y no ayudan para nada en el aprendizaje colaborativo.

\paragraph{5. ¡Progreso!}

La memoria es esencial para el aprendizaje (de hecho son dos procesos indisolubles) y lo que ocurre es que hay que hacer un uso adecuado de ella en cada tarea. Para que el progreso del alumno sea real se ha de poder integrar la nueva información con la ya conocida.

Puede parecer que solo queremos que los alumnos aprendan a base de juegos y emociones y acercarles las materias a cosas reales que ellos pueden relacionar. Sin embargo no hay que olvidar que es imposible no retener nada en la memoria. El ejemplo de saberse las tablas de multiplicación es muy descriptivo.

\paragraph{6. ¡Esto vale la pena!}

La satisfacción que produce al alumno el ver que va progresando y aprendiendo debe ser confirmada por la aplicación de criterios de evaluación claros

Sin duda, darte cuenta de que las cosas que aprendes tienen su utilidad en el mundo real te hace darte cuenta de que realmente merece la pena lo que estas estudiando.

\paragraph{7. ¡Soy útil!}

Como cualquier persona, el alumno tiene una necesidad de ser reconocido (el adolescente más) y se lo hemos de manifestar con naturalidad

Bajo mi punto de vista se ha hecho poco énfasis en esta parte de reconocimiento. Por suerte o por desgracia yo creo que me he educado en un entorno familiar en el que una de mis mayores motivcaiones era demostrar a mis padres y a mi familiar que yo merecía la pena y que se me reconocía mi esfuerzo incluso desde que estudiaba en primaria. 

También me ha gustado la disposición del aula para una clase cooperativa. Comparto la idea de un aprendizaje mucho más efectivo con este tipo de clases. Más cooperar y menos competir. Aunque visto de otro modo, lo que se puede estar fomentando es la competencia entre grupos en lugar de la competencia individual que podría fomentar la disposición de las mesas en una clase tradicional.


\subsubsection{Isaac Asimov en 1988. Pelos de punta}

Escribiendo este apartado me he encontrado con este video que me ha parecido alucinante.

Alucinante en el sentido de como Isaac Asimov ya en 1988 preveía que esto iba a suceder.

\url{https://www.youtube.com/watch?v=oIUo51qXuPQ}

La entrevista fue realizada por Bill Moyers para su programa televisivo "El Mundo de las Ideas"

Fuente: Bill Moyers Rewind: Isaac Asimov (1988)

\url{http://www.pbs.org/moyers/journal/blog/2008/03/bill\_moyers\_rewind\_isaac\_asimo\_1.htm}

Parece sensato que si hubo gente que supo entender lo que sucedería en años sucesivos, ahora podamos estar en lo cierto de lo que puede suceder en un futuro cercano en el mundo de la educación.

El enlace al video lo subí al foro general de la asignatura y tuvo mucho aceptación entre mis compañeros. A continuación muestro los comentarios que realizaron:

Raquel Navas Sanchez dijo: “Qué bueno el video, el tio ya visionaba cómo se transformaría la educación gracias a las nuevas tecnologías, gracias por compartirlo!”

Beatriz Mate Martínez dijo: “Muy interesante. Es increíble que lo que ahora nos parece innovación ya se comentara en 1988.” 

Helena Matesanz Marín dijo: “Gracias por abrir el foro con un vídeo tan interesante.  Cuánto cuesta que crezca un bosque y qué poco incendiarlo. Hay que intentar que mejore la educación.”


\end{opin}

\begin{opin}{\victorcolor}{Víctor}

En la tabla \ref{noevolucionEdu}  de mi compañero Gustavo se puede apreciar claramente la necesidad de innovación que necesita la educación y para ello, a lo largo de la asignatura, iremos viendo posibilidades con las que innovar en el aula.

\paragraph{Actividad cerebral del alumno durante la tradicional clase magistral}

El post \textit{Actividad cerebral del alumno durante la tradicional clase magistral} me resultó muy interesante.
%
Este post incluye un vídeo (\textit{Nuevas necesidades de la educación}), en el que se ven aulas de clase muy diferentes a las habituales. 
%
Aulas donde se potencia la creatividad, la energía, el talento.

Es necesaria una educación personalizada. 
%
Tal vez hace años tenía sentido una escuela en la que todos los alumnos salían muy parecidos, con los mismos conocimientos para realizar los mismos trabajos.
%
Pero los retos del mundo requieren nuevas soluciones que dependerán de lo creativas que sean las personas que se enfrenten a ellos, y esa creatividad, se puede ir potenciando desde la escuela.


\end{opin}

\begin{opin}{\pedrocolor}{Pedro}


Comenzamos con un breve repaso de lo visto hasta la fecha, con el único fin de dar empaque a toda la información vertida en los días anteriores. A día de hoy, he podido reflexionar sobre varias preguntas, a las que voy dando respuesta a lo largo del curso:

\begin{minipage}[hbtp]{1.0\linewidth}
\centering
\includegraphics[scale=0.5]{img/pedro2.png}
%\captionof{figure}{...}
\end{minipage}

Continuando un poco en la línea, y finalizando el Tema 2, afirmamos que resulta necesario vincular las emociones al aprendizaje. La única manera de lograr este hecho es mediante la utilización de recursos apropiados por parte del profesorado para captar la atención del alumnado. Para corroborar esta afirmación cito una frase: “La atención no se presta, se capta” (Javier Espinosa, jornada de Gamificación). Actualmente esta emoción resulta difícil debido a la obligatoriedad de las matemáticas, por eso tenemos que buscar la manera de divulgar los contenidos para lograr que el alumno sienta la curiosidad por descubrir.

 
Llegado este punto, me cuestiono mi labor como educador. ¿Tengo que limitarme a mis obligaciones como mero transmisor de conocimiento matemáticos, o tengo que hacer un esfuerzo de divulgación para sacar a la luz su verdadera utilidad?

\textbf{A medida que avanza la clase, saco mi propia conclusión.}

Tengo que ser capaz de enseñar competencias claves a mis alumnos, y para ello buscar la mejor forma de hacerlo utilizando diferentes recursos. De esta manera, lograré que ellos tomen conciencia de la utilidad de las matemáticas:

\begin{center}
\textbf{“HACER QUE LOS ALUMNOS PIENSEN SIN TENER CONCIENCIA DE ELLO”}
\end{center}

 Destacamos “divulgaMAT”, web de divulgación matemática con gran cantidad de recursos educativos, página que en un futuro me reportará ayuda para captar la atención del alumnado a través de actividades.

No hay que dejar pasar por alto el daño matemático que en ocasiones hacen los medios de comunicación. Hay casos en los que no se utilizan las matemáticas como una herramienta, sino que la manipulan para que nos creamos la información (Web curiosa sobre fallos matemáticos: malaprensa.com). En el otro extremo tenemos grandes aportes a las matemáticas en los medios, como por ejemplo las series “Universo Matemático” y “Más por menos” o el programa  “Orbita Laika”, entre otros.

Un buen divulgador matemático debe utilizar buenos ejemplos para mejorar la comprensión por parte de los alumnos, consiguiendo forjar un aprendizaje duradero. Es muy enriquecedor divulgar nuestros logros educativos, para que otros alumnos puedan beneficiarse de nuestros avances, creando una red de recursos cada vez mayor. (Objetivo de “divulgaMAT”)

\subsubsection{Jornada de formación complementaria: \textsc{Taller de Gamificación} con Javier Espinosa - 27/10/2016}

\vspace{2cm}
\end{leftbar}
\vspace{-2cm}
\begin{figure}[hbtp]
\begin{leftbar}{\pedrocolor}
\centering
\includegraphics[scale=0.07]{img/gamingpedro.jpg}
\caption{Javier Espinosa al comienzo de la Jornada de Formación.}
\vspace{2cm}
\end{leftbar}
\vspace{-2cm}
\end{figure}

\begin{leftbar}{\pedrocolor}

Javier fue capaz de mantenernos  sin parpadear durante dos horas. Bailamos, reímos  y aprendimos. Comenzó la ponencia con un formulario vía Google Forms para romper el hielo, muy buena idea para un primer día de clase.

Nos pidió que fuéramos “HACKERS” del sistema, como nuevos docentes tendremos que ser capaces de buscar los recursos necesarios para mejorar la educación. La principal forma de conseguirlo es cambiar de “PASIOFF” a “PASION” en nuestro interruptor de educador. “UN SIMPLE CLICK”

Recalcó que no todos íbamos a tener unos buenos comienzos en este mundillo, pero que debemos evitar guiarnos por esos pensamientos que minan nuestra moral y buscar la manera de captar la atención de nuestros alumnos. Javier logró ganarse a sus alumnos emocionándolos mediante el empleo de la gamificación.

Dejó bien claro el concepto de gamificación, no debemos confundirlo con el termino juego. El juego no tiene un fin concreto, mientras que la gamificación implica un proceso creado por el profesor con un fin didáctico.

Aquí es donde Javier nos invitó a crear una experiencia siguiendo una serie de pautas:

\begin{itemize}

\item En primer lugar analizar el aula, identificando los diferentes tipos de jugadores. 
%
En la figura \ref{fig:tiposJug} podemos ver un resumen de los más básicos.

\end{itemize}
\vspace{2cm}
\end{leftbar}
\vspace{-2cm}
\begin{figure}[hbt]
\begin{leftbar}{\pedrocolor}
\centering
\includegraphics[scale=0.4]{img/gamingpedro2.png}
\caption{Tipos básicos de jugadores en la gamificación.}
\label{fig:tiposJug}
\vspace{2cm}
\end{leftbar}
\vspace{-2cm}
\end{figure}
\vspace{-1cm}
\begin{leftbar}{\pedrocolor}

\begin{itemize}



\item Crea tu propia historia dando rienda suelta a tu creatividad, pero hay que ser cauteloso para no perder de vista los objetivos didácticos del juego. No siempre las historias que requieren más esfuerzo por parte del profesorado tienen una mayor aceptación entre el alumnado, así que no malgastes tus fuerzas innecesariamente. 

\item Tabla de clasificación de los alumnos, es muy importante que ellos vean su progreso en la actividad. 
%
Podemos ver un ejemplo en la figura \ref{fig:scoreboard}

\end{itemize}
\vspace{2cm}
\end{leftbar}
\vspace{-2cm}
\begin{figure}[hpbt]
\begin{leftbar}{\pedrocolor}
\centering
\includegraphics[scale=0.35]{img/gamingpedro3.png}
\caption{\textit{Scoreboard} de la clase o cuaderno de notas del profesor.}
\label{fig:scoreboard}
\vspace{4cm}
\end{leftbar}
\vspace{-4cm}
\end{figure}
\begin{leftbar}{\pedrocolor}

\begin{itemize}
\vspace{-1.2cm}
\item Crear diferentes niveles, para que los alumnos vayan progresando con sus logros. 

\end{itemize}



La mejor manera de comprobar el gran trabajo realizada por Javier, es darse una vuelta por su blog \url{http://gamificationspain.weebly.com/blog/mis-gamificaciones-ah-las-tenis} . En él nos habla de sus dos proyectos de gamificación:

\begin{itemize}

\item \textbf{\textsc{Earthxodus}}: Una aventura apocalíptica para 1º de la ESO en la asignatura de Naturales. Donde los alumnos tiene que recorrer el planeta estudiando la atmósfera, hidrosfera y geosfera. Incorpora una tienda donde intercambiar su dinero virtual (tokens) por tickets especiales. Todo adaptado al contenido curricular de la asignatura. El enlace es \url{iaravaca.wix.com/1eso}

\item \textbf{\textsc{The Hospital}}: Un hospital donde llegan casos de famosos y los alumnos son médicos que les tienen que curar a la vez que estudian anatomía. Desde Miley Cyrus con un problema digestivo hasta Lady Gaga con problemas respiratorios. El enlace es \url{iaravaca.wix.com/thehospital} 
\end{itemize}

Incluimos la imagen \ref{fig:gamingcerebro} como resumen del funcionamiento de un cerebro gamificado, y las consecuencias positivas que éste tiene:

\vspace{2cm}
\end{leftbar}
\vspace{-2.2cm}
\begin{figure}[hbtp]
\begin{leftbar}{\pedrocolor}
\centering
\includegraphics[scale=0.345]{img/gamingpedro4.png}
\caption{Características del cerebro gamificado.}
\label{fig:gamingcerebro}
%\vspace{2cm}
\end{leftbar}
%\vspace{-2cm}
\end{figure}
\begin{leftbar}{\pedrocolor}
\end{opin}



\begin{opin}{\virgicolor}{Virginia}

En el siguiente punto, se analiza por tanto la relación fundamental que existe entre \textbf{las emociones y el aprendizaje}.  Francisco Mora dice que las emociones son la base más importante sobre la que se sustentan todos los procesos de aprendizaje y memoria. Estoy completamente de acuerdo ya que a veces no nos acordamos lo que hicimos ayer y sin embargo recordamos algo que pasó años atrás por el simple hecho de que lo asociamos a un momento emocional. Por ejemplo, Raquel nos preguntó que estábamos haciendo el 11 de Septiembre de 2011 y enseguida me vino a la memoria lo que hice porque lo asocie a la tragedia de las torres gemelas. La cuestión es emocionarse independientemente de que sea por algo negativo como es el caso de las torres gemelas como si es por algo positivo. En este caso, recuerdo perfectamente lo que estaba haciendo el día que me enteré que había conseguido el trabajo que tengo actualmente. Por tanto, puesto que en nuestra vida personal los recuerdos los asociamos con las emociones y los retenemos mucho mejor, parece evidente que sí conseguimos que los alumnos se emocionen con aquello que están aprendiendo, el proceso de aprendizaje será más efectivo y los conocimientos que adquieran serán más duraderos.

Encontré una entrevista que se le hace a Francisco Mora en la 2 de TVE que me resultó muy interesante en las que destaca que \textit{“para aprender hay que evocar curiosidad”} y que \textit{“no hay razón sin emoción”.}

\url{http://www.rtve.es/alacarta/videos/para-todos-la-2/para-todos-2-entrevista-francisco-mora-aprender/1840491/}

La labor del profesor es fundamental en este sentido. Es decir, tenemos que hacer que los alumnos se emocionen con aquello que explicamos, hay que ser capaces de motivarles, manteniendo su curiosidad. Tenemos que conseguir que vean el aprendizaje como algo divertido ya que cuando nos divertimos el cuerpo segrega dopamina que activa el circuito cerebral y provoca “ganas de más”. En el blog de escuela con cerebro me gustó mucho cómo un grupo de alumnos hicieron una animación sobre un problema de física (ver Figura \ref{playmovil}). Estoy segura que al realizar la animación, se divirtieron, segregaron dopamina, y consiguieron establecer una conexión entre las emociones que sintieron al realizar el juego de animación y el aprendizaje del problema de física, afianzando de forma más duradera los nuevos conocimientos de física adquiridos.



\begin{minipage}[hbtp]{1.0\linewidth}
\centering
\includegraphics[scale=0.8]{img/playmovil.jpg}
\captionof{figure}{Animación realizada por alumnos de secundaria sobre un problema de física}
\label{playmovil}
\end{minipage}


Otro artículo importante de este blog es el de “Motivación escolar: siete etapas clave”. Las etapas clave son las siguientes:

\begin{itemize}

\item Qué curioso: hay que provocar curiosidad. 

\item Esto me interesa: demostrar que lo que se enseña es útil. 

\item Acepto el reto: no puede ser algo muy sencillo porque provocará aburrimiento ni algo demasiado complicado porque provocaría desmotivación y es lo contrario a lo que se busca. 

\item Soy el protagonista: el alumno tiene que ser un participante activo en el aprendizaje para que se sienta importante en dicho proceso. 

\item Progreso: es importante que el alumno se sienta valorado por su esfuerzo y que se cree un clima emocional positivo en la clase. 

\item Esto vale la pena: promover la autoevaluación para que el alumno se dé cuenta que su progreso es efectivo y sirve para algo. 

\item Soy útil: promover el trabajo cooperativo de forma que se muestre que cada participante del grupo es esencial en la consecución del objetivo del grupo en su conjunto. 
\end{itemize}

\end{opin}


\subsection{Innovación y recursos educativos (Tema 3.1 - 24/10/2016)}
\begin{opin}{\guscolor}{Gustavo}

\subsubsection{Divulgación de las Matemáticas como docentes}

En la clase de hoy Raquel hizo referencia a un debate que puede generar polémica. El asunto es si las matemáticas son asequibles para todo el mundo o solo para elegidos como algunos piensan. Raquel es partidaria de que todo el mundo podría aprender matemáticas correctamente si estas se explican cómo deberían. En mi opinión, yo no afirmaría ni una cosa ni la otra al 100\%, es decir, no todo es blanco ni negro. Relacionándolo con las inteligencias múltiples que explicó Helena López Casares en la conferencia que dio el 20 de septiembre en el Campus Vicálvaro, me atrevería a decir que un profesor puede intentar despertar el interés de un alumno por las matemáticas, pero es cierto que dentro de todas las posibles inteligencias que pueden existir, si un alumno no tiene bien desarrollada la inteligencia relacionada con las matemáticas, un profesor podrá ser capaz de despertasr el interés de un alumno hasta un límite.

En cualquier caso, ya sea para difundir las matemáticas a todo el mundo o no, lo que está claro es que hay que intentar quitar ese estigma existente dentro del mundo matemático acerca de que la gente a la que le gustan las matemáticas son “bichos raros”. Para ello hay que difundir y divulgar las matemáticas y qué mejor manera que hacerlo de un tiempo a esta parte que a través de la prensa y de los medios de comunicación. 

El problema está en que los periodistas históricamente han huido de las matemáticas desde bien entrada la Universidad y por tanto existen muchos errores en prensa y televisión acerca de las matemáticas como se pueden ver en las transparencias de clase o en el video visto en clase de Marilo Montero (\url{https://www.youtube.com/watch?v=zclITKd4ivQ}). Este error tiene que ver con el error o fallo que puede existir a la hora de representar ciertas expresiones matemáticas como puede ser la expresión 48/2(9+3) y que ha generado un debate en los foros de la asignatura que muestro a continuación dado que  me pareció muy interesante y no tuvimos tiempo para abordarlo en clase

En mi opinión, inicialmente el resultado era 2, clarísimamente. Pero después de razonar con mis compañeros de clase vi que podría haber más soluciones aparte de la mia. David Soria explicó lo siguiente: 

\begin{center}\rule{200pt}{0.2pt}\end{center}

El problema tiene dos opciones distintas dependiendo del orden de operación. Hay dos elementos de distinto orden que son PEMDAS Y BEDMAS, según en la escuela que te hayan enseñado puede ser uno u otro. El orden para aplicar las operaciones en PEMDAS (Paréntesis, Eponenciación, Multiplicación, División, Adición y Sustracción) mientras que el orden en BEDMAS (Paréntesis,, Exponenciación, División, Multiplicación, Adición y Sustracción). 

Con PEMDAS el resultado de la operación sería item 

\[
48÷2·(9+3) \to 48 ÷ 2·(12) \to 48÷2·12 \to 48÷24 = 2
\]

Con BEMDAS  el resultado de la operación seria item 

\[
48÷2·(9+3) \to 48 ÷ 2·(12) \to 24·12 = 288
\]

\begin{center}\rule{200pt}{0.2pt}\end{center}


Antonio Jesus Guerrero y Carlos Rodiño entendían que “que multiplicar y dividir están al mismo nivel, igual que sumar y restar; y que en tal caso la prioridad de operación es de izquierda a derecha.”

Posteriormente, Manuel Pulido compartió su forma de pensar con Carlos y con Antonio,  y además corroboró la información con un libro de texto indicando que es la manera más extendida de hacerlo.

El libro de texto decía lo siguiente:

\textit{En general:}

\begin{itemize}
\item \textit{En operaciones con paréntesis, primero hay que realizar las que están entre paréntesis y luego las demás.}
\item \textit{En operaciones sin paréntesis, primero se efectúan las multiplicaciones y divisiones y luego, las sumas y las restas.}
\item \textit{En operaciones de igual prioridad, primero la de más a la izquierda.}
\end{itemize}

\textit{Por lo tanto ellos lo calcularían así:}

\[
48÷2·(9+3) \to 48 ÷ 2·(12) \to 24·12 = 288
\]



Por último, intenté llegar a una conclusión que es la importancia que tiene como colocar las expresiones matemáticas para no dar lugar a ambigüedades de este estilo.


Lo que queda claro es que unos entienden la expresión como $\frac{48}{2(9+3)}$ y otros como $\frac{48}{2}(9+3)$.

Si ambas expresiones se reflejaran así no daría lugar a ninguna confusión.

En cualquier caso no estoy de acuerdo en aplicar las reglas que indican ciertos libros de texto dado que si aplicamos estas reglas indicadas por el libro de texto al que hacía referencia Manual, la expresión 

Podría interpretarse como si fuera igual a 288, cuando todos tenemos claro que debe ser igual a item 

Para terminar el debate Miriam Expóstio indicó que 
\textit{Desde mi punto de vista, para que se entendiese que hay que dividir todo entre "2(9+3)" se tendría que poner todo ese término entre corchetes, no?}


Con lo que estoy totalmente de acuerdo. Y por tanto, al no haber corchetes, el problema lo tiene quien escribe la expresión al haber varias interpretaciones.
Las matemáticas deben ser exactas y no dar lugar a interpretaciones. Para interpretaciones ya están las leyes ;-)
Además de los errores cometidos por los periodistas de manera involuntaria como puede ser el de Mariló también están los errores cometidos intencionadamente con el objetivo de manipular a esa parte de la sociedad menos documentada. Ejemplos pueden ser el número de asistentes a una manifestación que varía en función de diversos intereses.
También hablamos acerca de los errores comunes que tienen los alumnos a la hora de realizar ciertas operaciones básicas en matemáticas y que independientemente de ser de letras o de ciencia, no se deberían cometer. Es como si los de ciencia dijeran que no saben escribir gramaticalmente bien porque son de ciencia. No tiene sentido. La cultura es independiente de ser de ciencias o de letras.
Otra de las formas de divulgar las matemáticas en televisión es a través de series. Por ejemplo:

\begin{itemize}
\item “Universo matemático” era una serie, producida en el año 2000, que constaba de 10 capítulos y que abordaba distintos temas relacionados con la matemática. La obra obtuvo en el año 2002 el Premio a la divulgación científica del Festival Internacional Científico de Pekín. 

Se puede ver en 
\href{http://www.rtve.es/alacarta/videos/universo-matematico/} {http://www.rtve.es/alacarta/videos/universo-matematico/} 
\item “Más por menos”. Esta serie consta de 13 programas emitidos de septiembre de 1996 a enero de 1997 y de noviembre de 2002 a enero de 2003 en el programa de Televisión Educativa de TVE-2 "La Aventura del Saber".

Se puede ver en 
\href{http://www.rtve.es/television/la-aventura-del-saber/documentales/mas-por-menos/} {http://www.rtve.es/television/la-aventura-del-saber/documentales/mas-por-menos/} 
\item “Orbita Laika” sección de Matemáticas por Raúl Ibañez.
\item Matemáticas invisibles:
\begin{itemize}
\item Curiosidad sobre el tamaño de las hojas DINA-n (
\href{http://www.sabercurioso.es/2008/11/05/por-que-una-hoja-de-papel-din-a4-tiene-el-tamano-que-tiene/}{http://www.sabercurioso.es/2008/11/05/por-que-una-hoja-de-papel-din-a4-tiene-el-tamano-que-tiene/}
)
\item Explicación del digito de control de un código de barras. Todo tiene su explicación
\end{itemize}
\end{itemize}



Como vemos, hay muchas formas de divulgar las matemáticas y nosotros como futuros docentes debemos ser parte fundamental en el futuro de esta divulgación, ya sean:

\begin{itemize}
\item Artículos en revistas  
\item Convocatorias de actividades relacionadas con las matemáticas (fotografía, concursos, juegos…)
\item Concesiones de premios (Medalla Fields) 
\item Iniciativas como dar publicidad a los congresos internacionales de matemáticas 
\item Promover jornadas de popularización y divulgación  (“olimpiadas matemáticas”) 
\item Museos (Museo Nacional de Ciencia y tecnología, MUNCYT, antiguo Cosmocaixa)) 
\item Páginas web divulgativas
\item Divulgamat (
\href{http://www.divulgamat.net/}{http://www.divulgamat.net/}
)
\end{itemize}

\begin{minipage}[hbtp]{1.0\linewidth}
	\centering
	\includegraphics[width=0.6\linewidth]{img/chistegus.png}
	%\captionof{figure}{Coche de mediadios del siglo pasado.}
\end{minipage}

\end{opin}






\begin{opin}{\victorcolor}{Víctor}

En mi diario incluía también una mención sobre el ejemplo $48÷2·(9+3)$ tratado por mi compañero Gustavo, pero lo voy a omitir por evitar repeticiones innecesarias.

\subsubsection{Docentes como divulgadores}

Me ha resultado muy novedosa la idea de que \textit{los profesores deberíamos ser los primeros divulgadores de las Matemáticas} (Antonio Durán). 
%
No sólo en el aula con los alumnos, sino también fuera de ella.
%
Hacen falta personas apasionadas por las Matemáticas, capaces de ayudar a la gente a romper el mito ``Es que soy de letras'' 
%
\footnote{Al igual que personas apasionadas por las letras que rompan con el mito ``Es que soy de ciencias''. Qué pasa, ¿que por ser de ciencias no saber redactar ni expresarte por escrito?}
%
Y a veces, ese argumento se utiliza para escaquearse de llevar las cuentas en un viaje o en una cena, cuyos cálculos no pasan de simples divisiones y sumas que un estudiante de 5º de Primaria podría hacer.

\subsubsection{Construyendo el conocimiento matemático sin lagunas}

En verano estuve de voluntariado en Perú y una semana fue dar clase de mates a chavales de secundaria de allí. 
%
Les ocurría que sabían despejar las ecuaciones perfectamente. Hacían todos los pasos bien, hasta que al llegar al final, $3x = 24 \to x=\frac{24}{3} = ?$. 
%
El último paso no eran capaces de hacerlo. ¿Cómo es posible que lleguen al curso en el que están y sepan despejar sin saberse las tablas de multiplicar? 
%
¡Qué sistema educativo tan ineficiente! 
%
Pasan de curso sin los conocimientos necesarios... construyen el conocimiento con unas lagunas bestiales.

Viendo los errores de primero de grado propuestos por Raquel me doy cuenta que no hay tanta diferencia entre el modelo de allí, con el modelo de aquí; 
%
en cuanto a construir un conocimiento sólido sin lagunas.


La Ted Talk de Salman Khan me parece una charla que todo docente de Matemáticas debería ver (tanto es así, que mi primera aportación a los foros de la asignatura fue plantear esta charla).
%
¿Te imaginas construir el tejado de un edificio sin haber terminado los cimientos?
%
Nadie trabaja así. Ni siquiera nadie, exceptuando a Fito y Fitipaldis, sugiere trabajar así.
%
¿Porqué en Matemáticas enseñamos a integrar sin que se haya interiorizado bien la derivada? ¿Porqué tratamos de enseñar diagonalizar matrices sin que los alumnos tengan claro cómo se factoriza un polinomio con Ruffini?
%
No digo que haya que enseñar Ruffini en la universidad, sino que los docentes en la secundaria nos esforcemos por no dejar lagunas en el conocimiento de los alumnos. 
%
``Enseñar para la maestría'', como dice Salman Khan.

\paragraph{Khan Academy}

Al hilo de retomar esta charla este curso (ya la conocía anteriormente) estuve paseando por la web y viendo los recursos que tienen.
%
Es una pena que esté en inglés y tal vez no sea utilizable en clase, pero ayuda a hacerse una idea de cómo funcionar. 
%
Además, están trabajando en traducciones, para hacer llegar la academia a más países.

Otro aprendizaje, al hilo de Khan, es la importancia del inglés. 
%
Más allá de poder comunicarse, en internet hay infinidad de recursos (empezando por las charlas Ted) que merecen mucho la pena y que se podrían estar aprovechando mucho más, si supiéramos inglés.
%
A raíz de esto, cuando me preguntan qué es lo que más valoro del inglés, lo que siempre contesto es: ``poder aprender autodidactamente de lo que sea en internet y acceder a reflexiones y conocimiento de otras personas''.


\end{opin}

\begin{opin}{\pedrocolor}{Pedro}

\subsubsection{La clase invertida - jornada complementaria}
\vspace{2.0cm}
\end{leftbar}
\vspace{-2.0cm}
\begin{figure}[hbtp]
\begin{leftbar}{\pedrocolor}
\vspace{1cm}
\centering
\includegraphics[scale=0.12]{img/pedron.jpg}
\caption{Comienzo de la ponencia.}
\vspace{2.0cm}
\end{leftbar}
\vspace{-2.0cm}
\end{figure}
\begin{leftbar}{\pedrocolor}

\textbf{Jonathan Bergmann y Aaron Sams}, dos profesores de química, acuñaron el término de “Flipped Classroom”. En un principio estaba orientado para aquellos alumnos que frecuentemente faltaban a clase por algún motivo personal. Con el tiempo se dieron cuenta que este mismo modelo permitía que el profesor centrara la atención en las necesidades individuales de aprendizaje de cada estudiante.

Para poder entender en qué consiste esta metodología activa de “Flipped Classroom”, Chema nos hizo llegar la siguiente definición de Raúl Santiago:

Modelo pedagógico que transfiere el trabajo de determinados procesos de aprendizaje  fuera del aula y utiliza el tiempo de clase, junto con la experiencia del docente, para facilitar y potenciar otros procesos de adquisición y práctica de conocimientos dentro del aula. 

En pocas palabras: “Lleva tu clase a cada estudiante, en cualquier momento y en cualquier lugar”

Pero la pregunta que surgió era, ¿Cómo se controla si tus alumnos hacen su trabajo fuera del aula? Chema habló que era fundamental explicar a los alumnos en qué consistía eso de la clase invertida y lo importante que es seguir unos patrones  para alcanzar el éxito que se traduce en “aprendizaje”. Según sus palabras, ellos tienen que saber que son los protagonistas de su propia película. Todos pensamos que estas palabras sobre el papel suenan bien, pero la verdad resultante en el aula es bien distinta, la prioridad de los alumnos no es aprender. Ante esto, nos presento algunas herramientas como \textbf{Playposit y EDpuzzle}:

\begin{itemize}

\item  \textbf{Playposit} El profesor puede incluir preguntas dentro de un vídeo (de origen propio o de YouTube). Lo novedoso es que el estudiante tiene que contestar primero las preguntas antes de poder continuar y sin poder avanzar a partes que todavía no ha visionado (si puede rebobinar y luego avanzar). Por otra parte el profesor puede ver en su tablero que estudiantes han realizado la tarea y los resultados. Material muy útil para utilizar en clase invertida. 

\item  \textbf{EDpuzzle} Permite convertir cualquier video en tu propia lección educativa de una forma rápida y fácil. De este modo podremos cortar un video si nos interesa solo una parte, grabar nuestra propia voz encima del video.  El resultado serán videos muy interesantes y entretenidos. 
\end{itemize}

Personalmente creo que son dos herramientas muy bien pensadas para educadores. Playposit me permite poder evaluar a mis alumnos de manera individual, identificando sus puntos débiles sobre un determinado temario.

 
A continuación, Chema nos habló de los 10  errores más comunes a la hora de crear una Flipped Classroom:

\begin{itemize}

\item Videos demasiado largos ( Un minuto por año que tenga el alumno aproximadamente) 

\item Añadir más que remplazar. Los videos deben remplazar el contenido del texto, no añadir más. 

\item Dar clase cuando los alumnos no atienden. Al principio será necesario ver los vídeos en el aula, ya que normalmente les costará cambiar de rutina. 

\item Elaborar contenido de difícil acceso. Utilizar solo un soporte para elegir los videos y materiales. Explicar donde está todo alojado y cuál es su funcionamiento. Hacerlo fácil. 

\item Inactividad de los profesores en el aula. El profesor no deberá estar cómodamente sentado durante la clase. Deberá resolver dudas de los alumnos. 

\item Rendirse. Si el método no funciona, cambia de método. 

\item No tener en cuenta los alumnos con dificultades. 

\item No hacer clases interactivas. 

\item Utilizar videos de otras personas. Hay que personalizar los videos para tus alumnos. 

\item No hacer las clases divertidas y amenas. 
\end{itemize}
Me he querido centrar en esta metodología porque opino que permite al alumno marcar su ritmo de aprendizaje. Nos alejamos de los patrones que marca el sistema educativo actual, que hace que unos alumnos vayan con la lengua fuera, mientras que otros se aburren en clase. Aquí cada alumno elige su momento de estudio.




\end{opin}

\begin{opin}{\virgicolor}{Virginia}

\subsubsection{Divulgación de las matemáticas como docentes}

En este tema vimos la importancia de la correcta divulgación de las matemáticas.  Recalco lo de correcta porque de nada sirve divulgar matemáticas si está no se hace bien y por desgracia en los medios de comunicación se cometen errores bastante importantes. Teniendo en cuenta que los principales medios de divulgación son tales medios de comunicación como, televisión, periódicos, radio e Internet creo que es de vital importancia evitarlos: ¿cómo podemos pretender que los niños se expresen bien o les exijamos unos conocimientos básicos en matemáticas si los propios periodistas cometen fallos tremendos? Uno de los ejemplos que puso Raquel y que me dejo boquiabierta fue el de Marilo con el índice de masa corporal, ya no por el hecho de que no sea capaz de usar una calculadora aun cuando le estaban explicando paso a paso como hacerlo, sino de la propia imagen que da de desinterés y despreocupación como si no pasara nada el ser incapaz de hacer algo tan sencillo como una potencia y una división.

En mi opinión, el problema con las matemáticas empieza desde pequeños en el colegio donde comienzan a percibir la asignatura como algo difícil y aburrido. Para ello la labor del profesor es de vital importancia desde las primeras etapas del aprendizaje, de forma que tiene que ser capaz de establecer una conexión entre las emociones de los alumnos y la utilidad de las matemáticas en la vida real. A día de hoy con los avances de la tecnología creo que es más sencillo ya que existen multitud de juegos matemáticos o mentales que se usan en tablets y spmartphones y que los niños estarían encantados de utilizar.

A parte de la labor del profesor, como he comentado antes, los medios de comunicación son la principal fuente de información por lo que la divulgación por parte de los mismos es imprescindible para que llegue a las distintas edades. Es cierto que existen cada vez más programas relacionados con la ciencia y las matemáticas como el Hormiguero, Orbita Laika y Desafía tu Mente. Si bien el hormiguero es un programa de éxito porque lleva famosos de distintos campos y tiene diferentes secciones el programa, los otros dos programas me temo que su audiencia dista mucho de lo que tendría que ser y más sin los comparamos con programas basura como Gran Hermano o Mujeres Hombres y viceversa.  Es una pena que la gente se una más para ver programas basura que un programa de divulgación. Siempre escucho que si lo ponen es porque la gente ve esos programas, pero también pienso que, si desaparecieran todos esos programas basura y se pusieran más programas de entretenimiento relacionados con la ciencia y su divulgación y en un horario más apto para niños, seguro que se crearían unas inquietudes diferentes a esos niños, tendrían más ganas de aprender cosas curiosas e interesantes que buscar la forma de ganar más dinero trabajando menos.

Dentro de los libros, videos y/o artículos de divulgación matemática que vimos en clase quiero destacar los siguientes:

\begin{itemize}

\item La sección de Raúl Ibánez sobre matemáticas en Orbita Laika ya que Raúl consigue ganar el premio más importante de divulgación científica en España. Además es el creador y director de una página muy interesante, Divulgamat: \url{http://www.divulgamat.net/} 

\item Universo matemático que fue galardonada en el Festival Internacional Científico de Pekín con el Premio a la divulgación científica. 

\item El libro de Mateschef ya que el libro trata de hacer relaciones curiosas entre geometría y cocina, y las matemáticas y la cocina son dos cosas que me encantan así, que mejor forma que unir las dos. En el campo de la cocina y teniendo en cuenta el nombre de este libro, recuerdo siempre como en el programa de MasterChef cuando hacen postres recalcan siempre que “la repostería es matemática pura”. 

\item Matemáticas invisibles: muy interesante el artículo de las plantas carnívoras: \url{http://www.agenciasinc.es/Noticias/Las-plantas-carnivoras-utilizan-las-matematicas-para-cazar-a-sus-presas} 

\item El mundo con mirada matemática: a parte de lo que vimos en clase me gustaría añadir una web en la que se incluyen edificios famosos inspirados en las matemáticas: \url{http://www.metrocuadrado.com/noticias/especiales/edificios-famosos-inspirados-en-las-matematicas-664} 
\end{itemize}


Además de todos los libros que vimos en clase de los distintos divulgadores y todos los programas de los que hablamos, me gustaría aportar otros programas que yo he visto y que me parece muy interesantes como son: Mythbusters (Cazadores de mitos), por el Discovery Channel, y Brain Games (Juegos mentales) por National Geographic. El primero, está más relacionado con la divulgación científica y consistía en comprobar la veracidad de las leyendas urbanas y otras creencias usando para ello métodos científicos. El segundo, en realidad podría haberlo incluido cuando se trató la neurodidáctica ya que en este programa se explora los componentes del cerebro humano y su funcionamiento empleando expertos en ciencia cognitiva, neurociencia y psicología. No obstante, son dos programas muy interesantes a la vez que entretenidos, por lo que habría que promocionar más este tipo de programas.

A parte de todos los divulgadores que comentó Raquel quiero hablar de Eduardo Sáenz de Cabezón, un matemático que intenta relacionar las matemáticas con la magia y el humor. Además, es uno de los participantes de un grupo de matemáticos, biológos, científicos... que hacen monólogos. Aquí muestro el link de dos noticias sobre él y las matemáticas:



%%%%%%%%%%%%%%%%%%%%% LINK MUY LARGO
%\textcolor{red}{Link muy largo}

\url{http://www.abc.es/sociedad/abci-eduardo-sanchez-cabezon-matematico-desvela-magia-numeros-201512180154_noticia.html}

En este artículo Eduardo reflexiona acerca de la tendencia de la gente a renegar de las matemáticas: “Quizá solo falta perderles el miedo, acercarse de forma diferente a la que estamos acostumbrados”.

\url{http://www.nacion.com/vivir/ciencia/hombre-saca-chistes-matematica_0_1565443465.html}

En el vídeo que sale en este artículo me parece divertido su consejo para hacer un regalo a alguien que le quieres demostrar que tu amor es para siempre: “Si quieres decir a alguien que le quieres para siempre, regálale un teorema en lugar de un diamante”. Para finalizar recalca que hay que cambiar la educación matemática actual.

Y por último la página del grupo de monologuistas al que pertenece y de las actividades que realizan que me parece una labor muy importante, interesante y divertida para dar a conocer los diferentes entresijos de la ciencia:

\url{http://www.bigvanscience.com/tbvt.html}

En definitiva, hay multitud de formas de disfrutar de la ciencia en general y las matemáticas en particular, y para hacer uso de dichas formas, el deber de padres, profesores y medios de comunicación es conseguir inquietar y emocionar las mentes de los niños, para lo que antes son los propios adultos los que tendrían que conseguir emocionarse.

\end{opin}


\subsection{Clase cancelada (31/10/2016)}

\subsection{Pizarras digitales (Tema 4 - 07/11/2016)}
%\begin{opin}{\guscolor}{Gustavo}


El objetivo que pretende Manuel es el de concienciarnos de la utilidad práctica que tiene el uso de las pizarras digitales en el aula a la hora de la enseñanza.

Manuel indica que las claves para que algo tenga éxito en educación son 3:

\begin{itemize}
\item[1.]Que sea muy fácil. 
\item[2.]Que sea adaptable. Valido para cualquier asignatura. 
\item[3.]Que sea útil pedagógicamente. 
\end{itemize}

\subsubsection{Hoy Soy Feliz Porque Tengo Ayuda}

\textbf{H}oy \textbf{S}oy \textbf{F}eliz porque tengo \textbf{A}yuda.

El camino para conseguir ese éxito tiene que ver con el título del apartado.

\paragraph{H de Hardware}
Son las herramientas. Pero el importante es el que transmite que es EL PROFESOR. Tipos de hardware:

\begin{itemize}
\item Pizarra Digital Interactiva (PDI) 

\item Pizarra Digital Interactiva Portátil (PDiP). La que usa el profesor en su clase. Se puede usar desde cualquier lugar. Más económica, más flexible y con mayores posibilidades. 

\item Articlick. EVCD: Evaluacion continua Digital. 

\item Cámaras de documentos. Es barata y permite ver documentos y hacer anotaciones sobre ellos. 

\item Micrófonos y altavoces. 
\end{itemize}

\paragraph{S de Software}
Hay muchos tipos de software.

Las características del software asociado a la pizarra que nos mostró el profesor incluía:

\begin{itemize}
\item Una barra de herramientas personalizable y flotante sobre cualquier aplicación.  

\item Se permiten hacer anotaciones. Todas las anotaciones se guardan automáticamente.  

\item Se puede borrar lo que se quiera.  

\item Se pueden exportar a cualquier formato.  

\item Se puede guardar un video con sonido de cualquier explicación previo a darle a guardar. 

\item El lápiz hace de ratón: 
\begin{itemize}

\item Boton izquierdo: Punta 

\item Doble click: Botón central 

\item Botón derecho: último botón. 
\end{itemize}
\end{itemize}

Una vez explicadas las capacidades de la pizarra, Manuel nos mostró una serie de ejemplos de aplicación de simulaciones de mucha utilidad entre las que destacaría la “Graphing  calculator 3D” por encima de todas las demás ya que siempre cuesta mucho explicar cómo se representan los planos en el espacio y esta herramienta es absolutamente fantástica. Otras aplicaciones que también me resultaron interesantes fueron:
\begin{itemize}
\item[1.]La aplicación que te permite hacer simulaciones de caída de objetos y un péndulo. 

\item[2.]RM Easiteach con la que explica el movimiento de la tierra con respecto al sol y de la luna con respecto a la tierra. 

\item[3.]Otra aplicación explica los colores primarios y su mezcla que produce los colores complementarios que son los de la impresora 
\end{itemize}

\paragraph{F de Formación}
El profesor tiene que tener un curso presencial motivador

Tiene que existir un curso online

\paragraph{A de Ayuda}
Tienen que existir tanto el apoyo como las ayudas al profesor. IMC

\index{EVCD}\textbf{EVCD}: Evaluacion continua Digital
Por último, hicimos una demostración en clase pasa a explicar la evaluación continua con el programa VPAD. Encontré esta aplicación superútil dado que se pueden hacer evaluaciones interactivas de mucha utilidad.

Además añadiría que no hace falta descargarsela desde el móvil ya que se puede utilizar desde cualquier cliente web.

Hay otros programas como Flow, Plickers y Kahoot en internet en los que también se pueden hacer evaluaciones interactivas.

\paragraph{Street View y realidad virtual de una casa}
Con Manuel también aprendí que con el Street View de Google se puede hacer una foto de una habitación de casa y luego verla con unas gafas de realidad virtual o directamente con un móvil con visión de 360º. Sencillamente impresionante


\end{opin}

\begin{opin}{\victorcolor}{Víctor}

Hoy hemos visto utilizar una \index{PDi(P)}\textbf{PDi(P)}\textbf{ - Pizarra Digital interactiva (Portátil)} y unas cuantas aplicaciones muy útiles para trabajar con tics en el aula.

Entre ellas: Graphing calculator 3D (para dibujar superficies e intersecciones de superficies).
%
Esta herramienta complementa muy bien a geogebra, puesto que los problemas de geometría de segundo de bachillerato son en tres dimensiones y geogebra no tiene opción a dibujar en 3 dimensiones.

Otra herramienta simulaba el comportamiento de objetos atraídos por la fuerza de la gravedad. 
%
Lo primero es dibujar los objetos sin gravedad. Después, al activar la gravedad los objetos se mueven y se puede ver el comportamiento de un péndulo, de un lanzamiento vertical y estudiar a qué altura llega... ¡Los enunciados de los problemas pueden ser vídeos!

A la hora de dibujar en matemáticas, estas herramientas son geniales. Ayudan a visualizar mucho mejor y sobretodo, ahorran mucho tiempo de hacer los dibujos.

La explicación sobre los colores primarios de 15 segundos es la mejor explicación que he escuchado en la vida. 
%
Tener el \textit{flash} sobre los colores primarios de la luz permite mezclaros en el momento y moverlos sin apenas esfuerzo. 
%
Al mezclarlos, se comprueba claramente que son los colores complementarios (los colores que utilizan las impresoras para conseguir todos los demás).

Ha sido muy constructiva esta sesión, aunque hubiera estado mejor poder probarlas, ya que escribir sin mirar a dónde estás escribiendo no me parece algo muy intuitivo. 
%
¿Cuánto entrenamiento hace falta para acostumbrarse?
%
¿Cómo es la curva de aprendizaje?


\end{opin}

\begin{opin}{\pedrocolor}{Pedro}


Esta sesión ha resultado realmente enriquecedora. Ha sido mi primera aproximación a las pizarras digitales gracias a Manuel García. Su gran capacidad de síntesis me permitió tener una visión general de los cuatro campos que cubren las pizarras digitales: Hardware, Software, Formación y Ayuda.

\begin{itemize}

\item \textbf{Hardware:}
\begin{itemize}

\item \textbf{PDIP} (Pizarra Digital Interactiva Portátil). Nos ha mostrado la facilidad de utilización en el aula. Permite poder impartir clase desde cualquier sitio de la sala, cedérsela a los alumnos, y con posibilidad de conectar entre si hasta 9,  simultáneamente repartidas por la clase. 
%
Se puede ver un ejemplo en la figura \ref{fig:pdip}.

 
\item \textbf{PDI} (Pizarra Digital Interactiva) pese a no tener una disponible en el aula, nos ha explicado que los recursos que aporta una buena utilización de la misma son infinitos. 

\item Equipo Visualizador Cámara de Documentos nace como una evolución del proyector de diapositivas. Permite digitalizar cualquier material y compartirlo en clase. En algunos casos, permiten hasta realidad aumentada con presentaciones 3D. 
Se puede ver un ejemplo en la figura \ref{fig:EVCD}.

\begin{minipage}[hbtp]{1.0\linewidth}
	\begin{tabular}{cc}
		\begin{minipage}[hbtp]{0.4\linewidth}
			\centering
			\includegraphics[scale=0.4]{img/pdipedro.jpg}
			\captionof{figure}{Pizarra Digital Interactiva Portátil}
			\label{fig:pdip}
		\end{minipage}&
		\begin{minipage}[hbtp]{0.4\linewidth}
			\centering
			\includegraphics[scale=0.4]{img/pdipedro2.jpg}
			\captionof{figure}{Equipo Visualizador Cámara de Documentos}
			\label{fig:EVCD}
		\end{minipage}
	\end{tabular}
\end{minipage}
\end{itemize}
 
\item \textbf{Software:}
\begin{itemize}

\item \textbf{INTERWRITE Workspace} : Ofrece una gran cantidad de herramientas que permiten una mayor interactividad en el aula. Incluye reconocimiento de formas, herramientas de edición, etc. Considero que puede ser una herramienta bastante intuitiva. 

\begin{minipage}[hbtp]{1.0\linewidth}
\vspace{0.3cm}
\centering
\includegraphics[scale=0.2]{img/pdipedro3.jpg}
\vspace{0.3cm}
\captionof{figure}{Captura del programa INTERWRITE.}
\end{minipage}

 
\item \textbf{Flow} : Permite realizar preguntas improvisadas en clase, y conseguir al instante un conocimiento exacto del nivel de compresión de cada alumno. 
Podemos también crear test y exámenes en cualquier formato (Word, Power-Point, Write, etc ). Dentro de la evaluación en el aula, cabe también destacar Plickers  y Kahoot.
        
\begin{minipage}[hbtp]{1.0\linewidth}
\centering
\includegraphics[scale=0.45]{img/pdipedro4.png}
\vspace{0.3cm}
%\captionof{figure}{...}
\end{minipage}

 
\item \textbf{Physics Illustrator} : Este software nos brinda la posibilidad de ver cómo la ley de la gravedad actúa sobre los objetos…un impacto, un desplazamiento, etc. 

 
\begin{minipage}[hbtp]{1.0\linewidth}
\vspace{0.3cm}
\centering
\includegraphics[scale=0.2]{img/pdipedro5.png}
\vspace{0.3cm}
%\captionof{figure}{...}
\end{minipage}

\item Existen otros tipos de software considerados como recursos a la hora de impartir clases, como:  \textbf{easiteach} y \textbf{Graphing Calculator}
 
\end{itemize}
\end{itemize}
\end{opin}

\begin{opin}{\virgicolor}{Virginia}
Este día vino Manuel García Vuelta a darnos una clase muy interesante y productiva sobre los recursos tecnológicos que se puede utilizar en el aula como son las pizarras interactivas. El hecho de ver en directo el propio uso de esos recursos me pareció vital para entender su gran aplicación y la riqueza de este material en el aprendizaje de las distintas asignaturas.

En primer lugar se planteó tres preguntas típicas:

\begin{itemize}

\item Herramientas útiles 

\item Claves del éxito 

\item ¿Cómo puedo formarme? 
\end{itemize}

La respuesta es sencilla: que sean fáciles, adaptables y que sea útiles pedagógicamente.

Asimismo, Manuel nos explicó las distintas etapas para llevar a cabo el uso de estas herramientas en el aula. Con una frase muy sencilla, nos resume la inicial de cada una de ellas: \textbf{Hoy Soy Feliz porque tengo Ayuda:}

\begin{itemize}

\item H, Hardware: se encuentran las pizarras digitales interactivas portátiles o no. Las primeras son más útiles por la comodidad que ofrece la portabilidad. Articlik, que se utiliza para que los alumnos puedan contestar las preguntas del profesor con un mando a través de respuestas opcionales A, B, C o D. También hablo de cámara documentos para enfocar y hacer anotaciones. 

\item S, Software: tiene que ser multiusuario, con formato universal, que sea compatible con todas las pizarras digitales interactivas y que haya licencia para todo el centro. 

\item F, Formación: Hay que realizar un curso presencial motivacional y luego la posibilidad de realizar un curso online para potenciar en cualquier momento lo que se está haciendo. 

\item A, Ayuda: que el profesor pueda contar siempre con ayuda por si le surgiera, por ejemplo, algún problema con el software o el hardware. 
\end{itemize}

Estas fases son muy importantes pero siempre es imprescindible un buen profesor que aproveche bien estos recursos tecnológicos.

Había oído hablar de las pizarras interactivas pero nunca las había visto en funcionamiento, lo cual me pareció fascinante. En todo momento estábamos todos atentos a lo que Manuel nos mostraba, lo cual indica el interés que suscita estos recursos en el aula. Manuel empleó el programa Workspace y junto con otros programas nos enseñó entre otras cosas el movimiento relativo de la Luna con respecto el Sol, y el uso del péndulo. Al ver esto, imaginaba a mi sobrina de 8 años en una clase de este tipo y conociéndola me llamaría contándome lo que había visto en clase y cómo se lo habían enseñado, y desde luego, así no se le olvidaría jamás los conceptos aprendidos ya que los aprende de una forma visual y mediante emociones, y no a base de memorizar o de realizar ejercicios mecánicos que hasta ahora es lo que yo he visto a través de sus deberes.

Otro recurso muy práctico que nos mostró fue cuando se hizo la simulación de un examen y respondimos a las preguntas a través de la aplicación vpad. Creo que sería muy útil incluir por ejemplo al final de cada tema un seguimiento de los conceptos adquiridos por los alumnos realizando un examen tipo test de forma que respondan a través de la aplicación de sus smartphones. Creo que por el hecho de contestar más rápido que sus compañeros o ver quien ha respondido más respuestas correctas a través de la aplicación estarían más motivados en las clases y prestarían más atención para luego realizar este tipo de examen, ya que el usar esta tecnología no lo verían tanto como un examen típico sino incluso como un juego tipo trivial.

Buscando información en internet sobre las pizarras interactivas y sobre Manuel, encontré una presentación interesante del profesor Antonio Solano en el que nos indica que las pizarras se tienen que ver como una ventana al mundo que proyecten y que se utilicen para compartir. Aquí dejo el enlace de la presentación:

 \url{http://es.slideshare.net/ppitufo/pizarra-digital}

Uno de mis problemas a la hora de imaginarme dando clase es ver en las caras de los alumnos su aburrimiento y desmotivación. Desde luego, el uso de estas tecnologías provocaría seguro un cambio de actitud en los alumnos y creo que no hay nada mejor para un profesor que tener alumnos motivados y con ganas de ir a su clase para aprender de una forma diferente y más eficaz. No obstante, y aunque me encantaría que pudiera aplicarse en todas las aulas me temo que con el sistema educativo actual que tenemos hay mucho trabajo por hacer para cambiarlo y conseguir un sistema más progresivo, moderno y que aproveche los recursos tecnológicos.

Desde luego no hay nada como ver en directo a Manuel para motivarte en la educación haciendo uso de las pizarras interactivas, por ello quiero finalizar mostrando este vídeo suyo en el que de nuevo motiva a los profesores al uso de estos recursos tecnológicos:

\url{https://www.youtube.com/watch?v=0x6utJxuw2g}


\end{opin}


\subsection{Innovación y recursos educativos (Tema 3.2 - 14/11/2016)}
\begin{opin}{\guscolor}{Gustavo}

\subsubsection{Recursos educativos en el aula}
Durante la clase de hoy, Raquel nos hizo ver la gran cantidad de recursos didácticos existentes hoy en día en el mundo de las matemáticas. Veo una ventaja muy evidente respecto a buscarlo por nuestra cuenta y es que sin duda Raquel, como docente con años de experiencia, nos ha filtrado los recursos para evitar perdernos en la red.

Esto nos va a ayudar en nuestro futuro profesional como docentes ya que será un espacio dónde poder apoyarnos. Entre otro de los consejos que nos dio Raquel está el de que tenemos que asumir que es difícil estar al día de todo por lo tanto no hay que abrumarse por tal situación.

Los recursos a destacar son:

\subsubsection{Recursos del INTEF}
El Instituto Nacional de Tecnologías Educativas y de Formación  (INTEF) del Profesorado es la unidad del Ministerio de Educación, Cultura y Deporte responsable de la integración de las TIC en las etapas educativas no universitarias. Tiene rango de Subdirección General integrada en la Dirección General de Evaluación y Cooperación Territorial que, a su vez, forma parte de la Secretaría de Estado de Educación, Formación Profesional y Universidades (Fuente: \url{http://educalab.es/intef/introduccion})

Los siguientes recursos se encuentran disponibles desde la web del INTEF \url{http://educalab.es/recursos}, aunque también son interesantes los recursos específicos de matemáticas encontrados en el histórico de recursos \url{http://educalab.es/web/web/recursos/historico/asignaturas/matematicas}

\paragraph{Procomún}
\url{https://procomun.educalab.es/}

PROCOMÚN es el espacio que más destaca dentro de los recursos del INTEF y se describe como una red de Recursos Educativos Abiertos. Este espacio destaca por la gran cantidad de recursos disponibles y por la posibilidad de búsqueda de dichos recursos por medio de los metadatos.

Dentro de Procomún se hizo mención al Proyecto Gauss.

\paragraph{eXeLearning}
\url{http://exelearning.net/}

eXeLearning es una herramienta de autor de código abierto para ayudar a los docentes en la creación y publicación de contenidos web. Facilita la creación de contenidos educativos sin necesidad de ser experto en HTML o XML. Se trata de una aplicación multiplataforma que nos permite la utilización de árboles de contenido, elementos multimedia, actividades interactivas de autoevaluación… facilitando la exportación del contenido generado a múltiples formatos: HTML, SCORM, IMS, etc.

Características destacadas de Exelearning:
\begin{itemize}
\item Permite crear un árbol de navegación básico que facilitará la navegación.  

\item Permite escribir texto y copiarlo desde otras aplicaciones.  

\item Permite incluir imágenes, pero no es un editor de imágenes como Photoshop o Gimp.  

\item Permite incluir sonidos, pero deben estar grabados previamente con otra aplicación.  

\item Permite incluir vídeos y animaciones, pero no permite crearlas.  

\item Permite incluir actividades sencillas: preguntas de tipo test, de verdadero/falso, de espacios en blanco...  

\item Permite embeber elementos multimedia como vídeos, presentaciones, textos o audios.  

\item Permite incluir actividades realizadas con otras aplicaciones 
\end{itemize}


\paragraph{Proyecto Gauss}
\url{http://recursostic.educacion.es/gauss/proc/}

El Proyecto Gauss ha sido desarrollado por el INTEF. Es un proyecto específico de matemáticas y ofrece a los profesores varios centenares de ítems didácticos y de applets de GeoGebra, que cubren todos los contenidos de matemáticas de Primaria y de Secundaria.

\paragraph{Proyecto Agrega/Agrega2}
\url{http://www.agrega2.es/web/}

Este proyecto es una federación de repositorios de contenidos educativos digitales donde todo el mundo pueda buscar, visualizar y descargar material educativo digital no universitario.  Se pretende facilitar a la comunidad educativa una herramienta útil para integrar las Tecnologías de la Información y la Comunicación en el aula

Existe una segunda versión de este proyecto que mejora la anterior y se llama Agrega2. Es de los repositorios más complicados para encontrar cosas. De hecho, hay un curso específico para buscar contenidos en Agrega2.

\paragraph{Red de Buenas PracTICas 2.0}
\url{http://recursostic.educacion.es/buenaspracticas20/web/}

Red de Buenas Practicas 20 es una red social de profesores dentro del INTEF.

\paragraph{Internet en el aula}
\url{http://internetaula.ning.com/}

Otra red social para docentes

\paragraph{Educa con TIC}
\url{http://www.educacontic.es/recursos-educativos}

Es un blog especializado en el uso de las TIC en las aulas

\paragraph{Proyecto Descartes}
\url{http://proyectodescartes.org/}

El Proyecto Descartes comienza su andadura en el año 1998. Lleva mucho tiempo activo, por tanto es lógico pensar que hay mucha documentación y muchos recursos.

Hay una página antigua en una web oficial del ministerio (\url{http://recursostic.educacion.es/descartes/web/}) pero está sin mantenimiento. Esta página antigua lo tenían hecho en JAVA y daba tantos problemas a los usuarios que decidieron cambiarla. Aun así, todavía hay muchos recursos que redirigen de la nueva a la antigua.

La web actual del proyecto Descartes pertenece a la Red Educativa Digital Descartes, que explicamos a continuación.

\paragraph{Red Educativa Digital Descartes}
La Red Educativa Digital Descartes (RED Descartes) es una asociación no gubernamental que se ha constituido el 1 de junio de 2013. Los socios fundadores son profesoras y profesores que tienen una historia conjunta construida, durante quince años, desarrollando proyectos del Ministerio de Educación español, entre los que podemos citar el Proyecto Descartes, Educación Digital a Distancia, Proyecto Canals, Pizarra Interactiva, Newton, Experimentación Didáctica en el Aula, WikididácTICa y Buenas Practicas 2.0. (Fuente: \url{http://www.educacontic.es/blog/matematicas-interactivas-con-descartes-en-tablets-y-smartphones})

En la parte de arriba de la web hay un apartado de subproyectos que te lleva a la página web \url{http://proyectodescartes.org/indexweb.php} . Entre estos subproyectos destacan:
\begin{itemize}
\item Telesecundaria (\url{http://proyectodescartes.org/Telesecundaria/}): educación a través de videos. Ojeando esta aplicación, también hay ejercicios y explicaciones interactivas hechas con HTML5, lo cual hace que sea accesible a través de cualquier navegador moderno. 

Telesecundaria es además una modalidad en el sistema educativo de México.

 

\item Proyecto Canals (\url{http://proyectodescartes.org/canals/index.htm}) de la profesora catalana Maria Antònia Canals para infantil y primaria. 

 

\item Proyecto "EDAD" Educación Digital con Descartes (\url{http://proyectodescartes.org/EDAD/index.htm}) surge con el propósito de desarrollar recursos educativos digitales interactivos, para la Educación Secundaria Obligatoria (ESO) en las áreas curriculares de Matemáticas, Ciencias Naturales y Física y Química, que permitan su uso tanto en la enseñanza presencial como en la formación a distancia. 

 

\item Proyecto ASIPISA (\url{http://proyectodescartes.org/ASIPISA/index.htm}). ASIPISA es una palabra palíndroma, acrónimo de “Ayuda Sistemática Interactiva para PISA”, que da nombre a un proyecto de desarrollo de materiales educativos, digitales e interactivos, basados en las unidades liberadas del Programa internacional PISA 

 

\item Proyecto Competencias (\url{http://proyectodescartes.org/competencias/index.htm}): Pensado para formar en competencias como marcan los nuevos planes de estudios. Esta web recoge objetos de aprendizaje interactivos cuyo objetivo es la formación y evaluación competencial. Sus contenidos se basan en las unidades liberadas de PISA, en las de las Pruebas de Evaluación de Diagnóstico de diferentes Comunidades autónomas españolas de acuerdo a la Ley Orgánica de Educación (LOE) de 2006 y a las pruebas de Evaluación de diagnóstico establecidas por la Ley Orgánica para la Mejora de la Calidad Educativa (LOMCE) de 2013. 
\end{itemize}

\paragraph{Educarex}
Es el Portal con contenidos educativos de la Comunidad Extremadura. Extremadura estaba a la cola en educación e hizo un esfuerzo bestial para ponerse a la altura del resto de España. Este portal es parte del fruto de dichos esfuerzos.

\paragraph{Otros recursos}
Además de todo lo visto hasta ahora también existen otros recursos como revistas, blogs, páginas de internet, etc. A continuación se citan algunos de estos recursos para su conocimiento:
\begin{itemize}
\item Aulaplaneta es un sistema integrado de contenidos curriculares que pone al servicio del profesor una propuesta didáctica personalizable y gran variedad de recursos digitales para preparar sus clases, y a disposición de los alumnos todo lo que necesitan para aprender de forma motivadora y eficaz. Destacar el artículo ” Diez canales educativos imprescindibles de YouTube para alumnos y profesores” \url{http://www.aulaplaneta.com/2015/10/27/recursos-tic/diez-canales-educativos-imprescindibles-de-youtube-para-alumnos-y-profesores/}  

 

\item MisMates y TutorMates. Se trata de dos proyectos digitales de Oxford destinados para el área de Matemáticas.   

MisMates es una aplicación educativa de acceso exclusivamente on line y está enfocada al alumnado de entre 1º y 4º de la ESO que tiene a su disposición varias áreas de trabajo, un editor de expresiones matemáticas y una libreta digital.

TutorMates es una aplicación de escritorio para Windows, iOS o Linux, y trabaja con herramientas específicas los contenidos de cada bloque

 

\item Educacion 3.0 es una revista de elevado interés en el mundo educativo en el que destaca el artículo “15 recursos de Internet imprescindibles para cualquier profesor” \url{http://www.educaciontrespuntocero.com/recursos/recursos-para-educacion-profesor-imprescindibles/35931.html}  

 

\item ScolarTIC es un proyecto de la Fundación Telefónica. Es una Comunidad Educativa de ámbito hispano. Es un espacio social de aprendizaje, innovación y calidad educativa en el que se ofrecen cursos online gratis, recursos para el aula así como charlas, ponencias y talleres.  

 

\item Y además a nivel particular hay muchas webs de profesores que ponen su trabajo al servicio de los demás: 
\begin{itemize}
\item Pilarleku@ (\url{http://pilarlekunew.blogspot.com.es/}) 

\item Domingo Mendez \url{http://domingomendez.es/} y su blog “Educación y TIC” \url{http://domingomendez.blogspot.com.es/}  

\item Algebra con Papas \url{https://www.edu.xunta.es/espazoAbalar/sites/espazoAbalar/files/datos/1291360755/contido/index.htm}  

\item Thatquiz \url{https://www.thatquiz.org/es/} Web con cuestionarios de matemáticas 

\item Manuel Sada Allo y su web Ejemplos diversos de webs interactivas de Matemáticas \url{http://docentes.educacion.navarra.es/msadaall/geogebra/}  

\item Sectormatematica \url{http://www.sectormatematica.cl/}  

\item Antonio Perez Sanz \url{http://platea.pntic.mec.es/}~aperez4/. Antonio presentó los programas de TVE de “Más por menos” y “Universo Matemático”. Actualmente (Diciembre de 2016) es responsable de divulgamat. 

\item “Amo las mates” actualmente en la web \url{https://www.matematicasonline.es/}  

\item Disfruta las matemáticas \url{http://www.disfrutalasmatematicas.com/}  

\item Vitutor \url{http://www.vitutor.com/} también se usa para la universidad. Tiene un contenido de bachillerato bastante potente. es una plataforma de teleformación diseñada para el aprendizaje en línea de distintas materias. 

El proyecto comenzó con la especialización en contenidos de Matemáticas, y estamos trabajando en otras materias, como inglés.

\item  “Aula21” que en su página \url{http://www.aula21.net/primera/matematicas.htm}  recopila un listado de enlaces a recursos de interés en el mundo de las matemáticas. 

\item Banco de recursos de SM \url{http://www.smconectados.com/Banco_de_recursos.html} donde encontrarás recursos para ayudarte a hacer más fácil tu trabajo en el aula. 
\end{itemize}
\end{itemize}

\paragraph{9 cosas que los profesores digitalmente competentes hacen habitualmente}


\vspace{2cm}
\end{leftbar}
\vspace{-2cm}
\begin{figure}[hbt]
	\begin{leftbar}{\guscolor}
		\centering
		\includegraphics[width=0.8\linewidth]{img/9cosasdegus.jpg}
		\caption{9 cosas que los profesores digitalmente competentes hacen habitualmente.}
	\vspace{2cm}
	\end{leftbar}
\vspace{-2cm}
\end{figure}

\begin{leftbar}{\guscolor}
\vspace{-1.8cm} 
\paragraph{Uso del video en educación}
No cabe duda que el uso del video en la clase es una metodología innovadora. Está claro que tiene muchas ventajas como por ejemplo la de romper con la monotonía de la clase, pero también puede haber inconvenientes.

Los tipos de videos educativos según \url{http://www.uclm.es/profesorado/ricardo/Video/2002_2003/sld003.htm} son:

\begin{itemize}
\item \textbf{Documentales:} muestran de manera ordenada información sobre un tema concreto. 

\item \textbf{Narrativos:} tienen una trama narrativa a través de la cual se van presentando las informaciones relevantes para los estudiantes.  

\item \textbf{Lección monoconceptual:} son vídeos de muy corta duración que se centran en presentar un concepto.  

\item \textbf{Lección temática:} son los clásicos vídeos didácticos que van presentando de manera sistemática y con una profundidad adecuada a los destinatarios los distintos apartados de un tema concreto .  

\item \textbf{Vídeos motivadores:} pretenden ante todo impactar, motivar, interesar a los espectadores, aunque para ello tengan que sacrificar la presentación sistemática de los contenidos y un cierto grado de rigor científico. 
\end{itemize}

Un video motivador para poner a los alumnos puede ser el video de Tadeo Jones \url{http://www.telecinco.es/tadeojones/descubre-con-tadeo/Tadeo_Jones-Descubre_con_Tadeo-Matematicas_2_1697355179.html}

Otro video que impresiona a la hora de demostrar como la perspectiva puede engañar a como nuestro ojo le pasa la información a nuestro cerebro es \url{https://www.youtube.com/watch?v=U9PZizBDBZw} en el que colocando una serie de velas en un suelo plano y la posición de la cámara el autor nos muestra como da la sensación de que se acaba formando un cubo en 3 dimensiones sobre el que es capaz de sentarse.

Otro video fascinante es el de las potencias de 10 \url{https://www.youtube.com/watch?v=fbCwkfrKuaw} en el que nos muestran un “zoom out” con 10 elevado a n veces para salir al espacio y un “zoom in” con $\rfrac{1}{10}^n$ para adentrarnos en el organismo de las personas. El zoom out es otra manera de explicar los Sistemas de Información Geográfica como puede ser el de Google Maps.

Recursos de videos educativos pueden ser:
\begin{itemize}
\item El canal derivando de Youtube que son videos de Eduardo Sáenz de Cabezón \url{https://www.youtube.com/channel/UCH-Z8ya93m7_RD02WsCSZYA} 

\item “La pizarra de Fonemato” (www.matematicasbachiller.com) que contiene videos explicativos con una característica muy peculiar y es que lo explica todo muy despacio con un tono de voz serio que a la vez puede resultar cómico. 

\item El portal MatematicasIES \url{http://matematicasies.com} creado por Daniel López Avellaneda, Licenciado en Ciencias Matemáticas por la Universidad de Granada y Profesor de Matemáticas y Coordinador TIC en el IES Mar Serena. 

\item El portal Educacion 3.0 visto anterormente tiene recursos para crear videos como profesores. \url{http://www.educaciontrespuntocero.com/experiencias/recursos-para-grabar-lecciones-en-video/33017.html}  

\item Unicoos que es un portal de videos gratuitos para las asignaturas de ciencias. Enfocado a estudiantes de Secundaria, Bachillerato y universitarios.  

El portal Unicoos de YouTube proporciona algo más de 600 vídeos gratuitos para las asignaturas de Matemáticas, Física y Química. \url{https://www.youtube.com/user/davidcpv}
\end{itemize}

\paragraph{Matemáticas recreativas}
“La matemática recreativa se concentra en la obtención de resultados con actividades lúdicas, y a difundir o divulgar de manera entretenida y divertida los conocimientos o ideas o problemas matemáticos. Es un concepto tan viejo como lo son los juegos en los que interviene la lógica o de algún modo el cálculo” (Fuente: Wikipedia)

Es importante que los alumnos lleguen a ver que todos los juegos tienen una explicación matemática detrás. Llegar a sorprenderles es algo que logra captar su atención. Si recordamos en la página www.divulgamat.net hay una sección llamada Sorpresas Matemáticas en la que podremos encontrar bajo el menú principal recursos de este tipo


 
Un video que \textbf{llega a sorprender} a los alumnos es el de “crear chocolate de la nada” \url{https://www.youtube.com/watch?v=Y13tSEyOqGs.} En el video se consigue partir el chocolate y luego volver a reconstruir dando la sensación de que sobra una onza de chocolate. En realidad no se crea chocolate, es un truco creado que trabaja con diferentes pendientes de la recta que son cercanas y se puede manipular para aparentar que vuelve a su estado natural cuando no es cierto. La explicación detallada está en este otro video \url{https://www.youtube.com/watch?v=eb2hCmc2xso}

 

\paragraph{Anamorfismo y futbol\\}
 

\begin{minipage}[h]{1\linewidth}
	\centering
	\includegraphics[width=0.7\linewidth]{img/anamorf1.jpg}
\end{minipage}
 
Otro ejemplo de matemáticas recreativas es el del anamorfismo. Consiste en deformar la imagen a través de efectos ópticos o a través de un procedimiento matemático con perspectivas. Uno de los artistas más destacados utilizando esta técnica es Julian Beever. “Julian es un artista británico que se dedica a dibujar con tiza. Ha creado dibujos de tiza en 3D en el pavimento utilizando un método llamado anamorfosis que crea una ilusión óptica. Sus dibujos en las calles desafían las leyes de la perspectiva. Ha logrado una técnica que le da un gran realismo a la imagen” (Fuente: Wikipedia).

\begin{minipage}[h]{1\linewidth}
	\centering
	\includegraphics[width=0.7\linewidth]{img/anamorf2.jpg}
	
\end{minipage}
 
Un ejemplo de anaformismo de un cubo de Rubik en video se puede ver en  \url{https://www.youtube.com/watch?v=ooY7Mf0JlNM.} En este video nos permiten incluso acceder a la imagen que permite hace dicho anaformismo. Se puede descargar de \url{https://docs.google.com/file/d/0B3gyYFZJgwKVZ24zWDV1VVB5Wms/edit.} De hecho, me he descargado la imagen.

\begin{minipage}[h]{1\linewidth}
	\centering
	\includegraphics[width=0.45\linewidth]{img/anamorf3.png}
\end{minipage}
 
Por último, nos preguntaremos que tiene que ver el futbol con el anamorfismo. Pues bien, una vez visto que a la técnica que permite crear esta ilusión óptica se le llama anamorfismo, decir que en el futbol, principalmente en los partidos de primera división aprovechando el angulo de proyección de la grabación de las cámaras de televisión “colocan” la publicidad pintada en el plano para producir un efecto en 3D.

\begin{minipage}[h]{1\linewidth}
	\centering
	\includegraphics[width=0.7\linewidth]{img/anamorf4.jpg}
\end{minipage}
 
\paragraph{El cine y la literatura como recursos didacticos}
Por último, hay una gran cantidad de literatura matemática. No por ser matemáticos debemos olvidar la literatura. De hecho la convivencia entre ambas es fundamental.

Los siguientes enlaces tienen un montón de recursos literarios matemáticos:
\begin{itemize}
\item \url{http://aulamatematica.com/libros/libros_recomendados.htm}  

\item \url{http://www.librosmaravillosos.com/} en el que hay un buscador de libros gratuitos de difusión científica. 
\end{itemize}
El cine también ha servido como fuente de inspiración para muchos directores y guionistas a la hora de difundir las matemáticas:
\begin{itemize}
\item Una mente maravillosa 

\item El código Da Vinci 

\item Black Jack 

\item La vida es bella 

\item Cube 

\item Agora 

\item Contact 

\item Blade Runner 

\item El día de la bestia 

\item Moebius 

\item 3:19 

\item Granujas de medio pelo 
\end{itemize}
O como el caso de la “Jungla de Cristal 2” donde los protagonistas tienen que resolver el problema de las garrafas de 3 y 5 galones de agua. Para desactivar la bomba tienen que conseguir 4 galones exactos en una de las garrafas. ¿Cómo lo conseguirán? Aunque en la película no se explica claramente, se consigue (Fuente: \url{http://www.sociedadmatematicacantabria.es/Probl_Olimpiada/Sol_probl_3_2.htm}):
\begin{itemize}
\item 1º Llenas la de 5 y echas lo que puedas en la de tres.  

Quedan 3L en la de 3 y 2L en la de 5

\item 2º Vacías la de 3 y echas los 2L de la de 5 en la de 3.  

Quedan: 2L en la de 3 y 0L en la de 5

\item 3º LLenas la de 5 y echas lo que puedas en la de tres (1L)  

Quedan: 3L en la de 3 y 4L en la de 5

\item 4º Vacías la de 3 y ya tienes 4L en la de 5 
\end{itemize}

\paragraph{Khan Academy \url{https://es.khanacademy.org}}
Por último, no quería dejar sin mencionar en el portafolios la aportación realizada en el foro  por mi compañero de grupo Victor De Juan sobre la “Khan Academy”.

Khan Academy es una web que ofrece ejercicios de práctica, videos instructivos y un panel de aprendizaje personalizado que permite a los alumnos aprender a su propio ritmo, dentro y fuera del salón de clases. Y todo ello sin pagar ni un duro.

En este video de la Universidad Politécnica de Valencia nos explican cómo comenzar a usar esta web tanto si somos alumnos, profesores o padres. \url{https://www.youtube.com/watch?v=FvacPlqEw6g}

\end{opin}

\begin{opin}{\victorcolor}{Víctor}

Hoy ha sido un bombardeo de recursos para utilizar en clase. Paginas web, blogs, educalab... 
%
A día de hoy se me queda lejano porque no he profundizado sobre los recursos, porque creo que pueden caer en saco roto.
%
Parece que voy a tener que estudiarme y bucear por todos estos recursos para descubrir cuáles me gustan más, cuáles me parecen más útiles, etc.
%
Y este trabajo tendrá pleno sentido cuando tenga delante un verano en el que prepararme las clases del año siguiente y busque, más concretamente, como tratar algún contenido o me quiera apoyar en alguno de estos recursos para un tema específico que el año anterior no funcionó en clase la manera en la que lo hice.

Son tantísimos recursos que ahora no he podido profundizar en ellos, pero que me los he descargado todos para poder trabajar sobre ellos más adelante.

\subsubsection{Gamificación}

Este es un tema que no hemos llegado a ver en clase, pero sí en la charla de Javier de las jornadas complementarias.
%
Partiendo de la base de que \textit{La gamificación es como el ketchup. Hay veces que es genial, pero no soluciona absolutamente todos los problemas.} (Javier Espinosa), es un recurso muy muy interesante que, por como soy, creo que me encatará aplicar en el aula.
%
Tanto es así, que he elegido mi TFM a raíz de esto, para ser un docente gamificador. 
%
Creo que tengo mucho potencial en esta línea y espero aprender mucho con mi TFM.

\end{opin}

\begin{opin}{\pedrocolor}{Pedro}

Se presentan durante la clase varios recursos educativos, que permiten una mejor transmisión de conocimientos profesor-alumno. Cabe destacar la labor educativa del \index{INTEF} \textbf{INTEF (Instituto Nacional de Tecnologías Educativas y de Formación del Profesorado)}, encargado de la introducción de las TIC en las etapas educativas no universitarias. Destaca por:

\begin{itemize}

\item Su gran variedad de material de apoyo al profesorado, destinado a la continua actualización científica y didáctica del profesorado.
\item Elaboración y difusión de materiales en soporte digital y audiovisual, que permita hacer de las TICs un instrumento de trabajo cotidiano para el profesorado.
\item Mantenimiento del Portal de recursos educativos del Departamento y por la  creación de redes sociales que permita el intercambio de experiencias y recursos entre el profesorado.
\end{itemize}
Importante tener en cuenta que la web de INTEF ha sido sustituida por \url{http://educalab.es/home} dentro de la cual encontramos todos los recursos que pasamos a enumerar.


\begin{itemize}

\item \textbf{ Procomún:} Red de recursos educativos que permite compartir ideas y experiencias. Todos estos REAs están estructurados por niveles y materias. La variedad de recursos es múltiple y variada, desde fotografías, videos, problemas, ilustración, etc. Podemos participar en comunidades, redes sociales educativas, etc. 

\begin{minipage}[hbtp]{1.0\linewidth}
\centering
\includegraphics[scale=0.36]{img/intefpedro1.png}
%\captionof{figure}{...}
\end{minipage}

 
\item \textbf{ eXeLearning:} Considero que es una herramienta útil. Me permite, sin tener grandes conocimientos sobre lenguajes de programación, crear actividades de verdadero-falso, ejercicios de elección múltiple, actividades desplegables de relleno de huecos, etc. 
%
En la figura \ref{fig:exelearningPedro} encontramos un ejemplo.
\end{itemize}

\vspace{2cm}
\end{leftbar}
\vspace{-2cm}
\begin{figure}[hbtp]
\begin{leftbar}{\pedrocolor}
\centering
\includegraphics[scale=0.24]{img/intefpedro2.png}
\caption{Ejemplo de eXeLearning.}
\label{fig:exelearningPedro}
\vspace{2cm}
\end{leftbar}
\vspace{-2cm}
\end{figure}

\begin{leftbar}{\pedrocolor}
En el apartado de Documentación de \url{http://exelearning.net/category/documentacion/} encontramos desde el manual, hasta la versión portable del software.

\begin{itemize}


\item \textbf{Histórico de recursos:}  Permite el acceso a una gran cantidad de recursos educativos para la comunidad docente. Todos ellos ordenados por cursos, materias y unidades.  

\item \textbf{Proyecto Gauss:} Dentro de mis expectativas educativas, considero que es el recurso más interesante por centrarse en la asignatura de matemáticas. Brinda al profesorado varios centenares de ítems didácticos y de applets de GeoGebra (se pueden insertar en Geogebra), que cubren todos los contenidos de matemáticas de Primaria y de Secundaria. 

\item \textbf{EDA (Experiencia Didáctica en el Aula):} Proyecto que pretende mostrar al profesorado las ventajas e inconvenientes de utilizar estas nuevas tecnologías en el aula. 

\item \textbf{Agrega:} Repositorio de contenidos educativos. Su principal problema es la dificultad de buscar contenidos. Este buscador ha encontrado sustituto en Agrega2, permitiendo buscar unidades didácticas completas, recursos desagregados para la construcción de nuevos recursos, etc. 

\item \textbf{Red de Buenas Prácticas 2.0:} Red social de profesores que pone en común sus intereses. Destacar, dentro de la pestaña Inicio -> Recursos Educativos. Nos habla desde cómo crear un blog hasta como formarte en didáctica de TIC. 

\item \textbf{Proyecto Descartes:} Herramienta de trabajo para aquellos profesores que desean crear lecciones interactivas en el formato de páginas Web. 

\begin{itemize}

\item \textbf{Web inicial del Ministerio} -> \url{http://recursostic.educacion.es/descartes/web/} se encuentra ya sin mantenimiento. 

\item \textbf{Web \url{proyectodescartes.org}} -> Pertenece a una organización no gubernamental. 
\end{itemize}
\end{itemize}
 
Otros recursos educativos que merecen mención por su contenido son: Educarex y Educación 3.0. Esta última muy recomendable por su propósito de contribuir al cambio metodológico en las aulas a través de las TIC y de las metodologías activas.

Pese a todas estas web que recogen un contenido abrumador de recursos, no hay que subestimar el uso de videos educativos en el aula. Lo importante es la selección de los mismos:

\begin{itemize}

\item Contenido adecuado a la edad. 
\item Vocabulario cuidado. 
\item Claridad de la idea que se quiere transmitir. 
\item Duración media, que no llegue a cansar al alumno. 
 
\end{itemize} 
 
 
 
 
A continuación enumero los audiovisuales que más me han llamado la atención:
 

\begin{itemize} 

\item \textbf{La cuna de Halicarnaso} de José Antonio Lucero. Es sobre Historia, pero considero que el formato podría aplicarse perfectamente a la asignatura de matemáticas. Llegué a él gracias a Chema Lázaro, en su jornada de Flipped Classroom (Clase invertida). 
\item \textbf{Divermates }
\item \textbf{Descubre con Tadeo }
\item \textbf{Blog Mateomolivares} \url{http://matemolivares.blogia.com/}
\item \textbf{Fonemato} Videos de matemáticas de forma amena. Me aporta gran utilidad a la hora de ver como se podría impartir una clase, desde conceptos hasta la didáctica utilizada. 
\end{itemize} 
 
El profesor tiene que ser capaz de utilizar todo este material de una manera óptima, para captar la atención del alumno. Debemos ser capaces de asombrar, ofrecer desafíos y permitir que sean capaces de buscar estrategias de resolución. Por lo tanto, se hace necesario  saber \textit{dónde se encuentra la información, seleccionar la oportuna y dar el uso adecuado a la misma}.


\end{opin}

\begin{opin}{\virgicolor}{Virginia}


\subsubsection{Recursos educativos}

En primer lugar me gustaría destacar la frase que recalca Raquel en el tema: la labor del profesor no es demostrar todo lo que sabe sino saber cómo transmitir esos conocimientos. De nada sirve un ser un genio en una materia si luego nadie es capaz de entender lo que dice. Afortunadamente, hoy en día los profesores tienen la suerte de contar con multitud de recursos educativos para facilitar esa transmisión de conocimientos.

Me pareció una gran idea lo de la escuela 2.0 por el hecho de facilitar el aprendizaje de los alumnos a través de una herramienta básica de la era digital como son los ordenadores. A través de internet tenemos acceso a información ilimitada que puede ser de gran utilidad para mejorar el aprendizaje. Sin embargo, debido a que la inversión era muy alta, la idea se abandonó. Creo que habría que reconsiderar en que se invierte en este país ya que buenas inversiones a corto plazo pueden dar grandes resultados a largo plazo. Como bien dice el lema, sin educación no hay ciencia y sin ciencia no ha sanidad lo que muestra lo indispensable que es la educación para la buena marcha de un país con lo cual habría que reconsiderar y plantearse cómo mejorar el actual sistema educativo.

Cuando Raquel empezó a mostrar la cantidad de páginas webs que había con recursos tecnológicos para realizar las clases de una forma diferente, me quede bastante impresionada. También me animo a sentir la forma de enseñar con otra perspectiva. Reconozco que me da un poco de respeto el convertirme en profesora y caer en la rutina de la típica clase magistral en la que la mitad de los alumnos no atienden y la otra mitad no comprenden lo que se les cuenta. Todos estos recursos me hacen ver la educación de una forma más divertida y motivadora.

En primer lugar hablar del INTEF, Instituto Nacional de Tecnologías Educativas y de Formación a través del cual se llega a multitud de páginas con recursos educativos. Entre ellas llegamos a la siguiente página web \url{http://educalab.es/recursos} donde vimos un montón de recursos como procomún o exelearning aunque de especial interés me parece el histórico de recursos donde puedes ver aquellos que te interesen según la asignatura a impartir. Desde luego hay muchos interesantes, entre ellos el de laboratorio básico de azar, probabilidad y combinatoria donde las aplicaciones interactivas hacen más fácil la compresión del temario.

Dos recursos más conocidos e igualmente fundamentales son el proyecto descartes y el proyecto Gauss:

\begin{itemize}

\item \url{http://proyectodescartes.org/descartescms/} 

\item \url{http://recursostic.educacion.es/gauss/proc/} 
\end{itemize}

Las herramientas que tienen ambos proyectos es inmensa, creo que es muy interesante para cualquier profesor porque básicamente cualquier tema que se imaginen lo pueden encontrar en sus páginas webs. Dentro del proyecto Descartes me quedo con el proyecto EDAD ya que además de las matemáticas, por mi formación me gustaría poder dar clase también de física y química y precisamente el proyecto EDAD desarrolla recursos educativos digitales interactivos, para la educación secundaria obligatoria en las áreas curriculares de Matemáticas, Ciencias Naturales y Física y Química. En cuanto a la página del proyecto Gauss me gustó entre otras la aplicación de “La escalera de bomberos” para calcular las propiedades de los ángulos y la del “Tamaño relativo” para comprobar la exactitud de lo que vemos.

Otra de las páginas que me gusto fue la de educación de la comunidad autónoma de Extremadura donde por ejemplo para bachillerato hay un recurso interesante para el tema de los números complejos: \url{http://conteni2.educarex.es/mats/120077/contenido/}

Navegando un poco por Internet en las páginas que vimos en clase además de muchas otras que he descubierto buscando recursos educativos, me he dado cuenta de lo importante de 4 palabras que comenta Raquel al final de este apartado: Buscar-Analizar-Escoger-Adaptar. Esto es, son muchísimos los recursos de los que disponemos y por ello es necesario buscar aquellos que más nos interesen, analizarlos detenidamente para ver cuál de ellos es más útil en función de los alumnos que tengamos y de la unidad didáctica donde se quieran aplicar esos recursos, para finalmente escoger uno de ellos y adaptarlo en función de las necesidades educativas.

En este contexto, quiero poner el link de un blog que habla sobre los recursos tecnológicos e indica que para que dichos recursos sirvan de algo es necesario modificar las prácticas, es decir, el uso de tecnologías en educación implica nuevos planteamientos:
\url{http://matematicaeninicial5.blogspot.com.es/2010/01/los-recursos-tecnologicos-que.html}

\subsubsection{Uso del vídeo en educación}
También vimos este día el uso del vídeo en la educación que sirve para romper la monotonía de la clase pero hay que seguir una serie de pautas importantes en la elección del mismo ya que una elección o uso inadecuado del video podría ser más perjudicial que beneficioso. De la siguiente página web he recogido una imagen que resumen las funciones educativas del vídeo:
\url{http://miuras.inf.um.es/~oele/objetos/funciones_del_vdeo_en_la_educacin.html}


\begin{minipage}[hbtp]{1.0\linewidth}
	\centering
	\includegraphics[width=0.85\linewidth]{img/videoeducacion.jpg}
	\captionof{figure}{Funciones educativas del vídeo}
	\label{videoeducacion}
\end{minipage}

demás, Salman Khan muestra la importancia del uso del vídeo para reinventar la educación: \url{https://www.ted.com/talks/salman_khan_let_s_use_video_to_reinvent_education?language=es}

Me pareció muy motivador el video de Tadeo Jones y bastante divertido la parodia del 3x2 de José Mota. Vimos diferentes páginas donde ver multitud de videos relacionados con las matemáticas. Anteriormente yo hablé de Eduardo Saénz de Cabezón porque vi videos suyos bastante interesantes y curiosamente en este punto Raquel nos presenta también videos suyos a través del Canal Derivando. En este canal me resultó muy curioso el vídeo sobre el espesor de una página de papel de 0.01 mm de grosor tras doblarla 54 veces:

\url{https://www.youtube.com/channel/UCH-Z8ya93m7_RD02WsCSZYA}

Buscando más videos de él encontré una presentación muy interesante de cerca de 1 hora en el que relaciona las matemáticas con los juegos, el viaje y el amor:

\url{https://www.youtube.com/watch?v=4zjQNPlOpaI}

Muy útiles y muchos recursos audiovisuales aparecen en las páginas de Fonemato y Únicoos entre otros. Este último especialmente interesante porque hay recursos también para física y química, lo que me está aportando muchas ideas a la hora de dar yo clase ya que física y química es también una de las asignaturas que me va a tocar dar durante mi periodo de prácticas además de las matemáticas.

\subsubsection{Matemáticas recreativas}

Las matemáticas recreativas utilizan actividades lúdicas para transmitir los conocimientos matemáticos de una forma diferente y más divertida. Hay multitud de recursos para aplicar esta matemática recreativa en el aula como son acertijos, cuentos, juegos, chistes, jeroglíficos, trucos de magia… Al final como ya se ha ido viendo a lo largo de la asignatura el objetivo es llegar al alumno de una forma diferente a la tradicional conectando el aprendizaje con las emociones, haciendo que les guste lo que aprendan y tengan una mayor motivación por seguir ampliando sus conocimientos.
A continuación indicó algunos de los recursos de los que vimos en clase:

\begin{itemize}

\item www.divulgamat.net: en especial el apartado de sorpresas matemáticas donde se incluyen entre otros acertijos, chistes… Un ejemplo de los que vi es el siguiente acertijo: 
El director de un instituto, el día que comenzó el curso 2002-2003, reunió a todos los alumnos en el estupendo salón de actos y les dijo:
\begin{enumerate}
\item En el instituto hay 250 alumnos y 250 casilleros.
\item En estos momentos están todos cerrados.
\item El alumno nº 1 abrirá todos.
\item El alumno nº 2 cerrará todos los casilleros pares.
\item El alumno nº 3 cambiará el estado de los casilleros 3,6,9,12,... Es decir, el que esté abierto lo cierra y el que esté cerrado lo abre.
\item El alumno nº 4 cambiará el estado de los casilleros 4,8,12,16,...
\item El alumno nº 5 cambiará el estado de los casilleros 5,10,15,20,...
\end{enumerate}
Y así sucesivamente hasta el alumno nº 250.
Después de este entretenido comienzo de curso, ¿cuantos casilleros quedarán abiertos?
\item \url{http://www.eduardoochoa.com/joomla/}: El saco del dinero que desmiente una típica cadena de whatsapp, el mítico de los 5 hijos… 

\item \url{http://matemolivares.blogia.com/temas/matematicas-y-humor.php.} En concreto la sección de matemáticas y humor, que mejor que aprender riendo. 

\item Anamorfosis matemática: aparte de los que vimos en clase quiero hablar de Jonty Hurwitz que aplica este concepto a la escultura (ver Figura \ref{anamorfisVir}: Anamorfosis matemática aplicada en la escultura (Jonty Hurwitz)).  \url{http://culturacolectiva.com/jonty-hurwitz-esculturas-matematicas/} 
\end{itemize}


\vspace{2cm}
\end{leftbar}
\vspace{-2cm}

\begin{figure}[hbtp]
	\begin{leftbar}{\virgicolor}
	\centering
	\includegraphics[width=0.6\linewidth]{img/viryin.jpg}
	\captionof{figure}{Anamorfosis matemática aplicada en la escultura (Jonty Hurwitz)}
	\label{anamorfisVir}

\vspace{2cm}
\end{leftbar}
\vspace{-2cm}

\end{figure}
\begin{leftbar}{\virgicolor}

\subsubsection{El cine y la literatura como recursos didacticos}
Por último, se habló del cine y la literatura relacionada con las matemáticas. Creo que tanto la lengua y literatura como las matemáticas son dos asignaturas imprescindibles por lo que hay que inculcar su importancia desde edades tempranas. Una buena forma de ver la utilidad de las matemáticas es leer libros interesantes que apliquen de una u otra forma algún tema matemático, así, además de crearles pasión por las matemáticas estamos además inculcando la importancia de la lectura (\url{http://www.aulamatematica.com/libros/libros_recomendados.htm}). Me parece un recurso vital que aporta grandes beneficios en el futuro del estudiante en dos de las principales asignaturas.

Por otro lado, como el cine es algo que suele gustar mucho a los jóvenes, que mejor forma de emocionarlos poniendo películas de moda que apliquen matemáticas de forma amena y divertida. Raquel nos puso el video de La jungla de cristal que recordaba gratamente ya que el día que la vi recuerdo de estar entusiasmada con mi familia realizando el acertijo para ver quien acertaba primero, y la motivación que me supuso acertarlo rápidamente.

Entre las películas con trasfondo matemático las que más me han gustado son: una mente maravillosa, El código Da Vinci, Black Jack y Agora.

\end{opin}




\chapter{Trabajo de Innovación educativa}


\section{Introducción}


\subsection{Elección y Justificación}

Matemáticas es una asignatura que a bastante porcentaje del alumnado se le hace difícil. 
%
Hay algunos que optan por otros itinerarios sólo para evitar las Matemáticas, porque ya les parecen imposibles.
%
Hay otros que eligen Matemáticas "fáciles" (Matemáticas aplicadas a las Ciencias Sociales) porque no quieren ni Física ni Latín.

Sin saber con seguridad qué porcentaje del alumnado de esas "Matemáticas fáciles" la ha escogido por gusto o por ser la opción \textit{menos mala}, creemos que un alto porcentaje del alumnado de 1 de Bachillerato de Sociales no le gustan las Matemáticas y además pueden pensar que se les da realmente mal. 
%
Es por ello que queríamos hacer el trabajo para, de alguna manera, solucionar este problema que se da en el Bachillerato de Sociales.

Aunque una consigna de este trabajo era explicar un contenido didáctico de algún curso, pidiendo permiso, le hemos dado otro enfoque. 
%
Queríamos preparar la primera clase del curso de Matemáticas aplicadas a las Ciencias Sociales y queríamos hacerlo para motivarles y ayudarles a automotivarse.

A continuación, procedemos a definir más concretamente los objetivos.


\subsection{Objetivos}

\paragraph{Objetivo general}
\begin{itemize}
	\item Motivar al alumnado de Matemáticas para intentar romper la preconcepción que puedan tener sobre las Matemáticas. 
\end{itemize}

\paragraph{Objetivos específicos}
\begin{itemize}
	\item Explicar el concepto de la Indefensión Aprendida\footnote{Definida en \ref{defn::indefension}.}. Puede ser que algún alumno haya \textit{aprendido la indefensión} hacia las Matemáticas. El primer paso para romper esa indefensión es conocer que existe y que es aprendida.
	\item Introducir el curso de Matemáticas y motivar a los alumnos.
	\subitem Hacer hincapié en que estas no son las Matemáticas fáciles, sino las útiles. \footnote{Esto puede contribuir a que no se sientan "más tontos" (en las Matemáticas) que sus sus compañeros de las Matemáticas Académicas, ya que sus Matemáticas (CCSS) no son más fáciles, sino que tienen otro objetivo y son más útiles.}
	\item Mostrar el potencial de las Matemáticas aplicadas a las Ciencias Sociales para hacer más fuerte la motivación de los alumnos hacia la asignatura y complementar el objetivo anterior.
\end{itemize}

\section{Estructura}

Hemos estructurado el trabajo en torno a los objetivos específicos. 
%
Primero, tratamos la indefensión aprendida y lo haremos mediante un experimento, para que lo vean de primera mano.
%
Una vez interiorizado el concepto de indefensión aprendida, procederemos a introducir brevemente algunos aspectos del curso, recalcando la importancia de la automotivación.
%
Esta introducción se basará en material audiovisual para captar mejor la atención de los alumnos.
%
Por último, realizaremos un cálculo del número π por el método de Montecarlo. 
%
Este método se basa fundamentalmente en la Estadística y en la Probabilidad, temario específico de Matemáticas Aplicadas a las Ciencias Sociales.
%
Este cálculo se realizará en 2 partes. Una primera de cálculo manual, con material manipulativo (granos de arroz) y una segunda parte de cálculo simulado por ordenador.
%
El cálculo por ordenador es fundamental para que los alumnos puedan ver que realmente se obtiene el número π (ya que el método manual tiene algunos problemas, que trataremos en \ref{pimanual})

\subsection{Indefensión Aprendida}
\label{defn::indefension}

Por motivos de claridad de este escrito, exponemos primero el concepto y después el experimento, aunque después, en la exposición, primero realizaremos el experimento y después procederemos a explicar el concepto.

\subsubsection{Concepto}

Lo primero de todo es definir el fenómeno:
\begin{defn}[Indefensión Aprendida]
Condición de un ser humano o animal que ha aprendido a comportarse pasivamente, con la sensación subjetiva de no poder hacer nada y que no responde a pesar de que existen oportunidades reales de cambiar la situación aversiva, evitando las circunstancias desagradables o mediante la obtención de recompensas positivas.
\end{defn}

Creemos que este fenómeno se da en las Matemáticas de la siguiente manera:
%
Existe una creencia generalizada de que las Matemáticas son difíciles y que por mucho que me esfuerce, este pensamiento permea en mi y me lleva a comportarme pasivamente, motivado por la sensación de no poder hacer nada.
%
Otra posibilidad (un ejemplo más claro y fuerte de la indefensión) es: si yo he estudiado con mucho esfuerzo Matemáticas y no la he conseguido aprobar, aprenderé que no puedo sacar las Matemáticas.
%
Hay personas a las que les puede pasar esto porque realmente su inteligencia Logico-Matemática no sea muy alta, pero creemos que hay muchos que, con una inteligencia Logico-Matemática más que suficiente, se estrellan con las Matemáticas, ¿por qué? 
%
Porque a lo largo de su vida de estudiante, pueden haber aprendido la indefensión hacia las Matemáticas. 
%
Y esta indefensión se puede haber aprendido porque un año (o más) han tenido profesores más exigentes o que no han sabido transmitir los conocimientos y además en casa han recibido comentarios del tipo "Pues será que no vales para las Matemáticas".
%
Estos 2 factores creemos que son muy frecuentes que se den juntos o separados y que induzcan a los estudiantes una indefensión que se puede desaprender. Y para desaprenderla, el primer paso es conocer que existe.

Esta última conclusión, nos lleva al experimento para transmitir la Indefensión aprendida.

\subsubsection{Experimento}

Se dividirá a la clase en 2 grupos sin que ellos lo sepan.
%
Las llamaremos la mitad \textit{control} y la mitad \textit{experimental}.

Desde su perspectiva, la actividad a realizar será, dada una palabra, formar otra palabra en singular con las mismas letras.
%
Por ejemplo: dada la palabra \textit{\textsc{roma}} tendrán que escribir la palabra \textit{\textsc{amor}}. 
%
Cuando lo hayan conseguido, levantarán la mano y esperarán a que se pueda empezar la siguiente ronda.

El papel que juegan las 2 mitades es que no todos los alumnos van a recibir la misma palabra con la que trabajar. 
%
La mitad \textit{control} recibirá siempre palabras con las que se pueda hacer otra palabra, mientras que la mitad \textit{experimental} recibirá palabras con las que sea imposible escribir otra palabra con esas letras.

El desarrollo del experimento es el siguiente:

\begin{itemize}
	\item[0.-] Cada alumno tiene en su mesa 3 sobres de colores: uno rojo, otro amarillo y otro verde.
	\subitem Hay 2 tipos de sobres: los del grupo control y los del grupo experimental que habrán sido previamente diferenciados.
	\item[1.-] Cuando el profesor diga, abrirán el primer sobre, extraerán el papel con la palabra escrita y tendrán que escribir una nueva palabra con las mismas letras. Cuando lo consigan, les pediremos que levanten la mano y la mantengan levantada.
	\item[2.-] Una vez terminado el primer sobre (y haciendo ver a los que no lo han resuelto que hay muchos compañeros que sí), procederemos con el segundo sobre de la misma manera.
	\item[3.-] De la misma manera, procederemos al último sobre. 
	%
	En este último sobre el grupo control y el grupo experimental tiene exactamente la misma palabra. 
	%
	Es de esperar que los miembros del grupo control realicen con mayor rapidez el tercer sobre, mientras que los miembros del grupo experimental puedan incluso llegar a no ser capaces.
\end{itemize}

Una vez desarrollado el experimento, tendremos un rato de comentar la experiencia. 
%
Pediremos perdón por la pequeña "manipulación", ya que no todos tenían las mismas palabras y preguntaremos al grupo experimental cómo se han sentido.

Como decíamos anteriormente, una vez terminado el experimento en clase, procederemos a explicar el concepto, cómo se detecta y daremos algunas herramientas para ponerle solución. 
%
Para más información sobre estos puntos, consultar la presentación de la exposición.

\subsection{Motivación}


\subsection{Método Montecarlo para calcular π}
\label{pimanual}

\subsubsection{Simulación}



\chapter{Valoración de otros trabajos}

\section*{Grupo 1 - Cluedo matemático (21/11/2016)}

El primer grupo está formado por:
\begin{itemize}
\item Ana Reyes Camacho 

\item Antonio Jesús Guerrero Lobato 

\item Clara Rocío Lambas Magron 

\item Manuel Pulido Lopez 
\end{itemize}
\begin{figure}[h]
\centering
\includegraphics{img/cluedo1.jpg}
\end{figure}
La exposición estaba contextualizada en una clase para alumnos de 1º de la ESO de un IES el último día de clase antes de las vacaciones de Navidad. Durante la clase se tratará la unidad didáctica de la divisibilidad de los números primos y compuestos según aparece en el BOCM. Esta unidad didáctica suele generar mucha confusión entre los alumnos de la ESO y el grupo expositor considera que puede ser de gran interés para el alumnado.

Durante la clase se realizará un juego en el que se formarán varios equipos. Se trata de responder a una serie de preguntas en las que se premiará la rapidez en las respuestas siempre y cuando sean correctas. Por un lado el juego permitirá captar la atención de los alumnos. Por otro, el trabajo en equipo fomenta los valores de colaboración con los compañeros y el hecho de que sean preguntas cronometradas permitirá a las alumnas adaptarse a condiciones de trabajo de ciertas “presión”.

El juego se titula “Cluedo matemático”. El enunciado del juego dice así:

¡Esta noche el Señor Cero ha sido hallado asesinado en su mansión! Los detectives privados Simple y Compuesto han encontrado 99 sospechosos, pero no han podido resolver el caso. ¡Así que ahora la resolución del crimen depende de vosotros! Para ganar el juego debéis averiguar una sola cosa sobre el crimen”

Para la preparación del juego se:
\begin{itemize}
\item[1.] Coloca el tablero, hay un Hall donde empieza la acción y donde se coloca el peón, y 5 habitaciones dónde buscar pistas.

\item[2.] Prepara las tarjetas de preguntas, un tema para cada habitación.

 \item[3.] Prepara las tarjetas de pistas para descubrir al asesino, un pack de pistas por cada equipo.

\item[4.] Organiza a los alumnos en grupo, nombrando un portavoz
\end{itemize}

 \begin{figure}[h]
 \centering
\includegraphics[scale=0.6]{img/cluedo2.jpg}
\end{figure}

Las reglas del juego son:
\begin{itemize}
\item[1.] CÓMO GANAR: ¡Resuelve el crimen!  

\item[2.] CÓMO JUGAR: En cada turno se mueve el peón a una habitación contigua, se le hace a todos los grupos la misma pregunta, se dejan 120'' para calcular la respuesta y si contestan correctamente se les da una pista sobre el asesino.

\begin{figure}[h]
\centering
\includegraphics[scale=0.6]{img/cluedo3.jpg}
\end{figure}

\end{itemize}

 
La función principal del juego consistía en hacer preguntas por grupos y había que averiguar el número comprendido entre el 1 y el 99 a partir de una serie de pistas que se iban consiguiendo.

La exposición del grupo estuvo muy bien estructurada y demostraron que hicieron un gran trabajo de equipo.

Durante la misma explicaron que desestimaron la opción de tirar un dado, pero nosotros sí lo habríamos incluido incluyendo la posibilidad de que hubiese un rebote al tener un cronometro. Pero la opción del rebote también la descartaron.

La dificultad de las preguntas era variable por lo que es un juego muy versátil y de alto valor añadido para el aprendizaje.

Nos sorprendieron con algo novedoso al proponer analizar y evaluar el juego al final de la clase, lo que les permitirá que hubiese una mejora continua con los años. Para esta evaluación pidieron a los alumnos que contestaran a una serie de preguntas sobre lo que les ha parecido. Aunque sabemos que es difícil discernir cuando la exposición la realizas en el contexto de alumnos de secundario o en el contexto nuestro de alumnos de master, bajo nuestro punto de vista quizá sea demasiado pronto pedir la opinión a un niño de 1º de la ESO su opinión.

La exposición se ha ajustado perfectamente a los tiempos programados.

\newpage
\section*{Grupo 2 - Construyendo las matemáticas (21/11/2016)}


El segundo grupo está formado por:
\begin{itemize}
\item Helena Matesanz Marín 

\item Cristina Martínez Gonzalez 

\item Eliseo Virseda Alvaro 

\item Noemi Castillo Cumplido 

\item David Soria Castro 
\end{itemize}

\begin{figure}[hbtp]
\centering
\includegraphics[scale=1]{img/lego1.jpg}
\caption{Integrantes del grupo 2.}
\end{figure}

 
La exposición consistió en explicar con un material manipulativo conocido por los alumnos, que son las piezas de LEGO, los siguientes conceptos matemáticos que en el papel pueden parecer abstractos. Los conceptos matemáticos mostrados fueron:
\begin{itemize}
\item Fracciones 

\item Estadística 

\item Máximo común divisor y mínimo común múltiplo 

\item Ecuaciones lineales simples. Despejar la x. 

\item Teorema de Pitágoras 
\end{itemize}

\begin{figure}[hbtp]
\centering
\includegraphics[scale=1]{img/lego2.jpg}
\caption{Materiales manipulativos.}
\end{figure}
 
En nuestra opinión el uso de LEGO para varios conceptos matemáticos hizo que algunos de estos temas estuvieran un poco forzados. Es decir, para el concepto de fracciones las piezas de LEGO fueron de gran utilidad al permitir de una manera visual entender el concepto de lo que es una fracción como una parte de un todo. Sin embargo consideramos que explicar el concepto de M.c.d y m.c.m con piezas de LEGO quedó un poco forzado.

Por hacer una crítica constructiva, entendemos que era un grupo de 5 alumnos, es decir, de más alumnos que el resto de grupos de clase, pero quizá hubiésemos enfocado el trabajo en uno o dos conceptos matemáticos y habríamos trabajado un par de alumnos por cada concepto para evitar explicaciones de conceptos forzadas.

La exposición estuvo muy bien estructurada con una demostración conceptual por cada alumno pero se fueron de tiempo. Es cierto que se han explicado muchos conceptos pero los alumnos no hemos colaborado con lo que se puede perder eficacia al no hacer partícipes a los alumnos. Pero también entendemos que todos tenían que hablar y que eran temas muy interesantes.

Por último, la exposición también nos sirvió para aprender que a día de hoy hay un software en una web de internet (\url{http://www.publishyourdesign.com/}) que te permite realizar diseños de LEGO en 3D con el ordenador para luego poder incluso imprimirlos con impresoras 3D.

\newpage
\section*{Grupo 4 - La Oca Matemática (28/11/2016)}

Integrantes del grupo:

\begin{itemize}

\item Óscar Abelda
\item Miriam Expósito
\item Beatriz Mate
\item Pablo Saiz
\end{itemize}

\begin{minipage}[hbtp]{1.0\linewidth}
\centering
\includegraphics[scale=0.2]{img/grupo4_1.jpg}
\captionof{figure}{Material para cada grupo}.
\end{minipage}

Se presentó un juego clásico rediseñado para su aplicación en el campo de las matemáticas de 2º de la ESO. Desde un principio han sido capaces de meternos en el contexto de la situación en la cual sería aplicable la actividad. Está orientada a ese último día de clase donde podemos repasar contenidos de forma lúdica, sin que resulte aburrido para el alumnado. 

Nos sorprendió gratamente el gran trabajo realizado a la hora de elaborar el material manipulativo. La actividad contenía las instrucciones del juego, dado, tablero, tarjetas de contenidos, fichas, etc. Estábamos ante un juego de sobremesa con todo tipo de detalles.

Al poco tiempo de comenzar el juego, nos dimos cuenta de que pese a nuestra edad, poco a poco se iba incrementando la competitividad entre nosotros. No parábamos de comprobar si la pareja opuesta estaba resolviendo bien las preguntas, tiempo empleado, cuáles eran sus materias fuertes, etc. En todo momento contamos con un supervisor del grupo autor del juego, que nos guió durante la actividad.


\newpage
\section*{Grupo 5 - Pasapalabra-Pitágoras}

El grupo está formado por:

\begin{itemize}
\item Noelia Díaz Concepción
\item Borja González Blasco
\item Raquel Navas Sánchez
\item Carlos Rodiño Lourenco
\end{itemize}

Este grupo ha realizado su presentación mediante un recurso educativo de los que vimos en clase, el video, lo cual nos pareció muy interesante ya que de esta forma mostraron la utilidad de dicho recurso. 

Su trabajo de innovación se centra en una clase de primero de la E.S.O, en concreto, en la explicación del teorema de Pitágoras. En primer lugar definen las partes del teorema a través del juego de televisión de Pasapalabra. Después de esta parte introductoria, su trabajo se centra en la realización de tres tareas: la primera de ellas tiene como objetivo demostrar el teorema y las otras dos son para aplicar los conocimientos aprendidos del teorema. A continuación se muestra cada una de ellas.

\begin{itemize}

\item Actividad 1: Demostración que el cuadrado de la hipotenusa es igual a la suma del cuadrado de los catetos. Para ello se necesitan pajitas, tijeras, celo y garbanzos. 

\begin{figure}[hbtp]
\centering
\includegraphics{img/grupo5_1.jpg}
\caption{Actividad 1}
\end{figure}



\item Actividad 2: Aplicación del teorema para obtener el camino que tiene que seguir el conejo: uso de calculadora. El ganador de esta actividad recibe unas monedas de chocolate

\begin{figure}[hbtp]
\centering
\includegraphics{img/grupo5_2.jpg}
\caption{Actividad 2}
\end{figure}


\item Actividad 3: Aplicación del teorema en un edificio: sin usar la calculadora. El ganador de esta actividad recibe una tableta de chocolate. 

\begin{figure}[hbtp]
\centering
\includegraphics{img/grupo5_3.jpg}
\caption{Actividad 3}
\end{figure}

\end{itemize}

Nos ha parecido una actividad muy completa ya que utiliza diferentes recursos: vídeo, matemáticas recreativas mediante los juegos y manipulativa mediante el uso de las pajitas y garbanzos para demostrar el teorema. El hecho de introducir un premio después de las actividades de aplicación nos ha parecido una idea interesante y que además ninguno habíamos usado. Nos parece una gran forma de motivar e ilusionar a los estudiantes para que hagan la actividad de la mejor forma y lo más rápido posible, fomentando así mismo una competitividad sana. Nos ha gustado mucho el uso del vídeo y muy práctico. En general, nos ha parecido un gran trabajo y bastante divertido.


\newpage
\section*{Grupo 6 - Triangulo de Sierpinski}

El grupo 6 está formado por:

\begin{itemize}

\item Maria Rosa Calvo Arrojo 

\item Victor Fernandez Alcala 

\item Emilio Jose Hernandez Rodriguez 

\item José Miguel Hernando Santacruz 
\end{itemize}

El trabajo de innovación de este grupo se centra en una clase de tercero de la E.S.O. en el apartado de números y algebra, más concretamente en la explicación del tema de sucesiones. Su trabajo se divide en 4 partes y cada componente del grupo explica una de ellas. En primer lugar, nos muestran diferentes tipos de verdura para estudiar como hay una unidad que se repite, que es lo que se conoce como fractales. Su idea es que los alumnos al ver las verduras averigüen qué forma se está repitiendo en cada caso. Una vez averiguado se explica un poco la historia sobre este apartado para después comenzar con la explicación concreta del triángulo de Sierpinski.


\begin{figure}[hbtp]
\centering
\includegraphics[scale=1.2]{img/grupo6_1.jpg}
\end{figure}


En esta segunda parte, proporcionan material manipulativo (goma eva, cartulinas) para que se trabaje con él y se entienda mejor el concepto de repeticiones que están explicando. A continuación se muestra el resultado de las distintas actividades realizadas:



\begin{figure}[h]
\centering
\includegraphics{img/grupo6_2.jpg}
\caption{Triángulo de Sierpinski construido con materiales manipulativos.}
\end{figure}

\begin{figure}[h]
\centering
\includegraphics{img/grupo6_3.jpg}
\includegraphics{img/grupo6_4.jpg}
\end{figure}



Finalmente y una vez trabajado con el material proporcionado, se realiza la explicación en sí de las sucesiones.

Lo que más nos ha gustado de este trabajo es la posibilidad de trabajar con material manipulativo, siendo especialmente chulo e interesante las formas que se obtienen con el trabajo sobre las cartulinas. Quizás se ha echado en falta más material para que todos los alumnos puedan trabajar y realizar esas formas ya que la mejor forma de entusiasmar a los alumnos y hacer que estos estén pendientes de la explicación es que todos puedan trabajar con sus propias manos en lugar de observar. Por otro lado nos ha parecido muy interesante el hecho de realizar una conexión entre el tema matemático a tratar y la vida real, a través de la demostración de la forma de crecimiento de las verduras. Finalmente, les ha faltado incluir una hoja de valoraciones en su trabajo para que sepan las opiniones de los distintos grupos y las posibles mejoras que podrían aplicar en su trabajo de innovación.


\newpage
\section*{Grupo 7 - Clase de repaso 4º ESO B (académicas)}

El séptimo grupo está formado por:

\begin{itemize}
\item Lourdes Fuentes
\item Pedro Juan Valle
\item Daniel Camarero
\item David García
\end{itemize}


\begin{figure}[h]
\centering
\includegraphics[scale=0.20]{img/grupo7_1.jpg}
%\caption{.}
\end{figure}

La exposición del Grupo 7 se ha basado en la metodología de “aula invertida” y un ejemplo de gamificación mediante el programa Kahoot. 

En referencia a flipped classroom, nos han explicado el procedimiento a seguir en esta metodología. El profesor sube a la red un video sobre un tema en concreto, y los alumnos visionan el video en su casa. De esta manera, las clases en el aula estarán orientadas a solucionar las dudas de los alumnos sobre la materia. Para que lo viéramos con ejemplos, los integrantes se han grabado explicando temas como representación de funciones, trigonometría, etc. En esta parte, hemos echado en falta, que nos explicaran algunos ejemplos sobre los métodos mediante los cuales el profesor se asegura que el alumno ha visionado el video antes de ir a clase. Por ejemplo, nos hubiera gustado que los videos se interrumpieras con algún tipo de pregunta que fuera necesaria responder para seguir viendo la lección o que hubieran demostrado que no se puede avanzar en la reproducción del mismo.

La segunda parte ha estado orientada a la simulación de un proceso de evaluación mediante kahoot. Esta parte nos ha gustado, porque las preguntas estaban muy bien planteadas y creemos que potenciará la competitividad entre alumnos. Lo mejor de esta aplicación está en que no solo premia si has respondido correctamente, además tu puntación es mayor si eres más rápido que tus compañeros. 


\begin{figure}[hbt]
\centering
\includegraphics[scale=0.11]{img/grupo7_2.jpg}
\caption{Pregunta hecha con Kahoot sobre proporcionalidad.}
\end{figure}


Por nuestra parte, opinamos que todas estas metodologías deben ser puestas en práctica durante un período, con la finalidad de que los alumnos se vayan acostumbrando al ritmo de aprendizaje que marcan estas técnicas. Si pasado un tiempo vemos que no funcionan, podríamos ir modificando aspectos como el formato del video, la duración, utilizar otros recursos, etc.








%% Apendices (ejercicios, examenes)
%\appendix

%\chapter{---}% -*- root: ../Innovacion.tex -*-

\section{Conglomerado de recursos}

\begin{itemize}
	\item
	Estudios internacionales de Evaluación
	http://www.mecd.gob.es/inee/publicaciones/estudios-internacionales.html 
	\item 
	Instituto NAcional de Tecnologias Eduativas y Formación del profesorado 
	http://educalab.es/intef
	\item 
	Seamos gansos
	https://www.youtube.com/watch?v=K5G8gRvx7nQ
	\item 
	9 gestos cotidianos que los matemáticos hacemos de otra manera:
	http://verne.elpais.com/verne/2015/11/13/articulo/1447413460_147289.html?id_externo_rsoc=FB_CM
	\item 
	La escuela en 2030
	http://www.elmundo.es/espana/2014/10/21/54455b9f22601d22738b458e.html
	\item 
	El cerebro necesita emocionase para aprender
	http://economia.elpais.com/economia/2016/07/17/actualidad/1468776267_359871.html
	\item 
	La neuroeducación demuestra que emocion y conocimiento van juntos
	http://blogs.elpais.com/ayuda-al-estudiante/2013/12/la-neuroeducacion-demuestra-que-emocion-y-conocimiento-van-juntos.html
	\item 
	mates y neurociencia
	https://escuelaconcerebro.wordpress.com/2012/03/20/matematicas-y-neurociencia/
	\item 
	actividad cerebral del alumno durante la clase magistral
	http://ined21.com/actividad-cerebral-del-alumno-durante-la-tradicional-clase-magistral/
	\item 
	Aprendizaje cooperativo y neuroeducación: guiando la poda sináptica
	https://escuelaconcerebro.wordpress.com/2016/08/18/aprendizaje-cooperativo-y-neuroeducacion-guiando-la-poda-sinaptica/
	\item 
	blog de escuela con cerebro
	https://escuelaconcerebro.wordpress.com/
	\item
	Blog de Anna Fores
	https://annafores.wordpress.com/category/neurodidactica/
	\item 
	Neuroeducación por otra escuela
	https://www.youtube.com/watch?v=QiRqCKUiRDc&feature=youtu.be
	\item 
	Qué son las neurociencias:
	https://www.youtube.com/watch?v=kotroYR8p5A&feature=youtu.be
	\item 
	¿Cómo puede mejorar la neurociencia el aprendizaje? ¿y las nuevas tecnologías? 
	http://www.rtve.es/alacarta/videos/tres14/tres14-aprendizaje/1001759/
	\item 
	blog de Clara Grima (matemática, y antigua colaboradora en Orbita Laika)
	http://claragrima.com/
	\item 
	Salman Khan: Usemos el video para reinventar la educación
	https://www.ted.com/talks/salman_khan_let_s_use_video_to_reinvent_education?language=es
	\item 
	Tecnologías Educativas. Herramientas: Khan Academy web © UPV
	https://www.youtube.com/watch?v=FvacPlqEw6g
	\item 
	Construyendo una escuela en la nube
	https://www.ted.com/talks/sugata_mitra_build_a_school_in_the_cloud?language=es
	\item 
	Video juegos y educación
	http://ares.cnice.mec.es/informes/02/documentos/indice.htm
	\item 
	¿video juego para aprender Álgebra?
	http://one.elpais.com/jean-baptiste-huynh-te-muestra-el-videojuego-con-el-que-puedes-aprender-algebra-en-un-par-de-horas/
	\item
	TFM del proyecto Gauss
	https://repositorio.unican.es/xmlui/bitstream/handle/10902/1748/Ruiz%20Sar%C3%A1chaga%20Alexandra.pdf?sequence=1
	\item 
	Tecnologías Educativas. Herramientas: Vídeo, Youtube uso educativo @ UPV
	https://www.youtube.com/watch?v=2w9Yq76tFCg
	\item 
	Artículo El video en el aula
	http://www.lmi.ub.es/te/any93/ferres_cp/
	\item 
	Matemáticas de cine
	http://www.ceice.gva.es/web/innovacion-calidad
	\item
	talleres y juegos (infantil, primaria y secundaria)
	http://blogsaverroes.juntadeandalucia.es/
	\item 
	Gamificación: jugar para aprender
	https://educacioncuatropuntocero.wordpress.com/2015/02/05/gamificacion-jugar-para-aprender/
	\item 
	Inteligencias múltiples
	http://educarlasinteligenciasmultiples.blogspot.com.es/
	\item 
	PDI en mates
	http://tic.cardenalcisneros.es/2014/04/21/la-pizarra-digital-interactiva-en-matematicas/
	\item
	Web de Jose Dulac (PDI)
	http://www.dulac.es/
	\item 
	Multiclass
	http://www.multiclass.com/recursos-educativos
	\item 
	web de Pere Marqués (PDI)
	http://peremarques.net/
	\item 
	web de Artigraf (con formacion para el profesor)
	http://www.artigraf.com/default.php
	\item 
	tdah y tu
	http://www.tdahytu.es/
	\item
	Las vidas de Mario (estupendo video)
	http://www.lasvidasdemario.com/
	\item 
	Federacion Española de asociaciones de TDAH
	http://www.feaadah.org/es/
	\item 
	Fundacion CADAH
	http://www.fundacioncadah.org/web/
	\item
	canal youtube Educacion Activa TDAH URL
	https://www.youtube.com/user/Canal

\end{itemize}


\printindex
\end{document}
