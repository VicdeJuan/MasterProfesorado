\documentclass[palatino]{apuntes}

\title{Innovacion}
\author{}
\date{16/17 C1}

% Paquetes adicionales

% --------------------

\begin{document}
\pagestyle{plain}
\maketitle

\tableofcontents
\newpage
% Contenido.

\paragraph{Primera entrada del portafolio: ¿Qué es innovar en educación?}

Hacer las cosas de otra manera. "Si siempre haces lo mismo, no esperes resultados distintos." decía Einstein y para mi innovar es hacer de manera distinta teniendo como objetivo resultados distintos.

Mis expectativas para la asignatura son fundamentalmente ejercitar mi capacidad de innovar de cara a dar las clases. 

%Aprendido en clase

Otro aspecto importante es que sea un cambio duradero con mucha adoptación.

\begin{defn}[Educación\IS bulímica]
Estudio para vomitar en el examen.
\end{defn}


Informe ______:
Quien rechaza las mates se hace, no nace. En primaria nadie rechaza las mates, en Secundaria sí.

Todas las personas reconocen que las matemáticas son realmente útiles, quienes se les dan bien cómo quienes no se les dan bien.

A quien le gustan las mates, las consideran fáciles. A quién no le gustan, las consideran difíciles. 50\% de alumnos que rechazan las mates es por culpa de los malos profesores. 30\% de alumnos que no rechazan las mates es gracias a sus profesores. 

"No sabéis lo que nos hacéis falta", "si pongo ahora los exámenes que ponía hace 10 años, me denuncian al defensor del menor". Cada año dan menos y exigen menos porque quienes llegan a 1º de carrera llegan sabiendo menos.

La asignatura "Didáctica de las Mates" del grado en Magisterio incluye un apartado de contenidos mínimos, que cada año que pasa se come más parte de la asignatura. 
%
El examen tiene 2 partes, contenidos y didáctica. Quien suspende contenidos, suspende la asignatura y tienen un gran problema montado.


\paragraph{Segunda entrada del portafolio}


En verano estuve de voluntariado y una semana fue dar clase de mates a chavales de secundaria de allí. 
%
Les ocurría que sabían despejar las ecuaciones perfectamente. Hacían todos los pasos bien, hasta que al llegar al final, $3x = 24 \to x=\frac{24}{3} = ?$. 
%
El último paso no eran capaces de hacerlo. ¿Cómo es posible que lleguen al curso en el que están y sepan despejar sin saberse las tablas de multiplicar? 
%
¡Qué sistema educativo tan ineficiente! 
%
Pasan de curso sin los conocimientos necesarios... construyen el conocimiento con unas lagunas bestiales.

Viendo los errores de primero de grado (que no me lo puedo creer) me doy cuenta que no hay tanta diferencia entre el modelo (con su fallos) de allí, con el modelo (y sus fallos) de aquí.

\subparagraph{Ideas random:}

\begin{itemize}
	\item Como pastoralmusical pero con "unidad didáctica" relacionado con recurso interesante.
	\item ¿Porqué no, hablar todos los años de quién ha ganado la medalla Fields?
\end{itemize}





%% Apendices (ejercicios, examenes)
\appendix

\chapter{---}
% -*- root: ../Innovacion.tex -*-

\section{Conglomerado de recursos}

\begin{itemize}
	\item
	Estudios internacionales de Evaluación
	http://www.mecd.gob.es/inee/publicaciones/estudios-internacionales.html 
	\item 
	Instituto NAcional de Tecnologias Eduativas y Formación del profesorado 
	http://educalab.es/intef
	\item 
	Seamos gansos
	https://www.youtube.com/watch?v=K5G8gRvx7nQ
	\item 
	9 gestos cotidianos que los matemáticos hacemos de otra manera:
	http://verne.elpais.com/verne/2015/11/13/articulo/1447413460_147289.html?id_externo_rsoc=FB_CM
	\item 
	La escuela en 2030
	http://www.elmundo.es/espana/2014/10/21/54455b9f22601d22738b458e.html
	\item 
	El cerebro necesita emocionase para aprender
	http://economia.elpais.com/economia/2016/07/17/actualidad/1468776267_359871.html
	\item 
	La neuroeducación demuestra que emocion y conocimiento van juntos
	http://blogs.elpais.com/ayuda-al-estudiante/2013/12/la-neuroeducacion-demuestra-que-emocion-y-conocimiento-van-juntos.html
	\item 
	mates y neurociencia
	https://escuelaconcerebro.wordpress.com/2012/03/20/matematicas-y-neurociencia/
	\item 
	actividad cerebral del alumno durante la clase magistral
	http://ined21.com/actividad-cerebral-del-alumno-durante-la-tradicional-clase-magistral/
	\item 
	Aprendizaje cooperativo y neuroeducación: guiando la poda sináptica
	https://escuelaconcerebro.wordpress.com/2016/08/18/aprendizaje-cooperativo-y-neuroeducacion-guiando-la-poda-sinaptica/
	\item 
	blog de escuela con cerebro
	https://escuelaconcerebro.wordpress.com/
	\item
	Blog de Anna Fores
	https://annafores.wordpress.com/category/neurodidactica/
	\item 
	Neuroeducación por otra escuela
	https://www.youtube.com/watch?v=QiRqCKUiRDc&feature=youtu.be
	\item 
	Qué son las neurociencias:
	https://www.youtube.com/watch?v=kotroYR8p5A&feature=youtu.be
	\item 
	¿Cómo puede mejorar la neurociencia el aprendizaje? ¿y las nuevas tecnologías? 
	http://www.rtve.es/alacarta/videos/tres14/tres14-aprendizaje/1001759/
	\item 
	blog de Clara Grima (matemática, y antigua colaboradora en Orbita Laika)
	http://claragrima.com/
	\item 
	Salman Khan: Usemos el video para reinventar la educación
	https://www.ted.com/talks/salman_khan_let_s_use_video_to_reinvent_education?language=es
	\item 
	Tecnologías Educativas. Herramientas: Khan Academy web © UPV
	https://www.youtube.com/watch?v=FvacPlqEw6g
	\item 
	Construyendo una escuela en la nube
	https://www.ted.com/talks/sugata_mitra_build_a_school_in_the_cloud?language=es
	\item 
	Video juegos y educación
	http://ares.cnice.mec.es/informes/02/documentos/indice.htm
	\item 
	¿video juego para aprender Álgebra?
	http://one.elpais.com/jean-baptiste-huynh-te-muestra-el-videojuego-con-el-que-puedes-aprender-algebra-en-un-par-de-horas/
	\item
	TFM del proyecto Gauss
	https://repositorio.unican.es/xmlui/bitstream/handle/10902/1748/Ruiz%20Sar%C3%A1chaga%20Alexandra.pdf?sequence=1
	\item 
	Tecnologías Educativas. Herramientas: Vídeo, Youtube uso educativo @ UPV
	https://www.youtube.com/watch?v=2w9Yq76tFCg
	\item 
	Artículo El video en el aula
	http://www.lmi.ub.es/te/any93/ferres_cp/
	\item 
	Matemáticas de cine
	http://www.ceice.gva.es/web/innovacion-calidad
	\item
	talleres y juegos (infantil, primaria y secundaria)
	http://blogsaverroes.juntadeandalucia.es/
	\item 
	Gamificación: jugar para aprender
	https://educacioncuatropuntocero.wordpress.com/2015/02/05/gamificacion-jugar-para-aprender/
	\item 
	Inteligencias múltiples
	http://educarlasinteligenciasmultiples.blogspot.com.es/
	\item 
	PDI en mates
	http://tic.cardenalcisneros.es/2014/04/21/la-pizarra-digital-interactiva-en-matematicas/
	\item
	Web de Jose Dulac (PDI)
	http://www.dulac.es/
	\item 
	Multiclass
	http://www.multiclass.com/recursos-educativos
	\item 
	web de Pere Marqués (PDI)
	http://peremarques.net/
	\item 
	web de Artigraf (con formacion para el profesor)
	http://www.artigraf.com/default.php
	\item 
	tdah y tu
	http://www.tdahytu.es/
	\item
	Las vidas de Mario (estupendo video)
	http://www.lasvidasdemario.com/
	\item 
	Federacion Española de asociaciones de TDAH
	http://www.feaadah.org/es/
	\item 
	Fundacion CADAH
	http://www.fundacioncadah.org/web/
	\item
	canal youtube Educacion Activa TDAH URL
	https://www.youtube.com/user/Canal

\end{itemize}


\printindex
\end{document}
