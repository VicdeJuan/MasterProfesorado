
\section{Introducción}


\subsection{Elección y Justificación}

Matemáticas es una asignatura que a bastante porcentaje del alumnado se le hace difícil. 
%
Hay algunos que optan por otros itinerarios sólo para evitar las Matemáticas, porque ya les parecen imposibles.
%
Hay otros que eligen Matemáticas "fáciles" (Matemáticas aplicadas a las Ciencias Sociales) porque no quieren ni Física ni Latín.

Sin saber con seguridad qué porcentaje del alumnado de esas "Matemáticas fáciles" la ha escogido por gusto o por ser la opción \textit{menos mala}, creemos que un alto porcentaje del alumnado de 1 de Bachillerato de Sociales no le gustan las Matemáticas y además pueden pensar que se les da realmente mal. 
%
Es por ello que queríamos hacer el trabajo para, de alguna manera, solucionar este problema que se da en el Bachillerato de Sociales.

Aunque una consigna de este trabajo era explicar un contenido didáctico de algún curso, pidiendo permiso, le hemos dado otro enfoque. 
%
Queríamos preparar la primera clase del curso de Matemáticas aplicadas a las Ciencias Sociales y queríamos hacerlo para motivarles y ayudarles a automotivarse.

A continuación, procedemos a definir más concretamente los objetivos.


\subsection{Objetivos}

\paragraph{Objetivo general}
\begin{itemize}
	\item Motivar al alumnado de Matemáticas para intentar romper la preconcepción que puedan tener sobre las Matemáticas. 
\end{itemize}

\paragraph{Objetivos específicos}
\begin{itemize}
	\item Explicar el concepto de la Indefensión Aprendida\footnote{Definida en \ref{defn::indefension}.}. Puede ser que algún alumno haya \textit{aprendido la indefensión} hacia las Matemáticas. El primer paso para romper esa indefensión es conocer que existe y que es aprendida.
	\item Introducir el curso de Matemáticas y motivar a los alumnos.
	\subitem Hacer hincapié en que estas no son las Matemáticas fáciles, sino las útiles. \footnote{Esto puede contribuir a que no se sientan "más tontos" (en las Matemáticas) que sus sus compañeros de las Matemáticas Académicas, ya que sus Matemáticas (CCSS) no son más fáciles, sino que tienen otro objetivo y son más útiles.}
	\item Mostrar el potencial de las Matemáticas aplicadas a las Ciencias Sociales para hacer más fuerte la motivación de los alumnos hacia la asignatura y complementar el objetivo anterior.
\end{itemize}

\section{Estructura}

Hemos estructurado el trabajo en torno a los objetivos específicos. 
%
Primero, tratamos la indefensión aprendida y lo haremos mediante un experimento, para que lo vean de primera mano.
%
Una vez interiorizado el concepto de indefensión aprendida, procederemos a introducir brevemente algunos aspectos del curso, recalcando la importancia de la automotivación.
%
Esta introducción se basará en material audiovisual para captar mejor la atención de los alumnos.
%
Por último, realizaremos un cálculo del número π por el método de Montecarlo. 
%
Este método se basa fundamentalmente en la Estadística y en la Probabilidad, temario específico de Matemáticas Aplicadas a las Ciencias Sociales.
%
Este cálculo se realizará en 2 partes. Una primera de cálculo manual, con material manipulativo (granos de arroz) y una segunda parte de cálculo simulado por ordenador.
%
El cálculo por ordenador es fundamental para que los alumnos puedan ver que realmente se obtiene el número π (ya que el método manual tiene algunos problemas, que trataremos en \ref{pimanual})

\subsection{Indefensión Aprendida}
\label{defn::indefension}

Por motivos de claridad de este escrito, exponemos primero el concepto y después el experimento, aunque después, en la exposición, primero realizaremos el experimento y después procederemos a explicar el concepto.

\subsubsection{Concepto}

Lo primero de todo es definir el fenómeno:
\begin{defn}[Indefensión Aprendida]
Condición de un ser humano o animal que ha aprendido a comportarse pasivamente, con la sensación subjetiva de no poder hacer nada y que no responde a pesar de que existen oportunidades reales de cambiar la situación aversiva, evitando las circunstancias desagradables o mediante la obtención de recompensas positivas.
\end{defn}

Creemos que este fenómeno se da en las Matemáticas de la siguiente manera:
%
Existe una creencia generalizada de que las Matemáticas son difíciles y que por mucho que me esfuerce, este pensamiento permea en mi y me lleva a comportarme pasivamente, motivado por la sensación de no poder hacer nada.
%
Otra posibilidad (un ejemplo más claro y fuerte de la indefensión) es: si yo he estudiado con mucho esfuerzo Matemáticas y no la he conseguido aprobar, aprenderé que no puedo sacar las Matemáticas.
%
Hay personas a las que les puede pasar esto porque realmente su inteligencia Logico-Matemática no sea muy alta, pero creemos que hay muchos que, con una inteligencia Logico-Matemática más que suficiente, se estrellan con las Matemáticas, ¿por qué? 
%
Porque a lo largo de su vida de estudiante, pueden haber aprendido la indefensión hacia las Matemáticas. 
%
Y esta indefensión se puede haber aprendido porque un año (o más) han tenido profesores más exigentes o que no han sabido transmitir los conocimientos y además en casa han recibido comentarios del tipo "Pues será que no vales para las Matemáticas".
%
Estos 2 factores creemos que son muy frecuentes que se den juntos o separados y que induzcan a los estudiantes una indefensión que se puede desaprender. Y para desaprenderla, el primer paso es conocer que existe.

Esta última conclusión, nos lleva al experimento para transmitir la Indefensión aprendida.

\subsubsection{Experimento}

Se dividirá a la clase en 2 grupos sin que ellos lo sepan.
%
Las llamaremos la mitad \textit{control} y la mitad \textit{experimental}.

Desde su perspectiva, la actividad a realizar será, dada una palabra, formar otra palabra en singular con las mismas letras.
%
Por ejemplo: dada la palabra \textit{\textsc{roma}} tendrán que escribir la palabra \textit{\textsc{amor}}. 
%
Cuando lo hayan conseguido, levantarán la mano y esperarán a que se pueda empezar la siguiente ronda.

El papel que juegan las 2 mitades es que no todos los alumnos van a recibir la misma palabra con la que trabajar. 
%
La mitad \textit{control} recibirá siempre palabras con las que se pueda hacer otra palabra, mientras que la mitad \textit{experimental} recibirá palabras con las que sea imposible escribir otra palabra con esas letras.

El desarrollo del experimento es el siguiente:

\begin{itemize}
	\item[0.-] Cada alumno tiene en su mesa 3 sobres de colores: uno rojo, otro amarillo y otro verde.
	\subitem Hay 2 tipos de sobres: los del grupo control y los del grupo experimental que habrán sido previamente diferenciados.
	\item[1.-] Cuando el profesor diga, abrirán el primer sobre, extraerán el papel con la palabra escrita y tendrán que escribir una nueva palabra con las mismas letras. Cuando lo consigan, les pediremos que levanten la mano y la mantengan levantada.
	\item[2.-] Una vez terminado el primer sobre (y haciendo ver a los que no lo han resuelto que hay muchos compañeros que sí), procederemos con el segundo sobre de la misma manera.
	\item[3.-] De la misma manera, procederemos al último sobre. 
	%
	En este último sobre el grupo control y el grupo experimental tiene exactamente la misma palabra. 
	%
	Es de esperar que los miembros del grupo control realicen con mayor rapidez el tercer sobre, mientras que los miembros del grupo experimental puedan incluso llegar a no ser capaces.
\end{itemize}

Una vez desarrollado el experimento, tendremos un rato de comentar la experiencia. 
%
Pediremos perdón por la pequeña "manipulación", ya que no todos tenían las mismas palabras y preguntaremos al grupo experimental cómo se han sentido.

Como decíamos anteriormente, una vez terminado el experimento en clase, procederemos a explicar el concepto, cómo se detecta y daremos algunas herramientas para ponerle solución. 
%
Para más información sobre estos puntos, consultar la presentación de la exposición.

\subsection{Motivación}


\subsection{Método Montecarlo para calcular π}
\label{pimanual}

\subsubsection{Simulación}

