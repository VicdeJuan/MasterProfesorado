
\begin{opin}{\pedrocolor}{Pedro}
La asignatura me ha permitido aprender numerosos conceptos, metodologías y recursos educativos que desconocía anteriormente. Es indudable su gran aporte a mi formación como profesor. Me gustaría con el tiempo ir poco a poco analizando en profundidad cada una de las aportaciones realizadas en cada sesión, con la finalidad de potenciar al máximo su efectividad en el aula. 

El trabajo grupal me ha permitido tomar conciencia de lo importante que es saber organizarse, cooperar y escuchar las sugerencias del resto de compañeros. He aprendido mucho de sus aportaciones, vivencias educativas y sus puntos de vista como futuros educadores.

Para finalizar, y no por ello menos importante, destacar la gran importancia de las Jornadas de Formación Complementaria. Me han acercado a las metodologías activas y a las teorías y ciencias en las que se basan. Espero haber adquirido las herramientas necesarias para trabajar en el aula con mis futuros alumnos. 

Llegué al máster pensando que el mejor profesor era el que más sabía, pero como dijo Irene Ros en su Jornada, “No es mejor profesor quien más sabe, sino quien más consigue que aprendan sus alumnos”.
\end{opin}