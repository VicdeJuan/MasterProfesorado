\begin{opin}{\victorcolor}{Víctor}

En esta asignatura, más que quedarme con los contenidos concretos, me quedo con la motivación recibida.
%
Esta asignatura (debido a la motivación intrínseca de la profesora) me ha ayudado a tomar conciencia de la importancia que tendrá mi labor. 
%
Antes de entrar ya era consciente que las Matemáticas son la asignatura difícil con la que algunos se atragantan, pero no era consciente de la magnitud del problema que eso supone.

A día 11 de Diciembre, si tengo que resumir la asignatura en una palabra sería \textbf{Motivación}. 
%
Termino la asignatura muy motivado con la labor que me espera, con más ganas de empezar las prácticas y con una idea muy clara de que eso de 2 meses de vacaciones es mentira, porque tengo tantísimos recursos que consultar, tantas posibilidades con las que innovar, que, como poco, la mitad de cada verano lo acabaré invirtiendo en cómo innovar y mejorar mi labor docente.

Me llevo también el ejemplo de que desde la Universidad, aunque este máster se considere un trámite, existen profesores muy motivados que viven su trabajo muy vocacionalmente. 
%
He disfrutado mucho de la docencia impartida desde el convencimiento de que sirve para mucho, de que el cambio necesario en la educación empieza por formar a los profesores con un máster de verdad y no un mero trámite. Si alguna vez acabo formando a futuros docentes (que nunca me lo había planteado hasta ahora), será gracias a esta asignatura en primer lugar, más por la profesora que por la asignatura en sí. 

\end{opin}


\begin{opin}{\pedrocolor}{Pedro}
La asignatura me ha permitido aprender numerosos conceptos, metodologías y recursos educativos que desconocía anteriormente. Es indudable su gran aporte a mi formación como profesor. Me gustaría con el tiempo ir poco a poco analizando en profundidad cada una de las aportaciones realizadas en cada sesión, con la finalidad de potenciar al máximo su efectividad en el aula. 

El trabajo grupal me ha permitido tomar conciencia de lo importante que es saber organizarse, cooperar y escuchar las sugerencias del resto de compañeros. He aprendido mucho de sus aportaciones, vivencias educativas y sus puntos de vista como futuros educadores.

Para finalizar, y no por ello menos importante, destacar la gran importancia de las Jornadas de Formación Complementaria. Me han acercado a las metodologías activas y a las teorías y ciencias en las que se basan. Espero haber adquirido las herramientas necesarias para trabajar en el aula con mis futuros alumnos. 

Llegué al máster pensando que el mejor profesor era el que más sabía, pero como dijo Irene Ros en su Jornada, “No es mejor profesor quien más sabe, sino quien más consigue que aprendan sus alumnos”.
\end{opin}