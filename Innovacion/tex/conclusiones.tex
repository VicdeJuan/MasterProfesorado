

\subsection{Conclusiones de Gustavo}
\begin{leftbar}{\guscolor}
\subsubsection{Valoración de la asignatura}

Quería reflejar la autoevaluación de la asignatura y las conclusiones sin revisar el primer apartado que hice a principios de curso sobre las expectativas de la asignatura y así poder comprobar fielmente las diferencias entre lo que esperaba al empezar la asignatura y lo que he concluido al terminar la misma.

Durante la asignatura he aprendido muchas cosas, pero si tuviese que reflejar de manera clara y concisa las conclusiones principales que saco son:

\begin{itemize}
\item El aprendizaje va de la mano de las emociones y de la motivación.
\item Hay un nuevo campo reciente que está por investigar: la neurociencia.
\item Hay que buscar metodologías de enseñanza diferentes a la tradicional.
\end{itemize}

Existen infinidad de recursos para los docentes

En relación al número de horas de impartición de la asignatura he de decir que el tiempo de horas impartidas de clase a la asignatura no es tan elevado como debería para los contenidos que se pretenden. En total hemos tenido 9 días de clase de 2 horas. 18 horas de clase de las cuales 4 clases han sido “clases magistrales” impartidas por Raquel, 1 clase de pizarras digitales impartida por Manuel y 4 clases de exposición de trabajos.

10/10/2016 Clase 01. T01. Innovación en Educación

17/10/2016 Clase 02. T02. Introducción a la Neurodidáctica

24/10/2016 Clase 03. T02. Introducción a la Neurodidáctica

24/10/2016 Clase 03. T03. Innovación y recursos educativos

07/11/2016 Clase 04. Clase magistral de Pizarras digitales

14/11/2016 Clase 05. T03. Innovación y recursos educativos

21/11/2016 Clase 06. Exposición de trabajos Dia 1

28/11/2016 Clase 07. Exposición de trabajos Dia 2

05/12/2016 Clase 08. Exposición de trabajos Dia 3

12/12/2016 Clase 09. Exposición de trabajos Dia 4

A pesar de que este número de horas no es elevado, reconozco que el hecho de tener que trabajar el portafolio hace que las horas dedicadas a la asignatura aumenten considerablemente. Esta dedicación nos permite profundizar en los contenidos vistos en clase y es de aquí de donde saco la conclusión más importante y es que esta asignatura en general y el portafolio en particular es una herramienta muy útil para nuestro futuro como profesores de matemáticas.

También he de reconocer que me gustaron más las primeras clases de la asignatura por lo novedoso pero con el paso del tiempo, no sé si por rutina o porque realmente las clases fueron cambiando, me resultaron más tediosas. Las primeras clases fueron más colaborativas con preguntas y debates promovidos por los alumnos. Este tipo de clases me resultan mucho más atractivas. Sin embargo, el hecho de que tuviésemos tan pocas horas disponibles, las clases fueron tornando y me dio la sensación de que se limitaban a indicar en las transparencias enlaces a recursos didácticos en el mundo de las matemáticas. Reconozco que nos va a ser muy útil en el futuro el tener acceso a estos recursos, pero la utilidad real la obtendremos cuando seamos docentes y tengamos que hacer uso de estos recursos. A día de hoy no le podemos sacar el 100\% del jugo a la asignatura.

Por ser constructivo lo que yo haría en el futuro sería un par de cambios. El primero puede parecer un poco drástico. Sería eliminar los trabajos y las exposiciones lo que nos permitiría tener más horas de clase. Quizá al tener más tiempo se resolvería el segundo cambio que es el de que las clases fuesen más colaborativas. Generar debates y conocer la leftbarión de todos los alumnos sobre los distintos temas tratados en clase nos enriquecería mucho más. Esto es una conclusión y una evidencia que he aprendido de la propia asignatura así que, ¿por qué no emplearlo para nuestra propia clase?

Para terminar, el portafolio lo mantendría por lo que ya he comentado. Es una buena manera de repasar los conceptos vistos en clase.

Y llegados a este punto, acabo de leer cuales eran mis expectativas iniciales. La diferencia más clara entre las conclusiones que he sacado y las expectativas iniciales han sido que inicialmente esperaba que se hubiese hecho más hincapié en las TICs y sin embargo las TICs no han eclipsado a todo lo que hemos aprendido.

\subsubsection{Valoración de mis compañeros}

\subsubsection{Autoevaluación}


\end{leftbar}

\subsection{Conclusiones de Víctor}
\begin{leftbar}{\victorcolor}

\subsubsection{Valoración de la asignatura}

En esta asignatura, más que quedarme con los contenidos concretos, me quedo con la motivación recibida.
%
Esta asignatura (debido a la motivación intrínseca de la profesora) me ha ayudado a tomar conciencia de la importancia que tendrá mi labor. 
%
Antes de entrar ya era consciente que las Matemáticas son la asignatura difícil con la que algunos se atragantan, pero no era consciente de la magnitud del problema que eso supone.

A día 11 de Diciembre, si tengo que resumir la asignatura en una palabra sería \textbf{Motivación}. 
%
Termino la asignatura muy motivado con la labor que me espera, con más ganas de empezar las prácticas y con una idea muy clara de que eso de 2 meses de vacaciones es mentira, porque tengo tantísimos recursos que consultar, tantas posibilidades con las que innovar, que, como poco, la mitad de cada verano lo acabaré invirtiendo en cómo innovar y mejorar mi labor docente.

Me llevo también el ejemplo de que desde la Universidad, aunque este máster se considere un trámite, existen profesores muy motivados que viven su trabajo muy vocacionalmente. 
%
He disfrutado mucho de la docencia impartida desde el convencimiento de que sirve para mucho, de que el cambio necesario en la educación empieza por formar a los profesores con un máster de verdad y no un mero trámite. Si alguna vez acabo formando a futuros docentes (que nunca me lo había planteado hasta ahora), será gracias a esta asignatura en primer lugar, más por la profesora que por la asignatura en sí. 

\subsubsection{Valoración de mis compañeros}

El grupo de trabajo lo escogimos por comodidad, ya que éramos el mismo grupo en otra asignatura (Didáctica de las Matemáticas).
%
Como grupo hemos funcionado muy bien, creo que hemos sido muy comprensivos y hemos sabido trabajar en equipo.
%
Estoy muy gratamente sorprendido de lo bien que hemos trabajado. Haber formado los grupos por cómo estábamos sentados en otra asignatura ha dado muy buen resultado.

Mis compañeros son muy responsables, eficientes y trabajadores y estas dinámicas tan positivas me han animado a ser todavía más trabajador, eficiente y responsable de lo que era anteriormente. 
%
Hemos sabido aportar todos lo que era necesario y no ha hecho falta nunca llamar la atención a nadie por nada.

Por último, valoro muy positivamente los conocimientos previos de cada uno. Venir de carreras diferentes ha sido muy positivo, tanto para el ámbito profesional (del trabajo de la asignatura) como para el ámbito personal (de conocer personas con otras historias y entornos diferentes).

\subsubsection{Autoevaluación}

Como todo en esta vida, siempre se puede hacer mejor y siempre hay margen de mejora. 
%
Dicho esto, creo que he trabajado bastante bien. 
%
Sólo no he podido ir a una clase y en clase siempre he mantenido la concentración en lo que se estaba tratando.
%
La mayor dificultad ha sido saber priorizar lo importante sobre lo urgente. 
%
Debido al resto de mis responsabilidades (voluntariado, cursos y otros estudios) siempre tenía algo más urgente que hacer que revisar con profundida los recursos.
%
No obstante, creo que durante mi práctica laboral podré sacarle todo el jugo a los recursos facilitados por la profesora.
%
Eso sí, como las jornadas complementarias sólo ocurren una vez, y a los recursos puedo acceder en otro momento, he intentado ir a todas las jornadas complementarias (aunque no me ha sido posible ir a todas).

\end{leftbar}

\subsection{Conclusiones de Pedro}
\begin{leftbar}{\pedrocolor}

\subsubsection{Valoración de la asignatura}

La asignatura me ha permitido aprender numerosos conceptos, metodologías y recursos educativos que desconocía anteriormente. Es indudable su gran aporte a mi formación como profesor. Me gustaría con el tiempo ir poco a poco analizando en profundidad cada una de las aportaciones realizadas en cada sesión, con la finalidad de potenciar al máximo su efectividad en el aula. 

El trabajo grupal me ha permitido tomar conciencia de lo importante que es saber organizarse, cooperar y escuchar las sugerencias del resto de compañeros. He aprendido mucho de sus aportaciones, vivencias educativas y sus puntos de vista como futuros educadores.

Para finalizar, y no por ello menos importante, destacar la gran importancia de las Jornadas de Formación Complementaria. Me han acercado a las metodologías activas y a las teorías y ciencias en las que se basan. Espero haber adquirido las herramientas necesarias para trabajar en el aula con mis futuros alumnos. 

Llegué al máster pensando que el mejor profesor era el que más sabía, pero como dijo Irene Ros en su Jornada, “No es mejor profesor quien más sabe, sino quien más consigue que aprendan sus alumnos”.

\subsubsection{Valoración de mis compañeros}

Mis compañeros de grupo han respondido muy bien a la hora de realizar las actividades. Pese a tener todos responsabilidades fuera del Máster, en ningún momento han mostrado signos de flaqueza para sacar el trabajo adelante. A la hora de proponer una temática para el trabajo final, me ha sorprendido la gran capacidad creativa que tienen y la profesionalidad con la que tratan los temas.

\subsubsection{Autoevaluación}

Me he sentido como en “Howarts”, escuela a la cual asisten jóvenes matemáticos e ingenieros para desarrollar sus habilidades mágicas como educadores.  Eso ha sido para mí esta asignatura, dar nombre y definición a los problemas que vengo observando toda la vida como alumno y que ahora tengo la obligación como docente, de buscar solución para sacar lo mejor de cada alumno. 

\end{leftbar}

\subsection{Conclusiones de Virginia}
\begin{leftbar}{\virgicolor}


\subsubsection{Valoración de la asignatura}
Esta asignatura ha cumplido con creces mis expectativas iniciales al mostrarme nuevos métodos de aprendizaje aplicando técnicas como la neurodidáctica, y haciendo uso de múltiples recursos tecnológicos aportándome una información muy valiosa para mi futura carrera como profesional docente.

Mi visión como profesora ha dado un vuelco, ahora me siento más motivada y con ganas de hacer cosas nuevas, me ha animado a ser partícipe en la transmisión de conocimientos de una forma diferente a la típica clase magistral de forma que ahora el alumno sea un agente activo en lugar de pasivo, se trabaje de forma colaborativa en clase promoviendo el aprendizaje social y se haga uso de forma adecuada de la multitud de recursos tecnológicos que existen.

No obstante es una ardua tarea ya que en primer lugar el sistema educativo actual tiene que mejorar mucho para facilitar los nuevos métodos de aprendizajes. Así mismo, y dado toda la información y recursos de los que dispone el docente, es muy importante realizar cursos de formación del uso de los nuevos métodos y las nuevas tecnologías para realizar un buen uso de ello y conseguir así un aprendizaje permanente.

En cuanto al trabajo en grupo me ha parecido muy útil y divertido ya que he vivido en primera persona lo que es el trabajo colaborativo aprendiendo cosas de mis compañeros e intentando aportarles a ellos de igual forma otras cosas.

Para finalizar me quedo con la frase que dijo José Ramón Gamo en una de sus charlas: “el cerebro aprende emocionándose”. Por tanto, la labor del docente tiene que ser conseguir con su enseñanza emocionar a los alumnos de forma que estén más motivados y con ganas de más.

\subsubsection{Valoración de mis compañeros}


Con respecto a mis compañeros de trabajo, estoy muy contenta de haber trabajado con ellos ya que creo que nos hemos entendido muy bien y hemos trabajado a gusto.
%
El hecho de proceder de carreras diferentes creo que ha sido muy útil a la hora de aportarnos cosas diferentes.
%
 Además, en algún momento de agobio por la carga de trabajo que se nos presentaba me he sentido apoyada y aliviada por las palabras de mis compañeros.
%
En definitiva, son trabajadores, responsables y buenos compañeros, por lo que estoy muy contenta de haber formado parte de este grupo.



\subsubsection{Autoevaluación}
En general ceo que mi trabajo en esta asignatura ha sido bastante bueno ya que he asistido regularmente, salvo el último día que por motivos de trabajo y personales no pude.
%
Además, he intentado llevar al día el portafolio para facilitar un transcurso continuo de la asignatura lo que me ha permitido trabajar más cómoda y sin prisas.
%


Por otro lado, mi principal problema tanto en esta asignatura como en el resto ha sido la falta de tiempo ya que trabajo a jornada completa además de realizar también un curso de inglés.
%
Por tanto, no he podido dedicar todo el tiempo que hubiera querido a revisar toda la información complementaria que Raquel nos ha aportado y por desgracia tampoco he podido asistir a las jornadas complementarias que tanto me hubiera gustado, sobre todo, la de Javier Blumenfeld y José Ramón Gamo.
%
No obstante, gracias a Internet he podido asistir virtualmente a alguna charla de ellos.
%
Además, soy consciente de la multitud de recursos de los que disponemos, entre otros, toda la información complementaria que Raquel nos ha aportado y de los que iré haciendo uso durante el transcurso del máster y durante mi futuro profesional como docente ya que creo que es una información muy valiosa para conseguir que mis futuros alumnos se emocionen mientras aprendan.


\end{leftbar}

