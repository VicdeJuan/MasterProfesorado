\begin{opin}{\guscolor}{Gustavo}

\subsubsection{Recursos educativos en el aula}
Durante la clase de hoy, Raquel nos hizo ver la gran cantidad de recursos didácticos existentes hoy en día en el mundo de las matemáticas. Veo una ventaja muy evidente respecto a buscarlo por nuestra cuenta y es que sin duda Raquel, como docente con años de experiencia, nos ha filtrado los recursos para evitar perdernos en la red.

Esto nos va a ayudar en nuestro futuro profesional como docentes ya que será un espacio dónde poder apoyarnos. Entre otro de los consejos que nos dio Raquel está el de que tenemos que asumir que es difícil estar al día de todo por lo tanto no hay que abrumarse por tal situación.

Los recursos a destacar son:

\subsubsection{Recursos del INTEF}
El Instituto Nacional de Tecnologías Educativas y de Formación  (INTEF) del Profesorado es la unidad del Ministerio de Educación, Cultura y Deporte responsable de la integración de las TIC en las etapas educativas no universitarias. Tiene rango de Subdirección General integrada en la Dirección General de Evaluación y Cooperación Territorial que, a su vez, forma parte de la Secretaría de Estado de Educación, Formación Profesional y Universidades (Fuente: \url{http://educalab.es/intef/introduccion})

Los siguientes recursos se encuentran disponibles desde la web del INTEF \url{http://educalab.es/recursos}, aunque también son interesantes los recursos específicos de matemáticas encontrados en el histórico de recursos \url{http://educalab.es/web/web/recursos/historico/asignaturas/matematicas}

\paragraph{Procomún}
\url{https://procomun.educalab.es/}

PROCOMÚN es el espacio que más destaca dentro de los recursos del INTEF y se describe como una red de Recursos Educativos Abiertos. Este espacio destaca por la gran cantidad de recursos disponibles y por la posibilidad de búsqueda de dichos recursos por medio de los metadatos.

Dentro de Procomún se hizo mención al Proyecto Gauss.

\paragraph{eXeLearning}
\url{http://exelearning.net/}

eXeLearning es una herramienta de autor de código abierto para ayudar a los docentes en la creación y publicación de contenidos web. Facilita la creación de contenidos educativos sin necesidad de ser experto en HTML o XML. Se trata de una aplicación multiplataforma que nos permite la utilización de árboles de contenido, elementos multimedia, actividades interactivas de autoevaluación… facilitando la exportación del contenido generado a múltiples formatos: HTML, SCORM, IMS, etc.

Características destacadas de Exelearning:
\begin{itemize}
\item Permite crear un árbol de navegación básico que facilitará la navegación.  

\item Permite escribir texto y copiarlo desde otras aplicaciones.  

\item Permite incluir imágenes, pero no es un editor de imágenes como Photoshop o Gimp.  

\item Permite incluir sonidos, pero deben estar grabados previamente con otra aplicación.  

\item Permite incluir vídeos y animaciones, pero no permite crearlas.  

\item Permite incluir actividades sencillas: preguntas de tipo test, de verdadero/falso, de espacios en blanco...  

\item Permite embeber elementos multimedia como vídeos, presentaciones, textos o audios.  

\item Permite incluir actividades realizadas con otras aplicaciones 
\end{itemize}


\paragraph{Proyecto Gauss}
\url{http://recursostic.educacion.es/gauss/proc/}

El Proyecto Gauss ha sido desarrollado por el INTEF. Es un proyecto específico de matemáticas y ofrece a los profesores varios centenares de ítems didácticos y de applets de GeoGebra, que cubren todos los contenidos de matemáticas de Primaria y de Secundaria.

\paragraph{Proyecto Agrega/Agrega2}
\url{http://www.agrega2.es/web/}

Este proyecto es una federación de repositorios de contenidos educativos digitales donde todo el mundo pueda buscar, visualizar y descargar material educativo digital no universitario.  Se pretende facilitar a la comunidad educativa una herramienta útil para integrar las Tecnologías de la Información y la Comunicación en el aula

Existe una segunda versión de este proyecto que mejora la anterior y se llama Agrega2. Es de los repositorios más complicados para encontrar cosas. De hecho, hay un curso específico para buscar contenidos en Agrega2.

\paragraph{Red de Buenas PracTICas 2.0}
\url{http://recursostic.educacion.es/buenaspracticas20/web/}

Red de Buenas Practicas 20 es una red social de profesores dentro del INTEF.

\paragraph{Internet en el aula}
\url{http://internetaula.ning.com/}

Otra red social para docentes

\paragraph{Educa con TIC}
\url{http://www.educacontic.es/recursos-educativos}

Es un blog especializado en el uso de las TIC en las aulas

\paragraph{Proyecto Descartes}
\url{http://proyectodescartes.org/}

El Proyecto Descartes comienza su andadura en el año 1998. Lleva mucho tiempo activo, por tanto es lógico pensar que hay mucha documentación y muchos recursos.

Hay una página antigua en una web oficial del ministerio (\url{http://recursostic.educacion.es/descartes/web/}) pero está sin mantenimiento. Esta página antigua lo tenían hecho en JAVA y daba tantos problemas a los usuarios que decidieron cambiarla. Aun así, todavía hay muchos recursos que redirigen de la nueva a la antigua.

La web actual del proyecto Descartes pertenece a la Red Educativa Digital Descartes, que explicamos a continuación.

\paragraph{Red Educativa Digital Descartes}
La Red Educativa Digital Descartes (RED Descartes) es una asociación no gubernamental que se ha constituido el 1 de junio de 2013. Los socios fundadores son profesoras y profesores que tienen una historia conjunta construida, durante quince años, desarrollando proyectos del Ministerio de Educación español, entre los que podemos citar el Proyecto Descartes, Educación Digital a Distancia, Proyecto Canals, Pizarra Interactiva, Newton, Experimentación Didáctica en el Aula, WikididácTICa y Buenas Practicas 2.0. (Fuente: \url{http://www.educacontic.es/blog/matematicas-interactivas-con-descartes-en-tablets-y-smartphones})

En la parte de arriba de la web hay un apartado de subproyectos que te lleva a la página web \url{http://proyectodescartes.org/indexweb.php} . Entre estos subproyectos destacan:
\begin{itemize}
\item Telesecundaria (\url{http://proyectodescartes.org/Telesecundaria/}): educación a través de videos. Ojeando esta aplicación, también hay ejercicios y explicaciones interactivas hechas con HTML5, lo cual hace que sea accesible a través de cualquier navegador moderno. 

Telesecundaria es además una modalidad en el sistema educativo de México.

 

\item Proyecto Canals (\url{http://proyectodescartes.org/canals/index.htm}) de la profesora catalana Maria Antònia Canals para infantil y primaria. 

 

\item Proyecto "EDAD" Educación Digital con Descartes (\url{http://proyectodescartes.org/EDAD/index.htm}) surge con el propósito de desarrollar recursos educativos digitales interactivos, para la Educación Secundaria Obligatoria (ESO) en las áreas curriculares de Matemáticas, Ciencias Naturales y Física y Química, que permitan su uso tanto en la enseñanza presencial como en la formación a distancia. 

 

\item Proyecto ASIPISA (\url{http://proyectodescartes.org/ASIPISA/index.htm}). ASIPISA es una palabra palíndroma, acrónimo de “Ayuda Sistemática Interactiva para PISA”, que da nombre a un proyecto de desarrollo de materiales educativos, digitales e interactivos, basados en las unidades liberadas del Programa internacional PISA 

 

\item Proyecto Competencias (\url{http://proyectodescartes.org/competencias/index.htm}): Pensado para formar en competencias como marcan los nuevos planes de estudios. Esta web recoge objetos de aprendizaje interactivos cuyo objetivo es la formación y evaluación competencial. Sus contenidos se basan en las unidades liberadas de PISA, en las de las Pruebas de Evaluación de Diagnóstico de diferentes Comunidades autónomas españolas de acuerdo a la Ley Orgánica de Educación (LOE) de 2006 y a las pruebas de Evaluación de diagnóstico establecidas por la Ley Orgánica para la Mejora de la Calidad Educativa (LOMCE) de 2013. 
\end{itemize}

\paragraph{Educarex}
Es el Portal con contenidos educativos de la Comunidad Extremadura. Extremadura estaba a la cola en educación e hizo un esfuerzo bestial para ponerse a la altura del resto de España. Este portal es parte del fruto de dichos esfuerzos.

\paragraph{Otros recursos}
Además de todo lo visto hasta ahora también existen otros recursos como revistas, blogs, páginas de internet, etc. A continuación se citan algunos de estos recursos para su conocimiento:
\begin{itemize}
\item Aulaplaneta es un sistema integrado de contenidos curriculares que pone al servicio del profesor una propuesta didáctica personalizable y gran variedad de recursos digitales para preparar sus clases, y a disposición de los alumnos todo lo que necesitan para aprender de forma motivadora y eficaz. Destacar el artículo ” Diez canales educativos imprescindibles de YouTube para alumnos y profesores” \url{http://www.aulaplaneta.com/2015/10/27/recursos-tic/diez-canales-educativos-imprescindibles-de-youtube-para-alumnos-y-profesores/}  

 

\item MisMates y TutorMates. Se trata de dos proyectos digitales de Oxford destinados para el área de Matemáticas.   

MisMates es una aplicación educativa de acceso exclusivamente on line y está enfocada al alumnado de entre 1º y 4º de la ESO que tiene a su disposición varias áreas de trabajo, un editor de expresiones matemáticas y una libreta digital.

TutorMates es una aplicación de escritorio para Windows, iOS o Linux, y trabaja con herramientas específicas los contenidos de cada bloque

 

\item Educacion 3.0 es una revista de elevado interés en el mundo educativo en el que destaca el artículo “15 recursos de Internet imprescindibles para cualquier profesor” \url{http://www.educaciontrespuntocero.com/recursos/recursos-para-educacion-profesor-imprescindibles/35931.html}  

 

\item ScolarTIC es un proyecto de la Fundación Telefónica. Es una Comunidad Educativa de ámbito hispano. Es un espacio social de aprendizaje, innovación y calidad educativa en el que se ofrecen cursos online gratis, recursos para el aula así como charlas, ponencias y talleres.  

 

\item Y además a nivel particular hay muchas webs de profesores que ponen su trabajo al servicio de los demás: 
\begin{itemize}
\item Pilarleku@ (\url{http://pilarlekunew.blogspot.com.es/}) 

\item Domingo Mendez \url{http://domingomendez.es/} y su blog “Educación y TIC” \url{http://domingomendez.blogspot.com.es/}  

\item Algebra con Papas \url{https://www.edu.xunta.es/espazoAbalar/sites/espazoAbalar/files/datos/1291360755/contido/index.htm}  

\item Thatquiz \url{https://www.thatquiz.org/es/} Web con cuestionarios de matemáticas 

\item Manuel Sada Allo y su web Ejemplos diversos de webs interactivas de Matemáticas \url{http://docentes.educacion.navarra.es/msadaall/geogebra/}  

\item Sectormatematica \url{http://www.sectormatematica.cl/}  

\item Antonio Perez Sanz \url{http://platea.pntic.mec.es/}~aperez4/. Antonio presentó los programas de TVE de “Más por menos” y “Universo Matemático”. Actualmente (Diciembre de 2016) es responsable de divulgamat. 

\item “Amo las mates” actualmente en la web \url{https://www.matematicasonline.es/}  

\item Disfruta las matemáticas \url{http://www.disfrutalasmatematicas.com/}  

\item Vitutor \url{http://www.vitutor.com/} también se usa para la universidad. Tiene un contenido de bachillerato bastante potente. es una plataforma de teleformación diseñada para el aprendizaje en línea de distintas materias. 

El proyecto comenzó con la especialización en contenidos de Matemáticas, y estamos trabajando en otras materias, como inglés.

\item  “Aula21” que en su página \url{http://www.aula21.net/primera/matematicas.htm}  recopila un listado de enlaces a recursos de interés en el mundo de las matemáticas. 

\item Banco de recursos de SM \url{http://www.smconectados.com/Banco_de_recursos.html} donde encontrarás recursos para ayudarte a hacer más fácil tu trabajo en el aula. 
\end{itemize}
\end{itemize}

\paragraph{9 cosas que los profesores digitalmente competentes hacen habitualmente}


\begin{minipage}[h]{1\linewidth}
	\centering
	\includegraphics[width=0.8\linewidth]{img/9cosasdegus.jpg}
	\captionof{figure}{Coche de mediadios del siglo pasado.}
\end{minipage}

 
\paragraph{Uso del video en educación}
No cabe duda que el uso del video en la clase es una metodología innovadora. Está claro que tiene muchas ventajas como por ejemplo la de romper con la monotonía de la clase, pero también puede haber inconvenientes.

Los tipos de videos educativos según \url{http://www.uclm.es/profesorado/ricardo/Video/2002_2003/sld003.htm} son:

\begin{itemize}
\item \textbf{Documentales:} muestran de manera ordenada información sobre un tema concreto. 

\item \textbf{Narrativos:} tienen una trama narrativa a través de la cual se van presentando las informaciones relevantes para los estudiantes.  

\item \textbf{Lección monoconceptual:} son vídeos de muy corta duración que se centran en presentar un concepto.  

\item \textbf{Lección temática:} son los clásicos vídeos didácticos que van presentando de manera sistemática y con una profundidad adecuada a los destinatarios los distintos apartados de un tema concreto .  

\item \textbf{Vídeos motivadores:} pretenden ante todo impactar, motivar, interesar a los espectadores, aunque para ello tengan que sacrificar la presentación sistemática de los contenidos y un cierto grado de rigor científico. 
\end{itemize}

Un video motivador para poner a los alumnos puede ser el video de Tadeo Jones \url{http://www.telecinco.es/tadeojones/descubre-con-tadeo/Tadeo_Jones-Descubre_con_Tadeo-Matematicas_2_1697355179.html}

Otro video que impresiona a la hora de demostrar como la perspectiva puede engañar a como nuestro ojo le pasa la información a nuestro cerebro es \url{https://www.youtube.com/watch?v=U9PZizBDBZw} en el que colocando una serie de velas en un suelo plano y la posición de la cámara el autor nos muestra como da la sensación de que se acaba formando un cubo en 3 dimensiones sobre el que es capaz de sentarse.

Otro video fascinante es el de las potencias de 10 \url{https://www.youtube.com/watch?v=fbCwkfrKuaw} en el que nos muestran un “zoom out” con 10 elevado a n veces para salir al espacio y un “zoom in” con $\rfrac{1}{10}^n$ para adentrarnos en el organismo de las personas. El zoom out es otra manera de explicar los Sistemas de Información Geográfica como puede ser el de Google Maps.

Recursos de videos educativos pueden ser:
\begin{itemize}
\item El canal derivando de Youtube que son videos de Eduardo Sáenz de Cabezón \url{https://www.youtube.com/channel/UCH-Z8ya93m7_RD02WsCSZYA} 

\item “La pizarra de Fonemato” (www.matematicasbachiller.com) que contiene videos explicativos con una característica muy peculiar y es que lo explica todo muy despacio con un tono de voz serio que a la vez puede resultar cómico. 

\item El portal MatematicasIES \url{http://matematicasies.com} creado por Daniel López Avellaneda, Licenciado en Ciencias Matemáticas por la Universidad de Granada y Profesor de Matemáticas y Coordinador TIC en el IES Mar Serena. 

\item El portal Educacion 3.0 visto anterormente tiene recursos para crear videos como profesores. \url{http://www.educaciontrespuntocero.com/experiencias/recursos-para-grabar-lecciones-en-video/33017.html}  

\item Unicoos que es un portal de videos gratuitos para las asignaturas de ciencias. Enfocado a estudiantes de Secundaria, Bachillerato y universitarios.  

El portal Unicoos de YouTube proporciona algo más de 600 vídeos gratuitos para las asignaturas de Matemáticas, Física y Química. \url{https://www.youtube.com/user/davidcpv}
\end{itemize}

\paragraph{Matemáticas recreativas}
“La matemática recreativa se concentra en la obtención de resultados con actividades lúdicas, y a difundir o divulgar de manera entretenida y divertida los conocimientos o ideas o problemas matemáticos. Es un concepto tan viejo como lo son los juegos en los que interviene la lógica o de algún modo el cálculo” (Fuente: Wikipedia)

Es importante que los alumnos lleguen a ver que todos los juegos tienen una explicación matemática detrás. Llegar a sorprenderles es algo que logra captar su atención. Si recordamos en la página www.divulgamat.net hay una sección llamada Sorpresas Matemáticas en la que podremos encontrar bajo el menú principal recursos de este tipo


 
Un video que \textbf{llega a sorprender} a los alumnos es el de “crear chocolate de la nada” \url{https://www.youtube.com/watch?v=Y13tSEyOqGs.} En el video se consigue partir el chocolate y luego volver a reconstruir dando la sensación de que sobra una onza de chocolate. En realidad no se crea chocolate, es un truco creado que trabaja con diferentes pendientes de la recta que son cercanas y se puede manipular para aparentar que vuelve a su estado natural cuando no es cierto. La explicación detallada está en este otro video \url{https://www.youtube.com/watch?v=eb2hCmc2xso}

 

\paragraph{Anamorfismo y futbol\\}
 

\begin{minipage}[h]{1\linewidth}
	\centering
	\includegraphics[width=0.7\linewidth]{img/anamorf1.jpg}
\end{minipage}
 
Otro ejemplo de matemáticas recreativas es el del anamorfismo. Consiste en deformar la imagen a través de efectos ópticos o a través de un procedimiento matemático con perspectivas. Uno de los artistas más destacados utilizando esta técnica es Julian Beever. “Julian es un artista británico que se dedica a dibujar con tiza. Ha creado dibujos de tiza en 3D en el pavimento utilizando un método llamado anamorfosis que crea una ilusión óptica. Sus dibujos en las calles desafían las leyes de la perspectiva. Ha logrado una técnica que le da un gran realismo a la imagen” (Fuente: Wikipedia).

\begin{minipage}[h]{1\linewidth}
	\centering
	\includegraphics[width=0.7\linewidth]{img/anamorf2.jpg}
	
\end{minipage}
 
Un ejemplo de anaformismo de un cubo de Rubik en video se puede ver en  \url{https://www.youtube.com/watch?v=ooY7Mf0JlNM.} En este video nos permiten incluso acceder a la imagen que permite hace dicho anaformismo. Se puede descargar de \url{https://docs.google.com/file/d/0B3gyYFZJgwKVZ24zWDV1VVB5Wms/edit.} De hecho, me he descargado la imagen.

\begin{minipage}[h]{1\linewidth}
	\centering
	\includegraphics[width=0.7\linewidth]{img/anamorf3.png}
\end{minipage}
 
Por último, nos preguntaremos que tiene que ver el futbol con el anamorfismo. Pues bien, una vez visto que a la técnica que permite crear esta ilusión óptica se le llama anamorfismo, decir que en el futbol, principalmente en los partidos de primera división aprovechando el angulo de proyección de la grabación de las cámaras de televisión “colocan” la publicidad pintada en el plano para producir un efecto en 3D.

\begin{minipage}[h]{1\linewidth}
	\centering
	\includegraphics[width=0.7\linewidth]{img/anamorf4.jpg}
\end{minipage}
 
\paragraph{El cine y la literatura como recursos didacticos}
Por último, hay una gran cantidad de literatura matemática. No por ser matemáticos debemos olvidar la literatura. De hecho la convivencia entre ambas es fundamental.

Los siguientes enlaces tienen un montón de recursos literarios matemáticos:
\begin{itemize}
\item \url{http://aulamatematica.com/libros/libros_recomendados.htm}  

\item \url{http://www.librosmaravillosos.com/} en el que hay un buscador de libros gratuitos de difusión científica. 
\end{itemize}
El cine también ha servido como fuente de inspiración para muchos directores y guionistas a la hora de difundir las matemáticas:
\begin{itemize}
\item Una mente maravillosa 

\item El código Da Vinci 

\item Black Jack 

\item La vida es bella 

\item Cube 

\item Agora 

\item Contact 

\item Blade Runner 

\item El día de la bestia 

\item Moebius 

\item 3:19 

\item Granujas de medio pelo 
\end{itemize}
O como el caso de la “Jungla de Cristal 2” donde los protagonistas tienen que resolver el problema de las garrafas de 3 y 5 galones de agua. Para desactivar la bomba tienen que conseguir 4 galones exactos en una de las garrafas. ¿Cómo lo conseguirán? Aunque en la película no se explica claramente, se consigue (Fuente: \url{http://www.sociedadmatematicacantabria.es/Probl_Olimpiada/Sol_probl_3_2.htm}):
\begin{itemize}
\item 1º Llenas la de 5 y echas lo que puedas en la de tres.  

Quedan 3L en la de 3 y 2L en la de 5

\item 2º Vacías la de 3 y echas los 2L de la de 5 en la de 3.  

Quedan: 2L en la de 3 y 0L en la de 5

\item 3º LLenas la de 5 y echas lo que puedas en la de tres (1L)  

Quedan: 3L en la de 3 y 4L en la de 5

\item 4º Vacías la de 3 y ya tienes 4L en la de 5 
\end{itemize}

\paragraph{Khan Academy \url{https://es.khanacademy.org}}
Por último, no quería dejar sin mencionar en el portafolios la aportación realizada en el foro  por mi compañero de grupo Victor De Juan sobre la “Khan Academy”.

Khan Academy es una web que ofrece ejercicios de práctica, videos instructivos y un panel de aprendizaje personalizado que permite a los alumnos aprender a su propio ritmo, dentro y fuera del salón de clases. Y todo ello sin pagar ni un duro.

En este video de la Universidad Politécnica de Valencia nos explican cómo comenzar a usar esta web tanto si somos alumnos, profesores o padres. \url{https://www.youtube.com/watch?v=FvacPlqEw6g}

 


\end{opin}

\begin{opin}{\victorcolor}{Víctor}

Hoy ha sido un bombardeo de recursos para utilizar en clase. Paginas web, blogs, educalab... 
%
A día de hoy se me queda lejano porque no he profundizado sobre los recursos. 
%
Parece que voy a tener que estudiarme y bucear por todos estos recursos para descubrir cuáles me gustan más, cuáles me parece que pueden ser más útiles, etc.
%
La cantidad de tiempo invertido en filtrar 


\subsubsection{Ideas interesantes}

\begin{itemize}
	\item ¿Porqué no, hablar todos los años de quién ha ganado la medalla Fields?
\end{itemize}

\end{opin}

\begin{opin}{\pedrocolor}{Pedro}

.


\end{opin}

\begin{opin}{\virgicolor}{Virginia}
.


\end{opin}
