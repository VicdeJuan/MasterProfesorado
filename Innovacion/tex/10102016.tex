
\begin{opin}{\guscolor}{Gustavo}


\subsubsection{¿Qué es innovar en educación para ti?}
Innovar en educación para mí sería buscar cómo cambiar la forma tradicional de impartir las asignaturas para intentar que los alumnos se sientan más atraídos y puedan retener lo aprendido si no es para el resto de su vidas, al menos durante el mayor tiempo posible y así evitar lo que ocurre en muchos casos de que los alumnos aprueban los exámenes y se olvidan.


\subsubsection{Expectativas iniciales de la asignatura}
Personalmente considero que me encuentro entre ese perfil de alumno que aprueba un examen y se olvida casi por completo de  lo estudiado. Es cierto que esta situación se acentuaba cuando las asignaturas de las que me examinaban eran de memorizar. Con las matemáticas era algo diferente porque era necesario conocer la base para seguir avanzando en los cursos posteriores, los cuales además servían de recordatorio de lo estudiado.


Dicho esto, cuando leí el título de esta asignatura (\textit{Innovación Educativa y TICs aplicadas a la enseñanza de las Matemáticas}) me dio la sensación de que me iba a conocer cuáles son las nuevas metodologías de trabajo innovadoras que se están poniendo de moda y que son tan eficaces para el aprendizaje como las metodologías tradicionales. Estas nuevas metodologías docentes incluirían cambios drásticos con respecto a la metodología tradicional. También me imaginaba que se haría mucho hincapié en el uso de las nuevas tecnologías dado que facilita la labor para el docente y es una herramienta muy práctica en el aprendizaje.


Sobre el uso de las TICs en general me gustaría añadir una opinión personal. Tengo la sensación de que hay una tendencia generalizada de pensamiento que opina que con las TICs se va a poder hacer todo. Considero que hay que tener mucho cuidado con el uso de las tecnologías para todo. Las TICs sólo sirven cuando ayudan a realizar una determinada labor. Por poner un ejemplo simple, hay ocasiones en las que las TICs son tan complicadas de usar que no solo no ayudan sino que perjudican.

\end{opin}

\begin{opin}{\victorcolor}{Víctor}
.


\end{opin}

\begin{opin}{\pedrocolor}{Pedro}

.


\end{opin}

\begin{opin}{\virgicolor}{Virginia}
.


\end{opin}
