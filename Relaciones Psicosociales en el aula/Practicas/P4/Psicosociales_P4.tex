\documentclass[palatino,nochap]{apuntesURJC}

\title{Relaciones Psicosociales en el aula}
\subtitle{Práctica 4: La comunicación no verbal y su papel en las relaciones interpersonales y sociales.}
\author{Víctor de Juan}
\date{16/17 C2}

\newcommand{\makeheader}[1]{
\begin{center}
\Large \textbf{\textsc{Relaciones Psicosociales en el Aula}}\\
\end{center}
\begin{center}
\large Curso 2016-17 - 2 Cuatrimestre\\
Prof. Carlos María Alcover
\end{center}

\begin{center}
\Large \textbf{\textsc{Ficha de Prácticas}}
\end{center}

\begin{center}
\begin{tabular}{ll}
\hspace{2cm}\textbf{Nombre:} Víctor &  \hspace{1.5cm} \textbf{Apellidos:} de Juan Sanz\\
\vspace{0.3cm}&\\
\textbf{Grupo:} Matemáticas 	& \textbf{DNI:} %\input{../.DNI}}
 \hspace{3cm} \textbf{Fecha:} #1
\end{tabular}
\end{center}
}

% Paquetes adicionales

% --------------------

\begin{abstract}
En este documento se comenta el documental \textit{Los secretos del lenguaje no verbal} de \textit{Discovery Channel} como actividad de la asignatura \textit{Relaciones Psicosociales en el Aula}.
%
Se analizarán las posibles aplicaciones de este tema a la profesión docente.


\end{abstract}

\begin{document}

\pagestyle{plain}
\maketitle

\makeheader{\today}

\section{Breve contextualización teórico / conceptual}
\label{intro}

La investigación revela que el 93\% de la comunicación se produce por canales no verbales, es decir, que las palabras elegidas en la comunicación sólo transmiten un 7\% de la información.
%
Además, ese 93\% se divide en un 55\% del lenguaje corporal y un 38\% de la voz.

Esta afirmación llama la atención y alerta sobre la necesidad de estudiar el lenguaje corporal y aprender a controlarlo para poder comunicar mejor.
%
Los políticos y actores, que suelen ser observados minuciosamente, son un ejemplo para estudiar el lenguaje corporal. 
%
El documental \textit{Los secretos del lenguaje corporal} se basa en ellos para su exposición.

Actores, políticos y demás personas famosas trabajan y cuidan su lenguaje corporal. 
%
Los actores pueden haber aprendido durante su formación y tenerlo más fácil, pero los políticos necesitan preparadores. 
%
De hecho, esta preparación se puede entrever en las apariciones de determinados políticos, como sería Margaret Tacher y su tono de voz.

\section{Breve descripción del caso práctico / documental / ejercicio / experiencia / estudio}

\subsection{Descripción}

El documental, como ya hemos mencionado, expone la importancia de la comunicación no verbal estudiando diversos acontecimientos y conversaciones políticas.
%
Además, da ciertas claves sobre la importancia de algunos elementos.
%
Trata primero sobre el cuerpo (55\% de la comunicación) con temas como ¿Quién pasa primero por la puerta? ¿Quién tiene la mejor posición en un apretón de manos? ¿Cómo están sentados?

En el documental también cuentan una breve contextualización histórica explicando que fue a partir de 1960, tras el debate entre Nixon y Kenedy que se empezó a tener muy en cuenta la comunicación no verbal.


\subsection{Resultados o respuestas dadas}

A continuación exponemos una síntesis de las muestras de lenguaje corporal que se estudian en el documental.

\begin{itemize}
\item Las manos detrás de la espalda muestran tranquilidad y sinceridad.
\item El autocontacto, ya sea claro como tocarse la cara o sujetarse la muñeca cuando las manos están detrás de la espalda, indica nerviosismo e intento de autoconvencimiento.
\item Sujetarse al atril o lo que se tenga a mano de manera tensa o negar con la cabeza denota una actitud defensiva.
\item Si al afirmar rotundamente algo se echa el cuerpo hacia atrás se transmite una información contradictoria con el cuerpo y con las palabras. La afirmación podría no ser cierta o podría ser que el interlocutor estuviera incómodo.
\item Cruzar los brazos es sinónimo de protección porque puede haber algo escondido.
\end{itemize}


\subsection{Analiza las posibles aplicaciones prácticas en el aula}

Las aplicaciones reales al aula son varias.

\paragraph{Profesor como receptor:} 
%
En la comunicación con los alumnos puede resultar de gran importancia reconocer si puede haber algo más de lo que se  está transmitiendo con palabras.
%
Si el docente sospecha de que un alumno ha podido copiar en un examen, analizar el lenguaje no verbal para saber hacia dónde ir con las preguntas tras detectar posibles incongruencias es una gran habilidad a desarrollar por su potencial utilidad.
%
Al igual que en el caso de la copia en un examen, en cualquier otra infracción de la normativa.

\paragraph{Profesor como emisor:}
%
Es importante que nuestro lenguaje no verbal sea congruente con el verbal.
%
Este lenguaje no verbal puede ser analizado inconscientemente por los estudiantes creándoles una sensación de desconfianza si nuestro lenguaje no verbal transmite una información incongruente con nuestro lenguaje verbal.
%
En situaciones como tranquilizar y relajar los ánimos de los estudiantes en épocas de exámenes o explicar con asertividad las decisiones tomadas puede resultar clave la congruencia entre los lenguajes.

También puede resultar interesante, de cara a transmitir a los alumnos una decisión de la dirección del centro con la que el docente no está de acuerdo, tomar conciencia de que el lenguaje no verbal podría delatar la postura del docente y tal vez, que los estudiantes sepan que el docente no está de acuerdo con la dirección del centro, no está en los objetivos de nadie.



\section{Reflexión personal y comentarios sobre la práctica realizada}

Este documental me ha resultado muy llamativo. 
%
Abre la puerta al mundo del estudio del lenguaje no verbal, tema desconocido anteriormente.
%
Me ha motivado a seguir profundizando en el tema, estudiar el lenguaje no verbal en situaciones cotidianas, tanto el transmitido como el recibido.

\section{Sugerencias}



%\bibliographystyle{alpha}
%\bibliography{../../PsicoBib}  
%\label{bibliografia}

\printindex
\end{document}

