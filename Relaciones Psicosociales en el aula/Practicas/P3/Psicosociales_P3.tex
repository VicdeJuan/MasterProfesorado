\documentclass[palatino,nochap]{apuntesURJC}

\title{Relaciones Psicosociales en el aula}
\subtitle{Práctica 3: Detección, medida y reducción del prejuicio.}
\author{Víctor de Juan}
\date{16/17 C2}

\newcommand{\makeheader}[1]{
\begin{center}
\Large \textbf{\textsc{Relaciones Psicosociales en el Aula}}\\
\end{center}
\begin{center}
\large Curso 2016-17 - 2 Cuatrimestre\\
Prof. Carlos María Alcover
\end{center}

\begin{center}
\Large \textbf{\textsc{Ficha de Prácticas}}
\end{center}

\begin{center}
\begin{tabular}{ll}
\hspace{2cm}\textbf{Nombre:} Víctor &  \hspace{1.5cm} \textbf{Apellidos:} de Juan Sanz\\
\vspace{0.3cm}&\\
\textbf{Grupo:} Matemáticas 	& \textbf{DNI:} %\input{../.DNI}}
 \hspace{3cm} \textbf{Fecha:} #1
\end{tabular}
\end{center}
}

% Paquetes adicionales

% --------------------

%\begin{abstract}
%Este documento incluye una aplicación práctica de los conocimientos adquiridos en la teoría de la asignatura \thetitle, en concreto sobre la teoría de atribuciones de Kelley.
%\end{abstract}

\begin{document}

\pagestyle{plain}
\maketitle

\makeheader{\today}

\section{Breve contextualización teórico / conceptual}
\label{intro}

Esta práctica trata sobre los prejucios.

Sería aplicable al estudio del prejuicio el marco acerca de la estructura de las actitudes, tradado en el tema anterior \cite{UNEDPrejuicio}.
%
Desde esta aproximación, el prejuicio, como cualquier actitud, estaría formado por tres componentes: un componente afectivo (sentimientos y emociones que suscitan las personas a las que se dirige el prejuicio), un componente cognitivo (denominado estereotipo) y un componente conductual (conocido como discriminación).

Utilizaremos esta concepción del prejuicio como \textit{actitud prejuiciosa} y, dado que lo consideramos como una actitud, a la hora de medirlo se nos plantean algunos problemas.

Dado que el prejuicio es la base de la discriminación y dado que la discriminación es generadora de conflictos, parece importante y razonable estudiar los prejuicios. 
%
Además, para abordarlo desde un punto de vista científico es importante su medición, pero aquí, como con las actitudes encontramos un problema: ante una conducta no discriminadora existen 2 posibilidades: la ausencia de prejuicios o la disonancia.
%
En los términos manejados actualmente se distinguen 2 tipos de prejuicios: prejuicio manifiesto y prejuicio sutil.


¿Puede el prejuicio sutil desembocar en actitudes manifiestamente discriminadoras?


\subsection{Procesos y/o variables implicadas en la práctica}


\subsection{Breve definición teórica y características más destacadas}


\section{Breve descripción del caso práctico / documental / ejercicio / experiencia / estudio}

\subsection{Descripción}

Como en cualquier otro contexto social, en los contextos educativos los
prejuicios (raciales, sexistas, homófobos, etc.) pueden detectarse con relativa
frecuencia.

Esta práctica se centra en los prejuicios raciales y étnicos, que pueden
vincularse con conductas xenófobas, discriminatorias o agresivas y violentas
hacia las personas o los grupos que son objeto de dichos prejuicios.
Como material complementario a esta práctica se adjunta un ejemplo de
cuestionario que mide prejuicio racial sutil y explícito (Pettigrew y Meertens,
1995).


\subsection{Resultados o respuestas dadas}

\paragraph{1. Diseña una intervención para detectar posibles prejuicios raciales/étnicos en el aula.} Es decir, además de un cuestionario de auto-respuesta como el que se adjunta, qué otras herramientas utilizarías para detectarlo entre los/as estudiantes..

Es probable que la propia introducción de la actividad predisponga a los alumnos a contestar favorablemente. 
%
Esto es lo que se llama \textit{reactividad del sujeto} y es una amenaza a la validez de la investigación. 
%
Al exponer que vamos a realizar una medida del prejuicio en clase podría ocurrir que los alumnos contestaran más favorablemente a lo socialmente aceptado y establecido que es una actitud no prejuiciosa.
%
Por otro lado, ocultar el motivo de la investigación no es una práctica muy ética desde un punto de vista experimental.
%
Sin embargo, es verdad que, dada la edad de los sujetos, podrían no reaccionar al tema que tratamos y que no les sesgara las respuestas.

Lo ideal sería combinar la  escala (Pettigrew y Meertens,1995)  con actividades en las que se pusieran demanifiesto prejuicios sutiles en otros contextos diferentes a rellenar un cuestionario. 
%
Por ejemplo, podríamos crear un sociograma de la clase y ver si ese posible prejuicio sutil se traduce en una menor deseabilidad de estudiantes extranjeros como compañeros de trabajo.
%
De hecho se podrían realizar más de un sociograma con distintas preguntas, relativas a las familias de los compañeros, etc.





\paragraph{2. Suponiendo que hubieras detectado la existencia de prejuicios
raciales/étnicos en el aula, diseña una intervención para reducir o eliminar estos prejuicios} Es decir, actividades en el aula, fuera del aula, actividades académicas,
deportivas, lúdicas, etc.

Es necesario proceder con precaución a la hora de evidencia que se ha detectado la existencia de prejuicios.
%
No es lo mismo que se haya producido una discriminación manifiesta (en la que se tiene que trabajar evidenciándola y denunciándola) a que se haya descubierto prejuicio utilizando un cuestionario, en el que puede ser contraproducente tildar de racista la actitud de algún alumno.

En caso de que el prejuicio se haya manifestado en un incidente concreto, es necesaria una intervención.
%
En caso de que el prejuicio no se manifieste claramente, sino que sea sutil, creo que la mejor manera de trabajarlo es de una manera continuada. 
%
Es más efectivo el ejemplo (la actitud no prejuiciosa) y el envío continuo de estímulos no prejuiciosos a los alumnos que el desarrollo de actividades puntuales.
%
Ambas estrategias son complementarias pero las actividades puntuales no tienen sentido sino existe una intervención continuada.
%
Por ello, más que diseñar actividades puntuales, nos centraremos más en estrategias que pueden ser aplicadas de manera continuada (aunque alguna admita intervenciones puntuales).


Dovidio y Caertner (2004) sugieren diferentes estrategias para combatir el racismo, tanto a nivel individual como intergrupal.
%
Algunas son consideradas por los propios autores más eficaces que otras.
%
En general, los estudios ponen de manifiesto que \textbf{hacer que las personas sean conscientes de su prejuicio} es un método muy eficaz para reducirlo en quienes defienden explícitamente principios igualitarios y a la vez poseen sesgos implícitos.

Por otro lado, si el prejuicio genera problemas a nivel grupal, se puede adoptar la estrategia de la \textit{recategorización}. 
%
Esta estrategia consiste en inducir a miembros de grupos diferentes a imaginar que forman parte de un único grupo y no de grupos separados.
%
De esta manera estaríamos rompiendo el estereotipo en el que se basa el prejuicio.
%
Por ejemplo, tratar de generar una conciencia de clase. Todos los alumnos son de la misma clase y eso es lo que les une.
%
Es más importante su categoría de clase que la de etnia familiar.

Otra estrategia a aplicar sería la \textit{descategorización}, que consiste en procurar que los miembros de los grupos discriminados (normalmente en una clase ese grupo es minoritario) sean percibidos como individuos y no como miembros de un grupo.
%
Esto consigue crear ejemplares que se salen del estereotipo y reduciría el conflicto en clase.
%
Esta estrategia se conseguiría haciendo evidente el papel como alumnos de esos miembros y dejando de lado su condición racial/étnica.
%
Esta estrategia plantea un problema: no se trabaja el prejuicio a nivel general, sino a nivel ejemplar. 
%
Por ello, sería necesario complementarla con otras que ayudaran a renovar la categoría del estereotipo incluyendo como habituales las características de los ejemplares.

Por último, vídeos como el visto en clase sobre los estereotipos utilizan una \textit{categorización cruzada} en la que se evidencian categorías compartidas entre los miembros del grupo minoritario y los del grupo mayoritario.

\subsection{Analiza las posibles aplicaciones prácticas en el aula}

Este apartado ya ha sido comentada anteriormente.

\section{Reflexión personal y comentarios sobre la práctica realizada}


\section{Sugerencias}



\bibliographystyle{alpha}
\bibliography{../../PsicoBib}  
\label{bibliografia}

\printindex
\end{document}

