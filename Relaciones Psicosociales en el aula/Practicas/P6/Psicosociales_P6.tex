\documentclass[palatino,nochap]{apuntesURJC}

\title{Relaciones Psicosociales en el aula}
\subtitle{Práctica 4: La comunicación no verbal y su papel en las relaciones interpersonales y sociales.}
\author{Víctor de Juan}
\date{16/17 C2}

\newcommand{\makeheader}[1]{
\begin{center}
\Large \textbf{\textsc{Relaciones Psicosociales en el Aula}}\\
\end{center}
\begin{center}
\large Curso 2016-17 - 2 Cuatrimestre\\
Prof. Carlos María Alcover
\end{center}

\begin{center}
\Large \textbf{\textsc{Ficha de Prácticas}}
\end{center}

\begin{center}
\begin{tabular}{ll}
\hspace{2cm}\textbf{Nombre:} Víctor &  \hspace{1.5cm} \textbf{Apellidos:} de Juan Sanz\\
\vspace{0.3cm}&\\
\textbf{Grupo:} Matemáticas 	& \textbf{DNI:} %\input{../.DNI}}
 \hspace{3cm} \textbf{Fecha:} #1
\end{tabular}
\end{center}
}

% Paquetes adicionales

% --------------------

\begin{abstract}


\end{abstract}

\begin{document}

\pagestyle{plain}
\maketitle

\makeheader{\today}

\section{Breve contextualización teórico / conceptual}
\label{intro}

\section{Breve descripción del caso práctico / documental / ejercicio / experiencia / estudio}

\subsection{Descripción}

La tarea que se plantea en esta práctica consiste en la búsqueda de recursos
de información, asesoramiento o ayuda en Internet sobre el acoso escolar o
bullying, con un doble objetivo:
\begin{enumerate}
\item Que los/as docentes conozcan los recursos accesibles.
\item Que puedan, en su caso, plantear a los estudiantes de Secundaria y Bachillerato esta búsqueda como forma de sensibilización, concienciación e información sobre este fenómeno.
\end{enumerate}

\subsection{Resultados o respuestas dadas}


\subsection{Analiza las posibles aplicaciones prácticas en el aula}


\section{Reflexión personal y comentarios sobre la práctica realizada}


\section{Sugerencias}



%\bibliographystyle{alpha}
%\bibliography{../../PsicoBib}
%\label{bibliografia}

\printindex
\end{document}

