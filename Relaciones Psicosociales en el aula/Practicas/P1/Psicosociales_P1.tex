\documentclass[palatino,nochap]{apuntesURJC}

\title{Relaciones Psicosociales en el aula}
\subtitle{Práctica 1: Atribuciones y teoría de Weiner}
\author{Víctor de Juan}
\date{16/17 C2}

\newcommand{\makeheader}[1]{
\begin{center}
\Large \textbf{\textsc{Relaciones Psicosociales en el Aula}}\\
\end{center}
\begin{center}
\large Curso 2016-17 - 2 Cuatrimestre\\
Prof. Carlos María Alcover
\end{center}

\begin{center}
\Large \textbf{\textsc{Ficha de Prácticas}}
\end{center}

\begin{center}
\begin{tabular}{lll}
\textbf{Nombre:} Víctor &  \textbf{Apellidos:} de Juan &\\
\vspace{0.3cm}&&\\
\textbf{Grupo:} Matemáticas 	& \textbf{DNI:} \input{../.DNI} & \textbf{Fecha:} #1
\end{tabular}
\end{center}
}

% Paquetes adicionales

% --------------------

\begin{document}

\pagestyle{plain}
\maketitle

\makeheader{\today}

\section{Breve contextualización teórico / conceptual}

La teoría de Weiner sobre atribuciones se enmarca dentro de la cognición social.

Una atribución consiste en inferir las causas que han provocado una determinada acción de otra persona para poder predecir más eficazmente en el futuro.
%
Si una persona ha llevado a cabo una acción que nos ha resultado perjudicial es importante conocer cuál ha sido la causa para predecir futuros comportamientos.
%
No es lo mismo considerar que es muy probable que esa persona se vuelva a comportar así (lo que resulta perjudicial) que si es muy poco probable que esa conducta se vuelva a repetir (lo que no resultaría perjudicial para nosotros).

Este es un proceso importante que se puede observar desde distintos puntos de vista. %
Por ejemplo, desde un punto de vista evolucionista, el proceso de atribuciones resulta adaptativo (tal vez para nuestros ancestros supuso una mejora en la supervivencia y en la propagación de los genes). 

\subsection{Procesos y/o variables implicadas en la práctica}

En esta práctica nos centraremos en la teoría de Weiner sobre atribuciones, ya que es la más aceptada en la Psicología hoy en día.

\subsection{Breve definición teórica y características más destacadas}

Esta teoría establece que hay 3 variables que se evalúan a la hora de realizar una atribución. 
%
Éstas son: Consenso, Consistencia y Distintividad.

La primera establece si la conducta del autor es similar a la conducta de los demás autores potenciales.
%
La consistencia se refiere a si el autor realiza la misma conducta siempre que se dan condiciones similares.
%
Por último, la distintividad hace referencia al grado de similitud entre la conducta expresada por el autor hacia el sujeto de la acción y las conductas expresadas por el autor hacia otros sujetos de la acción. 
%
Hablaremos de \textit{distintividad alta} cuando la similitud sea escasa, es decir, la conducta realizada sea poco parecida a las conductas expresadas hacia otros sujetos en condiciones similares.

\section{Breve descripción del caso práctico / documental / ejercicio / experiencia / estudio}

\subsection{Descripción}
El caso práctico propone establecer conscientemente la valoración (alta o baja) de las 3 variables influyentes en el proceso de atribución según la teoría Weiner en una serie de casos prácticos preestablecidos.
%
En 2 de esos casos se valorarán las variables y en un último caso, se creará un posible escenario en el que las variables tomen unos valores (alto/bajo) previamente definidos.

\subsection{Resultados o respuestas dadas}

En la siguiente página incluimos las respuestas dadas al planteamiento.

\includepdf{SolucionP1.pdf}

\subsection{Analiza las posibles aplicaciones prácticas en el aula}

Esta teoría es aplicable a cualquier conducta realizada por cualquier autor, por lo que hay que distinguir varios puntos de vista:

\paragraph{Punto de vista del profesor: conductas de los alumnos} A la hora de atribuir causas a conductas (desde comportamiento en clase hasta rendimiento en exámenes) es importante no hacer estas atribuciones en procesos automáticos, sino dedicarle un esfuerzo cognitivo mayor. 
%
Esta teoría establece un marco en el que las atribuciones pueden realizarse de una manera menos automática y, de esta manera, se pueden llegar a realizar atribuciones más precisas que guíen la acción del profesor mejor. 

\paragraph{Punto de vista de los alumnos: conductas del profesor} Como profesores,  los alumnos realizarán atribuciones de nuestra conducta. 
%
Es importante transmitir que el trato dado a todos los alumnos es el mismo y que el grado de distintividad de nuestras conductas es bajo. 
%
Más importante todavía, como estamos estudiando en otras asignaturas, es fundamental la consistencia. 
%
Nuestras conductas tienen que tener un grado de consistencia muy alto.

Teniendo en cuenta esta teoría, y dado que las atribuciones que hagan nuestros alumnos sobre nuestras acciones podrían guiar su conducta, sería ideal que nuestra conducta pudiera guiarles a realizar (tal vez inconscientemente) atribuciones más precisas.

El grado de consenso no es tan fundamental. 
%
Sería ideal que todos los profesores trataran de forma similar a todos los alumnos, pero no todos los alumnos se comportan igual con todos los profesores ni todos los alumnos tienen las mismas capacidades para llevar a cabo con éxito las tareas de cada asignatura. 

\section{Reflexión personal y comentarios sobre la práctica realizada}

Es interesante el planteamiento de esta teoría sobre casos como el rendimiento de un alumno y las consecuencias que ésto pueda tener sobre la autopercepción del alumno.

Si un alumno obtiene un suspenso en un examen y la atribución causal que establece se debe a causas internas (se daría con un bajo consenso [no todos han suspendido], una alta distintividad [sólo suspendo matemáticas] y una alta consistencia [suspendo todos los exámenes]) pueden iniciarse procesos de descenso de autoeficacia percibida.
%
De cara a romper con esta posible espiral, poniendo un exámen fácil en el que pudiera aprobar (que sería lo primero que podríamos pensar),  no se solucionaría del todo el problema.
%
El alumno atribuiría su aprobado a causas externas: un examen más fácil de lo normal crea unas circunstancias diferentes a las anteriores, por lo que el proceso de atribución sería distinto. No estaríamos ante un examen de Matemáticas normal en el que el alumno creería que suspende, sino antes un examen fácil de Matemáticas.


\section{Sugerencias}



\bibliographystyle{alpha}
\bibliography{PsicoBib}  
\label{bibliografia}

\printindex
\end{document}

