\documentclass[palatino,nochap]{apuntesURJC}

\title{Relaciones Psicosociales en el aula}
\subtitle{Práctica 5: Conductas de liderazgo y participación en grupos.}
\author{Víctor de Juan}
\date{16/17 C2}

\newcommand{\makeheader}[1]{
\begin{center}
\Large \textbf{\textsc{Relaciones Psicosociales en el Aula}}\\
\end{center}
\begin{center}
\large Curso 2016-17 - 2 Cuatrimestre\\
Prof. Carlos María Alcover
\end{center}

\begin{center}
\Large \textbf{\textsc{Ficha de Prácticas}}
\end{center}

\begin{center}
\begin{tabular}{ll}
\hspace{2cm}\textbf{Nombre:} Víctor &  \hspace{1.5cm} \textbf{Apellidos:} de Juan Sanz\\
\vspace{0.3cm}&\\
\textbf{Grupo:} Matemáticas 	& \textbf{DNI:} %\input{../.DNI}}
 \hspace{3cm} \textbf{Fecha:} #1
\end{tabular}
\end{center}
}

% Paquetes adicionales

% --------------------

\begin{abstract}
En esta práctica se trabaja el concepto de Liderazgo como dinámica Psicosocial dentro del aula.
%
Para ello, se trabajará la Práctica 5 de la asignatura.
\end{abstract}


\begin{document}

\pagestyle{plain}
\maketitle

\makeheader{\today}

\section{Breve contextualización teórico / conceptual}
\label{intro}

\begin{defn}[Liderazgo]
El liderazgo es una interacción entre dos o más
miembros de un grupo que implica una
estructuración o reestructuración de la situación y
de las percepciones y expectativas de los miembros.
Los líderes son agentes de cambio –personas cuyos
actos afectan a otras personas–. El liderazgo se
produce cuando un miembro del grupo modifica la
motivación o competencias de los demás
miembros.
\end{defn}

Esta interacción se construye sobre una relación se construye sobre una relación de influencia en la que los líderes y los seguidores son actores persiguiendo cambios reales y desarrollando propósitos mutuos.

El tema del liderazgo se estudia mucho desde el ámbito de los recursos humanos, debido a que hay muy diversas formas de liderazgo.
%
Un líder puede ser más enfocado a la tarea y a los resultados, mientras que otros pueden estar enfocados al mantenimiento de las relaciones.
%
Estos 2 liderazgos no son excluyentes ni tampoco son los únicos.
%
Actualmente existen varios modelos de liderazgo que incorporan diversos tipos.
%
En el ámbito de la formación de Coordinadores de Tiempo libre también resulta importante tener esta formación sobre liderazgo y conocer cuál es tu liderazgo predilecto.

Además, desde el punto de vista de la labor docente es importante entender los procesos de liderazgo, como comentarles más adelante.


\section{Breve descripción del caso práctico / documental / ejercicio / experiencia / estudio}

\subsection{Descripción}

Las personas cuando participamos en grupos podemos realizar diferentes tipos de comportamientos, y en función de ellos ejercer en mayor o menor medida roles de liderazgo.

Como hemos estudiado en el Tema 5, la clasificación de las conductas de liderazgo orientadas a la tarea y orientas al mantenimiento (funcionamiento y ambiente social-interpersonal del grupo) es una de las más utilizadas, y su combinación permite identificar estilos de liderazgo diferenciados.

En el documento anexo se incluye un cuestionario de auto-aplicación dirigido a analizar este tipo de conductas. La práctica consiste en tomar como referente tu conducta habitual o general en situaciones grupales e identificar qué tipo de conductas tiendes a realizar y qué tipo de liderazgo implican.

Al final del documento aparece un cuadrante donde puedes situar, en función la puntuación obtenida en ambos tipos de conductas, el estilo de liderazgo que tiendes a adoptar cuando formas partes de grupos.


\subsection{Resultados o respuestas dadas}

\begin{table}[htbp]
\centering
\begin{tabular}{|l|c|}
\cline{1-1}
Conducta orientada a la tarea\\\hline\hline
Da información y opinión &5\\\hline
Solicita información y opinión &5\\\hline
Define dirección y rol &4\\\hline
Resume&4\\\hline
Da energía&4\\\hline
Comprueba comprensión&3\\\hline
\textsc{Total conductas de tarea:} & 25\\\hline
\end{tabular}
\end{table}

\begin{table}[htbp]
\centering
\begin{tabular}{|l|c|}
\cline{1-1}
Conducta orientada al mantenimiento\\\hline\hline
Fomenta la participación &5\\\hline
Facilita la comunicación&5\\\hline
Relaja tensión &4\\\hline
Observa procesos&5\\\hline
Resuelve problemas interpersonales&5\\\hline
Da apoyo y alaba&5\\\hline
\textsc{Total conductas de mantenimiento:}&29\\\hline
\end{tabular}
\end{table}


\paragraph{(30,30):} 
%
Cuando se planifica o se toman decisiones juntos, todos los miembros acaban comprometiéndose con la realización de la tarea al tiempo que establecen relaciones de confianza y respeto. 
%
Se da un gran valor a las decisiones creativas que generan compresión y acuerdo; se buscan y escuchan ideas y opiniones, incluso cuando son diferentes de las de uno.
%
El grupo en conjunto define la tarea y trabaja para realizarla.
%
Se fomenta la combinación creativa de tarea y mantenimiento.


\subsection{Analiza las posibles aplicaciones prácticas en el aula}


Como en otras prácticas, este contenido tiene 2 aplicaciones, dependiendo de actores de la acción.

\paragraph{Profesor como líder:} El profesor debe ejercer un liderazgo en la clase.
%
Si no lo ejerce él lo ejercerán otros alumnos, lo cual no es como debe funcionar el aula.
%
Es importante que el docente sepa liderar la clase, diferenciando en qué momento es necesario aplicar qué liderazgo.

\paragraph{Liderazgo entre los alumnos:} Entre los alumnos también se generan líderes que tienen una gran influencia.
%
Es importante que el docente sepa reconocerlos y tratar con ellos para poder controlar la clase.
%
Tal vez la mayoría de los alumnos no quiera aceptar lo que dice el profesor.
%
Será un escenario muy distinto si los líderes aceptan lo que el profesor dice (y podrán convencer al resto de alumnos) que si los líderes son quienes no aceptan (que pueden movilizar al resto de alumnos a protestar más de lo que harían por sí solos).



\section{Reflexión personal y comentarios sobre la práctica realizada}

Este tema me ha resultado poco profundo.
%
Este año he realizado el curso de coordinación de tiempo libre y hemos podido dedicarle mucho más tiempo, profundizando más y estudiando otros modelos, realizando cuestionarios más completos para conocernos mejor.
%
Esto me permitió conocerme más y reconocer qué aspectos de mi liderazgo son más fuertes y cuáles son más débiles.

Tal vez por mi formación previa no he aprendido tanto como me hubiera gustado con este tema, en particular con el cuestionario.



\section{Sugerencias}



%\bibliographystyle{alpha}
%\bibliography{../../PsicoBib}
%\label{bibliografia}

\printindex
\end{document}

