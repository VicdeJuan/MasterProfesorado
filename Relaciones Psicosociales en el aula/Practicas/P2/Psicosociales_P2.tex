\documentclass[palatino,nochap]{apuntesURJC}

\title{Relaciones Psicosociales en el aula}
\subtitle{Práctica 2: Composición y formación de actitudes y sus relaciones con la conducta.}
\author{Víctor de Juan}
\date{16/17 C2}

\newcommand{\makeheader}[1]{
\begin{center}
\Large \textbf{\textsc{Relaciones Psicosociales en el Aula}}\\
\end{center}
\begin{center}
\large Curso 2016-17 - 2 Cuatrimestre\\
Prof. Carlos María Alcover
\end{center}

\begin{center}
\Large \textbf{\textsc{Ficha de Prácticas}}
\end{center}

\begin{center}
\begin{tabular}{ll}
\hspace{2cm}\textbf{Nombre:} Víctor &  \hspace{1.5cm} \textbf{Apellidos:} de Juan Sanz\\
\vspace{0.3cm}&\\
\textbf{Grupo:} Matemáticas 	& \textbf{DNI:} %\input{../.DNI}}
 \hspace{3cm} \textbf{Fecha:} #1
\end{tabular}
\end{center}
}

% Paquetes adicionales

% --------------------

%\begin{abstract}
%Este documento incluye una aplicación práctica de los conocimientos adquiridos en la teoría de la asignatura \thetitle, en concreto sobre la teoría de atribuciones de Kelley.
%\end{abstract}

\begin{document}

\pagestyle{plain}
\maketitle

\makeheader{\today}

\section{Breve contextualización teórico / conceptual}

\label{intro}

Las actitudes son un componente fundamental del estudio de la Psicología Social ya que son la base de la relación entre los estímulos y las respuestas.

Las actitudes tienen varias funciones en un amplio espectro de las relaciones psicosociales que no vamos a describir en este documento.
%
Para más información se puede consultar \cite{Tema2}, material base para el estudio de este tema en la asignatura \thetitle.

Lo que sí vamos a desarrollar brevemente son los componentes de las actitudes. 
%
Tradicionalmente se trabajan 3 modelos: las actitudes tienen un único componente (Thurstone), las actitudes tienen 2 componentes (Allport) y las actitudes tienen 3 componentes (Rosenberg y Hovland).
%
Vamos a utilizar el último modelo ya que es el más aceptado y pueden explicar más fenómenos que los otros 2.


\subsection{Procesos y/o variables implicadas en la práctica}

Los 3 componentes de las actitudes, atendiendo al modelo de Rosenberg y Hovland: congnitivo, afectivo y conductual.

\subsection{Breve definición teórica y características más destacadas}

Consultar \ref{intro}.

\section{Breve descripción del caso práctico / documental / ejercicio / experiencia / estudio}

\subsection{Descripción}

Las variables a medir en esta práctica son los componentes de la actitud concreta del autor de este documento hacia el deporte.
%
Utilizando el modelo de Rosenberg y Hovland, diferenciaremos los 3 componentes de la actitud.

\subsection{Resultados o respuestas dadas}

En la siguiente página incluimos las respuestas dadas al planteamiento.

\includepdf{SolucionP2.pdf}

Esta relación está incluida en la reflexión personal y comentarios sobre la práctica (sección \ref{final}).


\subsection{Analiza las posibles aplicaciones prácticas en el aula}


Todo docente tiene claro el papel de la persuasión y la importancia de convencer a los alumnos de ciertas cosas.
%
Por ejemplo, intentar despertar una motivación intrínseca\footnote{Es mucho más potente una motivación intrínseca, basada en actitudes internas que una motivación extrínseca, basada en los premios de la familia o en las buenas calificaciones obtenidas.} para los estudios es fundamental y esto podría conseguirse mediante un cambio de actitudes:
%
persuadiendo para cambiar el componente cognitivo o para inducir una conducta opuesta a la manifestada. 
%
De esta última manera, dado que la conducta también influye en la actitud, se avanzar en la dirección del cambio de actitud.

Este razonamiento es aplicable, no sólo a la motivación sino también a los comportamientos disruptivos.
%
Como docente es importante saber tratarlos, para el bien de la propia persona disruptiva como para el bien del funcionamiento de la clase.



\section{Reflexión personal y comentarios sobre la práctica realizada}

El caso del deporte es un buen ejemplo de cómo la actitud no predice exactamente la conducta expresada.
%
Es tremendamente habitual tener actitudes positivas hacia el deporte (en concreto, sus componentes cognitiva y afectiva) pero no desarrollar una conducta deportiva habitual.

\label{final}
%
En el caso del redactor no es tan apreciada esa diferencia, ya que se podría considerar una conducta deportiva poco disonante con la actitud manifiesta.
%
Sin embargo, no es infrecuente la disonancia producida en muchas personas debida a la incongruencia de sus actitudes hacia el deporte (positivas) y la conducta expresada (nula o escasa)\footnote{Es razonable presuponer que la reducción de esta disonancia se realice mediante la infravaloración de la importancia dada al deporte.}.


\section{Sugerencias}

Resultaría interesante diferenciar entre medidas implícitas y medidas explícitas.
%
Tal vez se podría realizar una práctica contestando a 2 cuestionarios: uno de medidas explícitas y otro de medidas implícitas para después compararlos y ver que puede existir ambivalencia de las actitudes dependiendo de las circunstancias.
%
Es frecuente la disonancia entre, por ejemplo, el racismo tradicional y el racismo moderno.



\bibliographystyle{alpha}
\bibliography{../../PsicoBib}  
\label{bibliografia}

\printindex
\end{document}

