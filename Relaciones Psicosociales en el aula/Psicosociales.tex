\documentclass[palatino]{apuntesURJC}

\title{Relaciones Psicosociales en el aula}
\author{Víctor de Juan}
\date{16/17 C2}

\newcommand{\makeheader}[1]{
\begin{center}
\Large \textbf{\textsc{Relaciones Psicosociales en el Aula}}\\
\end{center}
\begin{center}
\large Curso 2016-17 - 2 Cuatrimestre\\
Prof. Carlos María Alcover
\end{center}

\begin{center}
\Large \textbf{\textsc{Ficha de Prácticas}}
\end{center}

\begin{center}
\begin{tabular}{lll}
\textbf{Nombre:} Víctor &  \textbf{Apellidos:} de Juan &\\
\vspace{0.3cm}&&\\
\textbf{Grupo:} Matemáticas 	& \textbf{DNI:} \input{../.DNI} & \textbf{Fecha:} #1
\end{tabular}
\end{center}
}

% Paquetes adicionales

% --------------------

\begin{document}
\pagestyle{plain}
\maketitle

\tableofcontents
\newpage
% Contenido.

\textit{’Nadie se baña dos veces en el mismo río’}

Esta frase atribuida a Heráclito de Éfeso tiene varias interpretaciones. 
%
El agua del río cada vez es distinta, porque fluye y cambia. 
%
Además, la persona que se baña también ha cambiado de un baño a otro.



%% Apendices (ejercicios, examenes)
\appendix

% -*- root: ../Psicosociales.tex -*-


\printindex
\end{document}
