\documentclass[palatino,nochap]{apuntesURJC}

\title{Aprendizaje y Desarrollo de la personalidad}
\author{Víctor de Juan Sanz}
\date{16/17 C1}

% Paquetes adicionales

% --------------------

\begin{document}
\pagestyle{plain}
\maketitle

\tableofcontents
\newpage
% Contenido.

\section{Tema 1}

\begin{defn}[Aprendizaje]
Cambio de conducta observable más o menos
estable, causado principalmente por eventos del
ambiente
\end{defn}

\subsection{Enfoque conductual}

Tiene 2 principios básicos:
\begin{itemize}
	\item \textbf{Equipotencialidad}: similitud de aprendizaje entre humanos y animales.
	\item \textbf{Principio de parsimonia:} Explicación mediante el menor número de principios de aprendizaje.
\end{itemize}

Presenta una objetividad alta y las leyes básicas estudias la relación entre el \textit{estímulo} y la \textit{respuesta}. 
%
Al sólo relacionar estos 2 conceptos, los procesos internos quedan excluidos del proceso científico.
%
Y, dado que las respuestas se pueden condicionar, la mayoría son aprendidas.


\begin{defn}[Condicionamiento clásico]
Pasando de definirlo porque todos lo conocemos.
\end{defn}

\begin{example}
\paragraph{Actividad 1}

\paragraph{Ejercicio 1}
María trabaja en una fábrica de muñecos. Su trabajo consiste en introducir piezas en
una máquina. Cada vez que una pieza se ha introducido, la máquina hace un “click”,
lo que significa que ha enganchado la pieza. La máquina no funciona muy bien
últimamente y tras el “click”, un chorro de aire sale disparado directamente a la cara
de María, lo que provoca que ella cierre los ojos y gire ligeramente la cabeza.
Un día, María se da cuenta de que pestañea y torna la cabeza al escuchar el “click”,
aunque todavía no haya salido el chorro de aire.

\begin{itemize}
\item \textbf{Estímulo neutro/condicionado:} Click.
\item \textbf{Estímulo incondicionado:} Aire.
\item \textbf{Respuesta Condicionada:} Pestañear.
\item \textbf{Respuesta Incondicionada:} Cerrar los ojos.
\end{itemize}

\paragraph{Ejercicio 2}

Guillermo vuelve todos los días a casa de la universidad en autobús. Desde la
parada del autobús hasta llegar al portal de su casa, debe andar un kilómetro más o
menos y atravesar un pequeño túnel. Un día, un hombre le para en mitad del túnel,
le amenaza con un puñal y le pide la cartera. Guillermo tuvo un ataque de ansiedad.
Ahora, cada vez que baja del autobús y va a cruzar el túnel, tiene exactamente la
misma respuesta de miedo y ansiedad

\begin{itemize}
\item \textbf{Estímulo neutro/condicionado:} Túnel.
\item \textbf{Estímulo incondicionado:} Puñal.
\item \textbf{Respuesta Condicionada/Incondicionada:} Miedo y ansiedad.
\end{itemize}


\paragraph{Ejercicio 3}

Pablo es un niño muy sensible. Cada vez que recibe un contacto físico suave (caricia), se relaja y se siente tranquilo. Últimamente, su profesora ha sido muy cariñosa con él y en un par de ocasiones en las que ella ha percibido que Pablo estaba nervioso, se ha aproximado a él y le ha dado una caricia con la que Pablo se ha relajado. Ahora, Pablo únicamente con ver a su profesora se siente relajado.

\begin{itemize}
\item \textbf{Estímulo neutro/condicionado:} Ver a la profesora.
\item \textbf{Estímulo incondicionado:} Caricia de la profesora.
\item \textbf{Respuesta Condicionada/Incondicionada:} Tranquilizarse.
\end{itemize}



\paragraph{Ejercicio 4}
Ante un chillido, Ana siente miedo y vergüenza. EL pasado Jueves, su profesor gritó repetidas veces a Ana cuando ella realizó mal las tareas de clase y ella sintió vergüenza ante sus compañeros. A día de hoy, Ana cada vez que ve a su profesor siente miedo. Además, también siente miedo ante el resto de profesores del colegio.

\begin{itemize}
\item \textbf{Estímulo neutro/condicionado:} Ver al profesor.
\item \textbf{Estímulo incondicionado:} Gritos del profesor.
\item \textbf{Respuesta Condicionada/Incondicionada:} Miedo y vergüenza.
\end{itemize}

El fenómeno que se ha producido es una generalización al sentir miedo de los demás profesores. 
%
Ha generalizado de su profesor a los profesores.

\paragraph{Ejercicio 5}

Pedro es un chico inseguro. El pasado martes en clase de matemáticas D. Alfredo le obligó a salir a la pizarra a resolver un ejercicio. Ante esta situación, Pedro se agobió, tuvo un ataque de ansiedad y se quedó completamente bloqueado. Fue una situación muy aversiva. Ahora, cada vez que tiene clase de matemáticas siente ansiedad y no quiere acudir. Al mismo tiempo, cada vez que ve a D. Alfredo siente tensión en el cuerpo y falta de aire. Son los mismos síntomas que experimentó el día del suceso.

\begin{itemize}
\item \textbf{Estímulo neutro/condicionado I} Clase de matemáticas.
\item \textbf{Estímulo neutro/condicionado II} Ver al profesor.
\item \textbf{Estímulo incondicionado:} Salir a la pizarra.
\item \textbf{Respuesta Condicionada/Incondicionada:} Ansiedad.
\end{itemize}

\textbf{¿Qué técnica aplicarías a Pedro? ¿Cómo? Razona la respuesta.} Desensibilización sistemática para eliminar la ansiedad condicionada.

Los pasos a seguir podrían ser:

\begin{enumerate}
	\item (Si la ansiedad generada es muy muy fuerte): Imaginar a Alfredo.
	\item (Si la ansiedad generada es muy muy fuerte): Imaginar a Alfredo dando clase.
	\item Contacto visual lejano con el profesor.
	\item Contacto visual cercano con el profesor.
\end{enumerate}

\paragraph{Ejercicio 6}

Juan es un alumno muy popular en el instituto. Le gustan varios grupos de música y, entre ellos, Fito y Fitipaldis es su preferido. Siempre lleva camisetas que representan a este grupo de música. El pasado lunes, primer día de curso, acudió a clase Ana, una nueva alumna vestida con una camiseta de Fito y Fitipaldis. Juan, nada mas ver a Ana y sin hablar con ella, le ha dicho a todo el mundo que es una “chica que promete”

\begin{itemize}
\item \textbf{Estímulo neutro/condicionado:} Chica.
\item \textbf{Estímulo incondicionado:} Camiseta de Fito.
\item \textbf{Respuesta Condicionada/Incondicionada:} Le mola la chica.
\end{itemize}

El fenómeno que se ha producido es la discriminación.

\paragraph{Ejercicio 7}


Santiago acude con ciertas reservas por primera vez al instituto. Con el ajetreo ha perdido el bocadillo que su madre le ha preparado. A la hora del descanso, cuando ya el hambre se deja sentir y aburrido, se ha sentado en la escalera del patio. Marta, una compañera de su clase, se dirige a él, le acaricia, le dice cosas agradables y le da una chocolatina que casualmente lleva en el bolsillo. Pablo se siente bien y calma momentáneamente su hambre. A partir del día siguiente, a Pablo le produce alegría encontrarse con Marta a la hora del recreo


\begin{itemize}
\item \textbf{Estímulo neutro/condicionado:} Ver a la chica.
\item \textbf{Estímulo incondicionado:} La chocolatina y el cariño.
\item \textbf{Respuesta Condicionada/Incondicionada:} Alegría.
\end{itemize}

El condicionamiento es clásico.

\paragraph{Ejercicio 8}
D. Julio, en la hora de educación física, suele incluir muchos ejercicios de alta dificultad y fatiga en las espalderas. Marcos, en esta hora suele hacer el payaso y remolonear descaradamente ya que le da miedo hacer estos ejercicios. D. Julio, molesto, suele echar a Marcos del gimnasio. Marcos se va al patio donde además de evitar el miedo de los ejercicios, encuentra algún amigo con el que jugar.

\textbf{¿Qué tipo de condicionamiento se está produciendo? Razona la respuesta}\textbf{¿Existen reforzadores? ¿De qué tipo?}
Es un condicionamiento operante. 
%
A la conducta de \textit{remolonear} recibe una respuesta (pensada como castigo) que resulta ser un reforzamiento. 
%
Además, son 2 los reforzadores. Uno positivo (jugar con algún amigo) y otro negativo (evitar el miedo de los ejercicios).



\paragraph{Actividad 2}

\paragraph{Pensar en posibles acciones metodológicas, basadas en el condicionamiento clásico, que podrían llevarse a cabo en el aula para implantar un determinado contenido de su área de conocimiento. Hacer una previsión de qué resultados se espera encontrar de su puesta en práctica.}



1. Exponer siempre una misma metodología al impartir la clase en la que se explique una primera parte de teoría una parte de problemas basados en juegos matemáticos. Justo antes de comenzar la segunda parte de juegos, se pondría una canción/melodía en la que todos los alumnos identifiquen dicha canción con la parte de juegos.
Para terminar, el día del examen se podría realizar el mismo procedimiento. 

Se prevee que los alumnos tengan la misma sensación que la parte de juegos. 

2. Para conseguir que un grupo de alumnos vuelva al silencio en caso de que se hayan desconcentrado y el murmullo sea muy ruidoso, existen técnicas como la de que el profesor levante la mano y a medida que los alumnos se van dando cuenta, tendrán que ir levantando la mano y permanecer callados de tal forma que al final se calle la clase por completo.

Se prevee que con el tiempo, asocien el levantar la mano con permanecer en silencio.

\end{example}


\begin{defn}[Generalización] 
Estímulos condicionados parecidos provocan una respuesta condicionada, en mayor o menor grado dependiendo del mayor o menor grado de parecido entre los estímulos.
\end{defn}

\begin{defn}[Discriminación]
Aprendizaje de diferenciación de estímulos parecidos. 
\end{defn}

\begin{example}
Ante el mordisco de un pitbul, se desarrolla miedo al pitbul y no a los perros.  
%
Antes otros perros, estaremos \textit{discriminando el estímulo} y no se manisfestará miedo.
\end{example}

\begin{defn}[Extinción]
Acabar con un condicionamiento.
\end{defn}

Para extinguir un condicionamiento hay principalmente 2 maneras. Exposición agresiva, contracondicionamiento (o desensibilización sistemática)




\begin{defn}[Condicionamiento operante]
Aplicar en un condicionamiento técnicas de refuerzo y castigo para crear una conducta.
%

\end{defn}

Tanto los refuerzos como los castigos pueden ser positivos o negativos. 
%
Los \concept{refuerzos} son recompensas a la conducta deseada para fomentarla.
%
Positivos si son una recompensa, negativos si son la desaparición de algo desagradable.
%
Los \concept{castigos} buscan eliminar una conducta no deseada.
%
Serán positivos si se obtiene algo desagradable o negativos si se evita la apración de algo agradable.

De él se derivan las siguientes leyes:
\begin{itemize}
\item \textbf{Ley del Efecto}: “ Cuando se establece una conexión entre un estímulo y una
respuesta y ésta va seguida de una consecuencia positiva dicha conexión se fortalece.
Si le sigue una consecuencia desagradable, la conexión se debilita”.
\item \textbf{Ley del Ejercicio}: “conexión entre estímulo y respuesta será mas fuerte mientras
más se practique y se irá debilitando a medida que se deje de utilizar”. Práctica implica
mejora solo cuando seguida consecuencia ( Sprinthall, Sprinthall, Oja 1996).
\item \textbf{Ley de la preparación}: “posesión del organismo de las capacidades y
condiciones necesarias (atención, motivación y desarrollo) para realizar aprendizaje.
\end{itemize}


¿Reforzamiento intermitente o continuo? El reforzamiento intermitente evita que se extinga la conducta por saciedad de reforzamiento (porque el sujeto se canse del refuerzo). Además, mantienen una tasa alta de conducta.


\subsection{Técnicas para instaurar, mantener o incrementar conductas}
\subsubsection{Moldeamiento}
También llamado de las aproximaciones sucesivas.
\subsubsection{PREMACK}
Utilizar como reforzador de una conducta de baja frecuencia otra conducta de alta frecuencia. 
\begin{example}
Para potenciar hacer los deberes (baja frecuencia), reforzamos con tiempo de juego al terminar los deberes (alta frecuencia).
\end{example}


\subsubsection{Contrato de contingencias}
\subsubsection{Economía de fichas}
Reforzar con dinero ficticio e inventado.

\subsection{Para disminución de conductas}
\subsubsection{Extinción}
Ya visto.
\subsubsection{Castigo positivo y negativo}
Ya visto.
\subsubsection{Coste de respuesta}
Retirada de reforzadores positivos recibidos anteriormente. 

\begin{example} Pierdes las fichas que habías ganado. \end{example}

\subsubsection{Tiempo fuera de reforzamiento}
Anular los reforzadores de la conducta.

\subsubsection{Sobre-corrección y practica positiva}
Restaurar el ambiente a un estado mucho mejor que el anterior, al producirse el
deterioro.

\subsubsection{Reforzamiento diferencial de baja tasa (RDB)}
Para reducir una conducta y no eliminarla, reforzaremos la conducta cuando su frecuencia haya sido baja. 

\subsubsection{Reforzamiento diferencial de otras conductas (RDO)}
Reforzaremos las demás conductas de un sujeto menos la que queremos eliminar. 


\paragraph{Actividad 3}

\paragraph{Elegir una técnica de mantenimiento/incremento de conductas y otra de disminución de conductas y poner un ejemplo de su empleo en el aprendizaje de algún contenido de los conocimientos que puedes impartir en un futuro.}

\textbf{Tiempo fuera de reforzamiento:} ignorar al alumno que busca continuamente llamar la atención. El refuerzo es que el profesor le haga caso, pero si éste ignora al alumno, éste reducirá su conducta al no verse reforzada.

\textbf{Economía de fichas} los positivos y negativos dentro de un aula.


\section{Tema 3}



Se recuerdan mejor los conceptos tangibles que los abstractos. 
%
Es esta una dificultad añadida para las matemáticas, porque trata con conceptos abstractos más que con conceptos tangibles.






\chapter{Cosas interesantes}

\section{Estudio de bioritmos}

La media de horas de sueño necesaria para adolescentes es 9,30 horas.

Cuando peor se rinde es a primera hora de la mañana.

Cuando mejor se aprende es entre las 11 y las 12:30 y de 16:00 a 17:30.

Los exámenes en lunes se suspenden más. Los exámenes a primera hora lo mismo.
Simplemente cambiando el día y la hora del examen, de media se aumenta casi 1 punto.

\section{Para profundizar}


El \href{https://es.wikipedia.org/wiki/Efecto\_Pigmali\%C3\%B3n}{Efecto Pigmalión} y la \href{https://es.wikipedia.org/wiki/Profec\%C3\%ADa\_autocumplida}{profecía autocumplida}. 
%
\textbf{If men define situations as real, they are real in their consequences.} 




%% Apendices (ejercicios, examenes)
\appendix

\chapter{---}
\input{tex/Psicologia_Ejs.tex}

\printindex
\end{document}
