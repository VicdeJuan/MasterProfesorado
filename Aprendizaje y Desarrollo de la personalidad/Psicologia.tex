\documentclass[palatino,nochap]{apuntesURJC}

\title{Aprendizaje y Desarrollo de la personalidad}
\author{Víctor de Juan Sanz}
\date{16/17 C1}

% Paquetes adicionales

% --------------------

\begin{document}
\pagestyle{plain}
\maketitle

\tableofcontents
\newpage
% Contenido.


\section{Condicionamiento clásico}

\subsection{Ejercicios de aplicación}

\paragraph{Ejercicio 1}
María trabaja en una fábrica de muñecos. Su trabajo consiste en introducir piezas en
una máquina. Cada vez que una pieza se ha introducido, la máquina hace un “click”,
lo que significa que ha enganchado la pieza. La máquina no funciona muy bien
últimamente y tras el “click”, un chorro de aire sale disparado directamente a la cara
de María, lo que provoca que ella cierre los ojos y gire ligeramente la cabeza.
Un día, María se da cuenta de que pestañea y torna la cabeza al escuchar el “click”,
aunque todavía no haya salido el chorro de aire.

\begin{itemize}
\item \textbf{Estímulo neutro/condicionado:} Click.
\item \textbf{Estímulo incondicionado:} Aire.
\item \textbf{Respuesta Condicionada:} Pestañear.
\item \textbf{Respuesta Incondicionada:} Cerrar los ojos.
\end{itemize}

\paragraph{Ejercicio 2}

Guillermo vuelve todos los días a casa de la universidad en autobús. Desde la
parada del autobús hasta llegar al portal de su casa, debe andar un kilómetro más o
menos y atravesar un pequeño túnel. Un día, un hombre le para en mitad del túnel,
le amenaza con un puñal y le pide la cartera. Guillermo tuvo un ataque de ansiedad.
Ahora, cada vez que baja del autobús y va a cruzar el túnel, tiene exactamente la
misma respuesta de miedo y ansiedad

\begin{itemize}
\item \textbf{Estímulo neutro/condicionado:} Túnel.
\item \textbf{Estímulo incondicionado:} Puñal.
\item \textbf{Respuesta Condicionada/Incondicionada:} Miedo y ansiedad.
\end{itemize}


\section{Para profundizar}


El \href{https://es.wikipedia.org/wiki/Efecto\_Pigmali\%C3\%B3n}{Efecto Pigmalión} y la \href{https://es.wikipedia.org/wiki/Profec\%C3\%ADa\_autocumplida}{profecía autocumplida}. 
%
\textbf{If men define situations as real, they are real in their consequences.} 





%% Apendices (ejercicios, examenes)
\appendix

\chapter{---}
% -*- root: ../Psicologia.tex -*-


\printindex
\end{document}
