\documentclass[palatino,nochap]{apuntesURJC}

\title{Aprendizaje y Desarrollo de la personalidad}
\author{Víctor de Juan Sanz}
\date{16/17 C1}

% Paquetes adicionales

% --------------------

\begin{document}
\pagestyle{plain}
\maketitle

\tableofcontents
\newpage
% Contenido.


\section{Condicionamiento clásico}

\subsection{Ejercicios de aplicación}

\paragraph{Ejercicio 1}
María trabaja en una fábrica de muñecos. Su trabajo consiste en introducir piezas en
una máquina. Cada vez que una pieza se ha introducido, la máquina hace un “click”,
lo que significa que ha enganchado la pieza. La máquina no funciona muy bien
últimamente y tras el “click”, un chorro de aire sale disparado directamente a la cara
de María, lo que provoca que ella cierre los ojos y gire ligeramente la cabeza.
Un día, María se da cuenta de que pestañea y torna la cabeza al escuchar el “click”,
aunque todavía no haya salido el chorro de aire.

\begin{itemize}
\item \textbf{Estímulo neutro/condicionado:} Click.
\item \textbf{Estímulo incondicionado:} Aire.
\item \textbf{Respuesta Condicionada:} Pestañear.
\item \textbf{Respuesta Incondicionada:} Cerrar los ojos.
\end{itemize}

\paragraph{Ejercicio 2}

Guillermo vuelve todos los días a casa de la universidad en autobús. Desde la
parada del autobús hasta llegar al portal de su casa, debe andar un kilómetro más o
menos y atravesar un pequeño túnel. Un día, un hombre le para en mitad del túnel,
le amenaza con un puñal y le pide la cartera. Guillermo tuvo un ataque de ansiedad.
Ahora, cada vez que baja del autobús y va a cruzar el túnel, tiene exactamente la
misma respuesta de miedo y ansiedad

\begin{itemize}
\item \textbf{Estímulo neutro/condicionado:} Túnel.
\item \textbf{Estímulo incondicionado:} Puñal.
\item \textbf{Respuesta Condicionada/Incondicionada:} Miedo y ansiedad.
\end{itemize}

\section{Estudio de bioritmos}

La media de horas de sueño necesaria para adolescentes es 9,30 horas.

Cuando peor se rinde es a primera hora de la mañana.

Cuando mejor se aprende es entre las 11 y las 12:30 y de 16:00 a 17:30.

Los exámenes en lunes se suspenden más. Los exámenes a primera hora lo mismo.
Simplemente cambiando el día y la hora del examen, de media se aumenta casi 1 punto.

\section{Para profundizar}


El \href{https://es.wikipedia.org/wiki/Efecto\_Pigmali\%C3\%B3n}{Efecto Pigmalión} y la \href{https://es.wikipedia.org/wiki/Profec\%C3\%ADa\_autocumplida}{profecía autocumplida}. 
%
\textbf{If men define situations as real, they are real in their consequences.} 


\section{Actividad 2}

\paragraph{Establecer entre los miembros del grupo un listado de palabras clave propias del condicionamiento clásico.}

\begin{itemize}
\item Estímulo incondicionado (EI)
\item Respuesta incondicionada (RI)
\item Estímulo neutro (EN)
\item Estímulo condicionado (EC)
\item Respuesta condicionada (RC)
\item Contracondicionamiento
\item Aversión
\item Recuperación expontánea
\item Manipulación
\end{itemize}

\paragraph{Esquema del condicionamiento clásico}

EI $\to$ RI

EN $\to$ No R

EN + EI $\to$ RI

EN pasa a ser EC

EC $\to$ RC

\paragraph{Pensar en posibles acciones metodológicas, basadas en el condicionamiento clásico, que podrían llevarse a cabo en el aula para implantar un determinado contenido de su área de conocimiento. Hacer una previsión de qué resultados se espera encontrar de su puesta en práctica.}



1. Exponer siempre una misma metodología al impartir la clase en la que se explique una primera parte de teoría una parte de problemas basados en juegos matemáticos. Justo antes de comenzar la segunda parte de juegos, se pondría una canción/melodía en la que todos los alumnos identifiquen dicha canción con la parte de juegos.
Para terminar, el día del examen se podría realizar el mismo procedimiento. 

Se prevee que los alumnos tengan la misma sensación que la parte de juegos. 

2. Para conseguir que un grupo de alumnos vuelva al silencio en caso de que se hayan desconcentrado y el murmullo sea muy ruidoso, existen técnicas como la de que el profesor levante la mano y a medida que los alumnos se van dando cuenta, tendrán que ir levantando la mano y permanecer callados de tal forma que al final se calle la clase por completo.

Se prevee que con el tiempo, asocien el levantar la mano con permanecer en silencio.


\section{Tema 3}

Se recuerdan mejor los conceptos tangibles que los abstractos. 
%
Es esta una dificultad añadida para las matemáticas, porque trata con conceptos abstractos más que con conceptos tangibles.


%% Apendices (ejercicios, examenes)
\appendix

\chapter{---}
\input{tex/Psicologia_Ejs.tex}

\printindex
\end{document}
