\documentclass[palatino]{apuntes}

\title{Seminarios}
\author{}
\date{16/17 C1}

% Paquetes adicionales

% --------------------

\begin{document}
\pagestyle{plain}
\maketitle

\tableofcontents
\newpage
% Contenido.

\section{TDAH}

Una de las cosas que impulsó a lo que se dedican es el descubrimiento de que se medicaba para que el chaval con un trastorno neurológico se adaptara al método. 
%
Tal vez el problema no está en el trastorno sino en la metodología.

\paragraph{El sabio que enseñó a hablar a su perro} (aunque el perro no aprendió nada).
%
El docente enseña, pero el aprendizaje es algo que depende de quien recibe la educación, es decir, \textbf{el aprendizaje es un acto intrínseco}. 

Sistema enciclopedista: exámenes en un momento determinado, clase magistral, libro de texto y tareas de repetición mecánica.

La neurociencia tiene mucho que decir a estos aspectos. 
%
¿Es eficaz este sistema mirando cómo son los procesos de aprendizaje del cerebro?


\subsection{Amigdala}
\paragraph{Dispositivos básicos del aprendizaje} que participan en todos los pasos de todos los procesos del aprendizaje.

\begin{itemize}
	\item Motivación: Si como docente digo "Mañana examen de tal", la respuesta será negativa y la predisposición para el estudio será, a lo sumo, ir a memorizar todo.
	%
	En cambio, plantear deberes que ilusionen (traer una foto que represente...) predispone positivamente y con ilusión.
	\item Atención:
	\item Memorias: No hay aprendizaje sin memoria. El aprendizaje tiene que ser consolidado.
\end{itemize}

Si tu testeas el nivel de conocimiento previo de un concepto y se hace en \textit{cerebro social} favorece los procesos de aprendizaje.

Un adulto explicando a un niño produce una activáción límbica de 2 ó 3 sobre 5. Si se lo explica alguien de su edad, la activación límbica es más fuerte y da lugar a un aprendizaje más eficiente. 
%
Este proceso es más fuerte porque la amígdala activan las memorias a corto y largo, la atención y la motivación.


\subsection{Hipocampo}

Dato interesante: los taxistas londinenses tenían un hipocampo el doble de grande.
%
Durante toda la vida, se siguen generando neuronas nuevas, aunque hay 2 ventanas más importantes: 0-2 años, adolescencia (nuevas conexiones)
%
En el caso de los taxistas, ocurría que las neuronas del hipocampo eran útiles y por eso se mantenían. ¿Las neuronas que no son útiles se destruyen?

El hipocampo es el conjunto de URLs en nuestro cerebro. 

Los recuerdos se tunean. No se recuerdan lo que se vivió, sino que se recuerda el recuerdo/la sensación de lo que vivió.
%
Cada vez que se activan las neuronas del recuerdo, se almacena una capa nueva sobre el recuerdo. Un tuneo nuevo. 

No hay aprendizaje sin emoción.

\subsection{TDAH}

Problema en el cortex (disco duro) prefrontal: un espesor inferior, con lo que hay menos neuronas. Sin embargo, el espesor de los adultos es todavía menor.
%
La diferencia está en las conexiones de las neuronas.

Los TDAH no tienen problema en recordar nada que le ilusiona (el sábado vamos al parque de atenciones). La información pasa al disco duro, se consolida y 2 días después, en plena bronca, le activa el recuerdo del parque de atracciones. Cuando llega a plano consciente, se activa el autodiálogo y gracias a la regulación emocional (función ejecutiva) es capaz de ser consciente de las consecuencias y callarse.


Hay una diferencia entre no aprender y aprender pero no ser capaz de retomar ese conocimiento aprendido. 
%
Pueden aprender, pero su frontal no es capaz de retomar la información.

Además, les falta la función ejecutiva \textit{automonitoreo}. No es capaz de cerrar las acciones. Se le olvida que había empezado a hacer algo y que lo ha dejado a medias (recoger).

Tampoco son perseverantes. La perseverancia no es un software del cerebro (como podría ser la regulación emocional, el conrol del tiempo, automonitoreo) .
%
La perseverancia es una acción que permite aplazar la recompensa/satisfacción

\textbf{Disfrontalizar} "no me voy a liar". Y al final, al llegar a casa: "Soy imbécil". Te secuestran las emociones.

Atender es un acto involuntario, la concentración es consciente. 
%
No podemos pedir que nos presten atención (porque es involuntario)
%
La concentración es la selección voluntaria de uno de los focos a los que atendemos.


Cuando la atención está a tope porque hay muchas cosas que llaman la atención entonces nadie se mueve. 
%
El TDAH es una disfunción que supone una desventaja, pero que con la metodología aplicada se puede minimizar, incluso anular.



Pregunta: 

Si nosotros venimos de esto, de clase magistral: ¿porqué hemos podido aprender a razonar y no sólo de memorizar? 



\textbf{Todo empieza en el cerebro límbico}

\textbf{Poner nombre al sentimiento hace que puedas regular la emoción.}

\section{KAHOOT}

\section{Gamificación}

\begin{defn}
Hackear a tu alrededor es modificar conductas para hacer la vida de la gente mejor. 
%
Crear dinámicas positivas que \textbf{contagien}.
\end{defn}

Objetivo: que cada alumn@ descubra para qué vale.

Nuestra generación no está siendo capaz de transformar el mundo en un sitio mejor. Nadie ha encontrado solución a la inmigración, al cambio climático... Necesitamos una educación distinta para que esto se pueda solucionar.

El alumno nunca se distrae en clase, sólo hace algo más interesante.

TED talk: Classroom game design.

El aprendizaje es una variación de las conexiones sinápticas.
En el aprendizaje influyen 4 elementos: percepción sensorial, Motivación , memoria, atención. 

\textbf{La atención se capta}, no se presta.


Entre el videojuego y la clase hay una analogía muy fuerte. 

\begin{defn}[DAS]
Deseo, acción, satisfacción -- Dopamina, adrenalina, serotonina
\end{defn}


La gamificación es como el ketchup. Hay veces que es genial, pero no soluciona absolutamente todos los problemas.



%% Apendices (ejercicios, examenes)
\appendix

\chapter{---}
\input{tex/Seminarios_Ejs.tex}

\printindex
\end{document}
