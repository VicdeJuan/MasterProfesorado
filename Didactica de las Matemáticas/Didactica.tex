\documentclass[palatino]{apuntes}

\title{Didactica de las Matemáticas}
\author{Víctor de Juan Sanz}
\date{16/17 C1}

% Paquetes adicionales

% --------------------

\begin{document}
\pagestyle{plain}
\maketitle

\tableofcontents
\newpage
% Contenido.

\section{Recomendaciones generales}

El rigor en las clases es muy importante.

Ejercicios realistas.

Instrucciones en los exámenes.

Analizar trabajos de otros y puesta en común, es una metodología didáctica muy interesante.


\section{Prueba de nivel}

Tipo test ha dado lugar a una prueba de nivel más amena de hacer. 


\paragraph{Criterios para valorar}

\begin{itemize}
	\item Claridad de los enunciados.
	\item No hay necesidad de cálculos difíciles para ver si lo saben o no.
	No hacen falta tampoco muchas repeticiones de lo mismo para evaluar si lo saben o no. 
	No van a usar calculadora.
	\item Dificultad.
	\item Tiempo suficiente.
	\item Formato: todo test o todo preguntas cortas... Y las opciones del tipo test que aporten información.
	\item Śecuenciación de los ejercicios (los más fáciles primero y los más difíciles después, para que no se desmotiven nada más empezar)
	\item Se adecua al temario y aporta información útil.
	\item \textit{Desde Enero, cada 2 meses me regalan 2 regalos. ¿Cuántos regalos tendré en Octubre?} Surgen dudas: ¿En Enero me dan 2 regalos? ¿Y entonces en Marzo tengo 4? Pero si en Febrero me dan 2, entonces en Octubre, ¿me han dado ya los 2 de Octubre o no? 
\end{itemize}

\section{Algebra}

En las ecuaciones, obligar a comprobar. Por ejemplo, no dar por válida una ecuación que no tenga comprobación.

%% Apendices (ejercicios, examenes)
\appendix

\chapter{---}
% -*- root: ../Didactica.tex -*-
\section{Actividad 1}


\paragraph{“Eres profesor de matemáticas 1º de ESO y es tu primer día de clase”
¿Qué les dirías a tus alumnos? ¿Qué harás durante esta primera clase?}

\textit{Nota:} Ya han tenido la presentación con su tutor y la tuya es la primera clase del día.


\begin{enumerate}
\item 	\textbf{Presentación del profesor}, incluyendo algunos aspectos como porqué me gustan las matemáticas o porqué soy profesor. 
\subitem Motivación y recordatorio de la diferencia del paso de primaria a secundaria.
\item \textbf{Presentación del alumnado:} cada estudiante se presenta diciendo su nombre y su asignatura favorita o sus hobbies.
\item \textbf{Presentación de la asignatura:} El temario, el horario, la evaluación y la metodología (incluyendo los materiales, tipo de cuaderno, boli y no típex, para que quede constancia de los errores).
\item \textbf{Actividad en grupo} Como un juego de lógica. Por grupos de 3 mejor que por parejas, ya que es más fácil que salga adelante si algún estudiante reacciona con timidez.
\subitem También puede ser buena idea hacer una competición, ya que ésta nos permitiría empezar a ver la personalidad de los chavales: quién se lo toma en serio, quién es más pasota, quién lidera y quién no, etc.
\end{enumerate}

Otros aspectos que podrían pensarse como la prueba de nivel, es mejor dejarlos para otro momento más avanzado, en el que ya se hayan desoxidado después de las vacaciones.


\documentclass[palatino,nochap]{apuntes}


\renewcommand{\organization}{URJC}

\title{Entrega 3 (Didáctica)}
\author{Grupo 6}
\date{16/17 C1}

% Paquetes adicionales

% --------------------

\begin{document}
\pagestyle{plain}

\documentclass[palatino,nochap]{apuntes}


\renewcommand{\organization}{URJC}

\title{Entrega 3 (Didáctica)}
\author{Grupo 6}
\date{16/17 C1}

% Paquetes adicionales

% --------------------

\begin{document}
\pagestyle{plain}

\input{../../tex/Actividad3.tex}

\printindex
	\end{document}


\printindex
	\end{document}

\section{Análisis de la Unidad Didáctica (UD) de SM\\\quad\quad\small{y comparativa con la UD de Editex}:}

\label{sec1}

\paragraph{Puntos a mejorar:}


\begin{itemize}
\item \textbf{Estructura:} no encontramos clara la estructura de la UD.
	\subitem No hay una descripción de los objetivos, aunque podemos encontrarlos dispersos en la ruta de aprendizaje.
	\subitem La planificación no está detallada. Sólo hay una sugerencia de temporalización, pero sin concretar qué parte del temario se podría tratar en cada sesión. \\
En este sentido, la UD que encontramos (de Editex) tenía una exposición clara de los objetivos, aunque la planificación es inexistente en ambas.

\item \textbf{Evaluación:} Los criterios de evaluación son muy escasos. En la UD de Editex este aspecto estaba mejor tratado porque estaban más especificados los criterios y tenían relación con los estándares de Aprendizaje definidos por el BOE.

\item \textbf{Contenido transversal:} Un aspecto muy útil que SM no tiene es incluir apartado de contenidos de otras asignaturas para los que la UD de Matemáticas es necesaria y útil, y viceversa.
\footnote{Este apartado ha sido incluido tras la revisión en clase.}

\end{itemize}


\paragraph{Puntos positivos:}

\begin{itemize}
\item La inclusión del apartado de \textbf{conocimientos previos}. Ahorra el tiempo al docente de buscar los conocimientos previos que deberían haber adquirido años anteriores.
Además, se sugiere en cada epígrafe algunos conceptos para repasar.
Este aspecto nos parece muy positivo y en la UD de Editex no aparecía. 

\item \textbf{Recursos:} Hay una muchas sugerencias de recursos (varias referencias a Geogebra, sugerencia de concursos para aplicar \textit{gamificación}) para utilizar durante las sesiones.
La UD de Editex no tenía un apartado de recursos, aunque sí sugiere escasas pautas metodológicas (\textit{Cómo trabajar la unidad}).
\end{itemize}


\newpage\section{¿Qué preguntas harías a la hora de analizar si las actividades y las metodologías de la ud son las adecuadas?}

\begin{itemize}
\item Cómo son las actividades? ¿Mecánicas y rutinarias? ¿Retadoras? ¿Trabajan conceptos o sólo procedimientos? ¿Son todas iguales cambiando los números, o cambian también el tipo y el tema?

\item ¿Qué objetivo tiene la unidad? ¿Se corresponde con el propuesto en el currículo oficial?

\item ¿El lenguaje es adecuado a la edad y la situación cotidiana del alumnado? ¿El rigor es adecuado o se tratan los temas de una manera demasiado laxa?

\item ¿Cuáles son los conocimientos previos necesarios antes de llegar a este tema? ¿Se tratan o se dan por supuestos?

\item ¿Qué errores espero? ¿Qué puedo hacer para solucionarlos?

\item ¿Se pueden plantear metodologías activas o cooperativas para trabajar la unidad? ¿Y recursos digitales o materiales manipulativos?

\item ¿Se trabajan las competencias transversales (comunicación, digital...)? ¿Sirven para trabajar todos los objetivos?

\item ¿Cómo es la evaluación a partir de las actividades? ¿Se puede evaluar la comprensión matemática adquirida con las actividades propuestas?

\item ¿Cuál es tu opinión personal? ¿te parecen adecuadas? ¿Qué cambiarías?
\end{itemize}





\newpage\section{Realizar un análisis de las actividades y metodologías propuestas en la ud de SM utilizando como guía el documento "Análisis de las actividades de la ud.docx"}

Basándonos en la U.D. de SM:

\begin{enumerate}
\item \textbf{¿Qué tipo de actividades son (demandan muchos conocimientos del alumno? ¿son rutinarias? ¿memorísticas? ¿trabajan conceptos o solo procedimientos? }

Son un cúmulo de tareas que sirven para afianzar (rutinarias) e interiorizar (memorísticas) los conocimientos del alumno. Algunas de ellas son más exigentes que demandan mayor conocimiento.

\item \textbf{¿Qué variables didácticas movilizan? ¿sólo cambian los números o cambia sustancialmente la actividad de manera que los alumnos necesitan otras estrategias distintas a la anterior? }

Las actividades propuestas se van alternando dependiendo del contenido impartido, por ejemplo la act.101 trata con la música y la act.97 con la economía para los procesos de modelización. La act.67 y otras hacen referencia a cálculo con calculadora y hay otra para resolver gráficamente con Geogebra.

\item \textbf{¿Qué errores o dificultades esperas que tengan los alumnos al hacer esas actividades? ¿qué harías para resolverlos? }

En una primera visión de la misma, esperamos errores de redondeo, los errores más comunes cometidos en Álgebra y cálculo del valor absoluto.
Para resolverlos proponemos PROHIBIR el lápiz y que los errores aparezcan en rojo para evitar que repitan el fallo.

\item \textbf{¿Qué objetivos tienen? }

Entender los números reales y aplicarlos en las diferentes situaciones e la vida cotidiana.

\item \textbf{¿Trabajan los alumnos en grupo para resolverlas? }

Esa parte queda a libre elección del profesor ya que no se menciona explícitamente.

\item \textbf{¿Planifican usar materiales manipulativos o tecnologías digitales? ¿Calculadora? }

Se utilizan bastante según viene especificado en criterio de evaluación B.1.12

\item \textbf{¿Se fomenta la comunicación con esas actividades? }


Los objetivos 01, 02, 05 06 hacen énfasis sobre la comunicación lingüística.

\item \textbf{¿Cómo están secuenciadas? ¿Cubren todos los objetivos?}

Las actividades están secuenciadas de una manera difícil de entender.\\
Si, y además hay una trazabilidad entre las actividades y las competencias y objetivos de la U.D.

\item \textbf{¿Cómo es la evaluación de la comprensión matemática a partir de las actividades?}


Nos resulta acorde a los objetivos y las competencias de la U.D.

\item \textbf{¿Cuál es tu opinión personal? ¿te parecen adecuadas? ¿Qué cambiarías?}

Nuestra opinión personal respecto a esta unidad didáctica está reflejada en la sección \ref{sec1}

\end{enumerate}




\printindex
\end{document}
