\documentclass[palatino,nochap]{apuntesURJC}

\title{Registros de representación}

\author{Víctor de Juan,
Pedro de la Mata Gómez,
Virginia Vadillo Lacasa,
Gustavo Adolfo Martínez Risque}

\date{16/17 C1}

% Paquetes adicionales

% --------------------



\begin{document}

\maketitle

\pagestyle{plain}

\begin{table}[hbtp]
\centering
\begin{tabular}{|c|c|}
\hline
N grupo & Apellidos, Nombre
\\\hline
Grupo 3 & Nombres
\\\hline
\end{tabular}
\caption{Descripción de los miembros del grupo}
\end{table}

\section{Introducción}
\section{Contextualización}

\section{Unidad didáctica}

\begin{figure}[hbtp]
%\includegraphics[scale=0.4]{img/img1.jpg}
\caption{\textcolor{red}{Tema 1, arriba a la izquierda} Conversión del registro natural al registro gráfico.}
\label{img1}
\end{figure}

\begin{table}[hbtp]
\begin{tabular}{cccccccc}
				&icónico & tabular & gráfico & aritmético & natural & página & imagen\\
Concepto de función & no & no & si & no & si & ??? & \ref{img1}\\
Funciones dadas por tablas de valores & no & no & si & si & si &235 & \ref{img2}\\
Proporcionalidad & si & si & si & si & si & ??? & \ref{img3}\\
Pendiente & no & no & si & si & si & 238 & \ref{img4}\\
Funciones lineales & si & si & si & si & si & 240 & \ref{img5} \\
Constantes & si & si & si & si & si & 242 & \ref{img6}\\
\end{tabular}
\caption{Cuadro resumen de los registros encontrados en la unidad didáctica.}
\label{Tablaresumen}
\end{table}


\section{Conclusiones}

\printindex

\bibliographystyle{abbrv}
\bibliography{angelica5}  % memoria.bib es el nombre del fichero que contiene


\end{document}