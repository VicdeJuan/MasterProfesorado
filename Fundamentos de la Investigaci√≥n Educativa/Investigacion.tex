\documentclass[palatino]{apuntesURJC}

\title{Fundamentos de la Investigación Educativa}
\author{Víctor de Juan Sanz}
\date{16/17 C1}




\def\citeapos#1{\citetitle{#1} (\citeauthor{#1}, \citeyear{#1}) \cite{#1}}
%\newcommand{\citeapos}[1]{\citetitle{#1} (\citeauthor{#1}, \citeyear{#1}) \cite{#1}}

% Paquetes adicionales

% --------------------

\begin{document}
\pagestyle{plain}
\maketitle

\tableofcontents
\newpage
% Contenido.


\chapter{Entrega 1}

\textbf{Tema de investigación. 
%
Pregunta de investigación que nos gustaría responder dentro de ese tema (mayor nivel de concreción).
%
Valoración de los mismos.}

\section{Tema} Un medio para beneficiar desde la programación al resto de disciplinas y viceversa.

\section{Preguntas de investigación: }

\begin{itemize}
\item ¿Es posible mejorar el rendimiento en otras asignaturas significativamente por aprender de cero a programar? ¿Tal vez se requiere una base previa con la que ejercitar conocimientos previos de programación?
\item ¿Aprender/ejercitar la programación en general o haría falta alguna restricción específica para que esta mejora tuviera lugar? ¿Es lo mismo programar páginas web, que bases de datos, que programación simple: condicional-bucles?
\item ¿En qué asignaturas tiene se puede producir más mejora en el rendimiento?
\item Utilizando la programación para mejorar en otras asignaturas, ¿mejora las competencias en programación significativamente más que los métodos tradicionales de enseñanza de la programación?
\end{itemize}

\section{Valoración:} Es un tema todavía grande y algunas de las preguntas son difíciles de contestar de manera absoluta. 
%
Lo que si se podría conseguir es acompañar estas preguntas de investigación con un experimento en algún grupo de estudiantes en lo que contestar parcialmente a las preguntas. 
%
Tal vez no lleguemos a clasificar todas las asignaturas del currículum según su mejora por aprender con la programación, pero sí algunas de ellas.

\chapter{Entrega 2}

\textbf{Justificación de la investigación (la valoración de la Tarea 1 puede servir de base)
%
Documentación bibliográfica sobre el tema.
%
Tres artículos relacionados. 
%
Sólo los títulos de los textos, referenciados según normativa APA o ISO.}

\section{Documentación bibliográfica}

La documentación bibliográfica se encuentra en la sección Bibliografía: \ref{bibliografia}

\section{Justificación}

Estos últimos años se caracterizan por la creciente demanda de puestos de trabajo relacionados con la Ingeniería Informática.
%
La Ingeniería Informática es una de las profesiones con menor tasa de paro. 
%
Incluso, muchas personas que no han estudiado Ingeniería Informática acaban desarrollándose laboralmente en el ámbito informático, como puede ser el caso de los matemáticos, físicos e Ingenieros de otras ramas (como Industriales y de Telecomunicaciones - aunque estos últimos están algo más relacionados con el tema).

Estas afirmaciones son compartidas por investigaciones científicas (\cite{CSIsImportant},\cite{CSArguing}). 
%
De hecho, \cite{CSArguing}, además de constatar la importancia de la Ingeniería Informática, trata si deberían incrementarse las competencias relativas a la Ingeniería Informática necesarias en la educación básica o secundaria.

Por otro lado, no sólo parece positivo incluir competencias relativas a la Ingeniería Informática (más allá de la Ofimática), en sistemas educativos que no las incluyan, sino que también es importante hacerlo bien.
%
Algunos sistemas educativos, debido a las metodologías aplicadas, no consiguen una satisfactoria adquisición de destrezas y competencias por parte de los estudiantes.
%
En esta investigación tratamos de aportar una metodología (y la evidencia sobre su funcionamiento) con la que los alumnos puedan desarrollar mejores competencias Informáticas. 
%
Debido al enfoque multidisciplinar de la metodología, estudiaremos también los beneficios producidos en las otras disciplinas.

El estudio \textit{The Impact of an Interdisciplinary Space Program on Computer Science Student Learning} \cite{Interdiscipline} realiza una experiencia parecida en E.E.U.U. Sin embargo, pretendemos completar esa investigación en dos sentidos: 
\begin{itemize}
	\item Realizar un estudio similar en un sistema educativo muy diferente, como es el Sistema Educativo Español. 
	%
	Además de las diferencias del sistema (como que en España los estudiantes no pueden elegir los \textit{major} que tienen en E.E.U.U.), las asignaturas referidas a la Ingeniería Informática de cada país tienen sus grandes diferencias.
	\item Estudiar el doble efecto de la interdisciplinalidad: el efecto desde la Informática a las otras competencias y viceversa.
\end{itemize}



\chapter{Entrega 3}

\textbf{Enunciar los objetivos del trabajo (aquellos que me llevan a poder responder a la pregunta de investigación).
%
Determinar (justificadamente) si la metodología que emplearíamos sería cuantitativa, cualitativa, o mixta.}

\section{Objetivos}

\begin{itemize}
	\item Dilucidar si el desarrollo por parte de los alumnos de un programa de preguntas y respuestas sobre el temario de otras asignaturas (como fechas de Historia, vocabulario de Inglés, aritmética básica o compleja) supone una mejora en el rendimiento en la asignatura sobre la que verse el programa desarrollado.
	%
	\subitem En una primera investigación, se intentará que cada estudiante programe sobre la asignatura que más le interese (entre unas previamente seleccionadas).
	\item Cómo mejora la autoeficacia en la programación (utilizando \citetitle{CPSES}  \cite{CPSES}) gracias a haber estado desarrollando un programa con una utilidad directa.
	\item Seleccionar un grupo de estudiantes de 1 de Bachillerato en un instituto donde se imparta Informática. 
	%
	El grupo será del bachillerato de ciencias, dado que el currículo de Informática de 1 de Bachillerato Tecnológico incluye la programación.

\end{itemize}

\section{Metodología}

La metodología empleada sería puramente cuantitativa. 
%
Primero mediremos las competencias informáticas con la escala \citetitle{CPSES} \cite{CPSES}, antes del experimento y después, con un grupo de control para comprobar si la mejora producida es significativa mediante un contraste de hipótesis estadístico.
%
Por otro lado, la misma comparativa con las otras asignaturas, con un cuestionario pre-experimento y otro post-experimento, con los que evaluar los conocimientos adquiridos.

\chapter{Entrega 4}

\textbf{Redacción del marco teórico, haciendo uso de los textos seleccionados. Se trata de redactar reelaborando ideas, definiciones, metodologías... que vamos a utilizar o adoptar, o que describen en qué ámbito teórico, o teórico-práctico, nos situamos.}

\textbf{
Por otra parte, descripción, en la medida de lo posible, de las técnicas de recogida de datos que vamos a utilizar en nuestra investigación (dentro de las vistas en clase).
}

\section{Marco teórico:}

Numerosas investigaciones (\cite{CSIsImportant},\cite{CSArguing} entre otras) evidencian la necesidad de la adquisición de competencias informáticas en la educación secundaria para un satisfactorio desempeño en la educación posterior y, a la larga, en la vida profesional.
%
A la hora de que los alumnos adquieran esas destrezas es imprescindible analizar cuál es la mejor metodología para que los estudiantes las adquieran.
%
En este sentido, \cite{StudentCenter} demuestra que la \concept{autoeficacia} (utilizando la escala \citetitle{CPSES},\cite{CPSES}) en \textit{Computer Science} aumenta significativamente cuando la metodología empleada es a base de problemas y tareas, siendo el docente una mera referencia a la que consultar las dudas que puedan surgir.
%
\label{studentbased}
%
Por otro lado, \cite{StudentCenterVSLectures} llega una conclusión parecida. 
%
La docencia basada en explicaciones del docente (\textit{Teacher based}) da lugar a una mejora en la autoeficacia peor que la mejora obtenida con una metodología centrada en el estudiante.
%
\label{groupsbased}
%
Otra investigación \cite{ABPCS} evidencia que en el ámbito de la programación también se aumentan las destrezas adquiridas trabajando en grupo sobre problemas y proyectos que los estudiantes encuentren interesantes y entretenidos. 


Por otro lado, Mitch Resnick, el investigador que desarrolló \concept{Scratch} \cite{scratch}, describe la programación como un medio por el que aprender y comprender.
%
Esta herramienta se desarrolló para que los alumnos de primeros cursos de primaria pudieran acercarse a la programación y esto les ayudara con el \textit{enfoque procedimental} necesario en la programación.
%
Podemos definir el \concept{pensamiento procedimental} como la consecución del modelo de solución de problemas de Polya \cite{Polya} basado en 4 pasos:
1) Comprender el problema. 2) Idear un plan (formular una estrategia general). 3) Ejecutar ese plan (formular una prueba detallada). 4) Verificar los resultados. 
%
Estos son los 4 pasos que se aplican continuamente en la resolución de problemas de la Ingeniería Informática y que, según Resnick, pueden resultar de ayuda para el resto del conocimiento académico.

El objeto de esta investigación es ampliar la idea de Resnick a la educación secundaria, con adolescentes que ya son capaces de un pensamiento más abstracto.
%
¿De qué manera la resolución de problemas informáticos (de programación) puede ayudar a mejorar el rendimiento en otras asignaturas que, a priori, no tienen relación como pueda ser la Historia o el Inglés?
%
Y ¿de qué manera, la modificación de la metodología de enseñanza de la programación en secundaria para enfocarla a mejorar el rendimiento en otras asignaturas redunda en una mejora significativa del aprendizaje de la programación?


\section{Recogida de datos: } Definiremos 
\begin{itemize}
	\item \textbf{Variable independiente:} Metodología impartida en la clase de Informática.
	\item \textbf{Variable dependiente:} Rendimiento en otra disciplina (Química Orgánica, Química Inorgánica, Historia o Inglés).
	\subitem Técnicamente se realizarán \textit{n} experimentos con las \textit{n} asignaturas elegidas por más de 5 estudiantes.
	\item \textbf{Grupos:}
	\subitem Dividiremos el grupo de Informática en 2. A los que aplicaremos la  metodología propuesta (grupo experimental) y a los que no (grupo control).
	%
	Los alumnos que vayan a recibir la metodología nueva escogerán una de las siguientes materias: Química Orgánica, Química Inorgánica, Historia o Inglés.
	\subitem Además, para poder realizar un análisis estadístico más profundo, tomaremos como referencia los alumnos del Bachillerato Tecnológico que no cursen Informática.
	%
	Así podremos contrarrestar el efecto de variables extrañas que afecten a todos por igual, como puede ser un cambio de profesor, avanzar el curso y en el temario, por lo que los alumnos cada vez habrán adquirido más destrezas...
\end{itemize}



La primera recogida de datos serán 2 cuestionarios a los 3 grupos definidos.
%
El primer cuestionario será uno de conocimientos sobre la materia elegida por el grupo experimental.
%
El segundo cuestionario será para evaluar la autoeficacia en Informática (según \citetitle{CPSES} \cite{CPSES}).

Después, a los alumnos del grupo experimental se les propondrá que hagan, por grupos (como ya hemos sugerido \ref{groupsbased}), un programa para evaluar sus conocimientos sobre la materia escogida.
%
El funcionamiento de la clase será \textit{basado en el estudiante} ya que, como hemos mencionado anteriormente (\ref{studentbased}), es una metodología más efectiva que la \textit{basada en el docente}.
%
Mientras el grupo experimental ha recibido esta propuesta, el grupo control recibirá las instrucciones que el docente considere oportunas para desarrollar un software similar (como podría ser el de un cajero automático).

Una vez finalizado el tiempo (especificado en el siguiente punto \ref{tiempos}), se volverán a pasar los 2 cuestionarios a los 3 grupos.

\section{Temporalización:}
\label{tiempos}
Un programa muy sencillo de preguntas y respuestas ha sido desarrollado para esta investigación a modo de ejemplo en aproximadamente media hora.
%
Esperamos que los alumnos sean capaces de aprender los rudimentos de la programación como para hacer un software con las caracterísitcas deseadas en tres semanas.


%% Apendices (ejercicios, examenes)

\bibliographystyle{abbrv}
\label{bibliografia}
\bibliography{tex/Bibliografia}  % memoria.bib es el nombre del fichero que contiene

%\printindex
\end{document}
