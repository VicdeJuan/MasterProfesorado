% -*- root: ./Centro.tex -*-

\section*{Actividad 1}
\paragraph{Enunciado: }
¿Qué es un centro?
\paragraph{Solución: }

\newpage
\section*{Actividad 2}
\paragraph{Enunciado: }
Reforma constitucional: da una nueva redacción, añadiendo, quitando o modificando su contenido, al artículo 27 de la Constitución:
\paragraph{Solución: }

\textcolor{white}{a}\\\\
\textbf{1.
} Todos tienen el derecho a la educación.
 \sout{Se reconoce la libertad de enseñanza.
}\\
\textbf{2.
} La educación tendrá por objeto el pleno desarrollo de la personalidad humana en el respeto a los principios democráticos de convivencia y a los derechos y libertades fundamentales.
 \\
\textbf{3.
} Los poderes públicos garantizan el derecho que asiste a los padres para que sus hijos reciban una formación \sout{religiosa y moral} que esté de acuerdo con sus propias convicciones.
 \\
\textbf{4.
} La enseñanza básica es obligatoria y gratuita.
\\
\textbf{5.
} Los poderes públicos garantizan el derecho de todos a la educación, mediante una programación general de la enseñanza, con participación efectiva de todos los sectores afectados y la creación de centros docentes.
 \\
\textbf{6.
} Se reconoce a las personas físicas y jurídicas la libertad de creación de centros docentes, dentro del respeto a los principios constitucionales.
 \\
\textbf{7.
} Los profesores, los padres y, en su caso, los alumnos intervendrán en el control y gestión de todos los centros sostenidos por la Administración con fondos públicos, en los términos que la ley establezca.
\\
\textbf{8.
} Los poderes públicos inspeccionarán y homologarán el sistema educativo para garantizar el cumplimiento de las leyes .
 \\
\textbf{9.
} Los poderes públicos ayudarán a los centros docentes que reúnan los requisitos que la ley establezca.
\\
\textbf{10.
} Se reconoce la autonomía de las Universidades , en los términos que la ley establezca.
\\


\newpage
\section*{Actividad 3}
\paragraph{Enunciado: }

\paragraph{Solución: }

\paragraph{¿En qué tiene razón Sofía y en qué no?}

\subparagraph{Tiene razón en:}

\begin{itemize}
	\item No dar francés.
 La autorización de la \textit{Dirección de Área Territorial} es para la profesora que está de baja.
 
	%
	Como Francés y Matemáticas son asignaturas que no comparten especialidad, tendrían que conseguir una nueva autorización para la interina, y sino, buscar una nueva interina.


	Sin embargo, si la profesora supiera francés bien y se viera cómoda, podría llegar a dar la clase esgrimiendo el argumento de que los alumnos necesitan un docente.

\end{itemize}


\subparagraph{No tiene razón en:}

\begin{itemize}
	\item No participar del blog.

	\item No ayudar con la olimpiada de Matemáticas.

	\item No participar en el expediente disciplinario.

\end{itemize}


\paragraph{¿Qué respuesta deben darle el director y el jefe de estudios?}



\newpage
\section*{Actividad 4}
\paragraph{Enunciado y solución:}


\begin{itemize}

	\item \textbf{Encomendar a un profesor del departamento que elabore una parte de la Memoria final de curso.
}
	%
	Sí puede realizarla.
 
	%
	Según el Reglamento Orgánico de Centros (RD 83/1996), en adelante ROC, en su artículo 51 se puede extraer del apartado a) que una de las competencias del jefe de departamento es \textit{“coordinar la elaboración de la memoria final de curso”}

	\item \textbf{Pedir a un profesor que le indique cómo lleva el desarrollo de sus programaciones didácticas.
}
	%
	Sí puede realizarla.

	%
	Según el ROC, en su artículo 51 se puede extraer del apartado f) que una de las competencias del jefe de departamento es “velar por el cumplimiento de la programación didáctica del departamento”
	\item \textbf{Entrar en el aula mientras un profesor imparte clase para observar cómo lo hace (por ejemplo, porque ha recibido quejas de los alumnos sobre él).
}
	%
	No, solamente el inspector tiene potestad para entrar en la clase del profesor.

	\item \textbf{Cambiar la calificación que ha puesto un profesor, si juzga que no ha evaluado adecuadamente a un alumno.
}
	%
	No.
 Como mucho el jefe de estudios, con una reclamación escrita puede
	\item \textbf{Sancionar a un profesor que no cumple con sus obligaciones, previo informe razonado.
}
	%
	No tiene ninguna autoridad sobre ningún profesor.
 
	%
	El director sería quien puede sancionar (además, obviamente, de la inspección)
	\item \textbf{Encargar a un profesor que se ocupe de organizar una actividad extraescolar del departamento.
}
	%
	Sí puede realizarla.

	%
	Según el ROC, en su artículo 51 el apartado b) dice que una de las competencias del jefe de departamento es \textit{"dirigir y coordinar las actividades académicas del departamento.
”}
	\item \textbf{Asignar a cada profesor del departamento los grupos y materias a los que debe impartir clase.
}
	%
	No.
 El reparto de los grupos se hace dentro del departamento.
 Hay una regulación acerca del reparto de grupos.

	\item \textbf{Determinar el día y la hora de reunión semanal del departamento.
}
	%
	No, lo hace la jefatura de estudios.
 
	\item \textbf{Decidir a qué se dedica el dinero asignado al departamento en los presupuestos del centro.
}
	%
	No.
 Se decide entre todo el departamento, de mutuo acuerdo.

	\item \textbf{Informar al director sobre irregularidades cometidas por los profesores de su departamento.
}
	%
	Sí.

\end{itemize}

\newpage
\section*{Actividad 5}

\paragraph{Enunciado: }
Violeta Pérez, profesora de Música, empieza a trabajar en un centro de Secundaria, impartiendo su materia a
alumnos de ESO.
 Varias semanas después de empezar el curso, oye rumores de que el director tiene una
academia, que gestiona conjuntamente con su mujer, y dedica la mayor parte de su tiempo a atender dicha
academia y a resolver asuntos relacionados con ella.
 Debido a ello, suele ausentarse del centro muchas de las
horas en las que debería estar en él y deja al jefe de estudios, que es íntimo amigo suyo, al frente de todo.
\\
En varias ocasiones en las que Violeta ha necesitado dirigirse al director y ha acudido a su despacho, se lo ha
encontrado cerrado y el jefe de estudios siempre le ha disculpado diciendo que “ha tenido que ausentarse por
un asunto urgente” y se ha ofrecido a encargarse del tema.
\\
Un empleado del centro, encargado de mantenimiento, con el que Violeta tiene mucha confianza, le ha
contado que el director suele dejar las luces del despacho y el ordenador encendidos, para simular que ha
salido un momento, pero que siempre desaparece y la mayor parte de las veces no vuelve hasta el día siguiente
e, incluso, se pasa varios días sin venir al centro.
\\
Finalmente, una alumna de Violeta le cuenta que tiene una amiga que acude a la academia Einstein y que allí
ha recibido clase del director.

\paragraph{Solución: }

\begin{itemize}
\item \textbf{Valora el comportamiento del director, en caso de tratarse de un centro privado (no concertado).
}

Para poder dar una respuesta más precisa necesitaríamos saber si el director es el propietario del centro privado o si por el contrario es un empleado del centro.
 Si el director es el propietario del centro no se podría valorar la actuación del director dado que él es el máximo responsable del Centro.
 En el segundo caso, sí que estaría cometiendo una dejación de funciones al estar incumpliendo su parte del contrato que tiene firmado con el centro privado.



\item \textbf{Valora el comportamiento del director, en caso de tratarse de un centro público.
}

En el caso de un centro público es diferente.
 En este caso claramente comete dos faltas graves:

\begin{itemize}
\item La primera falta grave que comete es el abandono de servicio y el incumplimiento de horarios.

\item La segunda falta grave que comete es trabajar en una academia privada es incompatible con su condición de funcionario.

\end{itemize}


\item \textbf{Indica qué debería hacer Violeta en uno u otro caso.
}
Si el Centro es público, la profesora tiene la obligación de ponerlo en conocimiento de la administración ya sea del inspector del centro o de la Dirección de Área Territorial correspondiente.

%
En el caso de que el centro sea privado y el director sea un empleado también podría comunicarlo a los responsables del centro, pero en este caso no estaría obligada.


\end{itemize}


\newpage
\section*{Actividad 6}
\paragraph{Enunciado: }

{David Malasaña, un alumno de 2o de bachillerato, tras solicitar una entrevista con el
director del centro, le expone a este que ha cumplido 18 años de edad y que, por
tanto, ya es mayor de edad.
 En consecuencia, acogiéndose a la Ley de protección de
datos, solicita al director que, a partir de ese momento, le sean entregados a él y solo
a él los boletines de notas y le hace saber que no desea que le sean comunicadas a sus
padres las calificaciones que obtenga.
 Además, le comunica que ha entregado en el
registro del centro un escrito en el que formula esa misma solicitud por escrito.
\\
Acabada la entrevista, el director, preocupado, llama por teléfono al padre del alumno
y le refiere lo que acaba de decirle su hijo.
 El padre, encolerizado, le dice al director
que David es su hijo, vive en su casa, es mantenido por él, se encuentra bajo su
responsabilidad y, por tanto, las notas deben seguir siendo entregadas a los padres,
como se ha venido haciendo hasta ahora.

\paragraph{Solución: }

\begin{itemize}
\item \textbf{¿Qué debe hacer el director?}
\subitem Darle las notas al alumno, ya que es mayor de edad.

\item \textbf{¿Qué respuesta debería haberle dado al alumno?}
\subitem Recomendarle con muy buen talante que arregle sus problemas familiares y que se mantenga en una situación normalizada.

%
En caso de que el alumno estuviera obcecado en ello, tendría que acabar cediendo.

\item \textbf{¿Qué respuesta debería darle al padre?}
\subitem Que, con independencia de la dependencia económica del alumno, la ley de protección de datos
\end{itemize}


\newpage
\section*{Actividad 7}
\paragraph{Enunciado: }

\paragraph{Solución: }

\textbf{Objetivo:}
Incrementar el número de alumnos que aprueban MATEMÁTICAS II en la PAU.


\textbf{Indicador de logro:}
Si el número de alumnos que aprueban Matemáticas II en la PAU en este curso es superior al número de aprobados en el curso anterior.


\textbf{Actuaciones:}
\vspace{-0.3cm}
\begin{itemize}
	\item Innovar en la metodología utilizada en ambos cursos de Bachillerato.

	\item Resolver exámenes de años anteriores.

	\item Los exámenes de evaluación sólo contendrán ejercicios de PAUs de años anteriores.

\end{itemize}

\textbf{Tareas:}
\vspace{-0.3cm}
\begin{itemize}
	\item Formar al profesorado durante el verano en nuevas tecnologías.

	\item Mandar ejercicios típicos de examen de deberes para todos los alumnos y dedicar rato de clase a la semana a la resolución de dudas del ejercicio de la semana.

\end{itemize}


\textbf{Temporalización:}
\vspace{-0.3cm}
\begin{itemize}
	\item Cursos de verano de formación para el profesorado.

	\item Mandar 1 ejercicio de examen de deberes a la semana para el último día de clase de la semana y dedicar un cuarto de hora de esa clase a la corrección del ejercicio y resolución de dudas.

\end{itemize}

\textbf{Responsable:}
Jefe del departamento de Matemáticas del centro.


\textbf{Indicador de seguimiento:}
\vspace{-0.3cm}
\begin{itemize}
	\item Cada alumno corregirá el ejercicio de un compañero (en base a unos criterios establecidos por el profesor) e informará al profesor de la calificación obtenida.

	\subitem En caso de que un alumno no haya realizado el ejercicio, no se computará como un 0, ya que esa medida falsearía la realidad.

	\item La calificación obtenida en el examen de evaluación 
\end{itemize}

\textbf{Responsable del control del cumplimiento de la tarea:}
Profesor de Matemáticas II en 2 de Bach.


\textbf{Resultado de la tarea:}
No podemos determinarlo.


\textbf{Recursos materiales:}
Exámenes de Matemáticas II en PAUs de años anteriores.


\textbf{Resultado final:}
No podemos determinarlo.



\newpage
\section*{Actividad 8}
\paragraph{Enunciado y solución:}

\begin{enumerate}
	\item \textbf{Fumar: } Grave, sancionable con quedarse terminada la jornada escolar.

	\subitem Esta categorización puede estar incluida como \textit{muy grave} en el RRI y, por lo tanto, puede depender del centro que se considere.

	\item \textbf{Examen final en blanco: } Sin más información, no es falta.
 Si se hiciera con actitud desafiante.

	\item \textbf{Empujón: } Sin más información leve o incluso nada, aunque hay muchos posibles agravantes con lo que podría llegar a considerarse muy grave.

	\item \textbf{Pelotillas: } Grave.

	\item \textbf{Maricón: } Sin más información leve o nada, aunque hay muchos posibles agravantes.

	\item \textbf{Romper un cristal: } Es un delito.
 Además, si es el domicilio de alguien de la comunidad educativa, es una falta grave.
 
	\item \textbf{Bocadillo: } Leve.

	\item \textbf{Porno: } Si se queda en el ámbito privado entre los alumnos, no es nada.

	\item \textbf{Robo: } Muy grave.

	\item \textbf{Beso: } Nada.

	\item \textbf{Chanclas: } 
	\item \textbf{Correo: } Muy grave.

	\item \textbf{Copiar: } Grave o muy grave, a especificar en el RRI.

\end{enumerate}


\newpage
\section*{Actividad 9}
\paragraph{Enunciado: }

El profesor Prudencio Maduro es director de un centro de Secundaria.
%
Los padres de una alumna de 1o de ESO se dirigen a él para informarle de que han presentado una denuncia (cuya fotocopia le entregan) en la Comisaría de Policía porque, según ellos, su hija ha sufrido abusos sexuales por parte de tres compañeros de clase.
%
Dichos abusos no se han producido en el centro ni en horario lectivo.
%
Al parecer – según el relato de los padres- los tres chicos la abordaron un sábado por la tarde, en plena calle, la obligaron a entrar en un portal y la sometieron a tocamientos y a diversos abusos.
%
Además, creen que los agresores pueden haber grabado un video con parte de la agresión.
%
Los padres solicitan que el centro tome medidas y expulse inmediatamente a los tres alumnos.
%
De no ser así –dicen- su hija no volverá al centro y permanecerá en casa.
%
El director llama a los padres de los tres alumnos y les dice que deben entrevistarse ese mismo día con él.
%
Recibe en su despacho a cada alumno, separadamente, acompañado de sus padres, le informa de la denuncia y le pide su versión de los hechos.
%
Los tres alumnos coinciden en la misma versión: el sábado en el que ocurrieron los supuestos abusos habían quedado para ir a dar una vuelta con la alumna supuestamente agredida y con otras dos chicas, las cuales no van al mismo centro que ellos.
%
Estuvieron comiendo una hamburguesa y luego deambularon por varias calles hasta que esas dos chicas decidieron irse a casa.
%
Cuando se quedaron solos con la chica supuestamente agredida, ella empezó a tontear con uno de ellos y ambos se metieron en un portal abierto y allí estuvieron besándose y tocándose, de forma totalmente voluntaria por parte de la chica.
%
Como tardaban en salir, los otros dos chicos se impacientaron y les pidieron que salieran.
%
Entonces la chica empezó a provocarles y a incitarles a que se unieran a su compañero.
%
Por ello, entraron los dos y, a modo de broma, los rodearon con los brazos y estuvieron un rato “frotándose y tocándose”.
%
También bromearon diciendo que habían grabado con el móvil a la chica cuando no se daba cuenta pero era totalmente falso que lo hubieran hecho.

\paragraph{Solución:}

\begin{itemize}
	\item Hay o no hay parte de lesiones. Indicios físicos de que haya habido una agresión sexual.
	\subitem\textit{No ha habido nada.}
	\item Es un problema de la palabra de la alumna contra la de los alumnos.
	\subitem\textit{En principio, no habría que dar más credibilidad a unos que a otros. 
	%
	Un buen director debería tener olfato para darse cuenta de lo que puede haber}
	\item Hablar con la policía (guardia civil). 
	\subitem\textit{No había indicios claros según la guardia civil}.
	\item Al final la chica acabó confesando que se había inventado que fue agredida, porque fue consentido.
\end{itemize}

\newpage
\section*{Actividad 10}
\paragraph{Enunciado: }

Natalia del Monte trabaja como profesora de Matemáticas de ESO en un centro concertado. Es su primer año de trabajo y cobra, según le ha comunicado la Dirección del centro, catorce pagas de 1.838,27 euros. Asimismo, el director le ha explicado que su jornada anual es de 1.180 horas, de las cuales 850 son lectivas y el resto complementarias. Las horas lectivas semanales son 25 (5 cada días, de lunes a viernes) y, además, debe cumplir 8 horas semanales complementarias, dedicadas a tutoría, reuniones, atención a padres, etc. También tiene que vigilar a los alumnos en el patio durante el recreo dos días a la semana. En cuanto a las vacaciones, el director le advierte que en verano el colegio celebra todos los años un “Campus” durante el mes de julio, en el cual se imparten clases de recuperación a alumnos que tienen materias pendientes para septiembre. Esas clases, por un acuerdo entre el claustro y la dirección, son impartidas por los últimos profesores que se han incorporado en cada especialidad, por lo cual le toca a ella. Tendrá que dar 20 clases semanales, durante todo el mes, y recibirá una compensación de 5 euros por clase.

\textbf{¿Cuáles de las condiciones de trabajo de Natalia son aceptables y cuáles no,
desde el punto de vista legal?}

\paragraph{Solución:}

Las 14 pagas es correcto. Sus 25 horas lectivas también (está especificado en el convenio colectivo). 
%
