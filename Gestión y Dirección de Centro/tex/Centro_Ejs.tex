% -*- root: ../Centro.tex -*-

\section{Actividad 1}

\paragraph{¿Qué es un centro?}


\section{Actividad 2}


\paragraph{Reforma constitucional: da una nueva redacción, añadiendo, quitando o modificando su contenido, al artículo 27 de la Constitución:}

\textcolor{white}{a}\\\\
\textbf{1.} Todos tienen el derecho a la educación. \sout{Se reconoce la libertad de enseñanza.}\\
\textbf{2.} La educación tendrá por objeto el pleno desarrollo de la personalidad humana en el respeto a los principios democráticos de convivencia y a los derechos y libertades fundamentales. \\
\textbf{3.} Los poderes públicos garantizan el derecho que asiste a los padres para que sus hijos reciban una formación \sout{religiosa y moral} que esté de acuerdo con sus propias convicciones. \\
\textbf{4.} La enseñanza básica es obligatoria y gratuita.\\
\textbf{5.} Los poderes públicos garantizan el derecho de todos a la educación, mediante una programación general de la enseñanza, con participación efectiva de todos los sectores afectados y la creación de centros docentes. \\
\textbf{6.} Se reconoce a las personas físicas y jurídicas la libertad de creación de centros docentes, dentro del respeto a los principios constitucionales. \\
\textbf{7.} Los profesores, los padres y, en su caso, los alumnos intervendrán en el control y gestión de todos los centros sostenidos por la Administración con fondos públicos, en los términos que la ley establezca.\\
\textbf{8.} Los poderes públicos inspeccionarán y homologarán el sistema educativo para garantizar el cumplimiento de las leyes . \\
\textbf{9.} Los poderes públicos ayudarán a los centros docentes que reúnan los requisitos que la ley establezca.\\
\textbf{10.} Se reconoce la autonomía de las Universidades , en los términos que la ley establezca.\\


\section{Actividad 3}


\paragraph{¿En qué tiene razón Sofía y en qué no?}

\subparagraph{Tiene razón en:}

\begin{itemize}
	\item No dar francés. La autorización de la \textit{Dirección de Área Territorial} es para la profesora que está de baja. 
	%
	Como Francés y Matemáticas son asignaturas que no comparten especialidad, tendrían que conseguir una nueva autorización para la interina, y sino, buscar una nueva interina.

	Sin embargo, si la profesora supiera francés bien y se viera cómoda, podría llegar a dar la clase esgrimiendo el argumento de que los alumnos necesitan un docente.
\end{itemize}


\subparagraph{No tiene razón en:}

\begin{itemize}
	\item No participar del blog.
	\item No ayudar con la olimpiada de Matemáticas.
	\item No participar en el expediente disciplinario.
\end{itemize}


\paragraph{¿Qué respuesta deben darle el director y el jefe de estudios?}



\section{Actividad 4}


\begin{itemize}

	\item \textbf{Encomendar a un profesor del departamento que elabore una parte de la Memoria final de curso.}
	%
	Sí. El jefe de departamento sólo es el responsable pero es algo que se redacta entre todos.
	\item \textbf{Pedir a un profesor que le indique cómo lleva el desarrollo de sus programaciones didácticas.}
	%
	Sí.
	\item \textbf{Entrar en el aula mientras un profesor imparte clase para observar cómo lo hace (por ejemplo, porque ha recibido quejas de los alumnos sobre él).}
	%
	No, solamente el inspector tiene potestad para entrar en la clase del profesor.
	\item \textbf{Cambiar la calificación que ha puesto un profesor, si juzga que no ha evaluado adecuadamente a un alumno.}
	%
	No. Como mucho el jefe de estudios, con una reclamación escrita puede
	\item \textbf{Sancionar a un profesor que no cumple con sus obligaciones, previo informe razonado.}
	%
	No tiene ninguna autoridad sobre ningún profesor. 
	%
	El director sería quien puede sancionar (además, obviamente, de la inspección)
	\item \textbf{Encargar a un profesor que se ocupe de organizar una actividad extraescolar del departamento.}
	%
	Sí. Se lo puede encargar pero no imponer.
	\item \textbf{Asignar a cada profesor del departamento los grupos y materias a los que debe impartir clase.}
	%
	No. El reparto de los grupos se hace dentro del departamento. Hay una regulación acerca del reparto de grupos.
	\item \textbf{Determinar el día y la hora de reunión semanal del departamento.}
	%
	No, lo hace la jefatura de estudios. 
	\item \textbf{Decidir a qué se dedica el dinero asignado al departamento en los presupuestos del centro.}
	%
	No. Se decide entre todo el departamento, de mutuo acuerdo.
	\item \textbf{Informar al director sobre irregularidades cometidas por los profesores de su departamento.}
	%
	Sí.

\end{itemize}



